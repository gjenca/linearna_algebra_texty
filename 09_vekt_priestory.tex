\section{Vektorové priestory}

Na začiatku tejto časti si skúsme uvedomiť, aké objekty nazývame v tejto chvíli
„vektory“:
\begin{itemize}
    \item $n$-tice z $\mathbb{R}^n$
    \item orientované úsečky so spoločným počiatkom v rovine (označme množinu všetkých týchto úsečiek $S$)
\end{itemize}
S oboma typmi vektorov môžeme robiť isté operácie:
\begin{itemize}
    \item sčítať
    \item násobiť skalárom
\end{itemize}
Teraz ideme spraviť toto: pokúsime sa zachytiť vlastnosti sčítania a násobenia skalárom pre „oba typy vektorov“ do abstraktného pojmu. Skôr, ako to urobíme, musíme si vyjasniť, aký majú operácie sčítania a násobenia skalárom dátový typ.

Sčítanie vektorov je toto: predpis, ktorý nám hovorí, ako z dvoch vektorov vyrobiť vektor. V jazyku matematiky je sčítanie teda zobrazenie:
$$ + : V \times V \longrightarrow V $$
kde $V$ je množina všetkých vektorov, o ktorých uvažujeme. Napríklad ak $V = \mathbb{R}^2$, máme $+ : \mathbb{R}^2 \times \mathbb{R}^2 \longrightarrow \mathbb{R}^2$.

$$ +((1,2), (-2,3)) = (-1, 5) $$

Samozrejme, súčet vektorov $\vec{u}, \vec{v}$ nezapisujeme bežne ako $+(\vec{u}, \vec{v})$, ale $\vec{u} + \vec{v}$.

Podobne násobenie vektora skalárom je zobrazenie typu
$$ . : \mathbb{R} \times V \longrightarrow V $$
kde $V$ je množina všetkých vektorov.

Základná idea definície vektorového priestoru spočíva v tom, že na vyjadrenie podstaty pojmu vektora nám stačí to, aby sme zachytili to, ako sa tieto operácie správajú.
Teda definícia vektorového priestoru nehovorí nič o tom, čo je vektor, ale vyjadruje len vlastnosti operácií $+, .$.

\begin{definicia}\label{def:vektorovyPriestor}
Nech $V$ je neprázdna množina, vybavená
\begin{itemize}
    \item operáciou $+ : V \times V \longrightarrow V$ (sčítanie vektorov)
    \item operáciou $. : \mathbb{R} \times V \longrightarrow V$ (násobenie vektora skalárom)
    \item fixným vybratým prvkom $\vec{0} \in V$ (nulový vektor)
\end{itemize}
pričom platia tieto rovnosti, pre všetky $a, b \in \mathbb{R}$ a pre všetky $\vec{x}, \vec{y}, \vec{z} \in V$:

\begin{enumerate}[(V1)]
    \item \textbf{asociativita sčítania vektorov}
    $$ (\vec{x} + \vec{y}) + \vec{z} = \vec{x} + (\vec{y} + \vec{z}) $$
   
    \item \textbf{komutativita sčítania vektorov}
    $$ \vec{x} + \vec{y} = \vec{y} + \vec{x} $$
   
    \item \textbf{nulový vektor je neutrálny vzhľadom na sčítanie}
    $$ \vec{x} + \vec{0} = \vec{x} $$
   
    \item \textbf{opačný vektor}
    $$ \text{existuje } -\vec{x} \text{ také, že } \vec{x} + (-\vec{x}) = \vec{0} $$
   
    \item \textbf{kompatibilita s násobením skalárom}
    $$ (a . b) . \vec{x} = a . (b . \vec{x}) $$
   
    \item \textbf{distributivita násobenia skalárom vzhľadom na sčítanie vektorov}
    $$ a . (\vec{x} + \vec{y}) = a\vec{x} + a\vec{y} $$
   
    \item \textbf{distributivita násobenia skalárom vzhľadom na sčítanie skalárov}
    $$ (a + b)\vec{x} = a\vec{x} + b\vec{x} $$
   
    \item \textbf{jednotkový zákon}
    $$ 1 . \vec{x} = \vec{x} $$
   
\end{enumerate}
Potom hovoríme, že $V$ je \emph{vektorový priestor} alebo (čo je to isté)
\emph{lineárny priestor}. Prvky množiny $V$ sa nazývajú \emph{vektory}. Rovnosti z
tejto definície sa volajú \emph{axiómy vektorového priestoru}.
\end{definicia}

\begin{priklad}
Množina $\mathbb{R}^n$, vybavená sčítaním a násobením po zložkách, nulový vektor
je $(0,\dots,0)$.
\end{priklad}

\begin{priklad}
Množina $\rho_O$ všetkých orientovaných úsečiek v rovine s fixným počiatkom $O$,
sčítanie je dané rovnobežníkovým pravidlom, násobenie skalárom je škálovanie
úsečky, nulový vektor je $\ora{OO}$.
\end{priklad}

\begin{priklad}
Množina všetkých funkcií z $\mathbb{R} \rightarrow \mathbb{R}$, sčítanie je súčet funkcií, násobenie skalárom je násobenie funkcie číslom.
\end{priklad}

\begin{priklad}
Jednoprvková množina, povedzme $\{\star\}$; uvedomme si, že nutne $\vec{0} = \star$ a že sčítanie a násobenie skalárom sú jediné možné:
$$ + : \{\star\} \times \{\star\} \rightarrow \{\star\} $$
$$ . : \mathbb{R} \times \{\star\} \rightarrow \{\star\} $$
\end{priklad}

\textbf{Odčítanie vektorov:} Na každom vektorovom priestore $V$ môžeme zaviesť odvodenú operáciu rozdielu vektorov
$$ - : V \times V \rightarrow V $$
danú predpisom
$$ \vec{x} - \vec{y} := \vec{x} + (-1)\vec{y} $$

\begin{veta}\label{veta:vlastnostiVP}
Nech $V$ je vektorový priestor. Potom pre všetky $\vec{x}, \vec{y}, \vec{z}, \vec{u} \in V$ a pre všetky $a \in \mathbb{R}$ platí:
\begin{enumerate}
    \item[a)] ak $\vec{x} + \vec{y} = \vec{x} + \vec{z}$, potom $\vec{y} = \vec{z}$
    \item[b)] ak $a\vec{x} = a\vec{y}$ a $a \neq 0$, potom $\vec{x} = \vec{y}$
    \item[c)] $a . \vec{0} = \vec{0}$ a $0 . \vec{x} = \vec{0}$
    \item[d)] ak $a\vec{x} = \vec{0}$, potom $a = 0$ alebo $\vec{x} = \vec{0}$
    \item[e)] $-\vec{x} = (-1)\vec{x}$
    \item[f)] $a(\vec{x} - \vec{y}) = a\vec{x} - a\vec{y}$
\end{enumerate}
\end{veta}

\begin{proof}
~
\begin{enumerate}
\item[a)]
Nech $\vec{x} + \vec{y} = \vec{x} + \vec{z}$, potom zrejme $-\vec{x} + (\vec{x} + \vec{y}) = -\vec{x} + (\vec{x} + \vec{z})$.
Asociativita sčítania (V1) aplikovaná na oboch stranách nám dá rovnosť
$$ (*) \quad (-\vec{x} + \vec{x}) + \vec{y} = (-\vec{x} + \vec{x}) + \vec{z} $$

Keďže sčítanie vektorov je komutatívne (V2), platí 
$\vec{x} + (-\vec{x})=-\vec{x}+\vec{x}$ a z toho a (*) dostávame
$$ (**) \quad (\vec{x} + (-\vec{x})) + \vec{y} = (\vec{x} + (-\vec{x})) + \vec{z} $$

Podľa axiómy o opačnom vektore $\vec{x} + (-\vec{x}) = \vec{0}$ a (**) dostaneme
$$ \vec{0} + \vec{y} = \vec{0} + \vec{z} $$

Aplikujeme na oboch stranách komutativitu sčítania (V2):
$$ \vec{y} + \vec{0} = \vec{z} + \vec{0} $$

a teraz už stačí iba na oboch stranách aplikovať axiómu o nulovom vektore (V3), čím dostaneme $\vec{y} = \vec{z}$.

\item[b)] (menej podrobne)
$$ a\vec{x} = a\vec{y} $$
$$ \Downarrow $$
$$ \frac{1}{a}(a\vec{x}) = \frac{1}{a}(a\vec{y}) $$
s využitím predpokladu $a\neq 0$. Následne kompatibilita (V5) nám dá
$$ (\frac{1}{a} . a)\vec{x} = (\frac{1}{a} . a)\vec{y} $$
$$ 1 . \vec{x} = 1 . \vec{y} $$
a potom podľa jednotkového zákona (V8)
$$ \vec{x} = \vec{y} $$

\item[c) ... f)] Dôkazy vynechávame.
\end{enumerate}
\end{proof}

Pointa tohto prístupu k veci je v tom, že keďže dôkaz vety \ref{veta:vlastnostiVP}
používa iba definíciu vektorového priestoru, veta platí pre všetky vektorové
priestory (rovinné, $n$-tice, funkcie...), ktoré spĺňajú definíciu. Všetko, čo
dokážeme pre vektorové priestory, bude platiť pre každý partikulárny prípad
vektorového priestoru.

\begin{tikzpicture}[
    % Style definitions for nodes and arrows
    node distance = 1.5cm and 1cm,
    every node/.style = {align=center},
    % Style for the circular/oval bubbles
    bubble/.style = {draw, ellipse, thick, minimum width=6em, minimum height=3em, inner sep=3pt},
    % Style for the main arrows going up
    up arrow/.style = {->, >={Latex[scale=1.3]}, thick}
]

    % --- Define and place nodes ---
    % Top main node
    \node[bubble] (def) {Definícia \ref{def:vektorovyPriestor}};
    
    % Bottom row of nodes relative to the top node
    \node[bubble, below=2.2cm of def] (ntice) {n-tice $\R^n$};
    \node[bubble, left=of ntice] (rv) {rovinné\\vektory $\rho_O$};
    \node[bubble, right=of ntice] (funkcie) {funkcie $\R^\R$};

    % --- Draw upward arrows ---
    % Left curved arrow
    \draw[up arrow, bend left=15] (rv.north) to (def.west);
    % Middle straight arrow
    \draw[up arrow] (ntice.north) to (def.south);
    % Right curved arrow
    \draw[up arrow, bend right=15] (funkcie.north) to (def.east);

    % --- Add hand-drawn style annotation ---
    % Node containing the annotation text
    \node[above right=0.8cm and 1.2cm of funkcie] (annot text) {spĺňajú\\definíciu};
    
    % Draw curved arrow from annotation to diagram
    \draw[->, >={Latex[scale=1]}, thin, bend right=15] (annot text.south west) to ($(funkcie)!0.6!(def)$);

\end{tikzpicture}

Preto budeme odteraz postupovať tak, že budeme formulovať pojmy a tvrdenia/vety v
jazyku určenom definíciou vektorového priestoru.

\subsection{Podpriestor vektorového priestoru}
Začnime pojmom podpriestoru vektorového priestoru.

\begin{definicia}\label{def:podpriestor}
Nech $V$ je vektorový priestor. Množina $U \subseteq V$ sa nazýva \textbf{podpriestor}, ak platí:
\begin{enumerate}
    \item $\vec{0} \in U$
    \item Pre všetky dvojice $\vec{x}, \vec{y} \in U$ platí, že $\vec{x} + \vec{y} \in U$ (uzavretosť na sčítanie)
    \item Pre všetky $\vec{x} \in U, a \in \mathbb{R}$ platí, že $a\vec{x} \in U$ (uzavretosť na násobenie skalárom)
\end{enumerate}
\end{definicia}

\textbf{POZOR!} Neexistuje nič také ako „$U$ je podpriestor“ samé osebe, to slovné
spojenie nemá žiaden zmysel; podobne ako „7 je menšie“. 
Slovné spojenie „je podpriestor“ totiž vyjadruje vzťah množiny a vektorového
priestoru, podobne ako „je menšie“ vyjadruje vzťah medzi dvoma číslami.

\begin{priklad}
Uvažujme podmnožinu $U = \{(1,2), (0,0)\}$ vektorového priestoru $\mathbb{R}^2$.
Je $U$ podpriestor $\mathbb{R}^2$? Nie, pretože (medzi iným)
$$ (1,2) + (1,2) = (2,4) \notin U $$
\end{priklad}

\begin{veta}
Ak $U$ je podpriestor vektorového priestoru $V$, potom množina $U$ vybavená operáciami zdedenými z $V$ je tiež vektorový priestor.
\end{veta}

\begin{priklad}
Uvažujme podmnožinu $U$ vektorového priestoru $\rho_O$ „vektory v rovine s počiatkom
$O$“ takú, že $U$ obsahuje všetky vektory, ktorých koncový bod leží na nejakej
polpriamke s počiatkom v $O$.
\begin{center}
\begin{tikzpicture}
    \node[label=left:O, circle, fill, inner sep=1.5pt] (O) at (0,0) {};
    \draw[dashed] (O) -- (5,2.5);
    \draw[thick, -{Stealth[]}] (O) -- (2,1) node[midway, above] {$\vec{x}$};
    \draw[thick, -{Stealth[]}] (O) -- (1,0.5) node[midway, above] {};
    \draw[thick, -{Stealth[]}] (O) -- (3.5,1.75) node[midway, above] {};
\end{tikzpicture}
\end{center}

Je $U$ podpriestor $S$? Nie, pretože ak $\vec{x} \in U$, potom $(-1) . \vec{x} \notin
U$, čo odporuje bodu 3. definície \ref{def:vektorovyPriestor}.
\end{priklad}

Všimnite si, že $U$ spĺňa obidve zvyšné podmienky definície.

\begin{priklad}
Ak nahradíme polpriamku z predošlého príkladu priamkou, taká množina vektorov $U$ už
podpriestorom vektorového priestoru $\rho_O$ bude:
\begin{center}
\begin{tikzpicture}
    \node[label=above:O, circle, fill, inner sep=1.5pt] (O) at (0,0) {};
    \draw[dashed] (O) -- (5,2.5);
    \draw[thick, -{Stealth[]}] (O) -- (2,1) node[midway, above] {$\vec{x}$};
    \draw[thick, -{Stealth[]}] (O) -- (1,0.5) node[midway, above] {};
    \draw[thick, -{Stealth[]}] (O) -- (3.5,1.75) node[midway, above] {};
    \draw[dashed] (O) -- (-5,-2.5);
    \draw[thick, -{Stealth[]}] (O) -- (-2,-1) node[midway, above] {$(-1)\vec{x}$};
    \draw[thick, -{Stealth[]}] (O) -- (-1,-0.5) node[midway, above] {};
    \draw[thick, -{Stealth[]}] (O) -- (-3.5,-1.75) node[midway, above] {};
\end{tikzpicture}
\end{center}
\end{priklad}

\textbf{Otázka:} priamky prechádzajúce počiatkom sú teda podpriestory $\rho_O$. Aké
ďalšie podpriestory $\rho_O$ existujú?
Zrejme dva: $\{\vec{0}\}$ a $\rho_O$.

\begin{priklad}
Nech $U$ je podmnožina vektorového priestoru $\R^3$
$$ U = \{(s+2t, -t, 2s+t) : s, t \in \mathbb{R}\} $$
Presvedčme sa, že $U$ je podpriestorom $\R^3$.
\begin{itemize}
    \item[1)] $\vec{0} \in U$: ak $s=t=0$, potom $(0+2.0, -0, 2.0+0) = (0,0,0) \in U$.
    \item[2)] Nech $\vec{x}_1, \vec{x}_2 \in U$.
    $\vec{x}_1 \in U$ znamená, že $\vec{x}_1 = (s_1+2t_1, -t_1, 2s_1+t_1)$ pre nejaké $s_1, t_1 \in \mathbb{R}$.
    Podobne $\vec{x}_2 = (s_2+2t_2, -t_2, 2s_2+t_2)$ pre nejaké $s_2, t_2 \in \mathbb{R}$.
    Počítajme $\vec{x}_1 + \vec{x}_2$:
    $$ ((s_1+s_2) + 2(t_1+t_2), -(t_1+t_2), 2(s_1+s_2) + (t_1+t_2)) $$
    ak položíme $S = s_1+s_2$ a $T = t_1+t_2$, vidíme, že $\vec{x}_1+\vec{x}_2 \in U$.
    \item[3)] Ak $\vec{x} \in U$, tj. $\vec{x} = (s+2t, -t, 2s+t)$ a $a \in \mathbb{R}$.
    Potom $a\vec{x} = (a(s+2t), a(-t), a(2s+t)) = (as+2at, -at, 2as+at)$.
    Položme $S' = as$ a $T' = at$.
    $$ a\vec{x} = (S'+2T', -T', 2S'+T') \in U $$
    a hotovo.
\end{itemize}
\end{priklad}

\begin{priklad}
Uvažujme vektorový priestor všetkých funkcií $\R^\R$. Nech $P$ je podmnožina tvorená všetkými párnymi funkciami, teda:
$$ P = \{f \in \R^\R \mid \forall x \in \R: f(-x) = f(x)\} $$
Tvrdíme, že $P$ je podpriestor $\R^\R$. Overíme podmienky z definície podpriestoru:

\begin{enumerate}
    \item Nulová funkcia $0(x)=0$ patrí do $P$, lebo pre každé $x \in \R$ platí
    $0(-x)=0=0(x)$.
    \item Nech $f, g \in P$. Potom pre každé $x \in \R$ platí:
    $$ (f+g)(-x) = f(-x) + g(-x) = f(x) + g(x) = (f+g)(x) $$
    Teda $f+g \in P$.
    \item Nech $f \in P$ a $c \in \R$. Potom:
    $$ (cf)(-x) = c \cdot f(-x) = c \cdot f(x) = (cf)(x) $$
    Teda $cf \in P$.
\end{enumerate}
Zistili sme, že množina párnych funkcií je uzavretá na súčet aj skalárny násobok a
obsahuje nulový vektor, a teda tvorí podpriestor vektorového priestoru $\R^\R$.
\end{priklad}

\subsection{Lineárny obal}

\begin{definicia} \label{def:10.1}
Nech $V$ je vektorový priestor, nech $X=(\vec{x}_{1},...,\vec{x}_{n})$ je $n$-tica vektorov z $V$. Potom \textbf{lineárny obal} je množina vektorov
$$ \Lo(X)=\{a_{1}\vec{x}_{1}+...+a_{n}\vec{x}_{n} \mid a_{1},...,a_{n}\in \R\} $$
\end{definicia}

Vidíme, že predpis pre vytvorenie lineárneho obalu $n$-tice $X$ sa dá vyjadriť slovne ako \textit{všetky lineárne kombinácie vektorov $X$}. Analogicky môžeme definovať lineárny obal množiny vektorov.

\textbf{Cvičenie:} Ak $X=(\vec{u}, \vec{v})$ je dvojica vektorov v rovine, ako môže vyzerať $\Lo(X)$? Uvažujte prípady ako $\vec{u}=\vec{0}$, $\vec{u}=c\vec{v}$, $\vec{u} \neq c\vec{v}$.


\begin{priklad}
Skúsme si napísať ako vyzerá lineárny obal dvojice vektorov $((-1,2,3), (2,0,2))$ v $\R^{3}$.
\begin{align*}
\Lo(X) &= \{a_{1}(-1,2,3)+a_{2}(2,0,2) \mid a_{1},a_{2}\in \R\} \\
&= \{(-a_{1},2a_{1},3a_{1})+(2a_{2},0,2a_{2}) \mid a_{1},a_{2}\in \R\} \\
&= \{(-a_{1}+2a_{2},2a_{1},3a_{1}+2a_{2}) \mid a_{1},a_{2}\in \R\}
\end{align*}
\end{priklad}

Označme vektorový priestor všetkých funkcií $\R \rightarrow \R$ ako $\R^\R$.

\begin{priklad}
Ako vyzerá lineárny obal $n$-tice funkcií $(1,x,x^{2},...,x^{n})$? 
Tuto (trochu lajdácky) stotožňujeme funkciu v $\R^\R$ s jej predpisom.
$$ \Lo(1,x,x^{2},...,x^{n}) = \{a_{0}+a_{1}x+...+a_{n}x^{n} \mid a_{0},a_{1},...,a_{n}\in \R\} $$
To je množina všetkých polynómov stupňa nanajvýš $n$, označujeme ju $\R^{n}[x]$.
\end{priklad}

\begin{veta} \label{veta:10.2}
Nech $V$ je vektorový priestor, nech $X=(\vec{x}_{1},...,\vec{x}_{n})$ je $n$-tica vektorov z $V$. Potom $\Lo(X)$ je podpriestor $V$.
\end{veta}

\begin{proof}
Overíme podmienky podpriestoru:
\begin{enumerate}
    \item $\vec{0}\in \Lo(X)$, lebo $\vec{0}=0\vec{x}_{1}+...+0\vec{x}_{n}\in \Lo(X)$.
    \item Nech $\vec{u}, \vec{v} \in \Lo(X)$. Potom
    $$ \vec{u}=a_{1}\vec{x}_{1}+...+a_{n}\vec{x}_{n} $$
    pre nejaké $a_{1},...,a_{n}\in \R$ a
    $$ \vec{v}=b_{1}\vec{x}_{1}+...+b_{n}\vec{x}_{n} $$
    pre nejaké $b_{1},...,b_{n}\in \R$.
    Máme dokázať, že $\vec{u}+\vec{v}\in \Lo(X)$. Počítajme:
    \begin{align*}
    \vec{u}+\vec{v} &= (a_{1}\vec{x}_{1}+...+a_{n}\vec{x}_{n}) + (b_{1}\vec{x}_{1}+...+b_{n}\vec{x}_{n}) \\
    &= (a_{1}+b_{1})\vec{x}_{1}+...+(a_{n}+b_{n})\vec{x}_{n}\in \Lo(X)
    \end{align*}
    \item Nech $\vec{u}$ je ako v bode 2), $c\in \R$. Potom
    $$ c\vec{u}=c(a_{1}\vec{x}_{1}+...+a_{n}\vec{x}_{n})=(ca_{1})\vec{x}_{1}+...+(ca_{n})\vec{x}_{n}\in \Lo(X) $$
\end{enumerate}
\end{proof}

Vidíme teda, že každý lineárny obal $n$-tice je podpriestorom priestoru, z ktorého vyberáme. Teda medziiným je $\R^{n}[x]=\Lo(1,x,...,x^{n})$ podpriestorom $\R^{\R}$.

\begin{definicia} \label{def:10.3}
Hovoríme, že vektorový priestor $V$ je \textbf{konečnorozmerný}, ak existuje $n$-tica $X=(\vec{x}_{1},...,\vec{x}_{n})$ vektorov z $V$ taká, že $\Lo(X)=V$. Ak $V$ nie je konečnorozmerný, je \textbf{nekonečnorozmerný}.
\end{definicia}

\begin{veta} \label{veta:10.4}
Vektorový priestor $\R^{n}$ je konečnorozmerný pre každé $n \ge 1$.
\end{veta}

\begin{proof}
Vezmime vektory
\begin{align*}
\vec{e}_{1} &= (1,0,0,...,0) \\
\vec{e}_{2} &= (0,1,0,...,0) \\
&\vdots \\
\vec{e}_{n} &= (0,0,0,...,1)
\end{align*}
Tvrdíme, že $\R^n = \Lo(\vec{e}_1, ..., \vec{e}_n)$. Nech $\vec{v}=(a_{1},...,a_{n})$ je nejaký vektor z $\R^{n}$. Máme:
\begin{align*}
a_{1}\vec{e}_{1}+...+a_{n}\vec{e}_{n} &= a_{1}(1,0,...,0)+...+a_{n}(0,0,...,1) \\
&= (a_{1},0,...,0)+...+(0,0,...,a_{n}) \\
&= (a_{1},...,a_{n})=\vec{v}
\end{align*}
Teda $\vec{v} \in \Lo(\vec{e}_{1},...,\vec{e}_{n})$.
\end{proof}

Uvedomme si, že $\R^n[x]$ je tiež konečnorozmerný (lebo sme ho definovali ako lineárny obal nejakej $n$-tice vektorov).

V tomto momente vzniká prirodzená otázka: existuje nejaký priestor, ktorý nie je konečnorozmerný?
Odpoveď je, samozrejme, ,,Áno'': napríklad vektorový priestor $\R^\R$ nie je konečnorozmerný. Dôkaz presahuje zamýšľaný rámec prednášky, preto ho nespravíme.
Skoro všetky vektorové priestory na tejto prednáške budú však konečnorozmerné.

