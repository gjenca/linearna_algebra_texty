\section{Vektory a operácie s nimi}

Sila lineárnej algebry spočíva v tom, že umožňuje viac pohľadov na rovnaký pojem. Tieto pohľady sú veľmi silne prepojené - niekedy sa medzi nimi ani nerozlišuje a plynule sa prechádza z jedného do druhého.

\noindent Vektor môže byť:
\begin{itemize}
    \item[(*)] množina všetkých orientovaných úsečiek v rovine/priestore, ktoré majú rovnakú veľkosť a smer.
    \item[(**)] Zvoľme bod $O$ v rovine/priestore. Vektor je orientovaná úsečka s počiatkom v tomto bode (môžeme ju stotožniť s jej druhým koncovým bodom; potom vektor = bod).
    \item[(***)] Usporiadaná n-tica reálnych čísel; $n=2$ pre rovinu a $n=3$ pre priestor.
    \item[(****)] Prvok vektorového priestoru.
\end{itemize}

Zostaneme pri výklade pojmov v rovine; zovšeobecnenie do priestoru je priamočiare.

\noindent \textbf{Typografické pravidlo:}
Vektory budeme písať so šípkou: $\vec{x}, \vec{y}, \vec{u}$.



\subsection{Prechody medzi definíciami}
Vysvetlíme prechody medzi pohľadmi na vektor:
\begin{center}
    (*) $\xrightarrow{\text{výber počiatku}}$ (**) $\xrightarrow{\text{voľba súradnicových osí}}$ (***)
\end{center}

Všetci asi vieme, čo je úsečka; orientovaná úsečka je úsečka s vybratým krajným bodom. Keďže úsečka nenulovej dĺžky má dva krajné body, každej úsečke nenulovej dĺžky zodpovedajú dve orientované úsečky:

\begin{center}
\begin{tikzpicture}
    \node[circle, fill, inner sep=1.5pt, label=left:A] (A) at (0,0) {};
    \node[label=right:B] (B) at (2,0) {};
    \draw[-{Stealth[]}] (A) -- (B) node[midway, below] {$\ora{AB}$};

    \node[label=left:A] (A2) at (4,0) {};
    \node[circle, fill, inner sep=1.5pt, label=right:B] (B2) at (6,0) {};
    \draw[-{Stealth[]}] (B2) -- (A2) node[midway, below] {$\ora{BA}$};
\end{tikzpicture}
\end{center}

Podľa (*) je (jeden!) vektor množina všetkých ($\infty$) orientovaných úsečiek, ktoré majú rovnakú veľkosť a smer.

\begin{itemize}
    \item Veľkosť orientovanej úsečky je jej dĺžka.
    \item Čo je smer, je akosi tiež jasné.
\end{itemize}

Asi najčistejší spôsob, ako na úrovni geometrie popísať veľkosť a smer je, že dve orientované úsečky $\ora{AB}$ a $\ora{CD}$ majú rovnakú veľkosť a smer práve vtedy, keď platí jedna z týchto možností:
\begin{itemize}
    \item $|AB|=|CD|=0$ - nulový vektor.
    \item $A \neq B, C \neq D$ a $ABDC$ je rovnobežník s uhlopriečkami $AD, BC$.
\end{itemize}

\begin{center}
\begin{tikzpicture}[scale=1.5]
    \node[circle, fill, inner sep=1.5pt, label=left:A] (A) at (0,0) {};
    \coordinate[label=right:B] (B) at (2,0);
    \node[circle, fill, inner sep=1.5pt, label=above:C] (C) at (0.5,1) {};
    \coordinate[label=above:D] (D) at (2.5,1);
    \draw[thick, -{Stealth[]}] (A) -- (B);
    \draw[thick, -{Stealth[]}] (C) -- (D);
    \draw (A) -- (C);
    \draw (B) -- (D);
\end{tikzpicture}
\end{center}

Teda vektor v zmysle (*) si môžeme predstaviť takto:
\begin{center}
\begin{tikzpicture}
    \foreach \x/\y in {0/0, 0.5/1, -0.2/0.8, 1.5/0.2, 1.8/1.2}
    {
        \node[circle, fill, inner sep=1.5pt] at (\x,\y) {};
        \draw[thick, -{Stealth[]}] (\x,\y) -- (\x+1,\y+0.7);
    }
    \node at (5.0,0.7) {$\left.\begin{array}{c} \\ \\ \end{array}\right\}$ všetky takéto šípky};
\end{tikzpicture}
\end{center}

Definícia (*) je úplne v poriadku, ale manipuluje sa s tým pojmom zle, pretože každý vektor je potom nekonečná množina.
Poradíme si takto: vyberieme v rovine jeden ľubovoľný bod, nazveme ho "počiatok" a budeme ho označovať $O$. V každej množine orientovaných úsečiek, ktorá je vektorom v zmysle (*), máme práve jednu orientovanú úsečku, ktorá má počiatok v $O$. Túto vyberieme z vektora v zmysle (*) a máme vektor v zmysle (**).

\begin{center}
\begin{tikzpicture}
    \node[label=left:O, circle, fill, inner sep=1.5pt] (O) at (0,0) {};
    \draw[thick, -{Stealth[]}] (O) -- (2,1) node[midway, above] {$\vec{u}$};
    \draw[thick, -{Stealth[]}] (O) -- (1,2) node[midway, left] {$\vec{v}$};
    \draw[thick, -{Stealth[]}] (O) -- (1,-2) node[midway, left] {$\vec{x}$};
\end{tikzpicture}
\end{center}

Všimnime si, že jeden z vektorov zodpovedá orientovanej úsečke $\ora{OO}$; hovoríme mu nulový vektor a značíme ho $\vec{0}$.

Umiestnime teraz v rovine dve kópie číselnej osi: vodorovnú a zvislú tak, aby sa pretínali v bode $O$.
\begin{center}
\begin{tikzpicture}[scale=0.8]
    \draw[<->] (-4.5,0) -- (4.5,0) node[right] {$x$};
    \draw[<->] (0,-4.5) -- (0,4.5) node[above] {$y$};
    \foreach \x in {-4,-3,...,-1,1,2,...,4}
    {
        \draw (\x, 0.1) -- (\x, -0.1) node[below] {\tiny \x};
    }
    \foreach \y in {-4,-3,...,-1,1,2,...,4}
    {
        \draw (0.1, \y) -- (-0.1, \y) node[left] {\tiny \y};
    }
    \node[circle, fill, inner sep=1.5pt] at (0,0) {};
    \draw[thick, -{Stealth[]}] (0,0) -- (3.5,2.5) node[midway, above left] {$\vec{u}$};
    \draw[dashed] (3.5,2.5) -- (3.5,0);
    \draw[dashed] (3.5,2.5) -- (0,2.5);
    \node at (7, 2.5) {$\longrightarrow \left(3\frac{1}{2}, 2\frac{1}{2}\right) \in \R^2$};
    \node[red] at (5,-2) {$\searrow$ ,,súradnicové osi''};
\end{tikzpicture}
\end{center}

Premietnutím koncového bodu orientovanej úsečky reprezentujúcej vektor v pravom uhle na osi určíme usporiadanú dvojicu reálnych čísel a naopak, z usporiadanej dvojice reálnych čísel vieme zrejmým spôsobom dostať vektor ako orientovanú úsečku s počiatkom v bode $O$. Pritom nulový vektor $\vec{0}$ zodpovedá dvojici $(0,0)$.

Voľba osí v rovine nám určuje bijekciu:
$$ \text{vektory v rovine} \longleftrightarrow \R^2 $$

To, že sa body v rovine dajú jednoducho (a užitočne) vyjadrovať ako dvojice čísel je
prekvapujúco mladý objav - pochádza z roku 1637 a vymyslel ho René
Descartes. Zaujímavé je, že v tom čase sa už vyše tisíc rokov používali sférické
súradnice pre určovanie polohy na Zemi pomocou rovnobežiek a poludníkov.

Zaveďme teraz terminológiu týkajúcu sa $\R^n$, ktorú budeme používať:
Pre $\vec{x}=(x_1, \dots, x_n) \in \R^n$, $x_i$ je $i$-ta zložka vektora $\vec{x}$.



\subsection{Operácie s vektormi}

\subsubsection{Násobenie vektora skalárom}
\begin{definition}[Násobenie geometrického vektora skalárom]
Ak $\alpha \in \R$ (skalár) a $\vec{v}$ je vektor, potom $\alpha\vec{v}$ je vektor $|\alpha|$-krát predĺžený/skrátený.
\begin{itemize}
    \item ak $\alpha > 0$, $(\alpha\vec{v})$ a $\vec{v}$ sú orientované rovnako,
    \item ak $\alpha < 0$, $(\alpha\vec{v})$ a $\vec{v}$ sú orientované opačne,
    \item ak $\alpha=0$, $\alpha\vec{v} = \vec{0}$.
\end{itemize}
\end{definition}

\begin{example}
~\\
\begin{center}
\begin{tikzpicture}
    \draw[gray] (-4,0) -- (4,0);
    \node[label=below:$O$, circle, fill, inner sep=1.5pt] at (0,0) {};
    \draw[thick, -{Stealth[]}] (0,0) -- (2,0) node[midway, above] {$\vec{v}$};
    \draw[thick, -{Stealth[]}] (0,0) -- (4,0) node[midway, above] {$2\vec{v}$};
    \draw[thick, -{Stealth[]}] (0,0) -- (-3,0) node[midway, above] {$-\frac{3}{2}\vec{v}$};
\end{tikzpicture}
\end{center}
\end{example}

\begin{definition}[násobenie vektora skalarom v $\R^n$]
Nech $\vec{x}=(x_1, \dots, x_n) \in \R^n$ a nech $\alpha \in \R$. Potom
$$ \alpha . \vec{x} = (\alpha x_1, \dots, \alpha x_n) $$
\end{definition}
\begin{example}
\begin{align*}
(-2).(2,-1,0)&=(-4,2,0)\\
\sqrt{3}.(\sqrt{3},-\frac{1}{\sqrt{3}},2)&=(3,-1,2.\sqrt{3})
\end{align*}
\end{example}
Ak si teraz zvolíme v rovine súradnicové osi, dostávame tým bijekciu medzi
geometrickými vektormi a prvkami $\R^2$. Táto bijekcia zachováva násobenie skalárom,
čo znamená, že 
\begin{center}
\begin{tabular}{m{6cm} l}
    % Row 1: Geometric vectors
    \begin{tikzpicture}
        \node[circle, fill, inner sep=1.5pt, label=below left:$O$] (O) at (0,0) {};
        \draw[thick, -{Stealth[]}] (O) -- (1.5,1) node[pos=0.7, above] {$\vec{v}$};
        \draw[thick, -{Stealth[]}] (O) -- (3,2) node[pos=0.7, above] {$2\vec{v}$};
    \end{tikzpicture}
    & Geometria \\
    
    % Row 2: Geometric vectors with axes
    \begin{tikzpicture}[scale=0.7]
        \draw[->] (-0.5,0) -- (6.5,0);
        \draw[->] (0,-0.5) -- (0,4.5);
        \node[below left, circle, fill, inner sep=1.5pt, label=below left:$O$] at (0,0) {};
        \foreach \x in {3,6} \draw (\x,0.1) -- (\x,-0.1) node[below] {\x};
        \foreach \y in {2,4} \draw (0.1,\y) -- (-0.1,\y) node[left] {\y};

        \draw[thick, -{Stealth[]}] (0,0) -- (3,2) node[pos=0.6, above] {$\vec{v}$};
        \draw[thick, -{Stealth[]}] (0,0) -- (6,4) node[pos=0.6, above] {$2\vec{v}$};
        \draw[dashed] (3,2) -- (3,0);
        \draw[dashed] (3,2) -- (0,2);
        \draw[dashed] (6,4) -- (6,0);
        \draw[dashed] (6,4) -- (0,4);
    \end{tikzpicture}
    & Voľba osí \\
    
    % Row 3: Algebraic operation
    $2.(3,2)=(6,4)$
    & Algebra
\end{tabular}
\end{center}Vidíme, že operácii škálovania (násobenie $\alpha \in \R$) (geometrická operácia) zodpovedá vynásobenie usporiadanej n-tice skalárom $\alpha$ vo všetkých zložkách n-tice.

Je dôležité si teraz uvedomiť, že korešpondencia medzi vektormi v geometrickom zmysle
a usporiadanými n-ticami závisí na voľbe osí, osi môžu mať rôznu mierku a môžu byť
dokonca trochu otočené.
\begin{center}
\begin{tikzpicture}[scale=0.9, rotate=20]
    \draw[<->] (-2.5,0) -- (3.5,0);
    \draw[<->] (0,-2.5) -- (0,5.5);
    \foreach \x in {-2,-1,1,2,3} \draw (\x,0.1) -- (\x,-0.1) node[below] {\tiny \x};
    \foreach \y in {-1,-0.5,1,2} \draw (0.1, \y*2) -- (-0.1, \y*2) node[left] {\tiny \y};
    
    \node[circle, fill, inner sep=1.5pt] at (0,0) {};
    \draw[thick, -{Stealth[]}] (0,0) -- (-0.5*2, 2*1) node[pos=0.7, right] {$\vec{u}$};
    \draw[dashed] (-0.5*2, 2*1) -- (-0.5*2, 0);
    \draw[dashed] (-0.5*2, 2*1) -- (0, 2*1);
\end{tikzpicture}
\end{center}
Inou voľbou osí sa bijekcia (vektory v rovine $\leftrightarrow \R^2$) zmení, ale to, že škálovanie zodpovedá násobeniu skalárom po zložkách bude stále platiť.

\subsubsection{Vlastnosti násobenia vektora skalárom}
Pre všetky vektory $\vec{x} \in \R^n$ a $a,b \in \R$ platí:
\begin{itemize}
    \item $(a . b) . \vec{x} = a . (b . \vec{x})$
\end{itemize}
\textbf{Prečo?} Nech $\vec{x}=(x_1, \dots, x_n)$.
\begin{align*}
(a . b) . \vec{x} &= (a . b) . (x_1, \dots, x_n) = ((ab)x_1, \dots, (ab)x_n) \\
&= (a(bx_1), \dots, a(bx_n)) = a(bx_1, \dots, bx_n) \\
&= a(b(x_1, \dots, x_n)) = a(b . \vec{x})
\end{align*}
\begin{itemize}
    \item Pre všetky $\vec{x} \in \R^n$ platí $0\vec{x} = \vec{0}$, $1\vec{x} = \vec{x}$.
\end{itemize}



\subsubsection{Sčítanie vektorov}
Geometricky sa operácia sčítania vektorov zavádza rovnobežníkovým pravidlom:
\begin{center}
\begin{tikzpicture}
    \coordinate[circle, fill, inner sep=1.5pt] (O) at (0,0);
    \coordinate (U) at (2,1);
    \coordinate (V) at (1,2);
    \coordinate (SUM) at (3,3);
    \draw[thick, -{Stealth[]}] (O) -- (U) node[below right] {$\vec{u}$};
    \draw[thick, -{Stealth[]}] (O) -- (V) node[above left] {$\vec{v}$};
    \draw[thick, -{Stealth[]}] (O) -- (SUM) node[above right] {$\vec{u}+\vec{v}$};
    \draw[dashed] (U) -- (SUM);
    \draw[dashed] (V) -- (SUM);
\end{tikzpicture}
\end{center}
Ak je jeden z vektorov $\vec{0}$, definujeme prirodzene $\vec{u}+\vec{0}=\vec{u}$.

Na algebraickej strane tejto geometrickej operácii zodpovedá sčítanie po zložkách.
\begin{center}
\begin{tikzpicture}[scale=0.8]
    \draw[->] (-0.5,0) -- (4.5,0);
    \draw[->] (0,-0.5) -- (0,4.5);
    \node[circle, fill, inner sep=1.5pt] at (0,0) {};
    \foreach \x in {1,2,3,4} \draw (\x,0.1) -- (\x,-0.1) node[below] {\tiny \x};
    \foreach \y in {1,2,3,4} \draw (0.1,\y) -- (-0.1,\y) node[left] {\tiny \y};
    \draw[thick, -{Stealth[]}] (0,0) -- (1,2) node[above left] {$\vec{u}$};
    \draw[thick, -{Stealth[]}] (0,0) -- (2,1) node[below right] {$\vec{v}$};
    \draw[thick, -{Stealth[]}] (0,0) -- (3,3) node[above right] {$\vec{u}+\vec{v}$};
    \draw[dashed] (1,2) -- (3,3);
    \draw[dashed] (2,1) -- (3,3);
\end{tikzpicture}
\end{center}
$\vec{u}=(1,2), \vec{v}=(2,1) \implies \vec{u}+\vec{v}=(1+2, 2+1)=(3,3)$.\\
Opäť, ako v prípade násobenia skalárom, tento vzťah medzi geometrickou operáciou sčítania vektorov a algebraickou operáciou sčítania po zložkách nezávisí na voľbe osí.

\begin{definition}[Sčítanie vektorov v $\R^n$]
Nech $\vec{x}=(x_1, \dots, x_n) \in \R^n$ a $\vec{y}=(y_1, \dots, y_n) \in \R^n$. Potom
$$ \vec{x}+\vec{y} = (x_1+y_1, \dots, x_n+y_n) $$
\end{definition}

\begin{example}
$(1,-3,2,0) + (-1,3,-1,4) = (0,0,1,4)$
\end{example}

\subsubsection{Vlastnosti sčítania vektorov v $\R^n$}
Pre všetky vektory $\vec{x}, \vec{y}, \vec{z} \in \R^n$ platia rovnosti:
\begin{itemize}
    \item $(\vec{x}+\vec{y})+\vec{z} = \vec{x}+(\vec{y}+\vec{z})$ (asociativita)
    \item $\vec{x}+\vec{y} = \vec{y}+\vec{x}$ (komutativita)
    \item $\vec{x}+\vec{0} = \vec{x}$
    \item $\vec{x}+(-1)\vec{x} = \vec{0}$
\end{itemize}
Dôkaz sa robí priamočiaro, napr. komutativita vektorov:
$$ \vec{x}+\vec{y} = (x_1+y_1, \dots, x_n+y_n) = (y_1+x_1, \dots, y_n+x_n) = \vec{y}+\vec{x} $$
pričom sme využili komutativitu sčítania reálnych čísel. Podobne pre ostatné rovnosti.



Násobenie skalárom a sčítanie vektorov sú navzájom prepojené pomocou distributivity.
\begin{itemize}
    \item Pre všetky $a \in \R$ a $\vec{x}, \vec{y} \in \R^n$ platí: $a(\vec{x}+\vec{y}) = a\vec{x} + a\vec{y}$
    \item Pre všetky $a,b \in \R$ a $\vec{x} \in \R^n$ platí: $(a+b)\vec{x} = a\vec{x} + b\vec{x}$
\end{itemize}
Násobenie skalárom má prednosť pred sčítaním vektorov.

\noindent\textbf{POZOR:} Neexistuje nič také ako násobenie vektorov medzi sebou!

Nič nám nebráni zaviesť odčítanie vektorov $\vec{x}-\vec{y}$ definované ako $\vec{x}-\vec{y} := \vec{x} + (-1)\vec{y}$. Je to teda odvodená operácia zavedená pomocou sčítania a násobenia $(-1)$. Samozrejme, ako ľahko vidieť, odčítanie vektorov prebieha tiež po zložkách:
$$ \vec{x}-\vec{y} = (x_1-y_1, \dots, x_n-y_n) $$



\subsection{Vektory z $\R^n$ ako stĺpce}
Odteraz až do konca letného semestra budeme na tomto predmete stotožňovať
usporiadané $n$-tice reálnych čísel so stĺpcovými vektormi (maticami typu $n \times 1$).
$$ \vec{x} = (x_1, \dots, x_n) \in \R^n \quad \iff \quad \vec{x} = \begin{pmatrix} x_1 \\ \vdots \\ x_n \end{pmatrix} $$
Teda prvok množiny $\R^n$ môže byť zapísaný ako riadok \underline{s čiarkami}, alebo
ako stĺpec, oba zápisy označujú tú istú vec:
\[
(1,-17,0,\frac{8}{3})=
\begin{pmatrix}
1\\-17\\0\\\frac{8}{3}
\end{pmatrix}
\]
Budeme plynule prechádzať medzi týmito dvoma spôsobmi zápisu usporiadaných $n$-tíc.


\subsection{Lineárne kombinácie}
\begin{definition}[Lineárna kombinácia]
Nech $\vec{v}_1, \dots, \vec{v}_m \in \R^n$ a $a_1, \dots, a_m \in \R$.
Potom \emph{lineárna kombinácia} vektorov $\vec{v}_1, \dots, \vec{v}_m$ s koeficientami $a_1, \dots, a_m$ je vektor
$$ a_1\vec{v}_1 + a_2\vec{v}_2 + \dots + a_m\vec{v}_m $$
\end{definition}

\begin{example}
$\vec{v}_1 = (1,3,4), \vec{v}_2 = (2,0,1)$ v $\R^3$.
$a_1 = 3, a_2 = 2$.
$a_1\vec{v}_1 + a_2\vec{v}_2 = 3(1,3,4) + 2(2,0,1) = (3,9,12) + (4,0,2) = (7,9,14)$.
\end{example}

\begin{example}
Zistite, či je $\vec{u}=(1,2) \in \R^2$ lineárnou kombináciou vektorov $\vec{v}_1=(1,-1)$ a $\vec{v}_2=(2,5)$ a určite koeficienty tejto lineárnej kombinácie.

Hľadáme $a_1, a_2 \in \R$ také, že platí:
$$ a_1 \vec{v}_1 + a_2 \vec{v}_2 = \vec{u} $$
Zapíšeme problém pomocou stĺpcových vektorov:
$$ a_1 \begin{pmatrix} 1 \\ -1 \end{pmatrix} + a_2 \begin{pmatrix} 2 \\ 5 \end{pmatrix} = \begin{pmatrix} 1 \\ 2 \end{pmatrix} $$
Podľa definície násobenia skalárom:
$$ \begin{pmatrix} a_1 \\ -a_1 \end{pmatrix} + \begin{pmatrix} 2a_2 \\ 5a_2 \end{pmatrix} = \begin{pmatrix} 1 \\ 2 \end{pmatrix} $$
Podľa definície sčítania vektorov:
$$ \begin{pmatrix} a_1 + 2a_2 \\ -a_1 + 5a_2 \end{pmatrix} = \begin{pmatrix} 1 \\ 2 \end{pmatrix} $$
Dva vektory sa rovnajú, ak sa rovnajú po zložkách:
\begin{align*}
    a_1 + 2a_2 &= 1 \\
    -a_1 + 5a_2 &= 2
\end{align*}
Aha! Sústava lineárnych rovníc.
Sčítaním oboch rovníc dostaneme:
$$ 7a_2 = 3 \implies a_2 = \frac{3}{7} $$
Dosadením do prvej rovnice:
$$ a_1 + 2\left(\frac{3}{7}\right) = 1 \implies a_1 + \frac{6}{7} = 1 \implies a_1 = 1 - \frac{6}{7} = \frac{1}{7} $$
Koeficienty sú $a_1=1/7$ a $a_2=3/7$.
Skúška:
$$ \frac{1}{7} \begin{pmatrix} 1 \\ -1 \end{pmatrix} + \frac{3}{7} \begin{pmatrix} 2 \\ 5 \end{pmatrix} = \begin{pmatrix} 1/7 \\ -1/7 \end{pmatrix} + \begin{pmatrix} 6/7 \\ 15/7 \end{pmatrix} = \begin{pmatrix} 7/7 \\ 14/7 \end{pmatrix} = \begin{pmatrix} 1 \\ 2 \end{pmatrix} $$
\end{example}
\begin{example}
Uvažujme vektory v rovine s počiatkom v bode $O$.
Nech $ABCD$ je štvorec so stredom $O$.
\begin{center}
\begin{tikzpicture}[rotate=15]
    % Define square vertices
    \coordinate (A) at (0, 4);
    \coordinate (B) at (0, 0);
    \coordinate (C) at (4, 0);
    \coordinate (D) at (4, 4);

    % Draw square and label vertices
    \draw (A) -- (B) -- (C) -- (D) -- cycle;
    \node[left=3pt] at (A) {A};
    \node[below=3pt] at (B) {B};
    \node[right=3pt] at (C) {C};
    \node[above=3pt] at (D) {D};

    % Define interior point O at the center (2,2)
    % Label 'O' is now on the left.
    \node[circle, fill, inner sep=1.5pt, label={[label distance=2pt]left:O}] (O) at (2, 2) {};
    
    % Define midpoint S(BC)
    \node[inner sep=0,label={[label distance=2pt]below right:S}] (S) at ($(B)!.5!(C)$) {};

    % Draw vectors (thick, with \bullet start, but no \vec labels)
    \draw[thick, -{Stealth[]}] (O) -- (C);
    \draw[thick, -{Stealth[]}] (O) -- (D);
    \draw[thick, -{Stealth[]}] (O) -- (S);
\end{tikzpicture}
\end{center}
Nech $S$ je stred strany $BC$. Vyjadrime vektor $\ora{OS}$ ako lineárnu kombináciu
vektorov $\ora{OC},\ora{OD}$.

Najskôr si uvedomme, že $\ora{OB}=(-1).\ora{OD}$. 
\begin{center}
\begin{tikzpicture}[rotate=15]
    % Define square vertices
    \coordinate (A) at (0, 4);
    \coordinate (B) at (0, 0);
    \coordinate (C) at (4, 0);
    \coordinate (D) at (4, 4);

    % Draw square and label vertices
    \draw (A) -- (B) -- (C) -- (D) -- cycle;
    \node[left=3pt] at (A) {A};
    \node[below=3pt] at (B) {B};
    \node[right=3pt] at (C) {C};
    \node[above=3pt] at (D) {D};

    % Define interior point O at the center (2,2)
    % Label 'O' is now on the left.
    \node[circle, fill, inner sep=1.5pt, label={[label distance=2pt]left:O}] (O) at (2, 2) {};
    
    % Define midpoint S(BC)
    \node[inner sep=0,label={[label distance=2pt]below right:S}] (S) at ($(B)!.5!(C)$) {};

    % Draw vectors (thick, with \bullet start, but no \vec labels)
    \draw[thick, -{Stealth[]}] (O) -- (C);
    \draw[thick, -{Stealth[]}] (O) -- (D);
    \draw[thick, -{Stealth[]},color=gray] (O) -- (B);
    \draw[thick, -{Stealth[]}] (O) -- (S);
\end{tikzpicture}
\end{center}
Vektory $\ora{OC}$ a $\ora{OB}$ sú kolmé a majú rovnakú dĺžku. Preto koncové vrcholy
vektorov $\ora{OB},\ora{OC},\ora{OB}+\ora{OC}$ spolu s bodom $O$ tvoria štvorec.
\begin{center}
\begin{tikzpicture}[rotate=15]
    % Define square vertices
    \coordinate (A) at (0, 4);
    \coordinate (B) at (0, 0);
    \coordinate (C) at (4, 0);
    \coordinate (D) at (4, 4);
    \coordinate (E) at (2, -2);


    % Draw square and label vertices
    \draw (A) -- (B) -- (C) -- (D) -- cycle;
    \node[left=3pt] at (A) {A};
    \node[below=3pt] at (B) {B};
    \node[right=3pt] at (C) {C};
    \node[above=3pt] at (D) {D};
    \draw (B) -- (E) -- (C);
    \node[below=3pt] at (E) {$\ora{OB}+\ora{OC}$};

    % Define interior point O at the center (2,2)
    % Label 'O' is now on the left.
    \node[circle, fill, inner sep=1.5pt, label={[label distance=2pt]left:O}] (O) at (2, 2) {};
    
    % Define midpoint S(BC)
    \node[inner sep=0,label={[label distance=2pt]below right:S}] (S) at ($(B)!.5!(C)$) {};

    % Draw vectors (thick, with \bullet start, but no \vec labels)
    \draw[thick, -{Stealth[]}] (O) -- (C);
    \draw[thick, -{Stealth[]},color=gray] (O) -- (E);
    \draw[thick, -{Stealth[]}] (O) -- (D);
    \draw[thick, -{Stealth[]},color=gray] (O) -- (B);
    \draw[thick, -{Stealth[]}] (O) -- (S);
\end{tikzpicture}
\end{center}
Pritom bod $S$ je stredom tohto štvorca, teda
\[
\ora{OS}=\frac{1}{2}(\ora{OB}+\ora{OC})
\]
a môžeme použiť pravidlá o počítaní s vektormi
\begin{multline*}
\frac{1}{2}.(\ora{OB}+\ora{OC})=\frac{1}{2}.\ora{OB}+\frac{1}{2}.\ora{OC}=
\frac{1}{2}.((-1).\ora{OD})+\frac{1}{2}.\ora{OC}=\\
(\frac{1}{2}.(-1)).\ora{OD}+\frac{1}{2}.\ora{OC}=-\frac{1}{2}\ora{OD}+\frac{1}{2}\ora{OC}
\end{multline*}
Teda
\begin{equation}\label{eq:squareCombination}
\ora{OS}=-\frac{1}{2}\ora{OD}+\frac{1}{2}\ora{OC},
\end{equation}
čo je hľadaná lineárna kombinácia.

Skúste si teraz rozmyslieť, aká je poloha vektorov 
$-\frac{1}{2}\ora{OD},\frac{1}{2}\ora{OC}$ v rovine. Keďže platí \eqref{eq:squareCombination},
spolu s bodmi $O$ a $S$ by mali ich
koncové body tvoriť
rovnobežník. Je to pravda?
\end{example}
