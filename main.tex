\documentclass[11pt,a4paper]{article}

% PACKAGES
\usepackage[utf8]{inputenc}
\usepackage[T1]{fontenc}
\usepackage[slovak]{babel}
\usepackage{amsmath, amssymb}
\usepackage{amsthm} % For theorem environments
\usepackage{graphicx}
\usepackage{enumerate}
\usepackage{tikz}
\usetikzlibrary{tikzmark}
\usetikzlibrary{arrows.meta}
\usetikzlibrary{calc}
\usepackage[all,pdf,2cell]{xy}
\usepackage{soul}
\usepackage{framed}
\usepackage{array}
\usepackage{todonotes}

% THEOREM ENVIRONMENTS SETUP
% Style for definitions and examples (upright text, bold title)
\theoremstyle{definition} 
\newtheorem{definition}{Definícia}[section]
\newtheorem{example}[definition]{Príklad}

% Style for theorems (italicized text, bold title)
\theoremstyle{plain} 
\newtheorem{tvrdenie}[definition]{Tvrdenie}
\newtheorem{veta}[definition]{Veta}

% Style for remarks and questions (upright text, italic title)
\theoremstyle{remark} 
\newtheorem*{question}{Otázka}

% Macros

% Simple abbreviations
\newcommand{\id}{\mathrm{id}}
\newcommand{\Nat}{\mathbb{N}}
\newcommand{\Set}{\mathbf{Set}}
\newcommand{\R}{\mathbb{R}}
\newcommand{\ora}[1]{\overrightarrow{#1}}

% More complex stuff
% Restriction of a function (from stackexchange)
\newcommand\restr[2]{{% we make the whole thing an ordinary symbol
  \left.\kern-\nulldelimiterspace % automatically resize the bar with \right
  #1 % the function
  \littletaller % pretend it's a little taller at normal size
  \right|_{#2} % this is the delimiter
  }}
\newcommand{\littletaller}{\mathchoice{\vphantom{\big|}}{}{}{}}

% DOCUMENT START
\begin{document}
\title{Lineárna algebra 1\\(texty k prednáškam)}
\author{Gejza Jenča}
\date{Verzia 1}
\maketitle
\input{01_mnoziny.tex}
\section{Zobrazenia}

Zobrazenia sa často nazývajú funkcie. Obe slová znamenajú to isté, obvykle však funkcia zobrazuje do čísel ($\mathbb{N}, \mathbb{R}, \mathbb{C}, \dots$).
Ktoré slovo sa použije je otázkou konvencie v danej časti matematiky.

\begin{definicia}[zobrazenie]\label{def:zobrazenie}
Nech $A, B$ sú množiny.
\emph{Zobrazenie} $f$ z $A$ do $B$ je predpis, ktorý každému prvku
$A$ priradí nejaký prvok $B$.
Zapisujeme 
\[
f \colon A \to B.
\]
$A$ je \emph{definičný obor}, $B$ je \emph{koobor}.
\end{definicia}

To znamená, že ak chceme špecifikovať nejaké zobrazenie $f$, musíme špecifikovať tri
veci:
\begin{enumerate}
    \item Z ktorej množiny sa zobrazuje (definičný obor).
    \item Do ktorej množiny sa zobrazuje (koobor).
    \item Predpis, ktorý nám určí, pre každý prvok definičného oboru ktorý prvok sa mu má zobraziť.
\end{enumerate}

Predpis môže byť daný rôzne.
Napríklad ak $f \colon \mathbb{R} \to \mathbb{R}$ môžeme predpis niekedy poznamenať pomocou vzorca, napr.
\[
f(x) = \sqrt{x^2+1}
\]
Ale $A, B$ vôbec nemusia byť množiny čísel, a predpis nemusí byť vzorec!
\begin{priklad}
Niekedy $A$ nemá číselnú povahu, $B$ áno.
\begin{itemize}
    \item $A = \text{všetky adresy v meste}$
    \item $B = \mathbb{R}$
\end{itemize}
Zobrazenie $d \colon A \to B$ môže byť \[
d(x) = \text{najkratšia vzdialenosť pri ceste peši medzi adresou $x$ a SvF STU, v minútach}.\]
Napríklad: $d(\text{moje bydlisko}) = 80$, $d(\text{Bernolákova 1})=8$.
\end{priklad}


\begin{priklad}
Zobrazenie $g \colon \mathbb{N} \rightarrow \mathbb{N}$ dané predpisom $g(n) = n^2 + 1$.
Toto nám hovorí, že $g$ zobrazuje z množiny všetkých prirodzených čísel do množiny všetkých prirodzených čísel.
Predpis je teda v tomto prípade daný vzorcom, ktorý nám umožňuje počítať hodnoty zobrazenia pre konkrétne prvky definičného oboru $g$ (t.j. prirodzené čísla) dosadením a výpočtom.
\begin{align*}
    g(2) &= 2^2 + 1 = 5 \\
    g(7) &= 7^2 + 1 = 50
\end{align*}
$g(-1) = ?$ Toto neexistuje, pretože $-1 \notin \mathbb{N}$ a nie je to teda prvok definičného oboru $g$.
\end{priklad}

\begin{priklad}\label{ex:jedlo}
Nech $A = \{\text{Jožko, Miško}\}$ a $B = \{\text{bageta, guláš, jablko}\}$.
Nech $j \colon A \rightarrow B$ je zobrazenie "najobľúbenejšie jedlo".
V tomto prípade je definičný konečná množina.
Preto nám stačí napísať hodnotu
zobrazenia $j$ v každom prvku množiny $A$:
\begin{itemize}
    \item $f(\text{Jožko}) = \text{bageta}$
    \item $f(\text{Miško}) = \text{guláš}$
    \item $f(\text{Anka}) = \text{guláš}$
\end{itemize}

Zobrazenie $j$ môžeme aj nakresliť:
\begin{center}
\includegraphics{figures/predn2_fig1.pdf}
\end{center}
\end{priklad}

Iný spôsob špecifikácie predpisu zobrazenia je napríklad tabuľkou:
\begin{center}
\begin{tabular}{c|ccc}
    $x$ & Jožko & Miško & Anka \\
    \hline
    $j(x)$ & bageta & guláš & guláš
\end{tabular}
\end{center}
Tu sa, samozrejme, nesmú prvky v hornom riadku opakovať.
\begin{priklad}
Nech $H$ je množina všetkých ľudí (aj z minulosti).
$\sigma \colon H \rightarrow H$ je zobrazenie dané predpisom 
\[
\sigma(x) = \text{otec človeka } x.
\]
\end{priklad}

\begin{priklad}
$S$ - množina všetkých občanov SR.
$\eta \colon S \rightarrow \mathbb{N}$ dané predpisom 
\[
\eta(x) = \text{rodné číslo}.
\]
\end{priklad}

\begin{priklad}
$B = \{\text{bageta, guláš, jablko}\}$. Nech $k \colon B \rightarrow \mathbb{R}$ je zobrazenie "koľko kalórií".
Keďže $B$ je konečná, stačí nám napísať:
$k(\text{guláš}) = 677$, $k(\text{bageta}) = 1148$, $k(\text{jablko}) = 301.4$.
\end{priklad}

\begin{priklad}
Poznáme nejaký príklad zobrazenia typu $A\times A\to A$, kde $A$ je nejaká množina? Samozrejme, už od prvého ročníka
základnej školy. Vezmime $A=\Nat$; sformulovať nejaký predpis pre zobrazenie 
$\Nat\times\Nat\to\Nat$ znamená povedať, ako vyrobiť z usporiadanej dvojice prirodzených čísel prirodzené
číslo. Napríklad môžeme definovať zobrazenie $+\colon A\times A\to A$ predpisom
\[
+(x,y)=\text{súčet čísel $x,y$}
\]
máme teda $+(1,3)=4$, $+(4,4)=8$. Samozrejme, zaužívaný spôsob zapisovania hodnoty zobrazenia $+$ v nejakej dvojici
$(a,b)\in\Nat\times\Nat$ je iný, nepíšeme obvykle $+(a,b)$, ale znak zobrazenia dáme medzi prvú a druhú zložku
usporiadanej dvojice, $a+b$. To je však detail, ktorý nič nemení na dôležitom náhľade, že sčítanie je zobrazením
z nejakej množiny do inej množiny.
\end{priklad}

Predošlý príklad je poučný v tom, že ukazuje ako jazyk postavený na pojmoch ,,množina'' a ,,zobrazenie'' umožňuje
popisovať matematické pojmy. Tento jazyk sa začal účinne používať na popis existujúcej a objavovanie novej matematiky v
20. storočí a dnes si už matematiku bez množín ani nevieme predstaviť.

Jeden zo spôsobov zapisovania zobrazení je ,,po prípadoch'', ako v nasledujúcich
dvoch príkladoch.
\begin{priklad}
\emph{Absolútna hodnota} je zobrazenie $\mathbb R\to\mathbb R$ dané predpisom
\[
|x|=
\begin{cases}
x& x\geq 0\\
-x& x<0
\end{cases}
\]
\end{priklad}
\begin{priklad}
\emph{Znamienková funkcia} alebo \emph{signum} je zobrazenie 
$\mathrm{sgn}\colon\R\to\R$ dané predpisom
\[
\mathrm{sgn}(x)=
\begin{cases}
1& x>0\\
0& x=0\\
-1& x<0
\end{cases}
\]
\end{priklad}

Voči Definícii \ref{def:zobrazenie} by bolo možné vzniesť (z istého hľadiska oprávnene) námietku o nepresnosti; používa nejasné slová ako
"priradí", "predpis".
Námietku je možné vyriešiť takto:

\begin{definicia}[formálna definícia zobrazenia]\label{def:zobrazenieFormal}
Nech $A$, $B$ sú množiny.  \emph{Zobrazenie} $f$ z $A$ do $B$ je množina $f \subseteq
A \times B$ taká, že pre každé $a \in A$ existuje práve jedno $b \in B$ také, že
$(a,b) \in f$.
\end{definicia}

$(a,b) \in f$ v zmysle Definície \ref{def:zobrazenieFormal} potom znamená $f(a)=b$ v zmysle Definície
\ref{def:zobrazenie}.
Aj keď je Definícia \ref{def:zobrazenieFormal} presnejšia, v skutočnosti ju bežne nikto nepoužíva, ani nikto bežne
nerozmýšľa o zobrazeniach ako o množinách usporiadaných dvojíc.
Niekedy však takéto presné uvažovanie nutne potrebujeme, a preto je dobré vedieť o existencii tohto pohľadu na pojem zobrazenia.
%\subsection*{Koniec formalistickej odbočky}

\begin{definicia}[Obor hodnôt]\label{def:oborHodnot}
Nech $f \colon A \rightarrow B$ je zobrazenie. \emph{Obor hodnôt} je množina
$$ \mathcal{H}(f) = \{f(a) | a \in A\} $$
\end{definicia}
Čiže máme $b \in \mathcal{H}(f)$ práve vtedy, keď existuje $a \in A$ také, že $f(a)=b$.
Je dôležité si uvedomiť rozdiel medzi oborom hodnôt a kooborom.
Ak napíšeme napríklad
$f \colon \mathbb{R} \rightarrow \mathbb{R}$ dané predpisom $f(x) = x^2 - x + 1$.
Koobor je $\mathbb{R}$, ale $\mathcal{H}(f) = \langle\frac{3}{4}, \infty\rangle$.
Určiť obor hodnôt zobrazenia môže byť teda ťažké a pre prácu so zobrazením to nemusí
byť nutné.
Čo potrebujeme o zobrazení nutne vedieť je koobor, nie obor hodnôt.
Našťastie, keď sa pred našim duševným zrakom zjaví nejaké zobrazenie, vždy je
vybavené kooborom.
Trochu mätúce môže byť, že koobor sa často neuvádza explicitne a
funkcia sa stotožňuje s predpisom, toto sa bežne bude diať na predmete
\emph{Matematická analýza}.
V tomto (a iných) smere sa konvencie v matematických
oblastiach líšia.
Pre profesionálneho matematika však spravidla nie je problém sa
odlišným konvenciám v prípade potreby prispôsobiť, ak potrebuje pracovať s
matematickou literatúrou a podobne.
\begin{definicia}[identické zobrazenie]\label{def:identickeZobrazenie}
Nech $A$ je množina. \emph{Identické zobrazenie} (na $A$) je zobrazenie $\id_A \colon A \rightarrow A$ dané predpisom $\id_A(a) = a$, pre každý prvok $a \in A$.
\end{definicia}

\begin{definicia}[rovnosť zobrazení]\label{def:rovnostZobrazeni}
Nech $A, B, C, D$ sú množiny, nech $f \colon A \rightarrow B$ a $g \colon C \rightarrow D$.
Hovoríme, že $f$ je \emph{rovné} $g$, ak $A=C$, $B=D$ a pre všetky $x \in A=C$ platí, že $f(x)=g(x)$.
\end{definicia}

Na lineárnej algebre budeme ohľadom pojmu rovnosti zobrazení trochu striktnejší ako
na iných predmetoch, budeme aplikovať definíciu \ref{def:rovnostZobrazeni} veľmi
presne.
Ilustruje to nasledujúci príklad.

\begin{priklad}\label{ex:rovnostZobrazeni}
~\par
\begin{enumerate}
    \item[A)] $f \colon \mathbb{Z} \rightarrow \mathbb{N}$ dané predpisom $f(k) = \sqrt{k^2}$ \\
    $g \colon \mathbb{Z} \rightarrow \mathbb{N}$ dané predpisom $g(k) = |k|$ \\
    Platí $f=g$.
    \item[B)] $f \colon \mathbb{N} \rightarrow \mathbb{N}$ dané predpisom $f(k) = \sqrt{k^2}$ \\
    $g \colon \mathbb{N} \rightarrow \mathbb{N}$ dané predpisom $g(k) = |k|$ \\
    Platí $f=g$.
    \item[C)] $f \colon \mathbb{Z} \rightarrow \mathbb{Z}$ dané $f(x) = |x|$ \\
    $g \colon \mathbb{Z} \rightarrow \mathbb{Z}$ dané $g(x) = x$ \\
    Platí $f \neq g$ (pretože pre $x=-1$ je $f(-1)=1$, ale $g(-1)=-1$).
    \item[D)] $f \colon \mathbb{N} \rightarrow \mathbb{N}$ dané $f(x) = x+1$ \\
    $g \colon \mathbb{N} \rightarrow \mathbb{Z}$ dané $g(x) = x+1$ \\
    Platí $f \neq g$ (pretože majú rôzne koobory).
    \item[E)] $f \colon \mathbb{Z} \rightarrow \mathbb{Z}$ dané $f(x) = x+1$ \\
    $g \colon \mathbb{N} \rightarrow \mathbb{Z}$ dané $g(x) = x+1$ \\
    Platí $f \neq g$ (pretože majú rôzne definičné obory).
\end{enumerate}
\end{priklad}

Uvažujme teraz nejaké zobrazenie $f\colon A\to B$ a množinu podmnožinu jeho 
definičného oboru $X\subseteq A$.
\emph{Zúženie $f$ na $X$} je zobrazenie $\restr{f}{X}\colon X\to B$ dané
predpisom
\[
(\restr{f}{X})(x)=f(x)
\]
Napríklad v E) Príkladu \ref{ex:rovnostZobrazeni} máme $f\neq g$, ale pritom
$g=\restr{f}{\Nat}$.

\subsection{Extrémne zobrazenia}

Nech $A$ je nejaká množina; koľko zobrazení existuje z $A$ do jednoprvkovej množiny $\{*\}$?
Je zrejmé, že zobrazenie $f\colon A\to \{*\}$ je iba jediné, jeho predpis je
$f(a)=*$ pre každé $a\in A$.

Ak uvažujeme zobrazenia opačným smerom, každé zobrazenie typu $\{*\}\to A$ zodpovedá výberu prvku z množiny
$A$, na ktorý sa zobrazí prvok $*$. Povedzme zobrazenia typu $\{*\}\to\{1,2\}$ sú presne dve.

Ako vyzerajú zobrazenia z prázdnej množiny $\emptyset$ do $A$? Toto je trochu horšie
pochopiteľné, ale ak zobrazenie definujeme formálne v zmysle Definície
\ref{def:zobrazenieFormal}, $f\colon\emptyset\to A$ znamená presne
$f\subseteq \emptyset\times A=\emptyset$, čo znamená, že $f=\emptyset$.
Zobrazenie $f\colon\emptyset\to A$ je teda iba jedno.
Jeho predpis pritom vôbec nemusíme formulovať, pretože v definičnom obore $f$ nie sú žiadne prvky.

Naproti tomu zobrazenie typu $A\to\emptyset$ neexistuje, ak $A\neq\emptyset$.
Podmienka $f\subseteq A\times\emptyset$ totiž implikuje $f=\emptyset$. Pokiaľ
$A$ je neprázdna množina, existuje nejaký prvok $a\in A$.
Podľa Definície \ref{def:zobrazenieFormal}, má pre toto $a\in A$ existovať (dokonca
práve jedno) $b$ také že $(a,b)\in f$, čo je ale v rozpore s tým, že $f=\emptyset$.

Ak aj definičný obor aj koobor zobrazenia je $\emptyset$, tento problém nenastane a
zobrazenie $f\colon \emptyset\to\emptyset$ existuje.

\subsection{tice ako zobrazenia}

Ak $A$ je nejaká množina, potom prvky $A^2=A\times A$ sú usporiadané dvojice prvkov množiny $A$.
\[
A^2=\{(a_1,a_2):a_1,a_2\in A\}
\]

Pre každú dvojicu $a=(a_1,a_2)\in A^2$ vieme vytvoriť zobrazenie $\hat a\colon \{1,2\}\to A$ dané predpisom
$\hat a(1)=a_1$, $\hat a(2)=a_2$. Naopak, pre každé zobrazenie $f\colon \{1,2\}\to A$ vieme vytvoriť
usporiadanú dvojicu $\bar f=(f(1),f(2))$. Zrejme
teda usporiadané dvojice prvkov $A$ môžeme chápať ako zobrazenia typu $\{1,2\}\to A$
a naopak.

Podobne pre iné hodnoty $n$ môžeme prvky $A^n$ stotožniť so zobrazeniami typu 
\begin{equation}\label{eq:An}
\{i\in\Nat : 1\leq i\leq n\}\to A
\end{equation}
Ak $n=0$, množina naľavo v \eqref{eq:An} je prázdna a dostávame typ $\emptyset\to A$. Takéto
zobrazenie je iba jedno, preto množina $A^0$ všetkých usporiadaných $0$-íc je
jednoprvková. Jej jediný prvok môžeme označiť $()$.

\subsection{Skladanie zobrazení}

Najdôležitejšou vecou na zobrazeniach je to, že sa dajú skladať.

\begin{definicia}
Nech $A, B, C$ sú množiny. Nech $f \colon A \rightarrow B$, $g \colon B \rightarrow C$.
Potom \emph{zložené zobrazenie} $g \circ f$ je zobrazenie $g \circ f \colon A \rightarrow C$ dané predpisom
\begin{equation}\label{eq:zlozeneZobrazenie}
(g \circ f)(x) = g(f(x))
\end{equation}
\end{definicia}

Vidíme, že nemôžeme ľubovoľné zobrazenie zložiť s ľubovoľným iným. Aby sme mohli vytvoriť zobrazenie $g\circ f$, musí
platiť, že \ul{koobor $f$ je rovnaká množina ako definičný obor $g$}. Ďalšia pasca je v tom, že hodnota
zobrazenia $g\circ f$ vzniká tak, že najskôr aplikujeme $f$ a potom aplikujeme $g$. Keďže píšeme a čítame zľava doprava,
vnímame v zápise $g\circ f$ písmeno $g$ ako prvé a $f$ ako druhé. Autor tohto textu používa pre zapamätanie si pravidla
o skladaní fakt, že predpis \eqref{eq:zlozeneZobrazenie} má písmená $f,g$ v rovnakom poradí na oboch stranách rovnosti.

\begin{priklad}
Zobrazenie "koľko kalórií má najobľúbenejšie jedlo":
Nech $j \colon A \rightarrow B$ je zobrazenie "najobľúbenejšie jedlo'' {a} $k \colon B \rightarrow \mathbb{R}$ je zobrazenie "koľko kalórií".
Potom $k \circ j \colon A \rightarrow \mathbb{R}$ je zobrazenie "koľko kalórií má najobľúbenejšie jedlo".
Napríklad: $(k \circ j)(\text{Miško}) = k(j(\text{Miško})) = k(\text{guláš}) = 677$.
\end{priklad}

\begin{priklad}
Nech $g \colon \mathbb{N} \rightarrow \mathbb{N}$ je dané $g(x) = x^2+1$ a $h \colon \mathbb{N} \rightarrow \mathbb{N}$ je dané $h(x) = 2x$.
\begin{itemize}
    \item $g \circ h \colon \mathbb{N} \rightarrow \mathbb{N}$ \\
    $(g \circ h)(x) = g(h(x)) = g(2x) = (2x)^2+1 = 4x^2+1$
    \item $h \circ g \colon \mathbb{N} \rightarrow \mathbb{N}$ \\
    $(h \circ g)(x) = h(g(x)) = h(x^2+1) = 2(x^2+1) = 2x^2+2$
\end{itemize}
Vidíme, že $g \circ h \neq h \circ g$, lebo napríklad $(g \circ h)(1) = 5$, ale $(h \circ g)(1) = 4$.
\end{priklad}

\begin{priklad}
Čo je zobrazenie $\sigma \circ \sigma \colon H \rightarrow H$, ak $\sigma(x)$ je otec človeka $x$?
Odpoveď: $\sigma(\sigma(x))$ je otcov otec, t.j. starý otec z otcovej strany.
\end{priklad}

Zavedieme teraz označenie, ktoré v základných kurzoch matematiky nie je príliš časté, ale autor tohto textu ho považuje za
užitočné. Pre dve množiny $A$, $B$ budeme ako $\Set(A,B)$ označovať množinu všetkých zobrazení
z množiny $A$ do množiny $B$. Okrem už zavedených množinových operácií tým dostávame nový spôsob, ako z dvoch množín
vyrobiť novú množinu. Na skladanie zobrazení sa môžeme pozerať ako na zobrazenie: pomocou skladania
vytvárame z usporiadanej dvojice zobrazení $(g,f)$, kde $g\in\Set(B,C)$ a $f\in\Set(A,B)$ zobrazenie
$g\circ f\in\Set(A,C)$, alebo inak povedané, pre každú trojicu množín $A,B,C$ máme zobrazenie typu
\[
\circ\colon\Set(B,C)\times\Set(A,B)\to\Set(A,C)
\]

Identické zobrazenia sa vo vzťahu na skladanie správajú špeciálne.
\begin{veta}[neutralita $\id$ vzhľadom na skladanie]\label{veta:neutralitaId}
Nech $A, B$ sú množiny, nech $f \colon A \rightarrow B$ je zobrazenie.
Potom platí $f \circ \id_A = f$ a $\id_B \circ f = f$.
\end{veta}

\begin{proof}
Máme dokázať, že dve zobrazenia sa rovnajú.
Čo je rovnosť dvoch zobrazení, o tom hovorí Definícia 1.9.
Pre $f \circ \id_A = f$:
Zobrazenie $\id_A$ je typu $A \rightarrow A$, zobrazenie $f$ je typu $A \rightarrow B$.
Teda $f \circ \id_A$ existuje a je typu $A \rightarrow B$. Majú rovnaký definičný obor aj koobor.
Pre všetky $x \in A$ platí:
$$ (f \circ \id_A)(x) = f(\id_A(x)) = f(x) $$
Teda $f \circ \id_A = f$ v zmysle Definície 1.9.
Dôkaz rovnosti $\id_B \circ f = f$ prenechávame čitateľovi ako cvičenie.
\end{proof}

\begin{veta}[Asociativita skladania zobrazeni]\label{veta:asocSkladania}
Nech $A, B, C, D$ sú množiny, nech $f \colon A \rightarrow B$, $g \colon B
\rightarrow C$, $h \colon C \rightarrow D$ sú zobrazenia.
Potom $h \circ (g \circ f)
= (h \circ g) \circ f$.
\end{veta}

\begin{proof}
Obe zobrazenia, $h \circ (g \circ f)$ aj $(h \circ g) \circ f$, majú definičný obor $A$ a koobor $D$.
Pre všetky $x \in A$:
\begin{align*}
    (h \circ (g \circ f))(x) &= h((g \circ f)(x)) = h(g(f(x))) \\
    ((h \circ g) \circ f)(x) &= (h \circ g)(f(x)) = h(g(f(x)))
\end{align*}
Keďže obe zobrazenia majú rovnaký definičný obor, koobor a vo všetkých bodoch nadobúdajú rovnakú hodnotu, rovnajú sa.
\end{proof}
Veta \ref{veta:asocSkladania} znamená, že vo výrazoch typu $h\circ g\circ f$ nemusíme písať zátvorky, aby sme
určili ktoré skladanie treba urobiť prvé.

\section{Injekcie, surjekcie a bijekcie}

V tejto časti si zavedieme dôležité vlastnosti zobrazení. Skladanie zobrazení
môžeme chápať ako nejaký typ binárnej operácie, pre ktoré sa identické zobrazenie chová neutrálne, viď
vety \ref{veta:asocSkladania} a \ref{veta:neutralitaId}.
Z dostatočného odstupu a zanedbávajúc isté rozdiely môžeme $g\circ f$ vidieť ako analógiu súčinu reálnych čísel a $\id$ ako analógiu
\footnote{Táto analógia sa dá spresniť, takže z istého abstraktného hľadiska sa dajú súčin a skladanie
naozaj pochopiť ako dve inštancie jediného abstraktného pojmu.}
jednotky:
\[
\begin{array}{c|c}
a.b & g\circ f\\
a.1 =a &  g\circ\id=g \\
1.b=b & \id\circ f= f
\end{array}
\]
Pre násobenie čísel vieme ku každému číslu $a\neq 0$ nájsť nejaké číslo $a^{-1}$ také, že $a.a^{-1}=a^{-1}.a=1$, voláme ho
prevrátená hodnota $a$. Prirodzene vzniká otázka, či a kedy vieme nájsť k nejakému zobrazeniu $f$ analógiu prevrátenej
hodnoty čísla, to znamená zobrazenie $g$ z vlastnosťou $g \circ f=\id$ alebo $f\circ g=\id$. Skúmanie tohto
problému vedie k pojmom injekcie, surjekcie a bijekcie. Situácia je však trochu komplikovanejšia ako v prípade čísel,
pretože zobrazenia sú trochu zložitejšie veci ako čísla. 

\begin{definicia}[injekcia]\label{def:injekcia}
Nech $A,B$ sú množiny. Zobrazeniu $f\colon A\to B$ hovoríme,
\emph{injekcia/injektívne} ak pre každé dva prvky $a_1,a_2\in A$ také, že $a_1\neq a_2$ platí, že $f(a_1)\neq f(a_2)$.
\end{definicia}

V jazyku formálnej logiky
\begin{equation}\label{eq:injekcia}
(\forall a_1,a_2\in A)\quad a_1\neq a_2\implies f(a_1)\neq f(a_2)
\end{equation}
\begin{priklad}
Zobrazenie $j\colon A\to B$ v príklade \ref{ex:jedlo} nie je injektívne. Platí totiž
\[
\text{Miško}\neq\text{Janka}\qquad j(\text{Miško})=j(\text{Janka})
\]
Dokázali sme teda negáciu formuly \eqref{eq:injekcia} (pre $f=j$, samozrejme), to znamená
\[
(\exists a_1,a_2\in A)\quad a_1\neq a_2\land j(a_1)=j(a_2)
\]
\end{priklad}
Všimnite si, že neinjektívnosť $j$ je vidno z obrázku.

Logicky ekvivalentná forma \eqref{eq:injekcia} je
\begin{equation}\label{eq:injekcia2}
(\forall a_1,a_2\in A)\quad f(a_1)=f(a_2)\implies a_1=a_2
\end{equation}
ktorá vznikne transpozíciou implikácie:
\[
a_1\neq a_2\implies f(a_1)\neq f(a_2)
\quad
\text{je to isté ako}
\quad
f(a_1)=f(a_2)\implies a_1=a_2
\]
\begin{definicia}[surjekcia]\label{def:surjekcia}
Nech $A,B$ sú množiny. Zobrazeniu $f\colon A\to B$ hovoríme,
\emph{surjekcia/surjektívne} ak pre každý prvok $b\in B$ existuje nejaký prvok $a\in A$ taký, že
$f(a)=b$.
\end{definicia}
\begin{priklad}
Zobrazenie $j\colon A\to B$ z príkladu \ref{ex:jedlo} nie je surjektívne. Na prvok jablko kooboru $B$
zobrazenia $j$ sa žiadny prvok definičného oboru $A$ zobrazenia $j$ nezobrazí. Inými slovami, pre všetky prvky $a\in A$
platí, že $j(a)\neq\text{jablko}$.
\end{priklad}
V tejto chvíli je užitočné uvedomiť si, že zobrazenie $f\colon A\to B$ je surjektívne práve vtedy, keď 
koobor $B$ je rovný oboru hodnôt $f$, $B=\mathcal H(f)$. To znamená, že z každého zobrazenia vieme spraviť surjektívne
zobrazenie, ak zúžime jeho koobor: tieto dve zobrazenia
\begin{align*}
f_1\colon\mathbb R\to \mathbb R&\qquad f_1(x)=x^2+1\\
f_2\colon\mathbb R\to \mathbb R&\qquad \langle 1,\infty)
\end{align*}
majú rovnaký definičný obor a predpis (ale nie koobor, teda sú to rôzne funkcie). Pritom $f_1$ nie je surjektívne, ale
$f_2$ je surjekcia. 
\begin{definicia}[bijekcia]\label{def:bijekcia}
Nech $A,B$ sú množiny. Zobrazeniu $f\colon A\to B$ hovoríme,
\emph{bijekcia/bijektívne} ak pre každý prvok $b\in B$ existuje nejaký prvok $a\in A$ taký, že
$f(a)=b$.
\end{definicia}
\subsection{Ľavé a pravé inverzné zobrazenie}

Definície injekcie a surjekcie vyzerajú veľmi odlišne. V tejto časti textu sa
naučíme, že sú prepojené istou skrytou symetriou, ktorá sa týka toho, ako sa správajú
vzhľadom na skladanie (operácia $\circ$) a identické zobrazenia.

Pre každé dve množiny $A$, $B$ máme tieto dve množiny:
\begin{itemize}
\item Zobrazenia z $A$ do $B$, teda množina $\Set(A,B)$.
\item Zobrazenia z $B$ do $A$, teda množina $\Set(B,A)$.
\end{itemize}

Ak $f\in\Set(A,B)$ (alebo $f\colon A\to B$, to je to isté), a $g\in\Set(B,A)$,
vieme z nich vytvoriť dve zložené zobrazenia, $g\circ f$ a $f\circ g$. Pritom
$f\circ g\colon B\to B$ a $g\circ f\colon A\to A$, alebo inak,
\[
f\circ g\in\Set(B,B)\quad g\circ f\in\Set(A,A),
\]
zobrazujú teda $B$ (respektíve $A$) do seba samej.
V množine $\Set(B,B)$ máme jeden význačný prvok, a to identické zobrazenie
$\id_B$; podobne samozrejme $\id_A\in\Set(A,A)$. Z týchto úvah nám akosi samovoľne vzniknú nasledujúce dva pojmy.
\begin{definicia}[zľava/sprava inverzné zobrazenie]\label{def:zlavaSpravaInverzne}
Nech $A,B$ sú množiny, nech $f\colon A\to B$. Hovoríme, že zobrazenie $g\colon B\to A$ je
\begin{itemize}
\item \emph{zľava inverzné k zobrazeniu $f$} ak platí, že $g\circ f=\id_A$
\item \emph{sprava inverzné k zobrazeniu $f$} ak platí, že $f\circ g=\id_B$
\end{itemize}
\end{definicia}

Všimnime si, že $f$ je zľava inverzné ku $g$ práve vtedy, keď $g$ je sprava inverzné
ku $f$ (rozmyslite si to).

\begin{veta}\label{veta:sekcie}
Nech $A,B$ sú množiny, nech $f\colon A\to B$. Potom
\begin{enumerate}[(a)]
\item $f$ je injekcia práve vtedy, ak existuje aspoň jedno zľava inverzné zobrazenie k $f$.
\item $f$ je surjekcia práve vtedy, ak existuje aspoň jedno sprava inverzné zobrazenie k $f$.
\end{enumerate}
\end{veta}
\begin{proof}~
\begin{enumerate}[(a)]
\item Nech $f$ je injekcia. Chceme nájsť nejaké zobrazenie $g\colon B\to A$, pričom $g$ má byť také,
že $g\circ f=\id_A$, teda pre všetky $a\in A$ má platiť
$(g\circ f)(a)=\id_A(a)$, to znamená $g(f(a))=a$. Keďže $f$ je injekcia, pre $b\in\mathcal H(f)$ existuje práve jedno
$a\in A$ také, že $f(a)=b$. Naozaj, ak by sme mali nejaké $a_1,a_2\in A$ také, že $a_1\neq a_2$ a zároveň
$f(a_1)=f(a_2)$, $f$ by nebola injekcia.
Pre $b\in B$, zvoľme $g(b)$ tak, že pre $b\in\mathcal H(f)$ máme $g(b)=a$, kde $f(a)=b$ a pre $b\in B\setminus\mathcal
H(f)$ zvolíme $g(b)$ ľubovoľne. Máme potom $g(f(a))=a$, pre každé $a\in A$.

Predpokladajme teraz, že existuje $g\colon B\to A$ také, že $g\circ f=\id_A$. Použijeme
charakterizáciu injekcie \eqref{eq:injekcia2}. Nech $a_1,a_2\in A$ sú také, že $f(a_1)=f(a_2)$. Z tohto
predpokladu máme dokázať, že $a_1=a_2$.
Podľa predpokladu zrejme $g(f(a_1))=g(f(a_2))$, čo znamená 
\begin{equation}\label{eq:veta:sekcie:1}
(g\circ f)(a_1)=(g\circ f)(a_2)\tag{*}
\end{equation}
Ale my sme predpokladali, že
$g\circ f=\id_A$, teda \eqref{eq:veta:sekcie:1} znamená, že $\id_A(a_1)=\id_A(a_2)$ a z toho máme ihneď
$a_1=a_2$
\item Dôkaz vynechávame.
\end{enumerate}
\end{proof}

\subsection{Inverzné zobrazenie}

\begin{definicia}[inverzné zobrazenie]\label{def:inverzneZobrazenie}
Nech $A,B$ sú množiny, nech $f\colon A\to B$. Hovoríme, že zobrazenie $g\colon B\to A$ je
\emph{inverzné} k zobrazeniu $f$ ak je zľava inverzné k $f$ a zároveň sprava inverzné k $f$.
\end{definicia}

\begin{veta}\label{veta:bijekciaInverzne}
Nech $A,B$ sú množiny. Potom $f\colon A\to B$ má inverzné zobrazenie práve vtedy, keď $f$ je bijekcia.
\end{veta}
\begin{proof}
Z definície bijekcie, inverzného zobrazenia a vety \ref{veta:sekcie} ihneď vidno, že ak má nejaké zobrazenie 
$f$ inverzné zobrazenie, potom $f$ je bijekcia.

Naopak, nech $f$ je bijekcia. Podľa vety \ref{veta:sekcie} má potom nejaké ľavé inverzné zobrazenie 
$g_L\colon B\to A$ a aj nejaké pravé inverzné zobrazenie $g_R\colon B\to A$. 
Ak dokážeme z týchto predpokladov že $g_L=g_R$, potom to je už inverzné zobrazenie k $f$.
Použijeme elegantný trik: vezmeme výraz $g_L\circ f\circ g_R$ a zjednodušíme ho dvoma
rôznymi spôsobmi:
\begin{align*}
g_L\circ f\circ g_R&=(g_L\circ f)\circ g_R=\id_A\circ g_R=g_R\\
g_L\circ f\circ g_R&=g_L\circ (f\circ g_R)=g_L\circ\id_B=g_L.
\end{align*}
Ale z toho zrejme vyplýva, že $g_L=g_R$.
\end{proof}
Všimnime si, že v dôkaze predošlej vety sme ukázali aj čosi navyše: pokiaľ $f$ je bijekcia, nielenže má
nejaké inverzné zobrazenie, ale toto inverzné zobrazenie je dokonca presne jedno. Z
toho vyplýva, že môžeme
zaviesť operáciu ,,invertuj zobrazenie''
\[
f\mapsto f^{-1}
\]
ktorá bude definovaná iba ak $f$ je bijekcia. Zobrazenie $f^{-1}$ je (vždy jediné) inverzné zobrazenie k zobrazeniu $f$. 


\section{Sústavy lineárnych rovníc}

\begin{definition}[Lineárna rovnica nad $\mathbb{R}$]
Lineárna rovnica o $n$ neznámych je rovnica tvaru
$$ (*) \quad a_{1}x_{1} + \dots + a_{n}x_{n} = c $$
kde $n \ge 1, n \in \mathbb{N}$. Koeficienty $a_1, \dots, a_n, c$ sú dané prvky $\mathbb{R}$. Riešenie tejto rovnice je taká $n$-tica $(x_1, \dots, x_n) \in \mathbb{R}^n$, že po dosadení do (*) je vzniknutý výrok pravdivý.
\end{definition}

\begin{example}
Daná je rovnica $3x_{1} + 2x_{2} + (-1)x_{3} = 7$, ktorú zvyčajne zapisujeme ako
$$ 3x_{1} + 2x_{2} - x_{3} = 7 $$
Niektoré jej riešenia sú napríklad $(x_{1}, x_{2}, x_{3}) = (1, 2, 0)$ alebo $(x_{1}, x_{2}, x_{3}) = (0, 0, -7)$. Táto rovnica má nekonečne veľa riešení.
\end{example}

\begin{definition}[Sústava lineárnych rovníc]
Sústava $m$ lineárnych rovníc o $n$ neznámych nad $\mathbb{R}$ je usporiadaná
$m$-tica rovníc o $n$ neznámych nad $\mathbb{R}$, kde $m, n \ge 1$. Neznáme sú
rovnaké pre všetky rovnice.
\end{definition}

\begin{equation}\label{eq:sustava}
\begin{array}{ccccccccc}
a_{11}x_{1} & + & a_{12}x_{2} & + & \dots & + & a_{1n}x_{n} & = & c_{1} \\
a_{21}x_{1} & + & a_{22}x_{2} & + & \dots & + & a_{2n}x_{n} & = & c_{2} \\
\vdots & & \vdots & & \ddots & & \vdots & & \vdots \\
a_{m1}x_{1} & + & a_{m2}x_{2} & + & \dots & + & a_{mn}x_{n} & = & c_{m}
\end{array}
\end{equation}

Riešenie sústavy je taká usporiadaná $n$-tica $(x_{1}, x_{2}, \dots, x_{n}) \in \mathbb{R}^{n}$, ktorá je riešením každej rovnice v sústave.

\begin{example}\label{ex:malaSustava}
Uvažujme sústavu rovníc:
\begin{align*}
    3x_1 + x_2 &= 1 \\
    x_1 - x_2 &= -5
\end{align*}
Ideme sa pokúsiť nájsť jej riešenie. Pripočítajme prvú rovnicu k druhej.
\begin{align*}
    3x_1 + x_2 &= 1 \\
    4x_1 - 0 &= -4
\end{align*}
Vynásobme druhú rovnicu číslom $\frac{1}{4}$.
\begin{align*}
    3x_1 + x_2 &= 1 \\
    x_1 - 0 &= -1
\end{align*}
Teraz už vieme, že $x_1=-1$, môžeme dosadiť túto hodnotu do prvej rovnice
a vyjadriť $x_2$. Ale môžeme postupovať aj ďalej a napríklad pripočítať
$-3$-násobok druhej rovnice k prvej.
V každom prípade, jediným riešením je $x_1 = -1$ a $x_2 = 4$.
\end{example}

Čo sme robili? Menili sme sústavu tak, aby zmenená sústava mala rovnakú množinu
riešení. Transformujeme teda v každom problém na iný, jednoduchší. Ale najviac
dôležité pri tom je to, že vždy tak, aby sa množina všetkých riešení nezmenila.
Aké úpravy môžeme robiť so sústavou lineárnych rovníc tak, aby sa nezmenila množina
všetkých riešení?

Môžeme napríklad:
\begin{enumerate}
    \item vymeniť dve rovnice v sústave medzi sebou
    \item vynásobiť rovnicu nenulovou konštantou (prečo nenulovou?)
    \item pripočítať ľubovoľný násobok jednej rovnice k druhej rovnici
\end{enumerate}

\subsection{Matice: základná terminológia a označenia}

Matica je typu $m\times n$ ($m,n\in\mathbb N)$ je obdĺžniková tabuľka reálnych čísel,
ktorá má $m$ riadkov a $n$ stĺpcov. Matice označujeme veľkými písmenami.
Všeobecnú maticu $A$ typicky zapisujeme napríklad takto:
\begin{equation}\label{eq:matrix}
A=
\left(
\begin{array}{cccc}
    a_{11} & a_{12} & \dots & a_{1n} \\
    a_{21} & a_{22} & \dots & a_{2n} \\
    \vdots & \vdots & \ddots & \vdots \\
    a_{m1} & a_{m2} & \dots & a_{mn}
\end{array}
\right)
\end{equation}
Konvencia je, že prvý index v $a_{ij}$ je číslo riadku a  
druhý je číslo stĺpca. Všimnime si, že \eqref{eq:matrix} obsahuje v zásade iba
informácie, že
\begin{itemize}
\item Matica sa volá $A$,
\item jej prvky sú značené $a_{ij}$,
\item jej typ je $m\times n$.
\end{itemize}
Toto budeme niekedy zapisovať krátko ako
\[
A=
\left(
\begin{array}{c}
a_{ij}
\end{array}
\right)_{m\times n}
\]


\subsection{Zápis sústavy lineárnych rovníc pomocou matice}

S maticami budeme na lineárnej algebre
pracovať často a budeme opakovane nachádzať ich nové významy. 
Ale v tejto chvíli, pre začiatok, použijeme maticu jednoducho pre zápis systému
lineárnych rovníc. Zapíšeme zo sústavy \eqref{eq:sustava} len to podstatné: koeficienty $(a_ij)$ a
pravú stanu $(c_i)$:
\[
\left( \begin{array}{cccc|c} a_{11} & a_{12} & \dots & a_{1n} & c_{1} \\ a_{21} & a_{22} & \dots & a_{2n} & c_{2} \\ \vdots & \vdots & \ddots & \vdots & \vdots \\ a_{m1} & a_{m2} & \dots & a_{mn} & c_{m} \end{array} \right)
\]

Zvislú stranu použijeme na oddelenie pravej strany. Je to čisto vizuálna pomôcka, nie
je naozaj súčasťou matice. Túto maticu nazývame \emph{rozšírená matica sústavy},
koeficienty $(a_{ij})$ tvoria \emph{maticu sústavy} a stĺpec $(c_i)$ je \emph{pravá
strana}.
$$
\underbrace{
    \overbrace{
        \left(
        \begin{array}{cccc}
            a_{11} & a_{12} & \dots & a_{1n} \\
            a_{21} & a_{22} & \dots & a_{2n} \\
            \vdots & \vdots & \ddots & \vdots \\
            a_{m1} & a_{m2} & \dots & a_{mn}
        \end{array}
        \right|
    }^{\text{matica sústavy}}
    \overbrace{
        \left.
        \begin{array}{c}
            c_{1} \\
            c_{2} \\
            \vdots \\
            c_{m}
        \end{array}
        \right)
    }^{\text{pravá strana}}
}_{\text{rozšírená matica sústavy}}
$$
Teda sústava $m$ lineárnych rovníc o $n$ neznámych sa bude zapisovať
pomocou matice typu $m\times(n+1)$.

V konkrétnom príklade to vyzerá takto.
Sústava
\begin{align*}
    3x_{1} + 2x_{2} - 7x_{3} &= 14 \\
    -x_{1} \qquad + 4x_{3} &= -7 \\
    \qquad x_{2} + x_{3} &= 0
\end{align*}
sa zapíše maticou
$$ \underbrace{
\overbrace{
\left(
\begin{array}{ccc}
3 & 2 & -7 \\
-1 & 0 & 4 \\
0 & 1 & 1 
\end{array}
\right|
}^{\text{matica sústavy}}
\overbrace{
\left.
\begin{array}{c}
-14 \\
-7 \\
0
\end{array}
\right)
}^{\text{pravá strana}}
}_{\text{rozšírená matica sústavy}}
$$
\subsection{Elementárne riadkové operácie}

Elementárna riadková operácia je zmena matice na inú maticu jedného z nasledujúcich
typov.
\begin{enumerate}
\item Výmena riadkov $k,l$:
\[
% --- First Matrix (before swap) ---
\begin{pmatrix}
a_{11} & \dots & a_{1n} \\
\vdots & \ddots & \vdots \\
% Mark the last element of row k
a_{k1} & \dots & \tikzmarknode{rowk}{a_{kn}} \\
\vdots & \ddots & \vdots \\
% Mark the last element of row l
a_{l1} & \dots & \tikzmarknode{rowl}{a_{ln}} \\
\vdots & \ddots & \vdots \\
a_{m1} & \dots & a_{mn}
\end{pmatrix}
% --- TikZ Arrow (placed after the first matrix) ---
\begin{tikzpicture}[remember picture, overlay]
    % Draw the C-shaped swap arrow
    \draw[<->, thick, shorten <=2pt, shorten >=2pt]
        ([xshift=1em]rowk.east) -- ++(1em,0)
        -- ([xshift=2em]rowl.east) -- ([xshift=1em]rowl.east);
\end{tikzpicture}
% --- Equivalence Symbol with adjusted spacing ---
\hspace{2em} \sim \quad % Increased space before ~ to account for the arrow
% --- Second Matrix (after swap) ---
\begin{pmatrix}
a_{11} & \dots & a_{1n} \\
\vdots & \ddots & \vdots \\
% Row l is now in the k-th position
a_{l1} & \dots & a_{ln} \\
\vdots & \ddots & \vdots \\
% Row k is now in the l-th position
a_{k1} & \dots & a_{kn} \\
\vdots & \ddots & \vdots \\
a_{m1} & \dots & a_{mn}
\end{pmatrix}
\]
\item Pripočítanie $\alpha$-násobku riadku $k$ k riadku $l$, kde $\alpha\in\mathbb
R$.
\[
% --- First Matrix (before operation) ---
\begin{pmatrix}
a_{11} & \dots & a_{1n} \\
\vdots & \ddots & \vdots \\
% Mark the last element of row k (the source row)
a_{k1} & \dots & \tikzmarknode{rowk}{a_{kn}} \\
\vdots & \ddots & \vdots \\
% Mark the last element of row l (the target row)
a_{l1} & \dots & \tikzmarknode{rowl}{a_{ln}} \\
\vdots & \ddots & \vdots \\
a_{m1} & \dots & a_{mn}
\end{pmatrix}
% --- TikZ Arrow for the operation ---
\begin{tikzpicture}[remember picture, overlay]
    % Draw an arrow from row k to row l with an unboxed alpha
    \draw[->, thick, shorten >=2pt]
        % Start a bit right of row k
        ([xshift=1.5em]rowk.east)
        % Draw a line to the right, and mark the corner
        -- ++(1em,0) coordinate (corner)
        % Draw a vertical line down to the level of row l
        -- (corner |- rowl.east)
        % Place the alpha to the right of the vertical line
        node [midway, right] {$\alpha$}
        % Draw the final segment pointing to row l
        -- ([xshift=1.5em]rowl.east);
\end{tikzpicture}
% --- Equivalence Symbol with adjusted spacing ---
\hspace{3em} \sim \quad
% --- Second Matrix (after operation) ---
\begin{pmatrix}
a_{11} & \dots & a_{1n} \\
\vdots & \ddots & \vdots \\
% Row k remains unchanged
a_{k1} & \dots & a_{kn} \\
\vdots & \ddots & \vdots \\
% Row l is updated: R_l <- R_l + alpha * R_k
a_{l1} + \alpha a_{k1} & \dots & a_{ln} + \alpha a_{kn} \\
\vdots & \ddots & \vdots \\
a_{m1} & \dots & a_{mn}
\end{pmatrix}
\]
\item Vynásobenie riadku $k$ číslom $\beta\in\mathbb R$, kde $\beta\neq 0$.
\[
% --- First Matrix (before operation) ---
\begin{pmatrix}
a_{11} & \dots & a_{1n} \\
\vdots & \ddots & \vdots \\
% Mark the last element of row k (the row being modified)
a_{k1} & \dots & \tikzmarknode{rowk}{a_{kn}} \\
\vdots & \ddots & \vdots \\
a_{m1} & \dots & a_{mn}
\end{pmatrix}
% --- TikZ Arrow for the operation ---
\begin{tikzpicture}[remember picture, overlay]
    % 1. Define the start and end coordinates for the arrow.
    \coordinate (arrow_start) at ([xshift=3.5em]rowk.east);
    \coordinate (arrow_end)   at ([xshift=1.5em]rowk.east);

    % 2. Draw the arrow between these points.
    \draw[->, thick, shorten >=2pt] (arrow_start) -- (arrow_end);

    % 3. Place the beta node slightly to the right of the arrow's start.
    \node[right=2pt] at (arrow_start) {$\beta$};
\end{tikzpicture}
% --- Equivalence Symbol with adjusted spacing ---
\hspace{4em} \sim \quad
% --- Second Matrix (after operation) ---
\begin{pmatrix}
a_{11} & \dots & a_{1n} \\
\vdots & \ddots & \vdots \\
% Row k is updated: R_k <- beta * R_k
\beta a_{k1} & \dots & \beta a_{kn} \\
\vdots & \ddots & \vdots \\
a_{m1} & \dots & a_{mn}
\end{pmatrix}
\]
\end{enumerate}
\begin{definition}
Hovoríme, že dve matice $A$, $B$ rovnakého typu sú \emph{riadkovo ekvivalentné},
ak existuje postupnosť elementárnych riadkových operácií, ktorou sa dá $A$ upraviť na
$B$.
\end{definition}
Elementárne riadkové operácie sú pre nás v tejto chvíli dôležité kvôli nasledujúcej
vete.
\begin{veta}\label{veta:sustavyMatice}
Dve sústavy $m$ lineárnych rovníc o $n$ neznámych majú rovnakú množinu riešení práve
vtedy keď sú ich rozšírené matice riadkovo ekvivalentné.
\end{veta}

Preto pri riešení sústavy lineárnych rovníc môžeme použiť nasledujúcu stratégiu:
\begin{enumerate}[(Krok 1)]
\item Napíšeme si rozšírenú maticu sústavy.
\item Pomocou elementárnych riadkových operácií maticu upravíme na jednoduchší tvar.
\item Nájdeme riešenie tej sústavy, ktorá zodpovedá tomuto jednoduchšiemu tvaru.
\end{enumerate}
Veta \ref{veta:sustavyMatice} nám hovorí, že tento postup je korektný.

Otázka je, čo budeme považovať za jednoduchší tvar; bude to takzvaný
\emph{stupňovitý tvar}, ktorý je naznačený na nasledujúcom obrázku.
\[
\left(
\begin{array}{ccccccccccc}
    0 & \cdots & 0 & \bullet &?& \cdots & \cdots & \cdots & \cdots & \cdots & ? \\
    0 & \cdots & 0 & 0 & 0 & \bullet &?& \cdots & \cdots & \cdots & ? \\
    \vdots & \ddots& \vdots & \vdots &\vdots&\vdots & \vdots & \vdots & \vdots & \vdots & \vdots \\
    0 & \cdots & 0 & 0 & 0 & 0 & 0& \cdots & 0 & \bullet & ? \\
    0 & \cdots & 0 & 0 & 0 & 0 &0 & \cdots & 0 & 0 & 0
\end{array}
\right)
\]
V tomto obrázku $\bullet$ znamená nenulový prvok (rôzny od 0), a prvok $?$ môže byť
ľubovoľný.

\begin{definition}[vedúci prvok riadku]\label{def:veduciPrvok}
Ak A je matica typu $m \times n$, potom vedúci prvok i-teho riadku matice je
najľavejší nenulový prvok toho riadku: $a_{ij} \ne 0$ a zároveň $a_{il} = 0$ pre
všetky $1 \le l < j$.
\end{definition}

\begin{definition}
Hovoríme, že matica A typu $m \times n$ je v stupňovitom tvare, ak
\begin{enumerate}[(a)]
    \item Ak $r_i(A) \ne (0, \dots, 0)$ a zároveň $r_k(A) = (0, \dots, 0)$, potom $i < k$. 
    \begin{framed}
    Každý nenulový riadok je nad každým nulovým riadkom.
    \end{framed}
    \item Ak $a_{ij}$ je vedúci prvok i-teho riadku a $a_{kl}$ je vedúci prvok k-teho
    riadku a $i<k$ potom aj $j<l$. 
    \begin{framed}
    Vedúci prvok vyššieho riadku leží viac vľavo ako
    vedúci prvok nižšieho riadku.
    \end{framed}
\end{enumerate}
\end{definition}
\begin{example}
\[
\begin{array}{rl}
\begin{pmatrix} 1 & 7 & 0 & -1 & 2 \\ 0 & 0 & 0 & 0 & 0 \\ 0 & 1 & 1 & -1 & 0
\end{pmatrix} & 
\quad \text{nie je v stupňovitom tvare (prečo?)} \\~\\
\begin{pmatrix} 0 & 1 & 2 & 3 & 4 \\ 0 & 0 & -1 & 0 & 1 
\end{pmatrix} &
\quad\text{je v stupňovitom tvare}\\~\\
\begin{pmatrix} 0 & 0 & 1 & 3 & 7 \\ 0 & 0 & 1 & 0 & 1 \\ 0 & 0 & 0 & 1 & 0
\end{pmatrix} &
\quad \text{nie je v stupňovitom tvare (prečo?)} \\~\\
\begin{pmatrix} 1 & 0 & 3 & 7 & 0 \\ 0 & 0 & 1 & 7 & 4 \\ 0 & 0 & 0 & 1 & 0 \\ 0 & 0 & 0 & 0 & 0 \\ 0 & 0 & 0 & 0 & 0 
\end{pmatrix} &
\quad\text{je v stupňovitom tvare}
\end{array}
\]
\end{example}

\section{Vektory a operácie s nimi}

Sila lineárnej algebry spočíva v tom, že umožňuje viac pohľadov na rovnaký pojem. Tieto pohľady sú veľmi silne prepojené - niekedy sa medzi nimi ani nerozlišuje a plynule sa prechádza z jedného do druhého.

\noindent Vektor môže byť:
\begin{itemize}
    \item[(*)] množina všetkých orientovaných úsečiek v rovine/priestore, ktoré majú rovnakú veľkosť a smer.
    \item[(**)] Zvoľme bod $O$ v rovine/priestore. Vektor je orientovaná úsečka s počiatkom v tomto bode (môžeme ju stotožniť s jej druhým koncovým bodom; potom vektor = bod).
    \item[(***)] Usporiadaná n-tica reálnych čísel; $n=2$ pre rovinu a $n=3$ pre priestor.
    \item[(****)] Prvok vektorového priestoru.
\end{itemize}

Zostaneme pri výklade pojmov v rovine; zovšeobecnenie do priestoru je priamočiare.

\noindent \textbf{Typografické pravidlo:}
Vektory budeme písať so šípkou: $\vec{x}, \vec{y}, \vec{u}$.



\subsection{Prechody medzi definíciami}
Vysvetlíme prechody medzi pohľadmi na vektor:
\begin{center}
    (*) $\xrightarrow{\text{výber počiatku}}$ (**) $\xrightarrow{\text{voľba súradnicových osí}}$ (***)
\end{center}

Všetci asi vieme, čo je úsečka; orientovaná úsečka je úsečka s vybratým krajným bodom. Keďže úsečka nenulovej dĺžky má dva krajné body, každej úsečke nenulovej dĺžky zodpovedajú dve orientované úsečky:

\begin{center}
\begin{tikzpicture}
    \node[circle, fill, inner sep=1.5pt, label=left:A] (A) at (0,0) {};
    \node[label=right:B] (B) at (2,0) {};
    \draw[-{Stealth[]}] (A) -- (B) node[midway, below] {$\ora{AB}$};

    \node[label=left:A] (A2) at (4,0) {};
    \node[circle, fill, inner sep=1.5pt, label=right:B] (B2) at (6,0) {};
    \draw[-{Stealth[]}] (B2) -- (A2) node[midway, below] {$\ora{BA}$};
\end{tikzpicture}
\end{center}

Podľa (*) je (jeden!) vektor množina všetkých ($\infty$) orientovaných úsečiek, ktoré majú rovnakú veľkosť a smer.

\begin{itemize}
    \item Veľkosť orientovanej úsečky je jej dĺžka.
    \item Čo je smer, je akosi tiež jasné.
\end{itemize}

Asi najčistejší spôsob, ako na úrovni geometrie popísať veľkosť a smer je, že dve orientované úsečky $\ora{AB}$ a $\ora{CD}$ majú rovnakú veľkosť a smer práve vtedy, keď platí jedna z týchto možností:
\begin{itemize}
    \item $|AB|=|CD|=0$ - nulový vektor.
    \item $A \neq B, C \neq D$ a $ABDC$ je rovnobežník s uhlopriečkami $AD, BC$.
\end{itemize}

\begin{center}
\begin{tikzpicture}[scale=1.5]
    \node[circle, fill, inner sep=1.5pt, label=left:A] (A) at (0,0) {};
    \coordinate[label=right:B] (B) at (2,0);
    \node[circle, fill, inner sep=1.5pt, label=above:C] (C) at (0.5,1) {};
    \coordinate[label=above:D] (D) at (2.5,1);
    \draw[thick, -{Stealth[]}] (A) -- (B);
    \draw[thick, -{Stealth[]}] (C) -- (D);
    \draw (A) -- (C);
    \draw (B) -- (D);
\end{tikzpicture}
\end{center}

Teda vektor v zmysle (*) si môžeme predstaviť takto:
\begin{center}
\begin{tikzpicture}
    \foreach \x/\y in {0/0, 0.5/1, -0.2/0.8, 1.5/0.2, 1.8/1.2}
    {
        \node[circle, fill, inner sep=1.5pt] at (\x,\y) {};
        \draw[thick, -{Stealth[]}] (\x,\y) -- (\x+1,\y+0.7);
    }
    \node at (5.0,0.7) {$\left.\begin{array}{c} \\ \\ \end{array}\right\}$ všetky takéto šípky};
\end{tikzpicture}
\end{center}

Definícia (*) je úplne v poriadku, ale manipuluje sa s tým pojmom zle, pretože každý vektor je potom nekonečná množina.
Poradíme si takto: vyberieme v rovine jeden ľubovoľný bod, nazveme ho "počiatok" a budeme ho označovať $O$. V každej množine orientovaných úsečiek, ktorá je vektorom v zmysle (*), máme práve jednu orientovanú úsečku, ktorá má počiatok v $O$. Túto vyberieme z vektora v zmysle (*) a máme vektor v zmysle (**).

\begin{center}
\begin{tikzpicture}
    \node[label=left:O, circle, fill, inner sep=1.5pt] (O) at (0,0) {};
    \draw[thick, -{Stealth[]}] (O) -- (2,1) node[midway, above] {$\vec{u}$};
    \draw[thick, -{Stealth[]}] (O) -- (1,2) node[midway, left] {$\vec{v}$};
    \draw[thick, -{Stealth[]}] (O) -- (1,-2) node[midway, left] {$\vec{x}$};
\end{tikzpicture}
\end{center}

Všimnime si, že jeden z vektorov zodpovedá orientovanej úsečke $\ora{OO}$; hovoríme mu nulový vektor a značíme ho $\vec{0}$.

Umiestnime teraz v rovine dve kópie číselnej osi: vodorovnú a zvislú tak, aby sa pretínali v bode $O$.
\begin{center}
\begin{tikzpicture}[scale=0.8]
    \draw[<->] (-4.5,0) -- (4.5,0) node[right] {$x$};
    \draw[<->] (0,-4.5) -- (0,4.5) node[above] {$y$};
    \foreach \x in {-4,-3,...,-1,1,2,...,4}
    {
        \draw (\x, 0.1) -- (\x, -0.1) node[below] {\tiny \x};
    }
    \foreach \y in {-4,-3,...,-1,1,2,...,4}
    {
        \draw (0.1, \y) -- (-0.1, \y) node[left] {\tiny \y};
    }
    \node[circle, fill, inner sep=1.5pt] at (0,0) {};
    \draw[thick, -{Stealth[]}] (0,0) -- (3.5,2.5) node[midway, above left] {$\vec{u}$};
    \draw[dashed] (3.5,2.5) -- (3.5,0);
    \draw[dashed] (3.5,2.5) -- (0,2.5);
    \node at (7, 2.5) {$\longrightarrow \left(3\frac{1}{2}, 2\frac{1}{2}\right) \in \R^2$};
    \node[red] at (5,-2) {$\searrow$ ,,súradnicové osi''};
\end{tikzpicture}
\end{center}

Premietnutím koncového bodu orientovanej úsečky reprezentujúcej vektor v pravom uhle na osi určíme usporiadanú dvojicu reálnych čísel a naopak, z usporiadanej dvojice reálnych čísel vieme zrejmým spôsobom dostať vektor ako orientovanú úsečku s počiatkom v bode $O$. Pritom nulový vektor $\vec{0}$ zodpovedá dvojici $(0,0)$.

Voľba osí v rovine nám určuje bijekciu:
$$ \text{vektory v rovine} \longleftrightarrow \R^2 $$

To, že sa body v rovine dajú jednoducho (a užitočne) vyjadrovať ako dvojice čísel je
prekvapujúco mladý objav - pochádza z roku 1637 a vymyslel ho René
Descartes. Zaujímavé je, že v tom čase sa už vyše tisíc rokov používali sférické
súradnice pre určovanie polohy na Zemi pomocou rovnobežiek a poludníkov.

Zaveďme teraz terminológiu týkajúcu sa $\R^n$, ktorú budeme používať:
Pre $\vec{x}=(x_1, \dots, x_n) \in \R^n$, $x_i$ je $i$-ta zložka vektora $\vec{x}$.



\subsection{Operácie s vektormi}

\subsubsection{Násobenie vektora skalárom}
\begin{definition}[Násobenie geometrického vektora skalárom]
Ak $\alpha \in \R$ (skalár) a $\vec{v}$ je vektor, potom $\alpha\vec{v}$ je vektor $|\alpha|$-krát predĺžený/skrátený.
\begin{itemize}
    \item ak $\alpha > 0$, $(\alpha\vec{v})$ a $\vec{v}$ sú orientované rovnako,
    \item ak $\alpha < 0$, $(\alpha\vec{v})$ a $\vec{v}$ sú orientované opačne,
    \item ak $\alpha=0$, $\alpha\vec{v} = \vec{0}$.
\end{itemize}
\end{definition}

\begin{example}
~\\
\begin{center}
\begin{tikzpicture}
    \draw[gray] (-4,0) -- (4,0);
    \node[label=below:$O$, circle, fill, inner sep=1.5pt] at (0,0) {};
    \draw[thick, -{Stealth[]}] (0,0) -- (2,0) node[midway, above] {$\vec{v}$};
    \draw[thick, -{Stealth[]}] (0,0) -- (4,0) node[midway, above] {$2\vec{v}$};
    \draw[thick, -{Stealth[]}] (0,0) -- (-3,0) node[midway, above] {$-\frac{3}{2}\vec{v}$};
\end{tikzpicture}
\end{center}
\end{example}

\begin{definition}[násobenie vektora skalarom v $\R^n$]
Nech $\vec{x}=(x_1, \dots, x_n) \in \R^n$ a nech $\alpha \in \R$. Potom
$$ \alpha . \vec{x} = (\alpha x_1, \dots, \alpha x_n) $$
\end{definition}
\begin{example}
\begin{align*}
(-2).(2,-1,0)&=(-4,2,0)\\
\sqrt{3}.(\sqrt{3},-\frac{1}{\sqrt{3}},2)&=(3,-1,2.\sqrt{3})
\end{align*}
\end{example}
Ak si teraz zvolíme v rovine súradnicové osi, dostávame tým bijekciu medzi
geometrickými vektormi a prvkami $\R^2$. Táto bijekcia zachováva násobenie skalárom,
čo znamená, že 
\begin{center}
\begin{tabular}{m{6cm} l}
    % Row 1: Geometric vectors
    \begin{tikzpicture}
        \node[circle, fill, inner sep=1.5pt, label=below left:$O$] (O) at (0,0) {};
        \draw[thick, -{Stealth[]}] (O) -- (1.5,1) node[pos=0.7, above] {$\vec{v}$};
        \draw[thick, -{Stealth[]}] (O) -- (3,2) node[pos=0.7, above] {$2\vec{v}$};
    \end{tikzpicture}
    & Geometria \\
    
    % Row 2: Geometric vectors with axes
    \begin{tikzpicture}[scale=0.7]
        \draw[->] (-0.5,0) -- (6.5,0);
        \draw[->] (0,-0.5) -- (0,4.5);
        \node[below left, circle, fill, inner sep=1.5pt, label=below left:$O$] at (0,0) {};
        \foreach \x in {3,6} \draw (\x,0.1) -- (\x,-0.1) node[below] {\x};
        \foreach \y in {2,4} \draw (0.1,\y) -- (-0.1,\y) node[left] {\y};

        \draw[thick, -{Stealth[]}] (0,0) -- (3,2) node[pos=0.6, above] {$\vec{v}$};
        \draw[thick, -{Stealth[]}] (0,0) -- (6,4) node[pos=0.6, above] {$2\vec{v}$};
        \draw[dashed] (3,2) -- (3,0);
        \draw[dashed] (3,2) -- (0,2);
        \draw[dashed] (6,4) -- (6,0);
        \draw[dashed] (6,4) -- (0,4);
    \end{tikzpicture}
    & Voľba osí \\
    
    % Row 3: Algebraic operation
    $2.(3,2)=(6,4)$
    & Algebra
\end{tabular}
\end{center}Vidíme, že operácii škálovania (násobenie $\alpha \in \R$) (geometrická operácia) zodpovedá vynásobenie usporiadanej n-tice skalárom $\alpha$ vo všetkých zložkách n-tice.

Je dôležité si teraz uvedomiť, že korešpondencia medzi vektormi v geometrickom zmysle
a usporiadanými n-ticami závisí na voľbe osí, osi môžu mať rôznu mierku a môžu byť
dokonca trochu otočené.
\begin{center}
\begin{tikzpicture}[scale=0.9, rotate=20]
    \draw[<->] (-2.5,0) -- (3.5,0);
    \draw[<->] (0,-2.5) -- (0,5.5);
    \foreach \x in {-2,-1,1,2,3} \draw (\x,0.1) -- (\x,-0.1) node[below] {\tiny \x};
    \foreach \y in {-1,-0.5,1,2} \draw (0.1, \y*2) -- (-0.1, \y*2) node[left] {\tiny \y};
    
    \node[circle, fill, inner sep=1.5pt] at (0,0) {};
    \draw[thick, -{Stealth[]}] (0,0) -- (-0.5*2, 2*1) node[pos=0.7, right] {$\vec{u}$};
    \draw[dashed] (-0.5*2, 2*1) -- (-0.5*2, 0);
    \draw[dashed] (-0.5*2, 2*1) -- (0, 2*1);
\end{tikzpicture}
\end{center}
Inou voľbou osí sa bijekcia (vektory v rovine $\leftrightarrow \R^2$) zmení, ale to, že škálovanie zodpovedá násobeniu skalárom po zložkách bude stále platiť.

\subsubsection{Vlastnosti násobenia vektora skalárom}
Pre všetky vektory $\vec{x} \in \R^n$ a $a,b \in \R$ platí:
\begin{itemize}
    \item $(a . b) . \vec{x} = a . (b . \vec{x})$
\end{itemize}
\textbf{Prečo?} Nech $\vec{x}=(x_1, \dots, x_n)$.
\begin{align*}
(a . b) . \vec{x} &= (a . b) . (x_1, \dots, x_n) = ((ab)x_1, \dots, (ab)x_n) \\
&= (a(bx_1), \dots, a(bx_n)) = a(bx_1, \dots, bx_n) \\
&= a(b(x_1, \dots, x_n)) = a(b . \vec{x})
\end{align*}
\begin{itemize}
    \item Pre všetky $\vec{x} \in \R^n$ platí $0\vec{x} = \vec{0}$, $1\vec{x} = \vec{x}$.
\end{itemize}



\subsubsection{Sčítanie vektorov}
Geometricky sa operácia sčítania vektorov zavádza rovnobežníkovým pravidlom:
\begin{center}
\begin{tikzpicture}
    \coordinate[circle, fill, inner sep=1.5pt] (O) at (0,0);
    \coordinate (U) at (2,1);
    \coordinate (V) at (1,2);
    \coordinate (SUM) at (3,3);
    \draw[thick, -{Stealth[]}] (O) -- (U) node[below right] {$\vec{u}$};
    \draw[thick, -{Stealth[]}] (O) -- (V) node[above left] {$\vec{v}$};
    \draw[thick, -{Stealth[]}] (O) -- (SUM) node[above right] {$\vec{u}+\vec{v}$};
    \draw[dashed] (U) -- (SUM);
    \draw[dashed] (V) -- (SUM);
\end{tikzpicture}
\end{center}
Ak je jeden z vektorov $\vec{0}$, definujeme prirodzene $\vec{u}+\vec{0}=\vec{u}$.

Na algebraickej strane tejto geometrickej operácii zodpovedá sčítanie po zložkách.
\begin{center}
\begin{tikzpicture}[scale=0.8]
    \draw[->] (-0.5,0) -- (4.5,0);
    \draw[->] (0,-0.5) -- (0,4.5);
    \node[circle, fill, inner sep=1.5pt] at (0,0) {};
    \foreach \x in {1,2,3,4} \draw (\x,0.1) -- (\x,-0.1) node[below] {\tiny \x};
    \foreach \y in {1,2,3,4} \draw (0.1,\y) -- (-0.1,\y) node[left] {\tiny \y};
    \draw[thick, -{Stealth[]}] (0,0) -- (1,2) node[above left] {$\vec{u}$};
    \draw[thick, -{Stealth[]}] (0,0) -- (2,1) node[below right] {$\vec{v}$};
    \draw[thick, -{Stealth[]}] (0,0) -- (3,3) node[above right] {$\vec{u}+\vec{v}$};
    \draw[dashed] (1,2) -- (3,3);
    \draw[dashed] (2,1) -- (3,3);
\end{tikzpicture}
\end{center}
$\vec{u}=(1,2), \vec{v}=(2,1) \implies \vec{u}+\vec{v}=(1+2, 2+1)=(3,3)$.\\
Opäť, ako v prípade násobenia skalárom, tento vzťah medzi geometrickou operáciou sčítania vektorov a algebraickou operáciou sčítania po zložkách nezávisí na voľbe osí.

\begin{definition}[Sčítanie vektorov v $\R^n$]
Nech $\vec{x}=(x_1, \dots, x_n) \in \R^n$ a $\vec{y}=(y_1, \dots, y_n) \in \R^n$. Potom
$$ \vec{x}+\vec{y} = (x_1+y_1, \dots, x_n+y_n) $$
\end{definition}

\begin{example}
$(1,-3,2,0) + (-1,3,-1,4) = (0,0,1,4)$
\end{example}

\subsubsection{Vlastnosti sčítania vektorov v $\R^n$}
Pre všetky vektory $\vec{x}, \vec{y}, \vec{z} \in \R^n$ platia rovnosti:
\begin{itemize}
    \item $(\vec{x}+\vec{y})+\vec{z} = \vec{x}+(\vec{y}+\vec{z})$ (asociativita)
    \item $\vec{x}+\vec{y} = \vec{y}+\vec{x}$ (komutativita)
    \item $\vec{x}+\vec{0} = \vec{x}$
    \item $\vec{x}+(-1)\vec{x} = \vec{0}$
\end{itemize}
Dôkaz sa robí priamočiaro, napr. komutativita vektorov:
$$ \vec{x}+\vec{y} = (x_1+y_1, \dots, x_n+y_n) = (y_1+x_1, \dots, y_n+x_n) = \vec{y}+\vec{x} $$
pričom sme využili komutativitu sčítania reálnych čísel. Podobne pre ostatné rovnosti.



Násobenie skalárom a sčítanie vektorov sú navzájom prepojené pomocou distributivity.
\begin{itemize}
    \item Pre všetky $a \in \R$ a $\vec{x}, \vec{y} \in \R^n$ platí: $a(\vec{x}+\vec{y}) = a\vec{x} + a\vec{y}$
    \item Pre všetky $a,b \in \R$ a $\vec{x} \in \R^n$ platí: $(a+b)\vec{x} = a\vec{x} + b\vec{x}$
\end{itemize}
Násobenie skalárom má prednosť pred sčítaním vektorov.

\noindent\textbf{POZOR:} Neexistuje nič také ako násobenie vektorov medzi sebou!

Nič nám nebráni zaviesť odčítanie vektorov $\vec{x}-\vec{y}$ definované ako $\vec{x}-\vec{y} := \vec{x} + (-1)\vec{y}$. Je to teda odvodená operácia zavedená pomocou sčítania a násobenia $(-1)$. Samozrejme, ako ľahko vidieť, odčítanie vektorov prebieha tiež po zložkách:
$$ \vec{x}-\vec{y} = (x_1-y_1, \dots, x_n-y_n) $$



\subsection{Vektory z $\R^n$ ako stĺpce}
Odteraz až do konca letného semestra budeme na tomto predmete stotožňovať
usporiadané $n$-tice reálnych čísel so stĺpcovými vektormi (maticami typu $n \times 1$).
$$ \vec{x} = (x_1, \dots, x_n) \in \R^n \quad \iff \quad \vec{x} = \begin{pmatrix} x_1 \\ \vdots \\ x_n \end{pmatrix} $$
Teda prvok množiny $\R^n$ môže byť zapísaný ako riadok \underline{s čiarkami}, alebo
ako stĺpec, oba zápisy označujú tú istú vec:
\[
(1,-17,0,\frac{8}{3})=
\begin{pmatrix}
1\\-17\\0\\\frac{8}{3}
\end{pmatrix}
\]
Budeme plynule prechádzať medzi týmito dvoma spôsobmi zápisu usporiadaných $n$-tíc.


\subsection{Lineárne kombinácie}
\begin{definition}[Lineárna kombinácia]
Nech $\vec{v}_1, \dots, \vec{v}_m \in \R^n$ a $a_1, \dots, a_m \in \R$.
Potom \emph{lineárna kombinácia} vektorov $\vec{v}_1, \dots, \vec{v}_m$ s koeficientami $a_1, \dots, a_m$ je vektor
$$ a_1\vec{v}_1 + a_2\vec{v}_2 + \dots + a_m\vec{v}_m $$
\end{definition}

\begin{example}
$\vec{v}_1 = (1,3,4), \vec{v}_2 = (2,0,1)$ v $\R^3$.
$a_1 = 3, a_2 = 2$.
$a_1\vec{v}_1 + a_2\vec{v}_2 = 3(1,3,4) + 2(2,0,1) = (3,9,12) + (4,0,2) = (7,9,14)$.
\end{example}

\begin{example}
Zistite, či je $\vec{u}=(1,2) \in \R^2$ lineárnou kombináciou vektorov $\vec{v}_1=(1,-1)$ a $\vec{v}_2=(2,5)$ a určite koeficienty tejto lineárnej kombinácie.

Hľadáme $a_1, a_2 \in \R$ také, že platí:
$$ a_1 \vec{v}_1 + a_2 \vec{v}_2 = \vec{u} $$
Zapíšeme problém pomocou stĺpcových vektorov:
$$ a_1 \begin{pmatrix} 1 \\ -1 \end{pmatrix} + a_2 \begin{pmatrix} 2 \\ 5 \end{pmatrix} = \begin{pmatrix} 1 \\ 2 \end{pmatrix} $$
Podľa definície násobenia skalárom:
$$ \begin{pmatrix} a_1 \\ -a_1 \end{pmatrix} + \begin{pmatrix} 2a_2 \\ 5a_2 \end{pmatrix} = \begin{pmatrix} 1 \\ 2 \end{pmatrix} $$
Podľa definície sčítania vektorov:
$$ \begin{pmatrix} a_1 + 2a_2 \\ -a_1 + 5a_2 \end{pmatrix} = \begin{pmatrix} 1 \\ 2 \end{pmatrix} $$
Dva vektory sa rovnajú, ak sa rovnajú po zložkách:
\begin{align*}
    a_1 + 2a_2 &= 1 \\
    -a_1 + 5a_2 &= 2
\end{align*}
Aha! Sústava lineárnych rovníc.
Sčítaním oboch rovníc dostaneme:
$$ 7a_2 = 3 \implies a_2 = \frac{3}{7} $$
Dosadením do prvej rovnice:
$$ a_1 + 2\left(\frac{3}{7}\right) = 1 \implies a_1 + \frac{6}{7} = 1 \implies a_1 = 1 - \frac{6}{7} = \frac{1}{7} $$
Koeficienty sú $a_1=1/7$ a $a_2=3/7$.
Skúška:
$$ \frac{1}{7} \begin{pmatrix} 1 \\ -1 \end{pmatrix} + \frac{3}{7} \begin{pmatrix} 2 \\ 5 \end{pmatrix} = \begin{pmatrix} 1/7 \\ -1/7 \end{pmatrix} + \begin{pmatrix} 6/7 \\ 15/7 \end{pmatrix} = \begin{pmatrix} 7/7 \\ 14/7 \end{pmatrix} = \begin{pmatrix} 1 \\ 2 \end{pmatrix} $$
\end{example}
\begin{example}
Uvažujme vektory v rovine s počiatkom v bode $O$.
Nech $ABCD$ je štvorec so stredom $O$.
\begin{center}
\begin{tikzpicture}[rotate=15]
    % Define square vertices
    \coordinate (A) at (0, 4);
    \coordinate (B) at (0, 0);
    \coordinate (C) at (4, 0);
    \coordinate (D) at (4, 4);

    % Draw square and label vertices
    \draw (A) -- (B) -- (C) -- (D) -- cycle;
    \node[left=3pt] at (A) {A};
    \node[below=3pt] at (B) {B};
    \node[right=3pt] at (C) {C};
    \node[above=3pt] at (D) {D};

    % Define interior point O at the center (2,2)
    % Label 'O' is now on the left.
    \node[circle, fill, inner sep=1.5pt, label={[label distance=2pt]left:O}] (O) at (2, 2) {};
    
    % Define midpoint S(BC)
    \node[inner sep=0,label={[label distance=2pt]below right:S}] (S) at ($(B)!.5!(C)$) {};

    % Draw vectors (thick, with \bullet start, but no \vec labels)
    \draw[thick, -{Stealth[]}] (O) -- (C);
    \draw[thick, -{Stealth[]}] (O) -- (D);
    \draw[thick, -{Stealth[]}] (O) -- (S);
\end{tikzpicture}
\end{center}
Nech $S$ je stred strany $BC$. Vyjadrime vektor $\ora{OS}$ ako lineárnu kombináciu
vektorov $\ora{OC},\ora{OD}$.

Najskôr si uvedomme, že $\ora{OB}=(-1).\ora{OD}$. 
\begin{center}
\begin{tikzpicture}[rotate=15]
    % Define square vertices
    \coordinate (A) at (0, 4);
    \coordinate (B) at (0, 0);
    \coordinate (C) at (4, 0);
    \coordinate (D) at (4, 4);

    % Draw square and label vertices
    \draw (A) -- (B) -- (C) -- (D) -- cycle;
    \node[left=3pt] at (A) {A};
    \node[below=3pt] at (B) {B};
    \node[right=3pt] at (C) {C};
    \node[above=3pt] at (D) {D};

    % Define interior point O at the center (2,2)
    % Label 'O' is now on the left.
    \node[circle, fill, inner sep=1.5pt, label={[label distance=2pt]left:O}] (O) at (2, 2) {};
    
    % Define midpoint S(BC)
    \node[inner sep=0,label={[label distance=2pt]below right:S}] (S) at ($(B)!.5!(C)$) {};

    % Draw vectors (thick, with \bullet start, but no \vec labels)
    \draw[thick, -{Stealth[]}] (O) -- (C);
    \draw[thick, -{Stealth[]}] (O) -- (D);
    \draw[thick, -{Stealth[]},color=gray] (O) -- (B);
    \draw[thick, -{Stealth[]}] (O) -- (S);
\end{tikzpicture}
\end{center}
Vektory $\ora{OC}$ a $\ora{OB}$ sú kolmé a majú rovnakú dĺžku. Preto koncové vrcholy
vektorov $\ora{OB},\ora{OC},\ora{OB}+\ora{OC}$ spolu s bodom $O$ tvoria štvorec.
\begin{center}
\begin{tikzpicture}[rotate=15]
    % Define square vertices
    \coordinate (A) at (0, 4);
    \coordinate (B) at (0, 0);
    \coordinate (C) at (4, 0);
    \coordinate (D) at (4, 4);
    \coordinate (E) at (2, -2);


    % Draw square and label vertices
    \draw (A) -- (B) -- (C) -- (D) -- cycle;
    \node[left=3pt] at (A) {A};
    \node[below=3pt] at (B) {B};
    \node[right=3pt] at (C) {C};
    \node[above=3pt] at (D) {D};
    \draw (B) -- (E) -- (C);
    \node[below=3pt] at (E) {$\ora{OB}+\ora{OC}$};

    % Define interior point O at the center (2,2)
    % Label 'O' is now on the left.
    \node[circle, fill, inner sep=1.5pt, label={[label distance=2pt]left:O}] (O) at (2, 2) {};
    
    % Define midpoint S(BC)
    \node[inner sep=0,label={[label distance=2pt]below right:S}] (S) at ($(B)!.5!(C)$) {};

    % Draw vectors (thick, with \bullet start, but no \vec labels)
    \draw[thick, -{Stealth[]}] (O) -- (C);
    \draw[thick, -{Stealth[]},color=gray] (O) -- (E);
    \draw[thick, -{Stealth[]}] (O) -- (D);
    \draw[thick, -{Stealth[]},color=gray] (O) -- (B);
    \draw[thick, -{Stealth[]}] (O) -- (S);
\end{tikzpicture}
\end{center}
Pritom bod $S$ je stredom tohto štvorca, teda
\[
\ora{OS}=\frac{1}{2}(\ora{OB}+\ora{OC})
\]
a môžeme použiť pravidlá o počítaní s vektormi
\begin{multline*}
\frac{1}{2}.(\ora{OB}+\ora{OC})=\frac{1}{2}.\ora{OB}+\frac{1}{2}.\ora{OC}=
\frac{1}{2}.((-1).\ora{OD})+\frac{1}{2}.\ora{OC}=\\
(\frac{1}{2}.(-1)).\ora{OD}+\frac{1}{2}.\ora{OC}=-\frac{1}{2}\ora{OD}+\frac{1}{2}\ora{OC}
\end{multline*}
Teda
\begin{equation}\label{eq:squareCombination}
\ora{OS}=-\frac{1}{2}\ora{OD}+\frac{1}{2}\ora{OC},
\end{equation}
čo je hľadaná lineárna kombinácia.

Skúste si teraz rozmyslieť, aká je poloha vektorov 
$-\frac{1}{2}\ora{OD},\frac{1}{2}\ora{OC}$ v rovine. Keďže platí \eqref{eq:squareCombination},
spolu s bodmi $O$ a $S$ by mali ich
koncové body tvoriť
rovnobežník. Je to pravda?
\end{example}

\section{Základy maticového počtu}

Pripomenutie: matica A typu $m \times n$ je obdĺžniková tabuľka s m riadkami a n stĺpcami; inak povedané: šírka je n a výška je m:

\begin{center}
\begin{tabular}{ccc}
 & \hspace{-2em} $n$ stĺpcov \\
 & \hspace{-2em} $\longleftarrow \dots \longrightarrow$ \\
\parbox{6em}{\centering $\big\uparrow$\\$m$ riadkov \\$\big\downarrow$}&
  $\begin{pmatrix}
    a_{11} & a_{12} & \dots & a_{1n} \\
    a_{21} & a_{22} & \dots & a_{2n} \\
    \vdots & \vdots & \ddots & \vdots \\
    a_{m1} & a_{m2} & \dots & a_{mn}
  \end{pmatrix}$ 
\end{tabular}
\end{center}
\vspace{1em}

Pozície v matici typu $m \times n$ sú usporiadané dvojice kladných prirodzených čísel $(i, j)$, kde $1 \le i \le m$, $1 \le j \le n$. Inými slovami, pozície v matici sú $\{1, \dots, m\} \times \{1, \dots, n\}$.

Číslo v matici na pozícii $(i, j)$ referencujeme pomocou dvojitého indexu, napr. $a_{ij}$, $b_{ij}$.

Množina všetkých (reálnych) matíc typu $m \times n$ sa značí $\mathbb{R}^{m \times n}$. Podľa pravidla, "vektory sú stĺpce" stotožňujeme $\mathbb{R}^n = \mathbb{R}^{n \times 1}$.
... je to isté ako ...

\subsection{Zápis matice pomocou predpisu}
Zápis matice pomocou predpisu je tvaru
% Annotations from page 2
\begin{center}
$A_{m \times n} = (\underbrace{\dots}_{\text{\parbox{8em}{\centering predpis pre prvok na pozícii
$i,j$}}})_{\underbrace{m \times n}_{\text{\parbox{6em}{\centering dve prirodzené čísla}}}}$
\end{center}
Predpis je obvykle výraz závislý na $i, j$.

\begin{example}
$(i+j)_{2 \times 3} = \begin{pmatrix} 2 & 3 & 4 \\ 3 & 4 & 5 \end{pmatrix}$
\end{example}

\begin{example}
$(i \cdot j)_{2 \times 3} = \begin{pmatrix} 1 & 2 & 3 \\ 2 & 4 & 6 \end{pmatrix}$
\end{example}

Tento zápis budeme používať mnoho razy na vyjadrenie maticových operácií.

\subsection{Riadky a stĺpce matice}
Nech

\[
A = (a_{ij})_{m \times n}
\]

Týmto zápisom špecifikujeme typ matice $A$ ako $m\times n$ ale aj to, že jej prvky označujeme $a_{ij}$.

$i$-ty riadok matice $A$ je
\[
r_i(A) = \begin{pmatrix} 
a_{i1} & a_{i2} & \dots & a_{in} 
\end{pmatrix}
\]

$j$-ty stĺpec matice $A$ je

\[s_j(A) = \begin{pmatrix} a_{1j} \\ a_{2j} \\ \vdots \\ a_{mj} \end{pmatrix}\]

\subsection{Pár druhov matíc}
\subsubsection{Štvorcové matice} 
sú matice typu $n \times n$. Ak $A = (a_{ij})_{n \times n}$ je štvorcová matica, potom (hlavná) diagonála matice je tvorená prvkami $(a_{11}, a_{22}, \dots, a_{nn})$.
% Image from page 4
\[ \begin{pmatrix} a_{11} & a_{12} & \dots & a_{1n} \\ a_{21} & a_{22} & \dots & a_{2n} \\ \vdots & \vdots & \ddots & \vdots \\ a_{n1} & a_{n2} & \dots & a_{nn} \end{pmatrix} \]

\subsubsection{Diagonálna matica} je štvorcová matica, ktorá má všetky prvky okrem diagonálnych rovné 0.
\[ \begin{pmatrix} a_{11} & 0 & \dots & 0 \\ 0 & a_{22} & \dots & 0 \\ \vdots & \vdots & \ddots & \vdots \\ 0 & 0 & \dots & a_{nn} \end{pmatrix} \]

\subsubsection{Nulová matica} je matica, ktorá má všetky prvky rovné 0. Značíme ju 0. Nemusí byť nutne štvorcová.

\subsubsection{Jednotková matica} je diagonálna matica, ktorá má na diagonále všade 1:
\[ \begin{pmatrix} 1 & 0 & 0 & \dots & 0 \\ 0 & 1 & 0 & \dots & 0 \\ 0 & 0 & 1 & \dots & 0 \\ \vdots & \vdots & \vdots & \ddots & \vdots \\ 0 & 0 & 0 & \dots & 1 \end{pmatrix} \]
Jednotkovú maticu typu $n \times n$ značíme $I_n$. Ak $n$ je nešpecifikované, značíme $I$.

\subsection{Operácie s maticami}

S maticami môžeme robiť to isté ako s vektormi: môžeme ich sčítať a násobiť
skalárom. Avšak na rozdiel od vektorov môžeme matice medzi sebou násobiť.

\subsubsection{Súčet matíc}
Nech $A = (a_{ij})_{m \times n}$
$B = (b_{ij})_{m \times n}$
sú dve matice rovnakého typu $m \times n$. Potom súčet matíc A, B je matica
\[ A + B = (a_{ij} + b_{ij})_{m \times n} \]

\begin{example}
$A = \begin{pmatrix} 1 & 2 & \sqrt{2} \\ -1 & 0 & 3 \end{pmatrix}$ $B = \begin{pmatrix} \frac{1}{2} & -1 & 0 \\ 0 & 0 & 4 \end{pmatrix}$
\[ A + B = \begin{pmatrix} \frac{3}{2} & 1 & \sqrt{2} \\ -1 & 0 & 7 \end{pmatrix} \]
\end{example}
Sčítať môžeme iba dvojice matíc rovnakého typu.

\subsubsection{Násobenie matice skalárom}
Nech $A = (a_{ij})_{m \times n}$, $c \in \mathbb{R}$. Potom
\[ c A = (c \cdot a_{ij})_{m \times n} \quad (\text{bodka sa nepíše vždy}) \]
\begin{example}
\[ -2 \begin{pmatrix} 1 & 2 & 3 \\ 0 & 1 & -7 \end{pmatrix} = \begin{pmatrix} -2 & -4 & -6 \\ 0 & -2 & 14 \end{pmatrix} \]
\end{example}

\subsection{Vlastnosti súčtu matíc a násobenia matice skalárom}
Zrejme doteraz zavedené operácie na maticiach manipulujú s maticami ako s vektormi $\mathbb{R}^{m \cdot n}$. V tomto zmysle nie sú pre nás ničím novým.
Samozrejme, platia pre sčítanie a násobenie skalárom rovnaké pravidlá ako pre vektory.

Pre každú trojicu matíc $A, B, C$ rovnakého typu a každú dvojicu $a, b \in \mathbb{R}$ platí
\begin{itemize}
    \item $A + B = B + A$ (komutativita +)
    \item $(A + B) + C = A + (B + C)$ (asociativita +)
    \item $A + O = O + A = A$ \quad (nulová matica, jej typ je nutne rovnaký ako typ A; O je neutrálna vzhľadom na +)
    \item $1A = A$
    \item $(ab)A = a(bA)$
    \item $(a+b)A = aA + bA$
    \item $a(A+B) = aA + aB$
\end{itemize}

\subsubsection{Transpozícia matice}
Nech $A = (a_{ij})_{m \times n}$. Potom A transponovaná (alebo transpozícia A) je matica
\[ A^T = (a_{ji})_{n \times m} \]
\begin{example}
\[ \begin{pmatrix} 1 & 2 & 4 \\ -1 & 0 & 3 \end{pmatrix}^T = \begin{pmatrix} 1 & -1 \\ 2 & 0 \\ 4 & 3 \end{pmatrix} \]
\end{example}

Pre každú dvojicu matíc A, B rovnakého typu a $c \in \mathbb{R}$ platí
\begin{itemize}
    \item $(A^T)^T = A$
    \item $(A + B)^T = A^T + B^T$
    \item $(cA)^T = c(A^T)$
\end{itemize}

\textbf{Symetrická matica} je taká matica, že
\[ A = A^T \]
Každá symetrická matica je štvorcová (prečo?). \\
\begin{example}
\[ \begin{pmatrix} 1 & 0 & 7 \\ 0 & 4 & 2 \\ 7 & 2 & 1 \end{pmatrix}\]
\end{example} 
je symetrická matica .
Je zrejmé, že každá diagonálna matica je symetrická.

\subsection{Súčin matíc}
Teraz ideme definovať súčin matíc; najskôr to urobíme pre špeciálny prípad matíc $1 \times n$ a $n \times 1$.
To nám umožní definovať všeobecný súčin matíc.

\subsubsection{Súčin riadku a stĺpca}
Uvažujme teraz dve matice, jedna typu $1 \times n$ (riadok) a druhá typu $n \times 1$ (stĺpec). Ich súčin je skalár daný
\[ \begin{pmatrix} y_1 & \dots & y_n \end{pmatrix} \begin{pmatrix} x_1 \\ \vdots \\ x_n \end{pmatrix} = y_1 x_1 + \dots + y_n x_n = \sum_{i=1}^n y_i x_i \]
\begin{example}
\[ \begin{pmatrix} 1 & -1 & 0 \end{pmatrix} \begin{pmatrix} 2 \\ 4 \\ 3 \end{pmatrix} = 1 \cdot 2 + (-1) \cdot 4 + 0 \cdot 3 = 2 - 4 + 0 = -2 \]
\end{example}

\subsubsection{Súčin matíc (všeobecne)}
Nech A je typu $m \times n$, B je typu $n \times k$. (počet stĺpcov A = počet riadkov B) \\
Potom súčin matíc A a B je matica $C$ typu $m \times k$ (počet riadkov A, počet stĺpcov B) \\
Prvok $C_{ij}$ matice $C$ na pozícii $(i, j)$ je daný ako súčin $i$-teho riadku A a $j$-teho stĺpca B:
\[ C_{ij} = r_i(A) \cdot s_j(B) \]

\begin{example}
\[ \underbrace{\begin{pmatrix} \dots \end{pmatrix}}_{\text{typ } 3 \times 2} \underbrace{\begin{pmatrix} \dots \end{pmatrix}}_{\text{typ } 2 \times 4} = \underbrace{\begin{pmatrix} \dots \end{pmatrix}}_{\text{typ } 3 \times 4} \]
\end{example}
Prechádzame postupne cez všetky usporiadané dvojice (riadok ľavej, stĺpec pravej). Pre každú dvojicu vyrobíme ich súčin a umiestnime to číslo do výslednej matice na pozíciu $(i, j)$.

\subsection*{Vlastnosti násobenia matíc}
\begin{itemize}
    \item $A(BC) = (AB)C$ (asociativita) \\
    Dôkaz nie je zrejmý, ale priamy dôkaz je prácny a neposkytne generický vhľad do veci, preto ho neurobíme.
    
    \item $A(B + C) = AB + AC$ (distributivita zľava)
    \item $(A + B) C = AC + BC$ (sprava)
    \item $(AB)^T = B^T A^T$
    \item $a(AB) = (aA) B = A (aB)$ (kompatibilita násobenia matice skalárom a násobenia
    matíc)
    \item Ak A je matica typu $m \times n$ potom 
    \[
    I_m A = A \quad A I_n = A
    \]
    kde $I_m$, $I_n$ sú jednotkové matice typu $m \times m$ resp. $n \times n$
    \item Čo sa týka násobenia nulovou maticou,
    \[
    0A = 0\quad A0 = 0
    \]
    kde $0$ označuje nulové matice správneho typu.
\end{itemize}

\textbf{POZOR!} Operácia násobenia matíc \textbf{nie je komutatívna!} \\
Vôbec nie je pravda, že pre matice platí $AB = BA$.
V prvom rade, ak existuje $AB$, musí byť počet stĺpcov A rovný počtu riadkov B.

A je typu $m \times n$, B je typu $n \times k$.
Aby súčin $BA$ vôbec existoval, musí byť $k = m$, ale to nie je vo všeobecnosti pravda.

Ale čo ak sú $A,B$ štvorcové matice rovnakého typu? Potom $AB$ aj $BA$ existujú a majú aj rovnaký typ.
Skúsme:
\[ \begin{pmatrix} 1 & 1 \\ 0 & 1 \end{pmatrix} \begin{pmatrix} 0 & 1 \\ 1 & 0 \end{pmatrix} = \begin{pmatrix} 1 & 1 \\ 1 & 0 \end{pmatrix} \]
\[ \begin{pmatrix} 0 & 1 \\ 1 & 0 \end{pmatrix} \begin{pmatrix} 1 & 1 \\ 0 & 1 \end{pmatrix} = \begin{pmatrix} 0 & 1 \\ 1 & 1 \end{pmatrix} \]
Teda vidíme, že $AB\neq BA$
Nič menej, pre niektoré dvojice matíc platí $AB = BA$, napríklad $A O = O A = O$.
Prirodzená otázka, pre ktoré dvojice štvorcových matíc rovnakého typu platí $AB = BA$
je dôležitá, ale nemá jednoduchú odpoveď.


\section{Matice ako zobrazenia vektorov}

Uvažujme maticu $A=(a_{ij})$ typu $m \times n$; sprava ju môžeme vynásobiť vektorom (t.j. stĺpcom) typu $n \times 1$;
výsledok je opäť stĺpec typu $m \times 1$ (t.j. vektor).

Príkl.
$$ \begin{pmatrix} 1 & -3 & \frac{1}{2} \\ 0 & 4 & -2 \end{pmatrix} \begin{pmatrix} 2 \\ 0 \\ 1 \end{pmatrix} = \begin{pmatrix} \frac{3}{2} \\ -2 \end{pmatrix} $$

Týmto spôsobom máme s každou maticou $A$ typu $m \times n$ asociované zobrazenie
$$ [[A]]: \mathbb{R}^{n} \rightarrow \mathbb{R}^{m} $$
dané predpisom $[[A]](\vec{x}) = A\vec{x}$ alebo (slovne) vynásob vektor $\vec{x} \in \mathbb{R}^{n}$ maticou $A$ zľava.
Máme teda nejaký typ matematického objektu (zobrazenie z $\mathbb{R}^n$ do $\mathbb{R}^m$) reprezentovaný iným objektom (matica typu $m \times n$).

Cieľom tohto textu je preskúmať väzbu medzi maticami a zobrazeniami, ktoré reprezentujú.
Teraz uvedieme niekoľko matíc a popíšeme zobrazenia, ktoré reprezentujú. Ak to budeme vedieť, sformulujeme aj význam toho zobrazenia - geometrický alebo iný.

\begin{priklad}[Nulové matice]
Uvažujme nulovú maticu typu $m \times n$. Aké zobrazenie $\mathbb{R}^{n} \rightarrow \mathbb{R}^{m}$ táto matica reprezentuje?
Počítajme:
$$ 
\underbrace{
\begin{pmatrix} 0 & 0 & \dots & 0 \\
0 & 0 & \dots & 0 \\
\vdots & \vdots & \vdots & \vdots \\
0 & 0 & \dots & 0 
\end{pmatrix}}
_{\R^{m\times n}}
\underbrace{
\begin{pmatrix} x_{1} \\
\vdots \\
x_{n} 
\end{pmatrix} 
}_{\in\R^n}
=
\underbrace{
\begin{pmatrix} 0 \\
\vdots \\
0 \end{pmatrix} 
}_{\R^m}
$$
$\in \mathbb{R}^{m \times n}$ \hspace{1em} $\in \mathbb{R}^{n}$ \hspace{1em} $\in \mathbb{R}^{m}$

Vidíme teda, že nulová matica reprezentuje zobrazenie z $\mathbb{R}^{n} \rightarrow \mathbb{R}^{m}$, ktoré zobrazí každý vektor z $\mathbb{R}^{n}$ na prvok $\vec{0}=(0,...,0) \in \mathbb{R}^{m}$, t.j.
konštantné zobrazenie s hodnotou $\vec{0}$. Asi nikoho neprekvapí, že toto zobrazenie sa značí $O$.
\end{priklad}

\begin{priklad}[Jednotkové matice a ich skalárne násobky]
Spomeňme si, že jednotková matica $I_n$ je diagonálna matica typu $n \times n$, ktorá má na diagonále samé 1. Preskúmajme, aké zobrazenie reprezentuje:
$$ 
\underbrace{
\begin{pmatrix} 1 & 0 & 0 & \dots & 0 \\
0 & 1 & 0 & \dots & 0 \\
\vdots & \vdots & \vdots & \ddots & \vdots \\
0 & 0 & 0 & \dots & 1 \end{pmatrix} 
}_{I_n\in\R^{n\times n}}
\underbrace{
\begin{pmatrix} a_{1} \\
a_{2} \\
\vdots \\
a_{n} \end{pmatrix}
}_{\in\R^n}
= 
\underbrace{
\begin{pmatrix} a_{1} \\
a_{2} \\
\vdots \\
a_{n} \end{pmatrix} 
}
_{\in\R^n}
$$

Vidíme teda, že $I_n \vec{x} = \vec{x}$ pre všetky vektory $\vec{x} \in \mathbb{R}^n$
a teda $I_n$ reprezentuje identické zobrazenie $\id_{\mathbb{R}^n}$:
$$
[[I_n]] = \id_{\mathbb{R}^n}.
$$
\end{priklad}
\begin{priklad}
Pozrime sa teraz na trochu všeobecnejší prípad; nech $D_{\alpha}$ je diagonálna matica typu $n \times n$, ktorá má na diagonále tú istú konštantu $\alpha \in \mathbb{R}$. Určme, aké zobrazenie takáto matica reprezentuje.
$$ \underbrace{\begin{pmatrix} \alpha & 0 & 0 & \dots & 0 \\ 0 & \alpha & 0 & \dots & 0 \\ \vdots & \vdots & \ddots & & \vdots \\ 0 & 0 & 0 & \dots & \alpha \end{pmatrix}}_{D_{\alpha} \in \mathbb{R}^{n \times n}} \underbrace{\begin{pmatrix} x_{1} \\ x_{2} \\ \vdots \\ x_{n} \end{pmatrix}}_{\in \mathbb{R}^{n}} = \underbrace{\begin{pmatrix} \alpha x_{1} \\ \alpha x_{2} \\ \vdots \\ \alpha x_{n} \end{pmatrix}}_{\in \mathbb{R}^{n}} = \alpha \vec{x} $$
(násobenie vektora sklárom)

Teda (symbolicky) 
$$
[[D_{\alpha}]](\vec{x})= \alpha\vec{x},
$$
pre všetky $\alpha \in \mathbb{R}$ a $\vec{x} \in \mathbb{R}^n$.
Všimnime si, že (keďže $D_0 = 0$ a $D_1 = I_n$) toto dostávame spätne ako špeciálne
prípady $\alpha=0, \alpha=1$:
\[
[[0]](\vec x)=[[D_0]](\vec x) = 0\vec{x} = \vec{0}
\]
\[
[[I_n]](\vec{x}) = [[D_1]](\vec{x}) = 1\vec{x} = \vec{x} = \id_{\mathbb{R}^n}(\vec{x})
\]
\end{priklad}

%D.Ú.: Popíšte zobrazenia reprezentované diagonálnymi maticami.

\begin{priklad}[Súčty a priemery]
Každá matica $A \in \mathbb{R}^{1 \times n}$ (riadok) reprezentuje zobrazenie
$[[A]]: \mathbb{R}^n \rightarrow \mathbb{R}^1$.
($\downarrow$ prvky $\mathbb{R}^1$ sú usporiadané 1-ice; $\mathbb{R}^1 = \mathbb{R}$)

Uvažujme špeciálne maticu z $\mathbb{R}^{1 \times n}$ obsahujúcu iba 1; máme
$$ (\begin{matrix} 1 & 1 & \dots & 1 \end{matrix}) \begin{pmatrix} x_1 \\ x_2 \\ \vdots \\ x_n \end{pmatrix} = x_1 + x_2 + \dots + x_n $$
Zobrazenie z $\mathbb{R}^n$ do $\mathbb{R}^1$ reprezentované touto maticou je teda dané jednoducho ako "súčet zložiek vektora".

Podobne $(\frac{1}{n} \dots \frac{1}{n})$ reprezentuje zobrazenie "priemer zložiek vektora".
$$ (\begin{matrix} \frac{1}{n} & \dots & \frac{1}{n} \end{matrix}) \begin{pmatrix} v_1 \\ v_2 \\ \vdots \\ v_n \end{pmatrix} = \frac{x_1}{n} + \dots + \frac{x_n}{n} = \frac{x_1 + \dots + x_n}{n} $$
\end{priklad}

\begin{priklad}[Pravouhlá projekcia na priamku]
Vezmime si teraz maticu
$$ P = \begin{pmatrix} 1/2 & 1/2 \\ 1/2 & 1/2 \end{pmatrix} $$
Táto reprezentuje zobrazenie $\mathbb{R}^2 \rightarrow \mathbb{R}^2$. Predpis tohto zobrazenia rozpísaný do zložiek nájdeme ľahko
$$ \begin{pmatrix} \frac{1}{2} & \frac{1}{2} \\ \frac{1}{2} & \frac{1}{2} \end{pmatrix} \begin{pmatrix} x_1 \\ x_2 \end{pmatrix} = \begin{pmatrix} \frac{1}{2}x_1 + \frac{1}{2}x_2 \\ \frac{1}{2}x_1 + \frac{1}{2}x_2 \end{pmatrix} $$
(obe zložky sú vždy rovnaké)

Pokúsme sa interpretovať toto zobrazenie geometricky; stotožníme vektory z $\mathbb{R}^2$ s vektormi v rovine prostredníctvom nejakých dvoch navzájom kolmých súradnicových osí
a určíme polohu obrazov niekoľkých vektorov.
Pre každý vektor $\vec{x}$ má $P\vec{x}$ obe zložky rovnaké. Geometricky toto znamená, že $P\vec{x}$ leží na priamke $q$ prechádzajúcej počiatkom
\begin{equation}\label{eq:q}
q = \{(t,t) : t \in \mathbb{R}\}
\end{equation}
\begin{align*}
[[P]](3,1) &= (2,2)\\
[[P]](0,0) &= (0,0)\\
[[P]](-1,-2) &= \left(-\frac{3}{2}, -\frac{3}{2}\right)
\end{align*}
\begin{center}
\begin{tikzpicture}[
    >=Stealth, % Sets the default arrow tip style
    x=1cm, y=1cm, % Sets the scale
    axis/.style={->, thick},
    vec/.style={->, thick, -{Stealth[length=2.5mm]}},
    dashline/.style={dashed, thick}
]

% --- Title Equation ---

% --- Axes and Ticks ---
\draw [axis] (-3,0) -- (4.5,0); % x-axis
\draw [axis] (0,-4) -- (0,4.5); % y-axis

% Ticks
\foreach \x in {-2,-1,1,2,3} {
    \draw (\x, 2pt) -- (\x, -2pt) node[below, yshift=-2pt] {$\x$};
}
\foreach \y in {-3,-2,-1,1,2,3} {
    \draw (2pt, \y) -- (-2pt, \y) node[left, xshift=-2pt] {$\y$};
}

% Origin Label
\node[below right, xshift=-2pt, yshift=-1pt] at (0,0) {$(0,0)$};
\node[above left, xshift=-1pt, yshift=1pt] at (0,0) {$[[P]](0,0)$};

% --- The Line q ---
\draw [red, thick] (-3.5,-3.5) -- (3.5, 3.5) node[right, xshift=2pt] {$q$};

% --- Point (3,1) and its Projection ---
\coordinate (A) at (3,1);
\coordinate (PA) at (2,2); % The projection [[P]](3,1) is at (3,3)

% Labels
\node[right, xshift=2pt] at (A) {$(3,1)$};
\node[right, xshift=3pt] at (PA) {$[[P]](3,1)$};

% Vectors
\draw [vec] (0,0) -- (A);     % Vector from origin to (3,1)
\draw [vec] (0,0) -- (PA);   % Vector from origin to [[P]](3,1)

% Dashed coordinate lines
\draw [dashline] (A) -- (A |- 0,0); % (3,1) to (3,0)
\draw [dashline] (A) -- (0,0 |- A); % (3,1) to (0,1)
\draw [dashline] (PA) -- (PA |- 0,0); % (3,3) to (3,0)
\draw [dashline] (PA) -- (0,0 |- PA); % (3,3) to (0,3)

% --- Point (-1,-2) and its Projection ---
\coordinate (B) at (-1,-2);
\coordinate (PB) at (-1.5,-1.5); % The projection is at (-1.5,-1.5)
\draw [dashline] (PB) -- (PB |- 0,0); % (3,3) to (3,0)
\draw [dashline] (PB) -- (0,0 |- PB); % (3,3) to (0,3)
\draw [vec] (0,0) -- (B);     % Vector from origin to (B)
\draw [vec] (0,0) -- (PB);     % Vector from origin to (PB)

% Label
\node[below, yshift=-2pt] at (B) {$(-1,-2)$};
\node[left, xshift=-2pt] at (PB) {$[[P]](-1,-2)$};

% Dashed coordinate lines
\draw [dashline] (B) -- (B |- 0,0); % (-1,-2) to (-1,0)
\draw [dashline] (B) -- (0,0 |- B); % (-1,-2) to (0,-2)

% Mapping
\draw [dashline,->, >=Stealth, blue] (A)--(PA);
\draw [dashline,->, >=Stealth, blue] (B)--(PB);

\end{tikzpicture}
\end{center}

Po vyskúšaní pár bodov dospejeme k hypotéze $[[P]](\vec{x})$ je priemet $\vec{x}$ na
$q$ v pravom uhle. Táto hypotéza je naozaj pravdivá. Je možné ju dokázať holými
rukami, ale rozumnejšie je odložiť jej dôkaz na neskôr, do druhého semestra, keď
budeme mať k dispozícii mocnejšie nástroje.
\end{priklad}

\begin{priklad}[Rovinná rotácia]
Uvažujme, pre $\alpha \in \langle 0, 2\pi \rangle$ maticu 
\[
L_{\alpha} = \begin{pmatrix} \cos\alpha & -\sin\alpha \\ \sin\alpha & \cos\alpha \end{pmatrix}
\]
Máme
$$ [[L_{\alpha}]](x_1, x_2) = \begin{pmatrix} \cos\alpha & -\sin\alpha \\ \sin\alpha & \cos\alpha \end{pmatrix} \begin{pmatrix} x_1 \\ x_2 \end{pmatrix} = \begin{pmatrix} x_1 \cos\alpha - x_2 \sin\alpha \\ x_1 \sin\alpha + x_2 \cos\alpha \end{pmatrix} $$
Význam tohto zobrazenia je rotácia
proti smeru hodinových ručičiek o uhol $\alpha$ vľavo okolo počiatku.
\begin{center}
\begin{tikzpicture}
    [dashline/.style={dashed, thick}]
    % Axes
    \draw ({-pi-0.5},0) -- ({pi+0.5},0);
    \draw (0,{-pi-0.5}) -- (0,{pi+0.5});

    % Vector x
    \coordinate (X) at (2, 1.5);
    \draw[-Stealth] (0,0) -- (X) node[below right] {$\vec{x}=(x_1,x_2)$};

    % Rotated Vector L_alpha(x)
    \pgfmathsetmacro{\angle}{atan2(1.5,2)}
    \pgfmathsetmacro{\rotatedangle}{\angle + 30} % Example rotation by 30 degrees
    \pgfmathsetmacro{\rotatedx}{2.5 * cos(\rotatedangle)}
    \pgfmathsetmacro{\rotatedy}{2.5 * sin(\rotatedangle)}
    \coordinate (LX) at (\rotatedx, \rotatedy);
    \draw[-Stealth] (0,0) -- (LX) node[above] {$[[L_\alpha]](\vec{x})$};

    % Angle alpha
    \pgfmathsetmacro{\startangle}{atan2(1.5,2)}
    \pgfmathsetmacro{\endangle}{\startangle + 30}
    \draw[dotted, thick, ->, >=Stealth, blue, bend left=15] (X) arc (\startangle:\endangle:2.5);
    \node at (2.25, 2.25) {$\alpha$};
    \node [below] at (2,0) {$x_1$};
    \node [left] at (0,1.5) {$x_2$};
    \draw[dashed](2,0) -- (X);
    \draw[dashed](0,1.5) -- (X);

\end{tikzpicture}
\end{center}
Toto nebudeme dokazovať teraz, ale dokážeme to neskôr.
\end{priklad}

\begin{priklad}[Osová súmernosť podľa priamky]
Uvažujme maticu $B = \begin{pmatrix} 0 & 1 \\ 1 & 0 \end{pmatrix}$ a počítajme pre $\vec{x}=(x_1, x_2)$ máme
$$ [[B]](\vec{x}) = B\vec{x} = \begin{pmatrix} 0 & 1 \\ 1 & 0 \end{pmatrix} \begin{pmatrix} x_1 \\ x_2 \end{pmatrix} = \begin{pmatrix} x_2 \\ x_1 \end{pmatrix} $$
Toto zobrazenie teda vymieňa zložky. Geometricky sa toto dá vyjadriť ako osová súmernosť
podľa priamky $q$, ktorú už poznáme \eqref{eq:q}
\begin{center}
\begin{tikzpicture}
[
    >=Stealth, % Sets the default arrow tip style
    x=1cm, y=1cm, % Sets the scale
    axis/.style={->, thick},
    vec/.style={->, thick, -{Stealth[length=2.5mm]}},
    dashline/.style={dashed, thick}
]
    % Osi
    \draw [->, thick] (-1,0) -- (4,0) node [right] {$x_1$};
    \draw [->, thick] (0,-1) -- (0,4) node [above] {$x_2$};
\foreach \x in {1,2,3} {
    \draw (\x, 2pt) -- (\x, -2pt) node[below, yshift=-2pt] {$\x$};
}
\foreach \y in {1,2,3} {
    \draw (2pt, \y) -- (-2pt, \y) node[left, xshift=-2pt] {$\y$};
}

    
    % Priamka q (os symetrie)
    \draw [red] (-1,-1) -- (4,4) node [right] {$q$};
    
    % Vektor x
    \coordinate (A) at (3,1);
    \draw [vec] (0,0) -- (A) node [right, black] {$(3,1)$};
    
    % Vektor B(x)
    \coordinate (B) at (1,3);
    \draw [vec] (0,0) -- (B) node [above, black] {$[[B]](3,1)$};
    
    % Pociatok
    \fill (0,0) circle (1.5pt) node [below left] {$(0,0)$};
    \draw [dashline] (A) -- (A |- 0,0); 
    \draw [dashline] (A) -- (0,0 |- A); 
    \draw [dashline] (B) -- (B |- 0,0); 
    \draw [dashline] (B) -- (0,0 |- B); 
    \draw [dashline,->, >=Stealth, blue] (A)--(B);
\end{tikzpicture}
\end{center}
\end{priklad}

\subsection{Ktoré zobrazenia sú reprezentovateľné maticami?}
V tomto momente vzniká prirodzená otázka, ktoré zobrazenia $\mathbb{R}^n \rightarrow \mathbb{R}^m$ sú reprezentovateľné maticami $R^{m \times n}$;
väčšina toho, čo sa budeme učiť vo zvyšku tohto semestra sa bude týkať hľadania systematickej odpovede na túto otázku.
Zatiaľ sa obmedzíme na konštatovanie, že nie všetky zobrazenia $\mathbb{R}^n \rightarrow \mathbb{R}^m$ sú reprezentovateľné maticou.
Napríklad iste platí, že $A\vec{0} = \vec{0}$ pre každú
maticu $A$. Teda ľubovoľné zobrazenie, ktoré
zobrazí vektor $\vec{0}$
na nenulový vektor iste
reprezentovateľné maticou nebude.

\subsection{Súčin matíc je reprezentácia zloženého zobrazenia}
\begin{veta}\label{veta:nasobenieMatic}
Nech $A \in \mathbb{R}^{m \times n}, B \in \mathbb{R}^{k \times m}$.
Uvažujme zobrazenia reprezentované maticami $A, B$.
$$ [[A]]: \mathbb{R}^n \rightarrow \mathbb{R}^m $$
$$ [[B]]: \mathbb{R}^m \rightarrow \mathbb{R}^k $$
Potom zložené zobrazenie $[[B]] \circ [[A]]$ je práve zobrazenie asociované s maticou
$BA$:
\[
[[B]] \circ [[A]] = [[B A]]
\]
\end{veta}
\begin{proof}
Máme dokázať, že $[[B]] \circ [[A]] = [[B A]]$.
Obe tieto zobrazenia sú typu $\mathbb{R}^n \rightarrow \mathbb{R}^k$. Nech $\vec{x} \in \mathbb{R}^n$.
\begin{align*}
([[B]] \circ [[A]])(\vec{x}) &= [[B]]([[A]](\vec{x})) & \text{(definícia zloženého zobrazenia)} \\
&= [[B]](A\vec{x}) & \text{(predpis pre $[[A]]$)} \\
&= B(A\vec{x}) & \text{(predpis pre $[[B]]$)} \\
&= (B A)\vec{x} & \text{(asociativita násobenia matíc)} \\
&= [[B A]](\vec{x}) & \text{(predpis pre $[[BA]]$)}
\end{align*}
Zobrazenia $[[BA]]$ a $[[B]] \circ [[A]]$ majú rovnaký definičný obor aj koobor a
každý vektor $\vec{x} \in \mathbb{R}^n$ zobrazia na rovnaký vektor v $\mathbb{R}^k$.
To znamená, že $[[B]] \circ [[A]] = [[BA]]$.
\end{proof}

\begin{priklad}
Uvažujme maticu rovinnej rotácie $L_{\frac{\pi}{2}}$:
$$L_{\frac{\pi}{2}}= \begin{pmatrix} \cos\frac{\pi}{2} & -\sin\frac{\pi}{2} \\ \sin\frac{\pi}{2} & \cos\frac{\pi}{2} \end{pmatrix} = \begin{pmatrix} 0 & -1 \\ 1 & 0 \end{pmatrix} $$
$$ L_{\frac{\pi}{2}} L_{\frac{\pi}{2}} = \begin{pmatrix} 0 & -1 \\ 1 & 0
\end{pmatrix} \begin{pmatrix} 0 & -1 \\ 1 & 0 \end{pmatrix} = \begin{pmatrix} -1 & 0
\\ 0 & -1 \end{pmatrix} 
=D_{-1}
$$
Táto matica by mala reprezentovať zložené zobrazenie $[[L_{\frac{\pi}{2}}]] \circ
[[L_{\frac{\pi}{2}}]]$, teda zobrazenie
$[[L_{\frac{\pi}{2}+\frac{\pi}{2}}]]=[[L_\pi]]$.
Matica $D_{-1}$ reprezentuje zobrazenie násobenie skalárom $-1$. Ale vynásobiť
rovinný vektor skalárom $-1$ je to isté ako otočiť ho o $\pi$ vľavo (alebo vpravo), teda $D_{-1} = L_{\pi}$.
\end{priklad}

\begin{priklad}
Ak vezmeme pravouhlú projekciu $P$ na priamku $q$ (viď vyššie), očakávame, že
$PP = P$, pretože $[[P]]\circ [[P]]=[[P]]$.
Naozaj:
$$ \begin{pmatrix} 1/2 & 1/2 \\ 1/2 & 1/2 \end{pmatrix} \begin{pmatrix} 1/2 & 1/2 \\ 1/2 & 1/2 \end{pmatrix} = \begin{pmatrix} 1/4+1/4 & 1/4+1/4 \\ 1/4+1/4 & 1/4+1/4 \end{pmatrix} = \begin{pmatrix} 1/2 & 1/2 \\ 1/2 & 1/2 \end{pmatrix} $$
\end{priklad}

%D.Ú.: Nájdite geometrickú interpretáciu zobrazenia $P \circ D_2$. Kreslite si obrázky.
%V akom vzťahu sú zobrazenia $P \circ D_2$ a $D_2 \circ P$?
%Z toho vyplýva, presvedčte sa $P D_2 \neq D_2 P$.

%D.Ú.: Ukážte, že $L_{\alpha} \circ P \neq P \circ L_{\alpha}$ a vypočítajte aj príslušné matice zložených zobrazení.

\subsection{Inverzná matica}
\begin{definicia}\label{def:inverzna}
Nech $A$ je štvorcová matica typu $n \times n$. \emph{Inverzná matica k
$A$} je taká matica $A^{-1}$ typu $n \times n$, pre ktorú platí 
$$ A A^{-1} = A^{-1} A = I_n $$
\end{definicia}
Pozor! Inverzná matica k štvorcovej matici nemusí existovať. Napríklad nulová matica iste nemá inverznú, prečo?

\begin{definicia}
Nech $A$ je štvorcová matica. Ak $A$ má inverznú, hovoríme, že $A$ je
\emph{regulárna}. V opačnom prípade hovoríme, že $A$ je \emph{singulárna}.
\end{definicia}
\begin{priklad}
Ak 
\[
A = \begin{pmatrix}
1 & 1 & 1 \\
1 & 2 & 1 \\
-3 & -6 & -2
\end{pmatrix}
\]
potom inverzná matica k $A$ je
\[
A^{-1} = \begin{pmatrix}
2 & -4 & -1 \\
-1 & 1 & 0 \\
0 & 3 & 1
\end{pmatrix}
\]
\end{priklad}

\begin{priklad}[Inverzná rotácia]
\[
L_{-\alpha} = \begin{pmatrix} \cos(-\alpha) & -\sin(-\alpha) \\ \sin(-\alpha) & \cos(-\alpha) \end{pmatrix} = \begin{pmatrix} \cos\alpha & \sin\alpha \\ -\sin\alpha & \cos\alpha \end{pmatrix}
\]
je inverzná matica k $L_{\alpha}$.
\end{priklad}

Keďže $A, A^{-1}$ sú štvorcové rovnakého typu, povedzme $A, A^{-1} \in \mathbb{R}^{n \times n}$, reprezentujú nejaké zobrazenia
$$ [[A]]: \mathbb{R}^n \rightarrow \mathbb{R}^n $$
$$ [[A^{-1}]]: \mathbb{R}^n \rightarrow \mathbb{R}^n $$
Definícia inverznej matice hovorí, že
$$ A A^{-1} = A^{-1} A = I_n $$
Keďže tieto sú matice, sú rovnaké aj nimi reprezentované zobrazenia sú rovnaké
$$ [[A A^{-1}]] = [[A^{-1} A]] = [[I_n]] $$
Podľa Vety \ref{veta:nasobenieMatic} ale
$$ [[A A^{-1}]] = [[A]] \circ [[A^{-1}]] $$
$$ [[A^{-1} A]] = [[A^{-1}]] \circ [[A]] $$
a vieme, že
$$ [[I_n]] = \id_{\mathbb{R}^n} $$
Z toho dostávame rovnosť zobrazení
$$ [[A]] \circ [[A^{-1}]] = [[A^{-1}]] \circ [[A]] = \id_{\mathbb{R}^n} 
$$
Ale to presne znamená, že $[[A^{-1}]]$ je zobrazenie inverzné k zobrazeniu $[[A]]$,
viď Definícia \ref{def:inverzneZobrazenie}.

\begin{veta}[Inverzia matice je inverzia zobrazenia]
Nech $A \in \mathbb{R}^{n \times n}$ je regulárna matica. Potom $[[A^{-1}]]$ je inverzné zobrazenie k zobrazeniu $[[A]]$.
Kompaktne to môžeme zapísať ako
$$ [[A^{-1}]] = [[A]]^{-1} $$
\end{veta}

\subsection{Inverzia súčinu matíc}

\begin{veta}
Nech $A, B$ sú regulárne matice rovnakého typu. Vtom aj $AB$ je regulárna matica a~platí $(AB)^{-1} = B^{-1}A^{-1}$.
\end{veta}

\begin{proof}
\begin{align*}
(AB) \cdot (B^{-1}A^{-1}) &= A (B B^{-1}) A^{-1} = A I A^{-1} = A A^{-1} = I \\
(B^{-1}A^{-1}) \cdot (AB) &= B^{-1} (A^{-1} A) B = B^{-1} I B = B^{-1} B = I
\end{align*}
\end{proof}

\textbf{POZOR:} Nič podobné neplatí pre súčet matíc.

\subsection{Počítanie inverznej matice}
Postup: napíšeme si vedľa seba maticu, ktorú chceme invertovať a jednotkovú maticu, oddelíme čiarou.
$$ ( A | I_n ) $$
(typu $n \times 2n$)
potom používame na obe matice (celé riadky) rovnaké elementárne riadkové operácie, ktorými sa snažíme upraviť maticu vľavo od čiary tak, aby sme tam dostali $I_n$. Ak sa nám to podarí, napravo od čiary máme $A^{-1}$.
$$ ( I_n | A^{-1} ) $$

\begin{priklad}
Takto sa to urobí pre konkrétnu maticu.
\[
\left[
\begin{array}{ccc|ccc}
1 & 1 & 1 & 1 & 0 & \tikzmarknode{r1a}{0} \\
1 & 2 & 1 & 0 & 1 & \tikzmarknode{r2a}{0} \\
-3 & -6 & -2 & 0 & 0 & 1
\end{array}
\right]
\begin{tikzpicture}[remember picture, overlay]
    \draw[->, thick, shorten >=2pt]
        ([xshift=1.5em]r1a.east)
        -- ++(1em,0) coordinate (corner)
        -- (corner |- r2a.east)
        node [midway, right] {$-1$}
        -- ([xshift=1.5em]r2a.east);
\end{tikzpicture}
\hspace{3.5em} \sim \quad
\left[
\begin{array}{ccc|ccc}
1 & 1 & 1 & 1 & 0 & \tikzmarknode{r1b}{0} \\
0 & 1 & 0 & -1 & 1 & 0 \\
-3 & -6 & -2 & 0 & 0 & \tikzmarknode{r3b}{1}
\end{array}
\right]
\begin{tikzpicture}[remember picture, overlay]
    \draw[->, thick, shorten >=2pt]
        ([xshift=1.5em]r1b.east)
        -- ++(1em,0) coordinate (corner)
        -- (corner |- r3b.east)
        node [midway, right] {$3$}
        -- ([xshift=1.5em]r3b.east);
\end{tikzpicture}
\hspace{3.5em} \sim \quad
\]
\[
\left[
\begin{array}{ccc|ccc}
1 & 1 & 1 & 1 & 0 & \tikzmarknode{r1c}{0} \\
0 & 1 & 0 & -1 & 1 & \tikzmarknode{r2c}{0} \\
0 & -3 & 1 & 3 & 0 & 1
\end{array}
\right]
\begin{tikzpicture}[remember picture, overlay]
    \draw[->, thick, shorten >=2pt]
        ([xshift=1.5em]r2c.east)
        -- ++(1em,0) coordinate (corner)
        -- (corner |- r1c.east)
        node [midway, right] {$-1$}
        -- ([xshift=1.5em]r1c.east);
\end{tikzpicture}
\hspace{3.5em} \sim \quad
\left[
\begin{array}{ccc|ccc}
1 & 0 & 1 & 2 & -1 & 0 \\
0 & 1 & 0 & -1 & 1 & \tikzmarknode{r2d}{0} \\
0 & -3 & 1 & 3 & 0 & \tikzmarknode{r3d}{1}
\end{array}
\right]
\begin{tikzpicture}[remember picture, overlay]
    \draw[->, thick, shorten >=2pt]
        ([xshift=1.5em]r2d.east)
        -- ++(1em,0) coordinate (corner)
        -- (corner |- r3d.east)
        node [midway, right] {$3$}
        -- ([xshift=1.5em]r3d.east);
\end{tikzpicture}
\hspace{3.5em} \sim \quad
\]
\[
\left[
\begin{array}{ccc|ccc}
1 & 0 & 1 & 2 & -1 & \tikzmarknode{r1e}{0} \\
0 & 1 & 0 & -1 & 1 & 0 \\
0 & 0 & 1 & 0 & 3 & \tikzmarknode{r3e}{1}
\end{array}
\right]
\begin{tikzpicture}[remember picture, overlay]
    \draw[->, thick, shorten >=2pt]
        ([xshift=1.5em]r3e.east)
        -- ++(1em,0) coordinate (corner)
        -- (corner |- r1e.east)
        node [midway, right] {$-1$}
        -- ([xshift=1.5em]r1e.east);
\end{tikzpicture}
\hspace{3.5em} \sim \quad
\left[
\begin{array}{ccc|ccc}
1 & 0 & 0 & 2 & -4 & -1 \\
0 & 1 & 0 & -1 & 1 & 0 \\
0 & 0 & 1 & 0 & 3 & 1
\end{array}
\right]
\]
\end{priklad}

\subsection{Sústavy lineárnych rovníc - nový pohľad}

Prizrime sa na sústavy lineárnych rovníc v kontexte zobrazení asociovaných z maticami.
Každá matica $A \in \R^{m \times n}$ nám určuje zobrazenie $[[A]]: \R^n \to \R^m$
predpisom $[[A]](\vec{x}) = A\vec{x}$.
Ak si predpis pre $[A]$ rozpíšeme do zložiek, dostaneme, pre $\vec{x} = (x_1, \dots, x_n) \in \R^n$, $A = (a_{ij})_{m \times n} \in \R^{m \times n}$:

$$ [[A]](\vec{x}) = \begin{pmatrix} a_{11} & a_{12} & \dots & a_{1n} \\ a_{21} & a_{22} & \dots & a_{2n} \\ \vdots & \vdots & \ddots & \vdots \\ a_{m1} & a_{m2} & \dots & a_{mn} \end{pmatrix} \begin{pmatrix} x_1 \\ x_2 \\ \vdots \\ x_n \end{pmatrix} = \begin{pmatrix} a_{11}x_1 + a_{12}x_2 + \dots + a_{1n}x_n \\ a_{21}x_1 + a_{22}x_2 + \dots + a_{2n}x_n \\ \vdots \\ a_{m1}x_1 + a_{m2}x_2 + \dots + a_{mn}x_n \end{pmatrix} $$

To je ale presne a~doslova to, čo poznáme pod menom ľavá strana sústavy lineárnych
rovníc! Ak teraz vezmeme nejaký vektor $\vec{b} = (b_1, \dots, b_m)$, tak
$[[A]](\vec{x}) = \vec{b}$ presne znamená:

\begin{align*}
 a_{11}x_1 + a_{12}x_2 + \dots + a_{1n}x_n &= b_1 \\
 a_{21}x_1 + a_{22}x_2 + \dots + a_{2n}x_n &= b_2 \\
 \vdots \qquad \qquad \qquad \qquad & \\
 a_{m1}x_1 + a_{m2}x_2 + \dots + a_{mn}x_n &= b_m
\end{align*}

čo je doslova sústava lineárnych rovníc. Vidíme teda, že (pre danú maticu $A$ a~vektor $\vec{b}$)
je to isté:

\begin{center}
    \begin{minipage}[c]{0.4\textwidth}
        \centering
        vyriešiť \\
        sústavu lineárnych \\
        rovníc $(A|\vec{b})$
    \end{minipage}
    \begin{minipage}[c]{0.1\textwidth}
        \centering
        $\Leftrightarrow$
    \end{minipage}
    \begin{minipage}[c]{0.4\textwidth}
        \centering
        nájsť množinu \\
        všetkých vektorov $\vec{x}$ \\
        takých, že $[A](\vec{x}) = \vec{b}$, resp. $A\vec{x} = \vec{b}$
    \end{minipage}
\end{center}

\subsection{Jednotkové vektory v $\R^n$}

Zaveďme novú notáciu: v~$\R^n$ budeme označovať $\vec{e}_1, \dots, \vec{e}_n$ vektory:
\begin{align*}
 \vec{e}_1 &= (1, 0, \dots, 0, 0) \text{} \\
 \vec{e}_2 &= (0, 1, \dots, 0, 0) \text{} \\
 \vdots \quad & \\
 \vec{e}_n &= (0, 0, \dots, 0, 1) \text{}
\end{align*}
Všimnime si, že sú to presne stĺpce jednotkovej matice $I_n$; $s_i(I_n) = \vec{e}_i$ a~môžeme teda písať
$$ I_n = (\vec{e}_1 \vec{e}_2 \dots \vec{e}_n) $$

\subsection{Prečo funguje algoritmus počítania inverznej matice}

Spomeňme si, ako sme počítali sústavy lineárnych rovníc: pomocou elementárnych riadkových operácií sme upravovali ľavú stranu na stupňovitý tvar.
Potom sme robili ,,spätné dosádzanie''. Mohli sme ale postupovať aj trochu inak: pred fázou spätného dosádzania vynulovať aj prvky nad diagonálou:
$$ \left( \begin{array}{ccccc|c}
\blacksquare & 0 & 0 & \dots & 0 & \blacksquare \\
0 & \blacksquare & 0 & \dots & 0 & \blacksquare \\
0 & 0 & \blacksquare & \dots & 0 & \blacksquare \\
\vdots & \vdots & \vdots & \ddots & \vdots & \vdots \\
0 & 0 & 0 & \dots & \blacksquare & \blacksquare
\end{array} \right) $$
to by nám spätné dosádzanie výrazne zjednodušilo.

Najlepší prípad je ten, keď máme $n$ rovníc o~$n$ neznámych a~na ľavej strane nám vyjde diagonálna matica. Vtedy môžeme postupovať ďalej a~podeliť každý riadok diagonálnym prvkom.
Vtedy nám na pravej strane vyjde priamo riešenie sústavy, a~na ľavej strane máme jednotkovú maticu.
$$ (A|\vec{b}) \sim \dots \sim (I|\vec{x}) $$
Ak sa nám toto teda podarí, našli sme (jediný) vektor taký, že $A\vec{x} = \vec{b}$ a~je to presne pravá strana po eliminácii matice sústavy na jednotkovú.

Pozrime sa teraz z~novej perspektívy na algoritmus počítania inverznej matice:
$$ (A|I_n) \sim \dots \sim (I_n|Y) $$

\begin{center}
na tento postup môžeme nahliadať ako na \\
súbežné riešenie $n$ sústav lineárnych rovníc \\
$(A|\vec{e}_1), \dots, (A|\vec{e}_n)$
\end{center}

\begin{center}
\begin{tabular}{p{0.45\textwidth} p{0.45\textwidth}}
o tejto matici $Y$ tvrdíme, že je inverzná k $A$; označme jej stĺpce $\vec{y}_1, \dots, \vec{y}_n$ & $\vec{y}_i$ je riešenie sústavy $(A|\vec{e}_i)$, pre $i \in \{1, \dots, n\}$ \\
$Y = (\vec{y}_1 \dots \vec{y}_n)$ & $\Downarrow$ \\
& znamená, že $A\vec{y}_i = \vec{e}_i$ pre $i \in \{1, \dots, n\}$
\end{tabular}
\end{center}

Ak si teraz uvedomíme, ako sa násobia matice (napravo postupujeme po stĺpcoch), môžeme nahliadnuť, že toto znamená:
$$ A \cdot \underbrace{(\vec{y}_1 \dots \vec{y}_n)}_{\text{Toto je } Y \text{}} = (A\vec{y}_1 \dots A\vec{y}_n) = \underbrace{(\vec{e}_1 \dots \vec{e}_n)}_{\text{Toto je } I_n \text{}} \implies AY = I_n $$

V poriadku, ale prečo platí aj $YA = I_n$?
Aby sme to dokázali, je vhodné zaviesť pojem elementárnej matice:

\begin{definicia}[Elementárna matica]
$E$ je elementárna matica, ak vznikla z~jednotkovej matice $I$ vykonaním jednej elementárnej riadkovej operácie.
\end{definicia}

Napríklad pre pripočítanie $\frac{1}{2}$-násobku prvého riadku $I_3$ k~druhému dostaneme:
$$ \begin{pmatrix} 1 & 0 & 0 \\ 0 & 1 & 0 \\ 0 & 0 & 1 \end{pmatrix} \xrightarrow{r_2 \leftarrow r_2 + \frac{1}{2}r_1} \begin{pmatrix} 1 & 0 & 0 \\ \frac{1}{2} & 1 & 0 \\ 0 & 0 & 1 \end{pmatrix} \to \text{elementárna matica} $$

Elementárne matice majú túto peknú vlastnosť: pre každú maticu $A$:

\begin{center}
\begin{tabular}{c c c}
$A \sim A'$ & $\Leftrightarrow$ & $A' = EA$ \\
$\uparrow$ & & $\uparrow$ \\
\parbox{4cm}{\centering elementárna riadková operácia} & & \parbox{5cm}{\centering toto je presne tá matica, ktorá vznikne z $A$ vykonaním elementárnej riadkovej operácie, ,,zakódovanej'' v $E$.}
\end{tabular}
\end{center}

Pri počítaní inverznej matice sa naľavo dialo toto:
$$ \underbrace{A \sim \dots \sim I_n}_{\text{elementárne riadkové operácie}} $$
Pomocou elementárnych matíc to môžeme zapísať takto:
\begin{equation} \label{eq:inv_left}
A \sim E_1 A \sim E_2 E_1 A \sim \dots \sim E_k \dots E_2 E_1 A = I_n
\end{equation}
Napravo sa zase dialo toto:
\begin{equation} \label{eq:inv_right}
I_n \sim E_1 I_n \sim E_2 E_1 I_n \sim \dots \sim E_k \dots E_2 E_1 I_n = E_k \dots E_2 E_1 = Y
\end{equation}
Z \eqref{eq:inv_left} a \eqref{eq:inv_right} zrejme dostávame $YA = I_n$.
Teda platí $AY = YA = I_n$
a~preto $Y = A^{-1}$, viď definícia inverznej matice \ref{def:inverzna}.



\end{document}
