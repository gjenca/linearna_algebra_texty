\subsection{Lineárny obal}

\begin{definicia} \label{def:10.1}
Nech $V$ je vektorový priestor, nech $X=(\vec{x}_{1},...,\vec{x}_{n})$ je $n$-tica vektorov z $V$. Potom \textbf{lineárny obal} je množina vektorov
$$ \Lo(X)=\{a_{1}\vec{x}_{1}+...+a_{n}\vec{x}_{n} \mid a_{1},...,a_{n}\in \R\} $$
\end{definicia}

Vidíme, že predpis pre vytvorenie lineárneho obalu $n$-tice $X$ sa dá vyjadriť slovne ako \textit{všetky lineárne kombinácie vektorov $X$}. Analogicky môžeme definovať lineárny obal množiny vektorov.

\textbf{Cvičenie:} Ak $X=(\vec{u}, \vec{v})$ je dvojica vektorov v rovine, ako môže vyzerať $\Lo(X)$? Uvažujte prípady ako $\vec{u}=\vec{0}$, $\vec{u}=c\vec{v}$, $\vec{u} \neq c\vec{v}$.


\begin{priklad}
Skúsme si napísať ako vyzerá lineárny obal dvojice vektorov $((-1,2,3), (2,0,2))$ v $\R^{3}$.
\begin{align*}
\Lo(X) &= \{a_{1}(-1,2,3)+a_{2}(2,0,2) \mid a_{1},a_{2}\in \R\} \\
&= \{(-a_{1},2a_{1},3a_{1})+(2a_{2},0,2a_{2}) \mid a_{1},a_{2}\in \R\} \\
&= \{(-a_{1}+2a_{2},2a_{1},3a_{1}+2a_{2}) \mid a_{1},a_{2}\in \R\}
\end{align*}
\end{priklad}

Označme vektorový priestor všetkých funkcií $\R \rightarrow \R$ ako $\R^\R$.

\begin{priklad}
Ako vyzerá lineárny obal $n$-tice funkcií $(1,x,x^{2},...,x^{n})$? 
Tuto (trochu lajdácky) stotožňujeme funkciu v $\R^\R$ s jej predpisom.
$$ \Lo(1,x,x^{2},...,x^{n}) = \{a_{0}+a_{1}x+...+a_{n}x^{n} \mid a_{0},a_{1},...,a_{n}\in \R\} $$
To je množina všetkých polynómov stupňa nanajvýš $n$, označujeme ju $\R^{n}[x]$.
\end{priklad}

\begin{veta} \label{veta:10.2}
Nech $V$ je vektorový priestor, nech $X=(\vec{x}_{1},...,\vec{x}_{n})$ je $n$-tica vektorov z $V$. Potom $\Lo(X)$ je podpriestor $V$.
\end{veta}

\begin{proof}
Overíme podmienky podpriestoru:
\begin{enumerate}
    \item $\vec{0}\in \Lo(X)$, lebo $\vec{0}=0\vec{x}_{1}+...+0\vec{x}_{n}\in \Lo(X)$.
    \item Nech $\vec{u}, \vec{v} \in \Lo(X)$. Potom
    $$ \vec{u}=a_{1}\vec{x}_{1}+...+a_{n}\vec{x}_{n} $$
    pre nejaké $a_{1},...,a_{n}\in \R$ a
    $$ \vec{v}=b_{1}\vec{x}_{1}+...+b_{n}\vec{x}_{n} $$
    pre nejaké $b_{1},...,b_{n}\in \R$.
    Máme dokázať, že $\vec{u}+\vec{v}\in \Lo(X)$. Počítajme:
    \begin{align*}
    \vec{u}+\vec{v} &= (a_{1}\vec{x}_{1}+...+a_{n}\vec{x}_{n}) + (b_{1}\vec{x}_{1}+...+b_{n}\vec{x}_{n}) \\
    &= (a_{1}+b_{1})\vec{x}_{1}+...+(a_{n}+b_{n})\vec{x}_{n}\in \Lo(X)
    \end{align*}
    \item Nech $\vec{u}$ je ako v bode 2), $c\in \R$. Potom
    $$ c\vec{u}=c(a_{1}\vec{x}_{1}+...+a_{n}\vec{x}_{n})=(ca_{1})\vec{x}_{1}+...+(ca_{n})\vec{x}_{n}\in \Lo(X) $$
\end{enumerate}
\end{proof}

Vidíme teda, že každý lineárny obal $n$-tice je podpriestorom priestoru, z ktorého vyberáme. Teda medziiným je $\R^{n}[x]=\Lo(1,x,...,x^{n})$ podpriestorom $\R^{\R}$.

\begin{definicia} \label{def:10.3}
Hovoríme, že vektorový priestor $V$ je \textbf{konečnorozmerný}, ak existuje $n$-tica $X=(\vec{x}_{1},...,\vec{x}_{n})$ vektorov z $V$ taká, že $\Lo(X)=V$. Ak $V$ nie je konečnorozmerný, je \textbf{nekonečnorozmerný}.
\end{definicia}

\begin{veta} \label{veta:10.4}
Vektorový priestor $\R^{n}$ je konečnorozmerný pre každé $n \ge 1$.
\end{veta}

\begin{proof}
Vezmime vektory
\begin{align*}
\vec{e}_{1} &= (1,0,0,...,0) \\
\vec{e}_{2} &= (0,1,0,...,0) \\
&\vdots \\
\vec{e}_{n} &= (0,0,0,...,1)
\end{align*}
Tvrdíme, že $\R^n = \Lo(\vec{e}_1, ..., \vec{e}_n)$. Nech $\vec{v}=(a_{1},...,a_{n})$ je nejaký vektor z $\R^{n}$. Máme:
\begin{align*}
a_{1}\vec{e}_{1}+...+a_{n}\vec{e}_{n} &= a_{1}(1,0,...,0)+...+a_{n}(0,0,...,1) \\
&= (a_{1},0,...,0)+...+(0,0,...,a_{n}) \\
&= (a_{1},...,a_{n})=\vec{v}
\end{align*}
Teda $\vec{v} \in \Lo(\vec{e}_{1},...,\vec{e}_{n})$.
\end{proof}

Uvedomme si, že $\R^n[x]$ je tiež konečnorozmerný (lebo sme ho definovali ako lineárny obal nejakej $n$-tice vektorov).

V tomto momente vzniká prirodzená otázka: existuje nejaký priestor, ktorý nie je konečnorozmerný?
Odpoveď je, samozrejme, ,,Áno'': napríklad vektorový priestor $\R^\R$ nie je konečnorozmerný. Dôkaz presahuje zamýšľaný rámec prednášky, preto ho nespravíme.
Skoro všetky vektorové priestory na tejto prednáške budú však konečnorozmerné.

Z dôkazu Vety \ref{veta:10.4} vieme, že $\R^{2}=\Lo((1,0),(0,1))$.
Aké sú iné $n$-tice $X$ také, že $\R^{2}=\Lo(X)$?
Môžeme napríklad vziať $X=((1,0),(0,1),(-2,1))$. Zrejme $\Lo(X)=\R^{2}$ (vektor $(-2,1)$ je ,,naviac''); ale môžeme napríklad vyhodiť z $X$ iný vektor a položiť si otázku, či $\Lo((1,0),(-2,1))=\R^{2}$? [áno]. Napríklad ale $\Lo((1,0),(2,0)) \ne \R^{2}$: prečo?

Pri rozmýšľaní o podobných otázkach vzniká prirodzene pojem z nasledujúcej definície.
(Veta: $\Lo(X)$ sa nezmení, ak sú vektory lineárne kombinované).

\begin{definicia} \label{def:10.5}
Nech $V$ je vektorový priestor, nech $X=(\vec{x}_{1},...,\vec{x}_{n})$ je $n$-tica vektorov z $V$. Hovoríme, že $X$ je \textbf{lineárne závislá}, ak existujú $a_1,..., a_n \in \R$, nie všetky rovné $0$, také, že
$$ a_{1}\vec{x}_{1}+a_{2}\vec{x}_{2}+...+a_{n}\vec{x}_{n}=\vec{0} $$
Ak $X$ nie je lineárne závislá, hovoríme, že $X$ je \textbf{lineárne nezávislá}.
\end{definicia}

Inými slovami: $n$-tica vektorov $(\vec{x}_{1},...,\vec{x}_{n})$ je lineárne nezávislá, ak má rovnica
\begin{equation}\label{eq:linkombJeNula}
a_{1}\vec{x}_{1}+...+a_{n}\vec{x}_{n}=\vec{0}
\end{equation}
iné riešenie ako $a_{1}=a_{2}=...=a_{n}=0$.


\begin{priklad}
Je $(\vec{e}_{1},\vec{e}_{2})=((1,0),(0,1))$ lineárne závislá v $\R^{2}$? Napíšme si,
čo znamená \eqref{eq:linkombJeNula} v tomto prípade.
    \begin{align*}
    a_{1}(1,0)+a_{2}(0,1) &= (0,0) \\
    (a_{1},0)+(0,a_{2}) &= (0,0) \\
    (a_{1},a_{2}) &= (0,0)
    \end{align*}
    Teda $a_1=a_2=0$ a $((1,0), (0,1))$ je lineárne nezávislá.
    Veľmi podobne sa môžeme presvedčiť, že $(\vec{e}_{1},\vec{e}_{2},...,\vec{e}_{n})$ je lineárne nezávislá v $\R^{n}$.
\end{priklad}
\begin{priklad}
    Je $((-1,3), (2,-6))$ lineárne závislá v $\R^2$?

    \begin{align*}
    a_{1}(-1,3)+a_{2}(2,-6) &= (0,0) \\
    (-a_{1},3a_{1})+(2a_{2},-6a_{2}) &= (0,0) \\
    (-a_{1}+2a_{2}, 3a_{1}-6a_{2}) &= (0,0)
    \end{align*}
    Posledná rovnosť presne zodpovedá sústave lineárnych rovníc
    \begin{align*}
    -a_{1}+2a_{2} &= 0 \\
    3a_{1}-6a_{2} &= 0
    \end{align*}
    Matica sústavy:
    $$ 
    \left(\begin{array}{rr|r}
    -1&2&0\\ 3&-6&0
    \end{array}\right)
    $$
    Riešme sústavu (Gaussovou elimináciou):
    $$ 
    \left(\begin{array}{rr|r}
    -1&2&0\\ 3&-6&0
    \end{array}\right)
    \sim 
    \left(\begin{array}{rr|r}
    -1&2&0\\
    0&0&0
    \end{array}\right)
    $$
    Vieme, že takáto sústava má nekonečne veľa riešení, teda iste aj aspoň jedno iné, než $a_{1}=a_{2}=0$.
    Z toho už vieme, že tá dvojica vektorov je lineárne závislá. Tým by sme mohli
    skončiť, takéto riešenie je v poriadku, ale (čisto pre ilustráciu pojmov): jedno z nenulových riešení je napríklad $a_1=2$, $a_2=1$:
    $$ 2.(-1,3)+1.(2,-6)=(0,0) $$
\end{priklad}

Teraz (na záver) pojem, ku ktorému smerujeme.

\begin{definicia}\label{def:baza}
Nech $V$ je vektorový priestor. \emph{Báza} $V$ je taká $n$-tica $X$ vektorov z $V$, že platí:
\begin{itemize}
    \item $X$ je lineárne nezávislá a zároveň
    \item $\Lo(X)=V$
\end{itemize}
\end{definicia}

Z horeuvedeného dostávame aspoň jeden príklad bázy vektorového priestoru: $(\vec{e}_{1},...,\vec{e}_{n})$ je bázou $\R^n$.

