\documentclass[varwidth=true]{standalone}
\usepackage{amsmath}
\usepackage{tikz}
% Load tikzmark and calc libraries
\usetikzlibrary{tikzmark, calc}

\begin{document}

\[
% --- First Matrix (before operation) ---
\left(
\begin{array}{rrr|r}
1 & -2 & 3 & 0 \\
% Mark row 2 (source)
0 & -1 & 5 & \tikzmarknode{row2}{3} \\
% Mark row 3 (target)
0 & 2 & 1 & \tikzmarknode{row3}{-8}
\end{array}
\right)
% --- TikZ Arrow for the operation ---
\begin{tikzpicture}[remember picture, overlay]
    % Draw an arrow from row 2 to row 3 with a '2'
    \draw[->, thick, shorten >=2pt]
        % Start a bit right of row 2
        ([xshift=1.5em]row2.east)
        % Draw a line to the right, and mark the corner
        -- ++(1em,0) coordinate (corner)
        % Draw a vertical line down to the level of row 3
        -- (corner |- row3.east)
        % Place the '2' to the right of the vertical line
        node [midway, right] {$2$}
        % Draw the final segment pointing to row 3
        -- ([xshift=1.5em]row3.east);
\end{tikzpicture}
% --- Equivalence Symbol with adjusted spacing ---
\hspace{3em} \sim \quad
% --- Second Matrix (after operation) ---
\left(
\begin{array}{rrr|r}
1 & -2 & 3 & 0 \\
0 & -1 & 5 & 3 \\
0 & 0 & 11 & -2
\end{array}
\right)
\]

\end{document}
