\documentclass[varwidth=true]{standalone}
\usepackage{amsmath}
\usepackage{tikz}
% Load tikzmark and calc libraries
\usetikzlibrary{tikzmark, calc}

\begin{document}

\[
% --- First Matrix (before operation) ---
\left(
\begin{array}{rrr|r}
% Mark row 1 (source)
1 & -2 & 3 & \tikzmarknode{row1}{0} \\
0 & 2 & 1 & -8 \\
% Mark row 3 (target)
-1 & 1 & 2 & \tikzmarknode{row3}{3}
\end{array}
\right)
% --- TikZ Arrow for the operation ---
\begin{tikzpicture}[remember picture, overlay]
    % Draw an arrow from row 1 to row 3 with a '1'
    \draw[->, thick, shorten >=2pt]
        % Start a bit right of row 1
        ([xshift=1.5em]row1.east)
        % Draw a line to the right, and mark the corner
        -- ++(1em,0) coordinate (corner)
        % Draw a vertical line down to the level of row 3
        -- (corner |- row3.east)
        % Place the '1' to the right of the vertical line
        node [midway, right] {$1$}
        % Draw the final segment pointing to row 3
        -- ([xshift=1.5em]row3.east);
\end{tikzpicture}
% --- Equivalence Symbol with adjusted spacing ---
\hspace{3em} \sim \quad
% --- Second Matrix (after operation) ---
\left(
\begin{array}{rrr|r}
1 & -2 & 3 & 0 \\
0 & 2 & 1 & -8 \\
0 & -1 & 5 & 3
\end{array}
\right)
\]

\end{document}
