\documentclass{article}

% Preamble
\usepackage[T1]{fontenc}
\usepackage[utf8]{inputenc}
\usepackage[slovak]{babel}
\usepackage{amsmath}
\usepackage{amssymb}
\usepackage{amsthm}
\usepackage{graphicx}

% Custom environments to match the document structure
\theoremstyle{definition}
\newtheorem{definition}{Definícia}
\newtheorem{example}{Príklad}

\theoremstyle{plain}
\newtheorem{theorem}[definition]{Tvrdenie}

% Style for remarks and questions (upright text, italic title)
\theoremstyle{remark} 
\newtheorem*{question}{Otázka}

% Set the proof symbol to a square, as in the document
\renewcommand{\qedsymbol}{$\Box$}

\begin{document}

\section*{Zobrazenia (Funkcie)}

\begin{definition}\label{def:zobrazenie}
Nech $A$, $B$ sú množiny. Zobrazenie $f$ z $A$ do $B$ je predpis, ktorý každému prvku $A$ priradí nejaký prvok $B$.
\end{definition}

Zapisujeme $f \colon A \rightarrow B$.

To znamená, že ak chceme špecifikovať nejaké zobrazenie $f$, musíme špecifikovať tri veci:
\begin{enumerate}
    \item Z ktorej množiny sa zobrazuje (definičný obor).
    \item Do ktorej množiny sa zobrazuje (koobor).
    \item Predpis, ktorý nám určí, pre každý prvok definičného oboru ktorý prvok sa mu má zobraziť.
\end{enumerate}

\subsection*{Terminológia}
Zobrazenia sa často nazývajú funkcie. Obe slová znamenajú to isté, obvykle však funkcia zobrazuje do čísel ($\mathbb{N}, \mathbb{R}, \mathbb{C}, \dots$). Ktoré slovo sa použije je otázkou konvencie v danej časti matematiky.

\begin{example}
$g \colon \mathbb{N} \rightarrow \mathbb{N}$ dané predpisom $g(n) = n^2 + 1$.
Toto nám hovorí, že $g$ zobrazuje z množiny všetkých prirodzených čísel do množiny všetkých prirodzených čísel.

Predpis je teda v tomto prípade daný vzorcom, ktorý nám umožňuje počítať hodnoty zobrazenia pre konkrétne prvky definičného oboru $g$ (t.j. prirodzené čísla) dosadením a výpočtom.
\begin{align*}
    g(2) &= 2^2 + 1 = 5 \\
    g(7) &= 7^2 + 1 = 50
\end{align*}
$g(-1) = ?$ Toto neexistuje, pretože $-1 \notin \mathbb{N}$ a nie je to teda prvok definičného oboru $g$.
\end{example}

\begin{example}
Nech $A = \{\text{Jožko, Miško}\}$ a $B = \{\text{bageta, guláš, jablko}\}$.
Nech $j \colon A \rightarrow B$ je zobrazenie "najobľúbenejšie jedlo".
V tomto prípade je definičný obor a koobor konečná množina. Zobrazenie môžeme nakresliť:
\begin{center}
\includegraphics{mnoziny_zobrazenia2_fig.pdf}
\end{center}
\end{example}

Iný spôsob špecifikácie predpisu zobrazenia je napríklad tabuľkou:
\begin{center}
\begin{tabular}{c|ccc}
    $x$ & Jožko & Miško & Anka \\
    \hline
    $j(x)$ & bageta & guláš & guláš
\end{tabular}
\end{center}
Tu sa, samozrejme, nesmú prvky v hornom riadku opakovať.

\begin{example}
Nech $H$ je množina všetkých ľudí (aj z minulosti).
$\sigma \colon H \rightarrow H$ je zobrazenie dané predpisom $\sigma(x) = \text{otec človeka } x$.
\end{example}

\begin{example}
$S$ - množina všetkých občanov SR.
$\eta \colon S \rightarrow \mathbb{N}$ dané predpisom $\eta(x) = \text{rodné číslo}$.
\end{example}

\begin{example}
$B = \{\text{bageta, guláš, jablko}\}$. Nech $k \colon B \rightarrow \mathbb{R}$ je zobrazenie "koľko kalórií".
Keďže $B$ je konečná, stačí nám napísať:
$k(\text{guláš}) = 677$, $k(\text{bageta}) = 1148$, $k(\text{jablko}) = 301.4$.
\end{example}

\subsection*{Formalistická odbočka}
Voči Definícii \ref{def:zobrazenie} by bolo možné vzniesť (z istého hľadiska oprávnene) námietku o nepresnosti; používa nejasné slová ako
"priradí", "predpis". Námietku je možné vyriešiť takto:

\begin{definition}[(formálna definícia zobrazenia)]\label{def:zobrazenieFormal}
Nech $A$, $B$ sú množiny. Zobrazenie $f$ z $A$ do $B$ je množina $F \subseteq A \times B$ taká, že pre každé $a \in A$ existuje práve jedno $b \in B$ také, že $(a,b) \in F$.
\end{definition}

$(a,b) \in f$ v zmysle Definície \ref{def:zobrazenieFormal} potom znamená $f(a)=b$ v zmysle Definície
\ref{def:zobrazenie}.

Aj keď je Definícia \ref{def:zobrazenieFormal} presnejšia, v skutočnosti ju bežne nikto nepoužíva, ani nikto bežne
nerozmýšľa o zobrazeniach ako o množinách usporiadaných dvojíc. Niekedy však takéto presné uvažovanie nutne potrebujeme, a preto je dobré vedieť o existencii tohto pohľadu na pojem zobrazenia.
\subsection*{Koniec formalistickej odbočky}

\begin{definition}[Obor hodnôt]\label{def:oborHodnot}
Nech $f \colon A \rightarrow B$ je zobrazenie. Obor hodnôt je množina
$$ \mathcal{H}(f) = \{f(a) | a \in A\} $$
\end{definition}
Čiže máme $b \in \mathcal{H}(f)$ práve vtedy, keď existuje $a \in A$ také, že $f(a)=b$.

Je dôležité si uvedomiť rozdiel medzi oborom hodnôt a kooborom. Ak napíšeme napríklad
$f \colon \mathbb{R} \rightarrow \mathbb{R}$ dané predpisom $f(x) = x^2 - x + 1$.
Koobor je $\mathbb{R}$, ale $\mathcal{H}(f) = \langle\frac{3}{4}, \infty\rangle$.
Určiť obor hodnôt zobrazenia môže byť ťažké; na druhej strane keď sa pred našim duševným zrakom zjaví nejaké zobrazenie, vždy je vybavené kooborom. Trochu mätúce môže byť, že koobor sa často neuvádza explicitne a funkcia sa stotožňuje s predpisom.

\section*{Identické zobrazenie}
\begin{definition}[1.8]
Nech $A$ je množina. Identické zobrazenie (na $A$) je zobrazenie $id_A \colon A \rightarrow A$ dané predpisom $id_A(a) = a$, pre každý prvok $a \in A$.
\end{definition}

\section*{Rovnosť dvoch zobrazení}
\begin{definition}[1.9]
Nech $A, B, C, D$ sú množiny, nech $f \colon A \rightarrow B$ a $g \colon C \rightarrow D$. Hovoríme, že $f$ je rovné $g$, ak $A=C$, $B=D$ a pre všetky $x \in A=C$ platí, že $f(x)=g(x)$.
\end{definition}

\begin{example}
\begin{enumerate}
    \item[A)] $f \colon \mathbb{Z} \rightarrow \mathbb{N}$ dané predpisom $f(k) = \sqrt{k^2}$ \\
    $g \colon \mathbb{Z} \rightarrow \mathbb{N}$ dané predpisom $g(k) = |k|$ \\
    Platí $f=g$.

    \item[B)] $f \colon \mathbb{N} \rightarrow \mathbb{N}$ dané predpisom $f(k) = \sqrt{k^2}$ \\
    $g \colon \mathbb{N} \rightarrow \mathbb{N}$ dané predpisom $g(k) = |k|$ \\
    Platí $f=g$.

    \item[C)] $f \colon \mathbb{Z} \rightarrow \mathbb{Z}$ dané $f(x) = |x|$ \\
    $g \colon \mathbb{Z} \rightarrow \mathbb{Z}$ dané $g(x) = x$ \\
    Platí $f \neq g$ (pretože pre $x=-1$ je $f(-1)=1$, ale $g(-1)=-1$).

    \item[D)] $f \colon \mathbb{N} \rightarrow \mathbb{N}$ dané $f(x) = x+1$ \\
    $g \colon \mathbb{N} \rightarrow \mathbb{Z}$ dané $g(x) = x+1$ \\
    Platí $f \neq g$ (pretože majú rôzne koobory).
\end{enumerate}
\end{example}

\section*{Skladanie zobrazení}
\begin{definition}
Nech $A, B, C$ sú množiny. Nech $f \colon A \rightarrow B$, $g \colon B \rightarrow C$. Potom zložené zobrazenie $g \circ f$ je zobrazenie $g \circ f \colon A \rightarrow C$ dané predpisom
$$ (g \circ f)(x) = g(f(x)) $$
(Je dôležité, aby platilo, že koobor zobrazenia $f$ je rovný definičnému oboru zobrazenia $g$).
\end{definition}

\begin{example}
Zobrazenie "koľko kalórií má najobľúbenejšie jedlo":
Nech $j \colon A \rightarrow B$ je zobrazenie "najobľúbenejšie jedlo" a $k \colon B \rightarrow \mathbb{R}$ je zobrazenie "koľko kalórií".
Potom $k \circ j \colon A \rightarrow \mathbb{R}$ je zobrazenie "koľko kalórií má najobľúbenejšie jedlo".
Napríklad: $(k \circ j)(\text{Miško}) = k(j(\text{Miško})) = k(\text{guláš}) = 677$.
\end{example}

\begin{example}
Nech $g \colon \mathbb{N} \rightarrow \mathbb{N}$ je dané $g(x) = x^2+1$ a $h \colon \mathbb{N} \rightarrow \mathbb{N}$ je dané $h(x) = 2x$.
\begin{itemize}
    \item $g \circ h \colon \mathbb{N} \rightarrow \mathbb{N}$ \\
    $(g \circ h)(x) = g(h(x)) = g(2x) = (2x)^2+1 = 4x^2+1$
    \item $h \circ g \colon \mathbb{N} \rightarrow \mathbb{N}$ \\
    $(h \circ g)(x) = h(g(x)) = h(x^2+1) = 2(x^2+1) = 2x^2+2$
\end{itemize}
Vidíme, že $g \circ h \neq h \circ g$, lebo napríklad $(g \circ h)(1) = 5$, ale $(h \circ g)(1) = 4$.
\end{example}

\subsection*{Cvičenie}
Čo je zobrazenie $\sigma \circ \sigma \colon H \rightarrow H$, ak $\sigma(x)$ je otec človeka $x$?
Odpoveď: $\sigma(\sigma(x))$ je otcov otec, t.j. starý otec z otcovej strany.

Identické zobrazenia sa vo vzťahu na skladanie správajú špeciálne.

\begin{tvrdenie}[1.10]
Nech $A, B$ sú množiny, nech $f \colon A \rightarrow B$ je zobrazenie. Potom platí $f \circ id_A = f$ a $id_B \circ f = f$.
\end{tvrdenie}

\begin{proof}
Máme dokázať, že dve zobrazenia sa rovnajú. Čo je rovnosť dvoch zobrazení, o tom hovorí Definícia 1.9.

Pre $f \circ id_A = f$:
Zobrazenie $id_A$ je typu $A \rightarrow A$, zobrazenie $f$ je typu $A \rightarrow B$.
Teda $f \circ id_A$ existuje a je typu $A \rightarrow B$. Majú rovnaký definičný obor aj koobor.
Pre všetky $x \in A$ platí:
$$ (f \circ id_A)(x) = f(id_A(x)) = f(x) $$
Teda $f \circ id_A = f$ v zmysle Definície 1.9.
Dôkaz rovnosti $id_B \circ f = f$ prenechávame čitateľovi ako cvičenie.
\end{proof}

\begin{tvrdenie}[1.11]
Nech $A, B, C, D$ sú množiny, nech $f \colon A \rightarrow B$, $g \colon B \rightarrow C$, $h \colon C \rightarrow D$ sú zobrazenia. Potom $h \circ (g \circ f) = (h \circ g) \circ f$. (Asociativita skladania).
\end{tvrdenie}

\begin{proof}
Obe zobrazenia, $h \circ (g \circ f)$ aj $(h \circ g) \circ f$, majú definičný obor $A$ a koobor $D$.
Pre všetky $x \in A$:
\begin{align*}
    (h \circ (g \circ f))(x) &= h((g \circ f)(x)) = h(g(f(x))) \\
    ((h \circ g) \circ f)(x) &= (h \circ g)(f(x)) = h(g(f(x)))
\end{align*}
Keďže obe zobrazenia majú rovnaký definičný obor, koobor a vo všetkých bodoch nadobúdajú rovnakú hodnotu, rovnajú sa.
\end{proof}

\end{document}
