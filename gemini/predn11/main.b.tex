\documentclass{elsarticle}
\usepackage{hyperref}
\usepackage[utf8]{inputenc}
\usepackage[T1]{fontenc}
\usepackage{bbm}
\usepackage{graphicx}
\usepackage{amssymb}
\usepackage{amsmath}
\usepackage{amsthm}
\usepackage{thmtools}
\usepackage{cleveref}
\usepackage{mathtools}
%\usepackage[notref,notcite]{showkeys}
\usepackage{url}
\usepackage{ifpdf}
\ifpdf
\usepackage[all,pdf,2cell]{xy}\UseAllTwocells\SilentMatrices
\else
\usepackage[all,xdvi,2cell]{xy}\UseAllTwocells\SilentMatrices
\fi
\usepackage{todonotes}
\usepackage{bm} %Package for bold math symbols
\usepackage{stackengine}
\usepackage{enumerate}
\usepackage{float}
\newtheorem{theorem}{Theorem}[section]
\newtheorem{lemma}[theorem]{Lemma}
\newtheorem{proposition}[theorem]{Proposition}
\newtheorem{corollary}[theorem]{Corollary}
\theoremstyle{definition}
\newtheorem{definition}[theorem]{Definition}
\newtheorem{example}[theorem]{Example}
\newtheorem{problem}{Problem}

\newcommand*{\vcbox}[1]{\begingroup
\setbox0=\hbox{#1}\parbox{\wd0}{\box0}\endgroup}

\newcommand{\id}{\mathrm{id}}
\newcommand{\newcategory}[1]{\expandafter\newcommand\csname #1\endcsname{\mathbf{#1}}}
\newcategory{Set}
\newcategory{Rel}
\newcategory{Fin}
\newcategory{FHilb}
\newcategory{FinSet}
\newcategory{FinRel}
\newcategory{Pos}
\newcategory{Frob}
\newcategory{DagFrob}
\newcategory{Act}
\newcategory{Sch}
\newcategory{Conf}
\newcategory{Cat}


\newcommand{\MatR}{\mathbf{Mat}_{\mathbb R_0^+}}
\newcommand{\MapR}{\mathbf{Map}_{\mathbb R_0^+}}
\newcommand{\MatS}{\mathbf{Mat}_\Sigma}
\newcommand{\MatRel}{\mathbf{M
at}_2}
\newcommand{\C}{\mathcal C}
\newcommand{\<}{\langle}
\newcommand{\val}[1]{\|{#1}\|}
\newcommand{\valphi}[1]{\|{#1}\|_\phi}
\renewcommand{\>}{\rangle}
\newcommand{\unit}{e}
\newcommand{\counit}{\raisebox{\depth}{\scalebox{1}[-1]{$\,e$}}}
\newcommand{\linv}{L^{\counit}}
\newcommand{\rinv}{R^{\counit}}
\newcommand{\nablaDot}{{\stackinset{c}{.1ex}{c}{.1ex}{$\cdot$}{$\bm{\nabla}$}}}
\newcommand{\DeltaDot}{{\stackinset{c}{.1ex}{c}{.1ex}{$\cdot$}{$\bm{\Delta}$}}}
\newcommand{\based}[1]{\underbracket[0.5pt][1.5pt]{#1}}
\newcommand{\supp}{\mathrm{supp}}
\newcommand{\theO}{\Dnabla\circ\Dnabla^\dagger}
\newcommand{\suggestion}[2][]{\todo[color=green!20 #1]{#2}}

%Notation for dagger nabla:
\newcommand{\Dnabla}{{\nablaDot}} % put dot in the middle
\newcommand{\dnabla}[3]{{\Dnabla}_{#1 #2}^{#3}}  %dagger nabla (with parameters)
\newcommand{\ddnabla}[3]{\sqrt{\frac{\val{{#3}}}
{\val{#1}\val{#2}}}\nabla_{{#1}{#2}}^{#3}} %definition of dagger nabla with parameters
\newcommand{\DDelta}{\DeltaDot}  %symbol for dagger Delta
\newcommand{\dDelta}[3]{{\DeltaDot}_{#1#2}^{#3}}  %dagger delta (with parameters)

\newenvironment{proposal}{\color{teal}}{}
\newenvironment{delete}{\color{purple}}{}
\newenvironment{remark}{\par\noindent{\bf Remark.}}{\par}


% see https://tex.stackexchange.com/questions/671112/how-to-write-a-xymatrix-like-latex-environment

\NewDocumentEnvironment{diagram}{ob}{
  \IfNoValueTF{#1}{\xymatrix{#2}}{\xymatrix@#1{#2}}%
}{}


\makeatletter
\newcommand*\savelabel[2]{%
   \immediate\write\@auxout{%
     \noexpand\global\noexpand\@namedef{mylabel@#1}{#2}}%
   #2%
}
 
% #1: id
\newcommand*\mycellref[1]{%
   \ifcsname mylabel@#1\endcsname
     \@nameuse{mylabel@#1}%
   \else
     ??%
   \fi
}
\makeatother

\newcounter{mycellno}[equation]
\DeclareRobustCommand{\mycell}[1]{\stepcounter{mycellno}\savelabel{#1}{(\theequation.\themycellno)}}
\newcommand{\putmycell}[2]{\ar@{}[#1]|{\mycell{#2}}}
\newcommand{\lscellrd}{\ar@{}[rd]|{\lesssim}}
\newcommand{\lscellr}{\ar@{}[r]|{\lesssim}}
\newcommand{\gscellrd}{\ar@{}[rd]|{\gtrsim}}
\newcommand{\gscellrdd}{\ar@{}[rdd]|{\gtrsim}}
\newcommand{\lscellrdd}{\ar@{}[rdd]|{\lesssim}}
\newcommand{\simeqcellrd}{\ar@{}[rd]|{\simeq}}
\newcommand{\simeqcellr}{\ar@{}[r]|{\simeq}}
\newcommand{\simeqcellrrd}{\ar@{}[rrd]|{\simeq}}
\newcommand{\simeqcellrdd}{\ar@{}[rdd]|{\simeq}}


\begin{document}
\title{Coherent configurations and Frobenius structures}
\author[1]{Gejza Jenča}
\ead{gejza.jenca@stuba.sk}
\affiliation{
\organization={Department of Mathematics and Descriptive Geometry,
Faculty of Civil Engineering,
Slovak University of Technology,},
addressline={Radlinského~11},
postcode={810 05},
postcodesep={},
city={Bratislava},
country={Slovakia}}
\fntext[1]{Supported by grants VEGA 2/0128/24 and 1/0036/23.}
\author[2]{Anna Jenčová}
\ead{jencova@mat.savba.sk}
\affiliation{
organization={Department of Mathematics and Descriptive Geometry,
Faculty of Civil Engineering,
Slovak University of TechnologyMathematical Institute, Slovak Academy of Sciences,},
city={Bratislava},
country={Slovakia}
}
\fntext[2]{Supported by the grant VEGA 2/0128/24.}
\author[3]{Dominik Lachman}
\ead{dominik.lachman@upol.cz}
\affiliation{
organization={Department of Algebra and Geometry,
Palacký University Olomouc,},
addressline={17.~Listopadu~12},
city={Olomouc},
country={Czech Republic}
}
\fntext[3]{The author received support from the Czech Science Foundation grant 25-20013L.}
\begin{abstract}
We prove that coherent configurations can be represented as modules over Frobenius structures in the category
of real nonnegative matrices. We generalize the notion of admissible morphism from association
schemes to coherent configurations. We show that the Frobenius structure associated to a coherent configuration
can be modified to become a dagger Frobenius structure, and use this to connect the coherent configurations to
groupoids and $H^*$-algebras. We examine the properties of the dagger Frobenius structure
with respect to admissible morphisms. We introduce the matrix $O$ obtained as the
composition of comultiplication and multiplication of the dagger Frobenius structure and
prove that we may obtain the valencies of colors, and thus recover the original coherent
configuration, as an eigenvector of $O$. In the last part of the paper, we examine the
spectrum of $O$  and apply it to generalize the Lagrange theorem from groups to association schemes. 

\end{abstract}
\maketitle
%\section{Meta}
%
%\begin{proposal}
%This is a proposed text.
%\end{proposal}
%
%Something.\todo{This is a TODO note}
%
%\begin{delete}
%Delete this, or not?
%\end{delete}

\section{Introduction}

The notion of an association scheme arose in the theory of experimental designs in
statistics in the first half of the twentieth century
\cite{bose1939partially,bose1952classification}. Later on, association schemes were
used in coding theory, graph theory, combinatorics, and elsewhere.  We refer to
\cite{bailey2004association,zieschang2005theory} for more information on association  
schemes and related topics. Every group gives rise to an association scheme, so 
the theory of association schemes can be considered as an extension of group theory.

Coherent configurations were introduced by Higman in \cite{higman1975coherent} as a
generalization of association schemes. They are a straightforward and natural
generalization of 2-orbits of a group acting on a finite set. Many
parts of the theory of association schemes can be extended  to the theory of coherent
configurations.

The notion of a morphism of association schemes was given by Hanaki
\cite{hanaki2010acategory}. Building on Hanaki's definition, French proposed in
\cite{french2013functors} a more restrictive definition of \emph{admissible
morphisms} and proved that this notion can be used to generalize some important
notions and theorems from groups to association schemes, for example the notion of a
normal subgroup and the first isomorphism theorem.

The main aim of this paper is to show that coherent configurations (and hence also
association schemes) can be represented as modules over Frobenius structures in the
category of non-negative real matrices $\MatR$ and examine several aspects of this
representation. The category $\MatR$ is a dagger compact category, just as the
category of finite dimensional Hilbert spaces $\FHilb$.
This is the main reason why our perspective and presentation is
strongly influenced by \emph{categorical quantum mechanics}, initiated by Abramsky 
and Coecke in their seminal paper \cite{abramsky2009categorical} and recently summarized in the books
\cite{heunen2019categories,coecke2017picturing}, in which the Frobenius structures in
dagger compact categories play an important role. Whenever appropriate, we try to use
the standard language of categorical quantum mechanics -- string diagrams.

The paper is structured as follows. In the preliminaries, we introduce the main definitions and basic results concerning
coherent configurations, Frobenius structures, and dagger compact categories. We also prove some basic results about
monoids in $\MatR$.  In Section 3, we prove that coherent configurations can be represented as modules over Frobenius
structures in $\MatR$. The most important result here is \Cref{thm:connections}, which
shows how some of the classes of coherent configurations studied in the literature 
connect to existing special classes in the theory of Frobenius structures. The following two sections are devoted to an attempt to generalize the notion of an admissible morphism from
association schemes to coherent configurations. It turns out that there are at least two possible meaningful
generalizations, \Cref{def:admorphism} and \Cref{def:stradmorphism}.  In Section 6, we prove that the Frobenius
structure we associated in Section 3 to a coherent configuration can always be ``normalized'' so that it becomes a
\emph{dagger Frobenius structure}. This allows us, in a straightforward way, to connect coherent configurations to
$H^*$-algebras in \Cref{coro:hstar} and present the connection of thin coherent configurations to groupoids (already
discovered by Hanaki in \cite{hanaki2015thin}) in an appropriate category-theoretical context in
\Cref{coro:thinCCIsGroupoid}. In section 7, we return to the notion of (strongly) admissible morphisms and examine how
they behave with respect to the dagger Frobenius structures associated with coherent configurations. It is clear that the
normalization procedure from section 6 can be reversed if we know the valencies of colors. In Section 8, we prove that
the information about valencies (and thus the structure constants of the coherent configuration) can be retrieved from
the dagger Frobenius structure in the form of a positive eigenvector of a certain symmetric matrix, which we denote by
$O$. In Section 9, we examine the spectral decomposition of $O$ and compute several examples.  In \Cref{thm:genLagrange}
we then show how the spectrum of the matrix $O$ can be used to give a generalization of the (normal subgroup case of)
Lagrange's theorem from finite groups to association schemes. In the last section, we sketch some possible directions of
future research, and we close the paper with several open problems.

\section{Preliminaries}

We assume some knowledge of the basics of category theory. For notions undefined here, see the standard monograph
\cite{mac1998categories}. 

\subsection{Dagger compact categories and string diagrams}

Let us review elements of the graphical language of string diagrams
that we will use throughout the paper. We mostly follow the conventions
established in the book \cite{heunen2019categories}. An object $A$ is denoted by
an arrow (called string), a morphism is denoted by a labeled box, circle or a triangle.
A morphism $f\colon A\to B$ is drawn like this:
\begin{center}
\includegraphics{strings_intro_f_A_to_B.pdf}
\end{center}
If the morphisms and objects are clear from the context, we omit their labels in the
diagram. Composition of morphisms $g\circ f$ of morphisms $f\colon A\to B$ and
$g\colon B\to C$ is drawn by connecting the codomain of $f$ to the
domain of $g$:
\begin{center}
\includegraphics{strings_intro_g_circ_f.pdf}
\end{center}
Moving to monoidal categories, the monoidal product $f\otimes g\colon A\otimes C\to B\otimes D$ of morphisms $f\colon A\to B$ and $g\colon C\to D$
is drawn like this:
\begin{center}
\includegraphics{strings_intro_f_otimes_g.pdf}
\end{center}
The monoidal unit object $I$ is drawn as the empty diagram. For an object $A$, a
morphism $s\colon I\to A$ is called \emph{a state}, or \emph{a vector} and drawn like this:
\begin{center}
\includegraphics{strings_intro_state.pdf}
\end{center}

An endomorphism of the unit object $t\colon I\to I$ is called \emph{a scalar}. The set of all
scalars forms a commutative monoid with respect to the composition, see \cite[Lemma 2.3]{heunen2019categories}.
There is a well-defined operation of multiplication of a morphism by a scalar:
for $f\colon A\to B$, $t\cdot f\colon A\to B$ is given by the composite
\[
\begin{diagram}
A
    \ar[r]^-{\lambda_A^{-1}}
&
I\otimes A
    \ar[r]^{t\otimes f}
&
I\otimes B
    \ar[r]^{\lambda_B}
&
B
\end{diagram}
\]
For a scalar $f$, this $t\cdot f$ coincides with $t\circ f$.
In string diagrams, we draw $t\cdot f$ like this:
\begin{center}
\includegraphics{strings_intro_t_times_f_A_to_B.pdf}
\end{center}
A {\em right dual object} to an object $L$ of a symmetric monoidal category $(\C,\otimes,I)$ is an object
$R$ such that there are morphisms
$\eta\colon I\to R\otimes L$ and $\varepsilon\colon L\otimes R\to I$ such
that the equations
\begin{equation}
\label{eq:snakes}
\begin{aligned}
\lambda_L\circ(\varepsilon\otimes\id_L)\circ(\id_L\otimes\eta)\circ\rho_L^{-1}&=\id_L\\
\rho_{R}\circ(\id_{R}\otimes\varepsilon)\circ(\eta\otimes\id_{R})\circ\lambda_{R}^{-1}&=\id_{R}
\end{aligned}
\end{equation}
are satisfied. If $R$ is a right dual to $L$, the $L$ is a \emph{left dual to $R$}.

Denoting the $\eta$ and $\varepsilon$ by

\begin{center}
~\par
\vcbox{\includegraphics{snake_eta.pdf}}
\vcbox{ and }
\vcbox{\includegraphics{snake_varepsilon.pdf}}
\vcbox{ , }
\end{center}
respectively, the string diagram form of the equations \eqref{eq:snakes} then is
\begin{center}
~\par
\vcbox{\includegraphics{snake_snake_left.pdf}}
\vcbox{$=$}
\vcbox{\includegraphics{snake_id_L.pdf}}
\vcbox{\hspace*{3em}}
\vcbox{\includegraphics{snake_snake_right.pdf}}
\vcbox{$=$}
\vcbox{\includegraphics{snake_id_R.pdf}}
\end{center}
This is why we refer to \eqref{eq:snakes} as the \emph{snake identities}.
The morphisms $\eta$ and $\varepsilon$ are called {\em coevaluation} (or \emph{cup}) and
{\em evaluation} (or \emph{cap}), respectively. We say that a quadruple
$(L,R,\eta,\varepsilon)$ satisfying \eqref{eq:snakes} is \emph{a duality situation}.

Let us note that in every duality situation, $\eta$ determines $\epsilon$ and vice versa.

A symmetric monoidal category $\C$ is {\em compact closed} 
if each of its objects has a chosen duality situation $(A,A^*,\eta_A,\varepsilon_A)$. 
The object-level mapping $A\mapsto A^*$ can be made to a functor $\C\to\C^{op}$: for
a morphism $f\colon A\to B$ we define $f^*\colon B^*\to A^*$ as
\[
f^*:=(\id_{A^*}\otimes\varepsilon_B)\circ(\id_{A^*}\otimes f\otimes\id_{B^*})\circ
    (\eta_A\otimes\id_{B^*}).
\]
For every triple of objects $A,B,C$ in a compact closed category $\C$, we have
$\C(A\otimes B,C)\simeq\C(A,B^*\otimes C)$. Therefore, every compact closed category
is a closed monoidal category.

A {\em dagger category} is a category $\C$ 
equipped with a functor $\dag\colon\C\to\C^{op}$ that is the identity
on objects and satisfies $f^{\dag\dag}=f$ for every morphism $f$ of $\C$. In
fact, the $\dag$ functor can be characterized as a mapping on the class of morphisms
of $\C$ that has the following properties:
\begin{itemize}
\item $(\id_H)^\dag=\id_H$
\item $(f\circ g )^\dag=g^\dag\circ f^\dag$
\item $f^{\dag\dag}=f$
\end{itemize}

We say that a morphism $f$ in a dagger category is \emph{self-adjoint} if and only if
$f^\dagger=f$.

\begin{definition}
\label{def:dagsymmonoidal}
Let $(\C,\otimes,I)$ be a symmetric monoidal category that is at the same time
a dagger category. We say that $\C$ is a {\em dagger symmetric monoidal category}
if the following equalities are satisfied, for all objects $A,B,C,D$ and
morphisms $f\colon A\to B$ and $g\colon C\to D$.
\begin{itemize}
\item $(f\otimes g)^\dag=f^\dag\otimes g^\dag$
\item $\alpha_{A,B,C}^{-1}=\alpha_{A,B,C}^\dag$
\item $\lambda_A^{-1}=\lambda_A^\dag$
\item $\rho_A^{-1}=\rho_A^\dag$
\item $\sigma_{A,B}^{-1}=\sigma_{A,B}^\dag$
\end{itemize}
\end{definition}

\begin{definition}
A dagger symmetric monoidal category $(\C,\otimes,I)$ is a {\em dagger compact category} if
 $\C$ is a compact closed category and the following diagram commutes
\begin{equation}
\label{diag:dagcompact}
\xymatrix{
I
	\ar[r]^-{\epsilon_A^{\dag}}
	\ar[rd]_{\eta_A}
&
A\otimes A^*
	\ar[d]^{\sigma_{A,A^*}}
\\
~
&
A^*\otimes A
}
\end{equation}

\end{definition}

\begin{example}\cite[Example 3.58]{heunen2019categories}
\begin{itemize}
\item The category of finite dimensional Hilbert spaces $\FHilb$ is dagger compact.
The monoidal structure is the tensor product.
For an object $X$, $X^*$ is the dual space, $\epsilon_X\colon X\otimes X^*\to\mathbb
C$ is the evaluation of a linear functional and this already determines the $\eta_X$.
For every linear mapping $f\colon X\to Y$, 
the mapping $f^*$ is the usual $Y^*\to X^*$ given by the precomposition of a linear functionals in $Y^*$ by $f$, and
$f^\dag\colon Y\to X$ is uniquely determined by the rule
$\langle f(x)|y\rangle=\langle x|f^\dag(y)\rangle$, for all $x\in X$ and $y\in Y$.
\item The category of sets and relations $\Rel$ is dagger compact. The monoidal
structure is the product of sets. For an object $X$, $X^*$ is equal to $X$ and 
the relation $\epsilon_X\colon X^*\otimes X\to 1$ is given by the rule 
\[
\epsilon_X((a,b),*)\iff a=b .
\]
The relation $\eta_X$ is given similarly.
For a relation $f\colon X\to Y$, $f^\dag=f^*$ is just the opposite relation.
\end{itemize}
\end{example}

\subsection{Monoids and comonoids}

Let $(\C,\otimes,I)$ be a monoidal category. A {\em monoid} in $\C$ is
a triple $(S,\nabla,\unit)$, where $S$ is an object of $\C$, $\unit\colon I\to S$ is the
{\em unit} and
$\nabla:S\otimes S\to S$ is the {\em multiplication}, such that the equations
\begin{equation}\label{eq:monoid}
\nabla\circ(e\otimes\id_S)=\lambda_S
\quad
\nabla\circ(\id_S\otimes e)=\rho_S
\quad
\nabla\circ(\nabla\otimes\id_S)=\nabla\circ(\id_S\otimes\nabla)
\end{equation}
are satisfied.
The two left equations in \eqref{eq:monoid} are called
\emph{left and right unit laws}, the right one is the \emph{associative law}.

In string diagrams, the equations can be expressed as

\begin{center}
\vskip 0.5em
\vcbox{\includegraphics{unit_left_unit.pdf}}
\vcbox{ $=$ }
\vcbox{\includegraphics{unit_id_unit.pdf}}
\vcbox{ $=$ }
\vcbox{\includegraphics{unit_right_unit.pdf}}
\hspace*{3em}
\vcbox{\includegraphics{assoc_left_assoc.pdf}}
\vcbox{ $=$ }
\vcbox{\includegraphics{assoc_right_assoc.pdf}}
\vskip 0.5em
\end{center}

A \emph{comonoid} $(S,\Delta,c)$ is defined in a dual way:
\begin{equation}\label{eq:comonoid}
(c\otimes\id_S)\circ\Delta=\lambda_S
\quad
(\id_S\otimes c)\circ\Delta=\rho_S
\quad
(\id_S\otimes\Delta)\circ\Delta=(\Delta\otimes\id_S)\circ\Delta
\end{equation}
\begin{center}
\vskip 0.5em
\vcbox{\includegraphics{counit_left_counit.pdf}}
\vcbox{ $=$ }
\vcbox{\includegraphics{counit_id_counit.pdf}}
\vcbox{ $=$ }
\vcbox{\includegraphics{counit_right_counit.pdf}}
\hspace*{3em}
\vcbox{\includegraphics{coassoc_left_coassoc.pdf}}
\vcbox{ $=$ }
\vcbox{\includegraphics{coassoc_right_coassoc.pdf}}
\vskip 0.5em
\end{center}
here $c\colon S\to I$ is called the {\em counit} and $\Delta\colon S\to S\otimes S$
is called the {\em comultiplication}.


Let $(S,\nabla_S,e_S)$, $(T,\nabla_T,e_T)$ be monoids in a monoidal category. A
\emph{monoid morphism}
is a morphism $f\colon S\to T$ preserving the multiplication and the unit:
\[
\nabla_S\circ(f\otimes f)=f\circ\nabla_T\qquad f\circ e_S=e_T
\]
\begin{center}
\vskip 0.5em
\vcbox{\includegraphics{monoid_morph_left_preserve_nabla.pdf}}
\vcbox{ $=$ }
\vcbox{\includegraphics{monoid_morph_right_preserve_nabla.pdf}}
\hspace*{3em}
\vcbox{\includegraphics{monoid_morph_left_preserve_unit.pdf}}
\vcbox{ $=$ }
\vcbox{\includegraphics{monoid_morph_right_preserve_unit.pdf}}
\vspace*{0.5em}
\end{center}

An invertible monoid morphism is called a \emph{monoid isomorphism}.
An invertible monoid morphism with the same domain and codomain is called a
\emph{monoid automorphism}.
A morphism that preserves just the multiplication and not necessarily the unit is
called a \emph{semigroup morphism}.



\subsection{Frobenius structures}

A {\em Frobenius structure} in a symmetric monoidal category $(\C,\otimes,I)$ is an object
$S$ equipped with a monoid structure $(S,\nabla,e)$ and a comonoid structure
$(S,\Delta,c)$ such that the equation
\begin{equation}
(\nabla\otimes\id_S)\circ(\id_S\otimes\Delta)
=\Delta\circ\nabla=(\id_S\otimes\nabla)\circ(\Delta\otimes\id_S)
\end{equation}
is satisfied.
In string diagrams, this is expressed as
\begin{center}
\vcbox{\includegraphics{frob_frob_left.pdf}}
\vcbox{ $=$ }
\vcbox{\includegraphics{frob_frob_mid.pdf}}
\vcbox{ $=$ }
\vcbox{\includegraphics{frob_frob_right.pdf}}
\end{center}

A Frobenius structure in a dagger monoidal category is a {\em dagger Frobenius
structure} if $\nabla=\Delta^\dag$ and $e=c^\dag$. Clearly, every dagger Frobenius
structure is completely determined by its monoid (or comonoid) structure.

Let us call $c\colon S\to I$ on a monoid $S$ a 
\emph{non-degenerate form} if there is a
(necessarily unique) coevaluation $\eta\colon I\to S\otimes S$ such that
$(S,S,\eta,c\circ\nabla)$ is a duality situation:
\begin{center}
\vcbox{\includegraphics{frob_frob_selfdual_left.pdf}}
\vcbox{ $=$ }
\vcbox{\includegraphics{frob_frob_selfdual_id.pdf}}
\vcbox{ $=$ }
\vcbox{\includegraphics{frob_frob_selfdual_right.pdf}}
\end{center}
\begin{theorem}\cite[Proposition 5.16]{heunen2019categories}. \label{thm:nondeg} 
Let $(S,\nabla,e)$ be a monoid in a monoidal category. 
There is a bijective correspondence between non-degenerate forms on $S$ and
comonoids $(S,\Delta,c)$ making the monoid-comonoid pair into a Frobenius structure.
\end{theorem}

The bijective correspondence goes as follows. For a given Frobenius structure, the non-degenerate form is simply the counit of the
comonoid part, the self-duality situation is then
$(S,S,\Delta\circ e,c\circ\nabla)$.
On the other hand, given a non-degenerate form $c$ on a monoid $S$, the related
coevaluation $\eta$ determines a comultiplication
\[
\Delta=(\nabla\otimes\id_S)\circ(\id_S\otimes\eta)\circ\lambda_S
\]
\begin{center}
\includegraphics{frob_frob_comul.pdf}
\end{center}
which gives us a comonoid $(S,\Delta,c)$ such that the Frobenius law for
$\nabla$ and $\Delta$ is satisfied.

It is easy to check that, in a compact closed category, the existence of the
duality situation $(S,S,\eta,c\circ\nabla)$ is equivalent to the fact that the morphism 
\[
(c\otimes\id_{S^*})\circ(\nabla\otimes\id_{S^*})\circ(\id_S\otimes\eta_S)\colon S\otimes I\to I\otimes S^*
\]
\begin{center}
\includegraphics{frob_frob_iso.pdf}
\end{center}
is an isomorphism. Note that here $\eta_S$ denotes the cup of the compact closed
structure.

We will use some additional properties of Frobenius structures 
that are considered in the literature. We say that a Frobenius structure
$S$ is \emph{special} if there is some scalar $t\colon I\to I$ such that
\[
\nabla\circ\Delta=\lambda_S\circ(t\otimes\id_S)\circ\lambda_S^{-1}.
\]
In string diagrams, this means
\begin{center}
\vcbox{\includegraphics{special_eye.pdf}}
\vcbox{ $=$ }
\vcbox{\includegraphics{special_eye_id.pdf}}
\end{center}
Let us remark that in the book \cite{heunen2019categories}, it is required
that $t=1$ for a Frobenius structure to be special.

\begin{example}\cite[Lemma 5.9]{heunen2019categories}\label{ex:pants}
For every object $X$ in a dagger compact category, there is a ``pair of pants'' Frobenius
structure on $X^*\otimes X$, with multiplication, comultiplication, unit and counit
given by the diagrams
\begin{center}
\vcbox{\includegraphics{pants_nabla.pdf}}
\vcbox{ $\quad$ }
\vcbox{\includegraphics{pants_delta.pdf}}
\vcbox{ $\quad$ }
\vcbox{\includegraphics{pants_unit.pdf}}
\vcbox{ $\quad$ }
\vcbox{\includegraphics{pants_counit.pdf}}
\vcbox{ $\quad$ }
\end{center}
\end{example}
\begin{example}\label{ex:groupoids}
\cite{heunen2013relative}
A \emph{groupoid} is a category in which every morphism is invertible.
There is a one-to-one correspondence between small groupoids and special dagger Frobenius
structures in $\Rel$.
For a groupoid $\mathbb G$, the underlying set of the dagger Frobenius structure 
is the set of all morphisms denoted by $\mathbb G_1$, the multiplication is the partial operation
of composition of morphisms, and the unit is the relation $\{*\}\to\mathbb G_1$ that
selects the set of all unit morphisms 
in $\mathbb G_1$.
\end{example}

We say that a Frobenius structure $S$ is \emph{symmetric} if
\[
c\circ\nabla\circ\sigma_{S,S}=c\circ\nabla,
\]
or, in string diagrams,
\begin{center}
\vcbox{\includegraphics{symmetric_symmetric_sigma.pdf}}
\vcbox{ $=$ }
\vcbox{\includegraphics{symmetric_symmetric_nosigma.pdf}}
\end{center}
\begin{definition}
An $H^*$ algebra is an algebra $A$ over $\mathbb C$ equipped with an anti-linear
involution $\dag\colon A\to A$, that is also a Hilbert
space satisfying
\[
\langle ab|c\rangle = \langle b|a^\dag c\rangle=\langle a|cb^\dag \rangle.
\]
\end{definition}
\begin{example}\label{ex:hStarAlgebras}
\cite{vicary2011acategorical} and \cite[Theorem 5.32]{heunen2019categories}
There is a one-to-one correspondence between finite dimensional $H^*$ algebras and
symmetric dagger Frobenius structures in $\FHilb$.
\end{example}

Every Frobenius structure $S$ in a compact closed category admits an automorphism
$\alpha\colon S\to S$ of its monoid reduct that is completely
determined by the property
\begin{equation}\label{eq:nakayama}
c\circ\nabla\circ(\id_S\otimes\alpha)\circ\sigma_{S,S}=c\circ\nabla
\end{equation}
\begin{center}
\vcbox{\includegraphics{symmetric_nakayama.pdf}}
\vcbox{ $=$ }
\vcbox{\includegraphics{symmetric_symmetric_nosigma.pdf}}
\end{center}
We say that $\alpha$ is the \emph{Nakayama automorphism}
\cite{nakayama1941frobeniusean} of $S$. Note that
$S$ is symmetric if and only if $\alpha$ is the identity.

We say that a Frobenius structure $S$ is \emph{extra} if $c\circ e=\id_I$. 


\subsection{The categories $\MatS$}

A \emph{commutative semiring} is a set $\Sigma$ equipped with a commutative monoid $(\Sigma, 0, +)$ and a commutative monoid $(\Sigma, 1, \cdot)$, such that $0$ is the zero element with respect to multiplication and multiplication distributes over addition. Let $\Sigma$ be a commutative semiring with unit, with $0 \neq 1$.
The category $\MatS$ is the
category whose objects are finite sets and whose morphisms from a set $X$ to a set $Y$ are matrices of elements of
$\Sigma$ with columns indexed by the set $X$ and rows indexed by the set $Y$. For $f\colon X\to Y$, we denote the
element of the matrix $f$ corresponding to a pair $x\in X$, $y\in Y$ by $f_x^y$.

The composition of morphisms
is the usual matrix multiplication: for $f\colon A\to B$ and
$g\colon B\to C$, we have
\[
(g\circ f)_a^c=\sum_{b\in B}f_a^b g_b^c,
\]
and the identity morphism $\id_A\colon A\to A$ is the unit
matrix. In computations, we shall write $\delta_a^{a'}$ for the elements
of the $\id_A$, this is simply the Kronecker delta:
\[
\delta_a^{a'}=
\begin{cases}
1&a=a'\\
0&\text{otherwise.}
\end{cases}
\]

There is a faithful functor $M$ from the category of finite sets $\FinSet$
to $\MatS$ that is identity on objects,
and for each morphism $f\colon A\to B$
\begin{equation}
M(f)_a^b=
\begin{cases}
1&b=f(a)\\
0&\text{otherwise}
\end{cases}
\end{equation}
for every $a\in A$, $b\in B$. In this paper, we identify $M(f)$ with $f$
whenever there is no danger of confusion.

For a morphism $f\colon A\to B$, the \emph{support of $f$} is the subset of $A$ given
by
\[
\supp(f)=\{a\in A:\text{$f_a^b\neq 0$ for some $b\in B$}\}.
\]
The \emph{support endomorphism of} $f$ is the matrix $\based f\colon A\to A$ given by
\begin{equation}\label{eq:supportendo}
{(\based f)}_a^{a'}=
\begin{cases}
1 & a=a'\text{ and $a\in \supp(f)$}\\
0 & \text{otherwise.}
\end{cases}
\end{equation}

Note that $\based f$ is an idempotent endomorphism of $A$ and that $f\circ\based f=f$.

The category $\MatS$ can be equipped with a symmetric monoidal structure
$(\MatS,I,\otimes,\alpha,\lambda,\rho,\sigma)$, where 
\begin{itemize}
\item $I$ is a one-element set (we write $I=\{*\}$ so that $*$ is the unique element of
the unit object $I$),
\item for finite sets $X,Y$, $X\otimes Y=X\times Y$,
\item for morphisms $f:A\to B$ and $g:C\to D$, $f\otimes g\colon A\otimes C\to B\otimes D$ is
the tensor (or Kronecker) product of matrices:
\[
(f\otimes g)_{ac}^{bd}=f_a^b\cdot g_c^d
\]
\item The $\alpha$, $\lambda$, $\rho$ and $\sigma$ are transferred from the canonical
cartesian monoidal structure on $\FinSet$ via the $M$ functor.
\end{itemize}

Let $\mathbbm 2=\{0,1\}$ denote the two-element totally ordered set, with
$0<1$.  Clearly, the bounded distributive lattice $(\mathbbm 2,0,1,\vee,\wedge)$ is a semiring and it is easy to
see that the category $\MatRel$ is the same thing as the category of finite
sets and relations $\FinRel$, a full subcategory of the category of sets and
relations $\Rel$.


In a very similar way to $\Rel$, we can equip $\MatS$ with a dagger compact
structure, as follows.  For every object $X$, we put $X^*=X$ and  define
$\eta_X\colon I\to X^*\otimes X$, $\epsilon_X\colon X\otimes X^*\to I$ by
$(\eta_X)_{*}^{ab}=\delta_a^b$ and $(\epsilon_X)_{ab}^*=\delta_a^b$. For every
morphism $f\colon A\to B$, it turns out that $f^*$ is simply the transpose of the
matrix $f$. We then put $f^\dagger=f^*$.

\subsection{The category $\MatR$}\label{sec:2.5}

Our basic category will be $\MatR$, where $\mathbb R_0^+:=[0,\infty)_{\mathbb R}$ is the semiring of
non-negative reals equipped with the usual addition and multiplication.  Morphisms in
$\MatR$ are matrices with nonnegative real entries; these will be called non-negative
matrices. The restriction to non-negative numbers has remarkable effects. For
example, isomorphisms in $\MatR$ can be characterized in a very simple way, as the
following theorem shows.
\begin{theorem}\cite{ding2014matrix}
A non-negative matrix has a non-negative inverse if and only if it is the product
of a diagonal matrix with strictly  positive diagonal entries and a permutation matrix.
\end{theorem}
As a consequence, we have the following.
\begin{corollary}\label{coro:chariso}
In $\MatR$, $f\colon A\to B$ is an isomorphism if and only if
for every $a\in A$ there is a unique $b\in B$ such that $f_a^b>0$ and
for every $b\in B$ there is a unique $a\in A$ such that $f_a^b>0$.
\end{corollary}
Therefore, two objects $X,Y$ of $\MatR$ are isomorphic if and only if
$|X|=|Y|$ and if we want to construct all isomorphisms in the homset
$\MatR(X,Y)$, we may simply take all permutation matrices and then replace the $1$'s in the matrix by 
arbitrary strictly positive real numbers.

On the other hand, there is an obviously defined functor $L\colon\MatR\to\FHilb$. On objects, we put $L(X)=\mathbb C^X$
and for a matrix $f\colon X\to Y$, $L(f)\colon\mathbb C^X\to\mathbb C^Y$ is the linear mapping that
is represented by the matrix $f$ in the standard bases. This functor is strictly monoidal and preserves the dagger
compact structure. Most of the time, we will suppress $L$ from our notations and treat the matrix $f$ and the associated
linear mapping $L(f)$ as the same thing.

Since
$\mathbb R_0^+$ is a partially ordered semiring, $\MatR$ can be enriched over
$\Pos$ by ordering the matrices in the homsets element-wise. But this 2-structure is 
uninteresting: for example, adjunctions are only the isomorphisms, and monads
are only the identity morphisms.

Nevertheless, there is another 2-structure on $\MatR$. Consider the functor
$R\colon\MatR\to\FinRel$ that is identity on objects and on morphisms
we have
\[
R(f)_a^b=1\iff f_a^b>0,
\]
for morphism $f\colon A\to B$ in $\MatR(A,B)$.
We may then preorder the morphisms in $\MatR(A,B)$ by the rule
$f\lesssim g\iff R(f)\leq R(g)$ or, equivalently,
\[
f\lesssim g\iff (\forall a\in A,b\in B: f_a^b>0\implies g_a^b>0).
\]
Note that an isomorphism class with respect to this 2-structure is quite large, for example, all diagonal matrices 
$f\colon A\to A$ that have all of their diagonal entries nonzero are isomorphic to
each other.

By equipping the $\MatR$ with such a 2-structure, we upgrade the functor $R\colon\MatR\to\FinRel$ to a 2-functor.
The internal left adjoints in the 2-structure on $\MatR$ are then those morphisms
that are mapped to left adjoints (i.e. mappings) in $\FinRel$ by the functor $R$.
In detail, $f\colon A\to B$ is left adjoint in $\MatR$ if and only if for every $a\in A$ there
is exactly one $b\in B$ such that $f_a^b>0$. We will call the left adjoints
in $\MatR$ \emph{internal maps} and denote the corresponding subcategory
by $\MapR$. Note that $f$ is an internal map if and only if $R(f)$ is a map.

The functor $R\colon\MatR\to\FinRel$ has very strong properties:
it is identity on objects, full, and strictly monoidal (hence also comonoidal).
Clearly, it also preserves the dagger compact structure. 
Note that \Cref{coro:chariso} can be simply phrased as ``$R$ reflects isomorphisms''.

\subsection{Monoids in $\MatR$}

Let us rewrite the associative law for a monoid $S$ in $\MatR$ in a concise form. For all
$a,b,c,d\in S$, we have
\[
(\nabla\circ(\nabla\otimes\id))_{abc}^d=
\sum_{(p,c')\in S\otimes S}\nabla_{pc'}^d\nabla_{ab}^p\delta_{c}^{c'}=
\sum_{p\in S}\nabla_{pc}^d\nabla_{ab}^p
\]
and
\[
(\nabla\circ(\id\otimes\nabla))_{abc}^d=
\sum_{(a',q)\in S\otimes S}\nabla_{a'q}^d\delta_a^{a'}\nabla_{bc}^{q}=
\sum_{q\in S}\nabla_{aq}^d\nabla_{bc}^q
\]
hence, the associative law means that
\begin{equation}\label{eq:assoc}
\sum_{p\in S}\nabla_{pc}^d\nabla_{ab}^p=
\sum_{q\in S}\nabla_{aq}^d\nabla_{bc}^q,
\end{equation}
for all $(a,b,c)\in S\otimes S\otimes S$ and $d\in S$.

For a monoid $(S,\nabla,\unit)$ in $\MatR$, we denote
$$
E_S=\{p\in S:e_*^p>0\}
$$
and refer to the elements of $E_S$ as \emph{units}. 
The unit laws mean that, for all $a,c\in S$, 
\begin{equation}\label{eq:unitsMatR}
\sum_{p\in E_S}e_*^p\nabla_{pa}^c=\delta_a^c=\sum_{q\in E_S}e_*^q\nabla_{aq}^c.
\end{equation}
The following two lemmas establish basic properties of units in a monoid in $\MatR$.
\begin{lemma}\label{lemma:units}
Let $(S,\nabla,\unit)$ be a monoid in $\MatR$. For every $a\in S$, we have
the following.
\begin{enumerate}
\item[(a)] There exists $p\in E_S$ such that $\nabla_{pa}^c>0$ for some $c\in S$.
\item[(b)]If $p\in E_S$, $\nabla_{pa}^c>0$, then $a=c$.
\item[(a')] There exists $p\in E_S$ such that $\nabla_{ap}^c>0$ for some $c\in S$.
\item[(b')]If $p\in E_S$, $\nabla_{ap}^c>0$, then $a=c$.
\item[(c)]The $p\in E_S$ with $\nabla_{pa}^a>0$ is unique and
\begin{equation}\label{eq:valueoflunit}
\nabla_{pa}^a=\frac{1}{e_*^p}\text{.}
\end{equation}
\item[(c')]The $p\in E_S$ with $\nabla_{ap}^a>0$ is unique and
\begin{equation}\label{eq:valueofrunit}
\nabla_{ap}^a=\frac{1}{e_*^p}\text{.}
\end{equation}
\end{enumerate}
\end{lemma}
\begin{proof}
By \eqref{eq:unitsMatR},
\begin{equation}\label{eq:leftunit}
\sum_{p\in E_S}e_*^p\nabla_{pa}^c=\delta_a^c
\end{equation}
for all $a,c\in S$.
The left-hand side expression here is positive if and only if $\nabla_{pa}^c>0$ for some $p\in E_S$, and 
$\delta_a^c$ is nonzero if and only if $a=c$. This proves simultaneously (a) and (b).
The proof of (a') and (b') is analogous.

To prove (c), suppose that $p,q\in E_S$ are such that $\nabla_{pa}^a>0$,
$\nabla_{qa}^a>0$.
By the associative law, for all $a\in S$,
\begin{equation}\label{eq:assocunit}
\sum_{u\in S}\nabla_{ua}^a\nabla_{pq}^u=\sum_{r\in S}\nabla_{pr}^a\nabla_{qa}^r.
\end{equation}
By (a) and (b), $r=a$ is the unique $r\in S$ for which $\nabla_{qa}^r>0$.
Hence the right-hand side of \eqref{eq:assocunit} equals $\nabla_{pa}^a\nabla_{qa}^a$
and, by our assumptions, is positive. Therefore, the left-hand side of
\eqref{eq:assocunit} is positive as well and, in particular, $\nabla_{pq}^u>0$ for
some $u\in S$. Since
$p\in E_S$, $q=u$ by (b). Since $q\in E_S$, $u=p$ by (b'). Therefore, $p=q$.
By the uniqueness of $p$ and the fact that 
\eqref{eq:leftunit} equals $1$ for $a=c$, we see that $e^p_*\nabla_{pa}^a=1$,
which yields \eqref{eq:valueoflunit}.

The proof of (c') is (up to a symmetry) the same as the proof of (c).
\end{proof}
By \Cref{lemma:units}, for every element $a$ of a monoid $S$ in $\MatR$, there is a unique
$s(a)\in E_S$ such that $\nabla_{s(a)a}^a>0$ and a unique $t(a)\in E_S$ such that
$\nabla_{at(a)}^a>0$. We say that $s(a)$ and $t(a)$ are the \emph{source and target
of $a$}, respectively. This terminology is justified by the following
lemma.

\begin{lemma}\label{lemma:sourcesTargets}
Let $(S,\nabla,e)$ be a monoid in $\MatR$, let $a,b,c\in S$ be such that
$\nabla_{ab}^c>0$. Then $t(a)=s(b)$, $s(a)=s(c)$ and $t(a)=t(c)$.
\end{lemma}
\begin{proof}
By associativity,
\[
\sum_{p\in S}\nabla_{at(a)}^p\nabla_{pb}^c=
\sum_{q\in S}\nabla_{aq}^c\nabla_{t(a)b}^q
\]
As $t(a)\in E_S$, \Cref{lemma:units} implies that $\nabla_{at(a)}^p>0$ if and only if $p=a$. Hence
the first sum equals $\nabla_{at(a)}^a\nabla_{ab}^c>0$. Therefore,
the second sum is positive so that  $\nabla_{t(a)b}^q>0$ for some
$q\in S$. By \Cref{lemma:units}, we see that $q=b$ and that $t(a)=s(b)$. The proofs of
$s(a)=s(c)$ and $t(b)=t(c)$ are very similar and are thus omitted.
\end{proof}

\subsection{Coherent configurations and association schemes}

The following definition seems to introduce a notational collision  with our use of
$\nabla$, $E_S$, $s$, and $t$ for monoids in $\MatR$. However, in the next section, it
will turn out that those notations are actually consistent.

\begin{definition}\cite{higman1975coherent}
\label{def:cc}
Let $X$ be a finite set, let $S$ be a system of subsets of $X\times X$
such that
\begin{enumerate}
\item[(CC1)] $S$ is a partition of $X\times X$.
\item[(CC2)] If $p\in S$ and $(x,x)\in p$, then $p$ is a subset of the identity relation $\id_X$.
\item[(CC3)]For $a,b,c\in S$  
 and $(x,y)\in c$, the number $\nabla_{ab}^c$ of $z\in X$ such that $(x,z)\in a$ and $(z,y)\in b$ does not depend on the choice of $(x,y)\in c$.
\item[(CC4)]If $c\in S$, then the opposite relation $c^{-1}=\{(y,x)\colon (x,y)\in c\}\in S$.
\end{enumerate}
Then $(X,S)$ is called {\em a coherent configuration on $X$}. The elements $p\in S$ with $p\subseteq\id_X$ are called {\em units}. The set of all units is denoted by $E_{S}$. The numbers $\nabla_{ab}^c$ are called \emph{the structural constants} of the
coherent configuration $(X,S)$.
A coherent
configuration is called an {\em association scheme} if $\id_X\in S$ (equivalently, $E_S$
is a singleton, containing only $\id_S$). The cardinality of the set $X$ is the \emph{degree} of $(X,S)$ and the
cardinality of the set $S$ is the \emph{rank} of $(X,S)$. The elements of $X$ are
\emph{vertices}. 
\end{definition}

It is instructive to visualize a coherent configuration $(X,S)$ as a complete digraph
on a set $X$, with edges colored by elements of the set $S$. The elements of
$S$ are therefore called \emph{colors}. The number $\nabla_{ab}^c$ is then the number of $(a,b)$
colored walks from the source vertex $x$ of any $c$-colored edge $(x,y)\in c$ to its target vertex $y$:
\[
\xymatrix{
~
&
\cdot
    \ar[rd]^b
\\
x
    \ar[rr]_c
    \ar[ur]^a
&
~
&
y
}
\]
which we sometimes call the $(a,b)$ triangles over $(x,y)$. If $a$ is a color such that
$(x,y)\in a$, we write $a=\langle x,y\rangle$.  A color $p\subseteq\id_X$ is called a
\emph{unit color}, or a \emph{loop color}. 

For every color $a$ of a coherent configuration $(X,S)$, there is a unique unit color
$q\in E_S$ such that $\nabla_{aa^{-1}}^q>0$. 
This $q$ is called \emph{the source of $a$} and is denoted by $s(a)$. 
The source of $a^{-1}$ is called \emph{the target of $a$} and we denote it by $t(a)$.

The number $\nabla_{aa^{-1}}^{s(a)}$ is called
\emph{the valency of a} and is denoted by $\val a$. Note that for every $x$ such that
$(x,y)\in a$ for some $y\in X$,
$\val a$ is the cardinality of the set
\[
\{y\in X:(x,y)\in a\}.
\]

Let $(X,S)$ be a coherent configuration. We say that $(X,S)$ is \emph{homogeneous}, 
or an \emph{association scheme} or simply a \emph{scheme} if $\id_X\in S$, that means, there is only one unit color.
We say that $(X,S)$ is \emph{thin} (or \emph{regular}) if $\val{a}=1$ for every color $a\in
S$. We say that $(X,S)$ is \emph{$\frac{1}{2}$-homogeneous} if every unit color (as a subset
of $X\times X$) has the same cardinality. Clearly, if $(X,S)$ is homogeneous or thin,
then $(X,S)$ is $\frac{1}{2}$-homogeneous. 

Let us mention that a coherent configuration $(X,S)$ is $\frac{1}{2}$-homogeneous
if and only if, for every $a\in S$, $\val{a}=\val{a^{-1}}$.
Indeed, for every 
color $a\in S$ the cardinality of $a\subseteq X\times X$ can be expressed as
\begin{equation}\label{eq:valencies}
|a|=|s(a)|\cdot\val{a}
\end{equation}
and it is clear from (CC4) that $|a|=|a^{-1}|$. Hence $|s(a)|=|s(a^{-1})|$ implies 
$\val{a}=\val{a^{-1}}$. For the opposite implication, whenever  $p,q\in E_S$, we may pick
some color $a$ with $s(a)=p$ and $t(a)=s(a^{-1})=q$ and then cancel the valencies from the
equation \eqref{eq:valencies}.

\begin{remark}
In this paper, we are slightly diverging from the prevailing notations. The usual
notation for the opposite relation is $c^*$ -- this collides with the standard
notation for duals in a compact category.  The usual notation for valency is $n_a$,
we think that $\val a$ is more readable.
\end{remark}

\begin{example}
Let $X$ be a finite set. Put $S=\{\{(x,y)\}:(x,y)\in X\times X\}$. Then $\mathcal
D_X:=(X,S)$ is always a coherent configuration, called \emph{discrete coherent configuration on $X$}.
\end{example}
\begin{example}
The unique coherent configuration on a singleton $X=\{*\}$ is called the \emph{terminal
coherent configuration} and denoted simply by $1$.
\end{example}

Let $(\Gamma,\cdot,e)$ be a group acting on a finite set $X$ from the right, where
$\odot\colon X\times\Gamma\to X$ denotes the action.
We may extend the action to $X\times X$ in an obvious way:
\[
(x_1,x_2)\odot a:=(x_1\odot a,x_2\odot a).
\]
Clearly, this is an action of $\Gamma$, so it induces a decomposition
of $X\times X$ into orbits.
Each of these orbits is a subset of $X\times X$, that means, a relation
on the set $X$ (these relations are sometimes called \emph{orbitals}, or
\emph{2-orbits}). 
This way, from a group acting on a set
we obtain data in the form $(X,(X\times X)/\Gamma)$, where $(X\times X)/\Gamma$ is
the decomposition of $X\times X$ into 2-orbits. 
This system is always a coherent configuration. The 
coherent configurations arising from an action of a group in this way are called
\emph{Schurian coherent configurations}.
    \begin{figure}
        \centering
        \includegraphics[width=0.3\linewidth]{pentagon.pdf}
        \caption{The Schurian scheme $D_5$}
        \label{fig:pentagon}
    \end{figure}
\begin{example}
Consider the dihedral group $D_5$ acting on the $5$ vertices of a regular pentagon in
the usual way. Write $X$ for the set of vertices. This situation
induces a decomposition of $X\times X$ into three classes: the set of all loops $l$ 
(this is the unique unit color, so we obtain an association scheme in this case), the
set of all diagonals $d$ and the set of all sides $f$, see \Cref{fig:pentagon}.
Every color is a
symmetric relation. Some values of the structural constants are
\begin{align*}
\nabla_{dd}^f&=1\\
\nabla_{dd}^d&=0\\
\nabla_{dd}^l&=2
\end{align*}
and the valencies are $\val{l}=1$, $\val{f}=\val{d}=2$.
\end{example}
\begin{example}\label{ex:cayleyScheme}
Every finite group $\Gamma$ acts on itself via multiplication. It is clear that the Schurian coherent configuration given
by this action is a thin association scheme. Moreover, every thin association scheme arises in this way.
\end{example}

\section{Coherent configurations and Frobenius structures}

Consider a pair $(X,S)$, where $X$ is a finite set equipped with a system $S$
of subsets such that $S$ is a partition of $X\times X$, that means, the
condition (CC1) of \Cref{def:cc} is satisfied. For each such a pair we may
construct a non-negative matrix (a $\MatR$ morphism) $\gamma\colon X\otimes S\to X$ given by

\[
\gamma_{xa}^y=
\begin{cases}
1&\text{if $(x,y)\in a$}\\
0&\text{otherwise.}
\end{cases}
\]

Clearly, this matrix $\gamma$ faithfully represents the pair $(X,S)$.

A \emph{pointed magma} in $\MatR$ is a triple $(S,\nabla,\unit)$, 
where $S$ is a finite set, $\nabla\colon S\otimes S\to S$ and $\unit\colon I\to S$.

Note that for a pair $(X,S)$ satisfying (CC3), it is natural to consider the
structural constants $\nabla_{ab}^c$ as elements of a matrix $\nabla\colon
S\otimes S\to S$. Similarly, for $(X,S)$ satisfying (CC2), we may encode the
information about unit colors by a $\MatR$ morphism $\unit\colon I\to S$ given by

\begin{equation}\label{eq:defunit}
\unit_{*}^a
=
\begin{cases}
1 & \text{if $a\subseteq\id_X$}\\
0 & \text{otherwise.}
\end{cases}
\end{equation}

A \emph{module over a pointed magma $(S,\nabla,\unit)$} is a finite set $X$
and a morphism $\gamma\colon X\otimes S\to X$ such that the equations

\begin{equation}
\label{eq:module}
\gamma\circ(\gamma\otimes\id_S)=\gamma\circ(\id_X\otimes\nabla)
\qquad
\gamma\circ(\id_X\otimes e)=\lambda_X
\end{equation}

are satisfied. In string diagrams, this is expressed by
\begin{center}
\vskip 0.5em
\vcbox{\includegraphics{module_gamma_gamma_module.pdf}}
\vcbox{ $=$ }
\vcbox{\includegraphics{module_nabla_gamma_module.pdf}}
\hspace*{3em}
\vcbox{\includegraphics{module_unit_nabla_module.pdf}}
\vcbox{ $=$ }
\vcbox{\includegraphics{module_id_module.pdf}}
\vskip 0.5em
\end{center}

where the gray dot represents the $\gamma$.


\begin{theorem}\label{thm:monAction}
Let $X$ be a finite set, let $S$ be a partition of $X\times X$,
so that $(X,S)$ satisfies (CC1).
Let $\gamma\colon X\otimes S\to X$ be the $\MatR$-morphism
associated with $(S,X)$. 
Then $\gamma$ is a module over some pointed magma $(S,\nabla,\unit)$ if
and only if $(X,S)$ satisfies (CC2) and (CC3) of
\Cref{def:cc}.
Moreover, $(S,\nabla,e)$ is then a monoid.
\end{theorem}
\begin{proof}
Let us prove that $(X,S)$ satisfies (CC3) if and only if
\begin{equation}\label{eq:CC3}
\gamma\circ(\gamma\otimes\id_S)=\gamma\circ(\id_X\otimes\nabla).
\end{equation}
Suppose that we have some magma $\nabla\colon
S\otimes S\to S$ such that \eqref{eq:CC3} is satisfied. This means that,
for all $(x,a,b)\in X\otimes S\otimes S$ and $y\in X$,

\begin{equation}\label{eq:modulesqone}
(\gamma\circ(\id\otimes \nabla))_{xab}^y
    =\sum_{(z,c)\in X\otimes S}\gamma_{zc}^y\delta_x^z\nabla_{ab}^c
    =\sum_{c\in S}\gamma_{xc}^y\nabla_{ab}^c
\end{equation}
is equal to
\begin{equation}\label{eq:modulesqtwo}
(\gamma\circ(\gamma\otimes\id))_{xab}^y
    =\sum_{(w,p)\in X\otimes S}\gamma_{wp}^y\gamma_{xa}^w\delta_b^p
    =\sum_{w\in X}\gamma_{wb}^y\gamma_{xa}^w.
\end{equation}

Consider the rightmost term in \eqref{eq:modulesqtwo} first. Obviously, it just
counts the number of $w\in X$ such that $(x,w)\in a$ and $(w,y)\in b$, that is, the number
of $(a,b)$ triangles over $(x,y)$. Further, in the
rightmost term of \eqref{eq:modulesqone}, there is exactly one $c\in S$ such that
$\gamma_{xc}^y$ is nonzero, namely $c=\<x,y\>$,  and then $\gamma_{xc}^y=1$, so
\eqref{eq:modulesqone} equals $\nabla_{ab}^c$, irrespective of $x$ and $y$.  Hence
\eqref{eq:CC3} is equivalent to (CC3), and $\nabla$ is necessarily given by the structural
constants.

Let us prove that the equality 
\begin{equation}\label{eq:CC2}
\gamma\circ(\id_X\otimes e)=\lambda_X
\end{equation}
for some $\unit\colon I\to X$ is equivalent the the fact that $(X,S)$
satisfies (CC2). This equality  means that, for all $u\in X$
and $w\in X$

\begin{equation}\label{eq:moduletrone}
\delta_u^w=(\gamma\circ(\id_X\otimes\unit))_{u*}^w=
    \sum_{(p,v)\in X\otimes S}\gamma_{vp}^w\delta_u^v\unit_{*}^p=
    \sum_{p\in S}\gamma_{up}^w\unit_{*}^p.
\end{equation}
Note that, for all $u\in X$ and $w\in X$, the very definition
of $\gamma$ implies that 
\begin{equation}\label{eq:isnabla}
\sum_{p\in S}\gamma_{up}^w\unit_{*}^p=
\unit_*^c
\end{equation}
where $(u,w)\in c$. Hence equation \eqref{eq:CC2} is equivalent to
\begin{equation}\label{eq:CC2eq}
e_*^c=\delta_u^w,\qquad \forall (u,w)\in c.
\end{equation}
If this is true for some $e:I\to S$, then it is clear that $e_*^c\in\{0,1\}$ for all $c$,
moreover, $e_*^c=1$ if and only if $(u,u)\in c$ for some $u$, in which case we must have
$u=w$ for all $(u,w)\in c$. This means that (CC2) holds and $e$ must be given by
\eqref{eq:defunit}.

On the other hand, suppose that (CC2) is satisfied. We may then define
$\unit$ as in \eqref{eq:defunit}, which clearly satisfies \eqref{eq:CC2eq}.

%Assume that \eqref{eq:moduletrone} equals $\delta_u^w$.
%Suppose that $(x,x)\in c$; we shall prove that this implies that $c\subseteq\id_X$,
%in other words, that if $(y,z)\in c$ then $y=z$. 
%As $\delta_x^x=1$, \eqref{eq:moduletrone}
%equals $1$ for $w=u=x$. By \eqref{eq:isnabla}, we see that $\unit_*^c=1$.
%Put $u:=y$ and $w:=z$ in \eqref{eq:moduletrone}.
%As $(y,z)\in c$ and we already know that $\unit_*^c=1$, 
%\eqref{eq:moduletrone} equals $1$, 
%but then $1=\delta_y^z$ so $y=z$ and $(X,S)$ satisfies (CC2).
%
%On the other hand, suppose that (CC2) is satisfied. We may then define
%$\unit$ as in \eqref{eq:defunit}. We claim that then \eqref{eq:moduletrone} equals
%$\delta_u^w$. 
%
%Suppose that $u=w$, we need to prove that \eqref{eq:moduletrone} equals $1$. 
%Put $x:=u=w$ and let $c\in S$ be such that
%$(x,x)\in c$, so that \eqref{eq:isnabla} equals $\unit_*^c$. 
%As $c$ is a unit color, $\unit_*^c=1$ and hence \eqref{eq:moduletrone} equals $1$.
%Suppose that $u\neq w$, we need to prove that \eqref{eq:moduletrone} equals $0$.
%Clearly, $u\neq w$ and $(u,w)\in c$ imply that $c$ is not a unit color, hence $\unit_*^c=0$ and we see that 
%\eqref{eq:moduletrone} is $0$.
%
Let us prove that $\nabla$ is associative; that is for all
$(a,b,c)\in S\otimes S\otimes S$, and $d\in S$, the equality \eqref{eq:assoc} is satisfied.
Fix any $(x,y)\in d$. Then $\nabla_{pc}^d$ is the number of $(p,c)$-triangles over
$(x,y)$. For each of those triangles, $\nabla_{ab}^p$ is the number of $(a,b)$ triangles
along the $p$ edge. Thus, the number $\nabla_{pc}^d\cdot\nabla_{ab}^p$ is the number of squares
of the form
$$
\xymatrix{
\cdot
	\ar[r]^b
&
\cdot
	\ar[d]^c
\\
x
	\ar[u]^a
	\ar[ru]^p
	\ar[r]_d
&
y
}
$$
The sum on the left hand side of \eqref{eq:assoc} then runs over all possible colors $p$ of the
diagonal. Therefore, the left hand side of \eqref{eq:assoc} is equal to the number of all squares of the form
\begin{equation}
\label{eq:squareassoc}
\xymatrix{
\cdot
	\ar[r]^b
&
\cdot
	\ar[d]^c
\\
x
	\ar[u]^a
	\ar[r]_d
&
y
}
\end{equation}
Note that this number does not depend on the particular choice of $(x,y)\in d$.
For $(x,y)\in d$, the number $\nabla_{aq}^d\cdot\nabla_{bc}^q$ is the number of squares of the form
$$
\xymatrix{
\cdot
	\ar[r]^b
	\ar[rd]^q
&
\cdot
	\ar[d]^c
\\
x
	\ar[u]^a
	\ar[r]_d
&
y
}
$$
Summing over all possible colors $q$ of the diagonal gives us the number of all squares of the
form \eqref{eq:squareassoc} and we have proved the associativity of $\nabla$.

Let us prove that $\unit$ is a right unit with respect to $\nabla$. By
\eqref{eq:unitsMatR}, this means that
\begin{equation}
\label{eq:runit}
\sum_{q\in E_S}\nabla_{aq}^b =\delta_a^b,\qquad \forall a,b\in S.
\end{equation} 
So let $a$ and $(x,y)\in a$. Then for $q\in E_S$, $\nabla_{aq}^a>0$ if and only if
$(y,y)\in q$. Since there is a unique unit color with this property, and $(x,y,y)$ is the
unique $(a,q)$-colored walk from $x$ to $y$, the sum on the left hand side of
\eqref{eq:runit} is equal to 1 for $a=b$. 
If $\nabla_{aq}^b>0$ for $b\in S$, then $(y,y)\in q$ and $(x,y)\in b$. Since $S$ is a partition of
$X\times X$, $a\cap b\neq\emptyset$ implies that $a=b$. 
%
%This amounts to proving that for every $a\in S$
%there is exactly one pair $(q,b)$ with $q\in E_S$ and $b\in S$ 
%such that $\nabla_{aq}^b>0$, and that, for this pair, we have $b=a$ and
%$\nabla_{aq}^b=\nabla_{aq}^a=1$. 
%Suppose that $\nabla_{aq}^b>0$ and
%let $(x,y)\in a$. Then $(y,y)\in q$ and $(x,y)\in b$. Since $S$ is a partition of
%$X\times X$, $a\cap b\neq\emptyset$ implies that $a=b$. Clearly,
%$\nabla_{aq}^a=1$, because $(x,y,y)$ is the unique $(a,q)$-colored
%walk from $x$ to $y$. Finally, $q$ is unique, because for every $p\in E_S$,
%the only $(a,p)$-colored walk from $x$ to $y$ is $(x,y,y)$, so
%$(y,y)\in p\cap q$ and $p=q$.
\end{proof}

We have proved that every pair of finite sets satisfying (CC1)-(CC3) can be represented
as a module over a monoid in $\MatR$. We say that $(S,\nabla,e)$ is the 
\emph{associated monoid of the coherent configuration $(S,X)$} and that $\gamma$ is
\emph{the module of $(S,X)$}.

A natural question arises: what does (CC4) mean in this context?
This question is answered by the following theorem.

\begin{theorem}\label{thm:frobIsC4}
Let $(X,S)$ be a pair of finite sets satisfying (CC1)-(CC3). Let $f: S\to I$ be such that
$f^*_a>0$ iff $a\subseteq \id_X$. 
Then $f$ is a non-degenerate form on the associated monoid 
$(S,\nabla,\unit)$ if and only if $(X,S)$ is a coherent configuration.
\end{theorem}
\begin{proof}
Suppose that $(X,S)$ is a coherent configuration. We need to prove that
there is some $\eta\colon I\to S\otimes S$ such that
$(S,S,\eta,f\circ\nabla)$ is a duality situation, that is, that the identities in
\eqref{eq:snakes} hold.  The first identity has the form
\begin{equation}\label{eq:leftfrobsnake}
\lambda_S\circ((f\circ\nabla)\otimes\id_S)\circ(\id_S\otimes\eta)\circ\rho_S^{-1}=\id_S.
\end{equation}
On the left-hand side, we have, for all $a,b\in S$,
\begin{equation}\label{eq:lsnakeelement}
(\lambda\circ(f\otimes\id)\circ(\nabla\otimes\id)\circ(\id\otimes\eta)\circ\rho^{-1})_a^b=
\sum_{p,q\in S}f_q^*\nabla_{ap}^q\eta_*^{pb}.
\end{equation}
Observe that $f_q^*\nabla_{ap}^q\eta_*^{pb}>0$ implies that $q\in E_S$ and
$\nabla^q_{ap}>0$. As $q$ is a unit color, this means that $p=a^{-1}$, $q=s(a)$, and
$\nabla^q_{ap}=\nabla^{s(a)}_{aa^{-1}}=\|a\|$. Hence the right-hand side of
\eqref{eq:lsnakeelement}  reads
\[
f_{s(a)}^*\val a\eta_*^{a^{-1}b}.
\]
It is now clear that this is equal to $\delta_a^b$, so that the identity
\eqref{eq:leftfrobsnake} is satisfied,  if 
$\eta\colon I\to S\otimes S$ is given by the rule
\begin{equation}\label{eq:etafrob}
\eta_*^{pb}=
\begin{cases}
\frac{1}{f_{s(b)}^*\val b} & \text{if $p=b^{-1}$}\\
0 & \text{otherwise}.
\end{cases}
\end{equation}
We now check the second identity of \eqref{eq:snakes}. Similarly as above, we need to
prove that for all $a,b\in S$,
\[
\delta_a^b=(\rho_S\circ(\id_S\otimes (f\circ\nabla))\circ(\eta\otimes
\id_S)\circ\lambda_S^{-1})_a^b=\sum_{p,q}f_p^*\nabla_{qa}^p\eta_*^{bq}.
\]
This time, the right-hand side is equal to
\[
f_*^{t(a)}\nabla^{t(a)}_{a^{-1}a}\eta_*^{ba^{-1}}=f_*^{t(a)}\val{a^{-1}}\eta_*^{ba^{-1}}=\delta_a^b,
\]
note that $s(a^{-1})=t(a)$.


Let us now prove the converse. Suppose that $f$ is a non-degenerate form. By the remarks
below \Cref{thm:nondeg}, this means that
\[
M:=(\id_S\otimes\eta_S)\circ(\nabla\otimes\id_{S^*})\circ(f\otimes\id_{S^*})\colon
S\otimes I\to I\otimes S^*
\]
is an isomorphism in $\MatR$, 
here $\eta_S$ denotes the canonical cup from the compact structure. The matrix elements
of this isomorphism are
\[
M_a^b=\sum_{p\in S}\nabla_{ab}^pf_p^*=\sum_{p\in E_S}\nabla_{ab}^p
f_p^*=\nabla_{ab}^{s(a)}f^*_{s(a)},
\]
since the expressions under the sum are nonzero only if $p\in E_S$ and $\nabla^p_{ab}>0$,
which implies that $p=s(a)$. By \Cref{coro:chariso}, we obtain that for every $a\in S$ there is a unique $b\in S$ such
that $\nabla^{s(a)}_{ab}$ is nonzero.  Let $(x,y)\in a$.
As $s(a)$ is a unit color, $(x,x)\in s(a)$ and, by the uniqueness of $b$,
$(y,x)\in b$. Therefore $b$ is an inverse relation to $a$, and we see that the set of
relations $S$ is closed with respect to taking inverses.
\end{proof}

A natural choice of a non-degenerate form in the above theorem is $f=\unit^\dag$, this choice will be used in our paper.
Our main motivation for choosing $e^\dag$ is \Cref{thm:connections} below.

Let $(S,X)$ be a coherent configuration. As $e^\dag$ is a non-degenerate form on $(S,\nabla,e)$, 
the bijective correspondence in \Cref{thm:nondeg}
gives us a Frobenius structure comultiplication.  
\[
\Delta=(\id_S\otimes\nabla)\circ(\eta\otimes\id_S)\circ\rho_S,
\] 
where $\eta$ is given by \eqref{eq:etafrob} (with $f=\unit^\dag$). Unwinding this definition of $\Delta$,
for all $a,b,c\in S$
\begin{equation}\label{eq:comul}
\Delta_a^{bc}=\sum_{p\in
S}\nabla_{pa}^c\eta_*^{bp}=\frac{1}{\val{b^{-1}}}\nabla_{b^{-1}a}^c
\end{equation}
By \cite[Proposition 5.16]{heunen2019categories}, the same $\Delta$ is also given by
the equation
\begin{equation}\label{eq:comulAlt}
\Delta=(\nabla\otimes\id_S)\circ(\id_S\otimes\eta)\circ\lambda_S
\end{equation}
which in our situation gives us
\[
\Delta_a^{bc}=\frac{1}{\val{c}}\nabla_{ac^{-1}}^b
\]
so there are two ways of expressing the $\Delta$ in terms of $\nabla$ and the
valencies.

Let us remark that, alternatively, it is possible to define $\Delta$ by
\eqref{eq:comul} or \eqref{eq:comulAlt} and then prove directly by a
combinatorial argument that $(S,\Delta,e^\dag)$ is a comonoid and that
$\Delta,\nabla$ satisfies the Frobenius laws.

The following theorem shows that some properties of a coherent configuration
are reflected in the properties of the associated Frobenius structure.

\begin{theorem}\label{thm:connections}
For every coherent configuration $(X,S)$, the associated Frobenius structure
is
\begin{enumerate}[(a)]
\item special if and only if $(X,S)$ is thin,
\item symmetric if and only if $(X,S)$ is $\frac{1}{2}$-homogeneous,
\item extra if and only if $(X,S)$ is an association scheme.
\end{enumerate}
\end{theorem}
\begin{proof}
\begin{enumerate}[(a)]
\item
Suppose that $S$ is special, meaning that
for all $a,d\in S$,
\begin{multline}\label{eq:eye}
(\nabla\circ\Delta)_a^d=
\sum_{(b,c)\in S\otimes S}\nabla_{bc}^d\Delta_a^{bc}=\\
\sum_{(b,c)\in S\otimes S}\nabla_{bc}^d\frac{1}{\val{b^{-1}}}\nabla_{b^{-1}a}^c=
\sum_{b\in S}\frac{1}{\val{b^{-1}}}\sum_{c\in S}\nabla_{bc}^d\nabla_{b^{-1}a}^c
\end{multline}
is equal to $t\cdot\delta_a^d$, for some $t\in\mathbb R_0^+$.
We want to prove that this implies that $(X,S)$ is thin. Assume the contrary,
and denote the color with valency greater than one by $b^{-1}$. That means, that
there is a vertex that is a source of at least two $b^{-1}$-colored edges, so there
must be some edges and vertices as in the following picture
\[
\xymatrix{
\cdot
    \ar[r]^{b^{-1}}
    \ar[d]_{b^{-1}}
&
\cdot
    \ar@(ur,ul)_a
\\
\cdot
    \ar[ur]_{d}
}
\]
Note that $d$ is a non-unit color and $a$ is a unit color, which means that $a\neq
d$ and that the picture can be drawn as
\[
\xymatrix{
\cdot
    \ar[r]^{b^{-1}}
&
\cdot
    \ar@(ur,ul)_a
\\
\cdot
    \ar[ur]_{d}
    \ar[u]^{b}
}
\]
The existence of such an arrangement of vertices and edges means that
\[
\nabla_{bb^{-1}}^d\nabla_{b^{-1}a}^{b^{-1}}>0,\] 
and this means (put $c:=b^{-1}$)
that the inner sum (and thus the whole sum) in the rightmost term of \eqref{eq:eye} is positive.
Therefore, $(\nabla\circ\Delta)_a^d>0$ for some $a\neq d$, so
$\nabla\circ\Delta$ is not diagonal.

Suppose that $(X,S)$ is thin. Let us prove that all off-diagonal entries
of $\nabla\circ\Delta$ are zero. Assuming the contrary, let $a,d\in S$ be such that
$a\neq d$ and $(\nabla\circ\Delta)_a^d>0$. Then there must be some  $b,c\in S$ for which 
$\nabla_{bc}^d\nabla_{b^{-1}a}^c>0$.
This means that 
for every $(x,y)\in d$ there exist $z,w\in X$ such that we have edges and colors
as in the following picture:
\[
\xymatrix{
z
    \ar[r]^{b^{-1}}
    \ar[rd]^c
&
w
    \ar[d]^a
\\
x
    \ar[r]_d
    \ar[u]^b
&
y
}
\]
The assumption $a\neq d$ means that $x\neq w$. Therefore, $(z,w),(z,x)$ are two
distinct edges colored by $b^{-1}$ with the same source vertex. 
Hence, $\val{b^{-1}}>1$ and $(X,S)$ is not thin.
We have proved that $\nabla\circ\Delta$ is a diagonal matrix. We now claim that every diagonal entry is equal to the
cardinality of $X$. As $(X,S)$ is thin, we have 
\[
 (\nabla\circ\Delta)_a^a=\sum_{b\in S}\sum_{c\in S}\nabla_{bc}^a\nabla_{b^{-1}a}^c.
\]
For every $b\in S$, the inner sum counts the number of squares
\begin{equation}\label{diag:collapsingSquare}
\xymatrix{
z
    \ar[r]^{b^{-1}}
&
w
    \ar[d]^a
\\
x
    \ar[r]_a
    \ar[u]^b
&
y
}
\end{equation}
over an arbitrary fixed $(x,y)\in a$. If $w\neq x$ in one of such squares, 
then $(y,x),(y,w)\in a^{-1}$ are
distinct, hence $\val{a^{-1}}\ge 2$ which contradicts the assumption that $(X,S)$ is
thin. Thus $w=x$ and we see that every square of the type \eqref{diag:collapsingSquare} collapses to
\[
\xymatrix{
z
    \ar@<-.75ex>[r]_-{b^{-1}}
&
x
    \ar[l]_-b
    \ar[r]_-a
&y
}
\]
Summing the number of such collapsed squares over all $b\in S$ actually means to
count the number of all $z\in X$, because each $z\in X$ gives us exactly one such
collapsed square. Therefore, $(\nabla\circ\Delta)_a^a=|X|$ for every $a\in S$ and we have proved that
the Frobenius structure $S$ is special.
\item
For every $a,b\in S$
\begin{equation}\label{eq:symmetrysigma}
(e^\dag\circ\nabla\circ\sigma_{S,S})_{ab}^*=
\sum_{p\in{E_S}}\nabla_{ba}^p=
\begin{cases}
\val{a^{-1}}&b=a^{-1}\\
0&\text{otherwise}
\end{cases}
\end{equation}
and 
\begin{equation}\label{eq:symmetrynosigma}
(e^\dag\circ\nabla)_{ab}^*=
\sum_{p\in{E_S}}\nabla_{ab}^p=
\begin{cases}
\val{a}&b=a^{-1}\\
0&\text{otherwise.}
\end{cases}
\end{equation}
Clearly, the equality of \eqref{eq:symmetrysigma} and \eqref{eq:symmetrynosigma}
is equivalent to $\val{a}=\val{a^{-1}}$, for all $a\in S$.
\item
It suffices to observe that $e^\dag\circ e$ is a $1\times 1$ matrix containing the number
\[
\sum_{p\in S}(e^\dag)_p^*e_*^p=\sum_{p\in E_S}1,
\]
which is equal to the cardinality of $E_S$.
\end{enumerate}
\end{proof}

\begin{remark} Note that the above results depend crucially on the choice of the
non-degenerate form $f=\unit^\dag$. Other choices yield different Frobenius structures
with different properties. For example, as in the proof of (b) we obtain that the
structure associated with a non-degenerate form $f$ is symmetric if and only if  
\[
f_{s(a)}^*\val a=f^*_{s(a^{-1})}\val{a^{-1}},\qquad a\in S.
\]
Putting 
\[
f_p^*=
\begin{cases}
    |p| & p\in E_S\\
    0 & \text{otherwise}
\end{cases}
\] gives us 
\[
f^*_{s(a)}\val a=|s(a)|\cdot\val a=|a|=|a^{-1}|=f^*_{s(a^{-1})}\val{a^{-1}},
\]
so that this choice leads to a symmetric Frobenius structure, for every coherent
configuration.
\end{remark}

\begin{theorem}
For every coherent configuration $(X,S)$, the Nakayama
automorphism $\alpha\colon S\to S$ 
of the associated Frobenius structure is a diagonal matrix given
by the rule
\[
\alpha_a^q=
\begin{cases}
\frac{\val{a}}{\val{a^{-1}}}&q=a\\
0&\text{otherwise.}
\end{cases}
\]
\end{theorem}
\begin{proof}
It suffices to prove that the morphism $\alpha$ given by the above rule satisfies \eqref{eq:nakayama}. On the left-hand side
of \eqref{eq:nakayama} we obtain, for all $(a,b)\in S\otimes S$,
\begin{multline*}
(e^\dag\circ\nabla\circ(\id_S\otimes\alpha)\circ\sigma_{S,S})_{ab}^*=\\
\sum_{\substack{q\in S\\r\in E_S}}\alpha_a^q\nabla_{bq}^r=
\sum_{r\in E_S}\frac{\val{a}}{\val{a^{-1}}}\nabla_{ba}^r=
\begin{cases}
\frac{\val{a}}{\val{a^{-1}}}\val{a^{-1}}&b=a^{-1}\\
0&\text{otherwise}
\end{cases}
\end{multline*}
and this is equal to \eqref{eq:symmetrynosigma}, which is the same as the right-hand
side of \eqref{eq:nakayama}.

\end{proof}
\begin{corollary}\label{coro:nakayama}
Let $(X,S)$ be a coherent configuration. For every $a,b,c\in S$ such that
$\nabla_{ab}^c>0$,
\[
\frac{\val{a}}{\val{a^{-1}}}\cdot
\frac{\val{b}}{\val{b^{-1}}}=
\frac{\val{c}}{\val{c^{-1}}}.
\]
\end{corollary}
\begin{proof}
The Nakayama automorphism $\alpha\colon S\to S$ preserves $\nabla$:
\begin{center}
\vskip 0.5em
\vcbox{\includegraphics{monoid_morph_left_nakayama.pdf}}
\vcbox{ $=$ }
\vcbox{\includegraphics{monoid_morph_right_nakayama.pdf}}
\end{center}
meaning that, for all $(a,b)\in S\otimes S, c\in S$
\[
\sum_{(q,r)\in S\otimes S}\nabla_{qr}^c\alpha_a^q\alpha_b^r=
\sum_{p\in S}\nabla_{ab}^p\alpha_p^{c}.
\]
Both  sums are reduced to a single term
\[
\nabla_{ab}^c\cdot\frac{\val{a}}{\val{a^{-1}}}.
\frac{\val{b}}{\val{b^{-1}}}=
\nabla_{ab}^c\cdot\frac{\val{c}}{\val{c^{-1}}}
\]
and we may cancel $\nabla_{ab}^c>0$ from both sides.

\end{proof}
\section{Morphisms and admissible morphisms}

In this section we will use $(X,S,\gamma)$ to denote the fact that
$\gamma$ is the module associated with the coherent configuration $(X,S)$. We do not 
distinguish between mappings in $\FinSet$ and their matrix representations
in $\MatR$.

The following definition is a straightforward extension of the definition of a morphism of association
schemes from \cite{hanaki2010acategory}.
\begin{definition}

A {\em  morphism} of coherent configurations $(X,S)\to (X',S')$ is
a pair of mappings $\phi_0\colon X\to X'$ and $\phi_1\colon S\to S'$ such that
for all $x,y\in X$ and $a\in S$, $(x,y)\in a$ implies $(\phi_0(x),\phi_0(y))\in \phi_1(a)$.

\end{definition}

We write $(\phi_0,\phi_1)\colon (X,S)\to (X',S')$ for a morphism of
coherent configurations.

\begin{proposition}\label{prop:laxmonoidmorphism}
Let $(\phi_0,\phi_1)\colon (X,S)\to (X',S')$ be a morphism of coherent
configurations.
\begin{enumerate}[(a)]
\item For every $p\in E_S$, $\phi_1(p)\in E_{S'}$.
\item For every $a\in S$, $\phi_1(s(a))=s(\phi_1(a))$.
\item For every $a\in S$, $\phi_1(t(a))=t(\phi_1(a))$.
\item For all $a,b,c\in S$, if $\nabla_{ab}^c>0$, then 
$\nabla_{\phi_1(a)\phi_1(b)}^{\phi_1(c)}>0$.
\item For all $a\in S$, $\phi_1(a^{-1})={\phi_1(a)}^{-1}$.
\end{enumerate}
\end{proposition}
\begin{proof}~
\begin{enumerate}[(a)]
\item If $p\in E_S$, then $(x,x)\in p$ for some $x\in X$. If $(x,x)\in p$, then $(\phi_0(x),\phi_0(x))\in\phi_1(p)$, so $\phi_1(p)\in
E_S$.
\item If $(x,y)\in a$, then $(x,x)\in s(a)$ and
$(\phi_0(x),\phi_0(x))\in\phi_1(s(a))$. On the other hand, 
$(\phi_0(x),\phi_0(y))\in\phi_1(a)$, so $(\phi_0(x),\phi_0(x))\in s(\phi_1(a))$.
\end{enumerate}
The remaining statements (c)-(e) can be proved in a very similar way.
\end{proof}

Let us link the notion of morphism with the 2-structure of $\MatR$. The (a)
and (d) of \Cref{prop:laxmonoidmorphism} mean that $\phi_1$ is a \emph{lax morphism}
of monoids in $\MatR$, that means that the diagrams
\[
\begin{diagram}
I
    \ar@{}[rrd]|(.35){{\gtrsim}}
    \ar[r]^{e_S}
    \ar[rd]_{e_{S'}}
&
S
    \ar[d]^{\phi_1}
\\
~
&
S'
&
~
\end{diagram}
\quad
\begin{diagram}
S\otimes S
    \gscellrd    
    \ar[r]^{\nabla}
    \ar[d]_{\phi_1\otimes\phi_1}
&
S
    \ar[d]^{\phi_1}
\\
S'\otimes S'
    \ar[r]_{\nabla}
&
S'
\end{diagram}
\]
commute.
The pair of mappings $(\phi_0,\phi_1)$ then is a lax module morphism, as the following
proposition shows.

\begin{proposition}\label{prop:charmorphism}

Let $(X,S,\gamma)$, $(X',S',\gamma')$ be coherent configurations. A pair
of mappings $\phi_0\colon X\to X'$ and $\phi_1\colon S\to S'$
is a morphism if and only if
\begin{equation}\label{diag:morphism}
\begin{diagram}
X\otimes S
    \ar[r]^\gamma
    \ar[d]_{\phi_0\otimes \phi_1}
    \gscellrd
&
X
    \ar[d]^{\phi_0}
\\
X'\otimes S'
    \ar[r]_{\gamma'}
&
X'
\end{diagram}
\end{equation}
\end{proposition}
\begin{proof}

We see that
\[
(\phi_0\circ\gamma)_{xa}^{y'}=
\sum_{y\in X}(\phi_0)_y^{y'}\gamma_{xa}^y
\]
is nonzero if and only if there is $y\in X$ such that $(x,y)\in a$ and $\phi_0(y)=y'$.
On the other hand,
\[
(\gamma'\circ(\phi_0\otimes \phi_1))_{xa}^{y'}=
\sum_{(w,p)\in X'\otimes S'}(\gamma')_{wp}^{y'}(\phi_0)_x^w(\phi_1)_a^p
\]
is nonzero if and only if $(\phi_0(x),y')\in \phi_1(a)$.

$(\Rightarrow)$: Suppose that \eqref{diag:morphism} holds. Assume $(x,y)\in a$,
we need to prove that $(\phi_0(x),\phi_0(y))\in \phi_1(a)$. Putting $y'=\phi_0(y)$
means that $(\phi_0)_{y}^{y'}\gamma_{xa}^y>0$, hence
$(\phi_0\circ\gamma)_{xa}^{y'}>0$. By the diagram, this means that
$(\gamma\circ(\phi_0\otimes \phi_1))_{xa}^{y'}>0$, so
$(\phi_0(x),y')=(\phi_0(x),\phi_0(y))\in \phi_1(a)$.

$(\Leftarrow)$: Suppose that $(\phi_0,\phi_1)$ is a morphism.
Assume that $(\phi_0\circ\gamma)_{xa}^{y'}>0$, this means that
there is $y\in X$ such that $\phi_0(y)=y'$ and $(x,y)\in a$. As $(x,y)\in a$,
$(\phi_0(x),\phi_0(y))\in \phi_1(a)$. Putting $w=\phi_0(x)$ and $p=\phi_1(a)$ we obtain
$\gamma_{wp}^{y'}=1$, so $\gamma_{wp}^z(\phi_0)_x^w(\phi_1)_a^p>0$ and thus
$(\gamma\circ(\phi_0\otimes \phi_1))_{xa}^{y'}>0$.

\end{proof}

In \cite{french2013functors} C. French introduced and studied a subclass of morphisms of association schemes called \emph{admissible morphisms}.
When we attempted to extend this notion to coherent configurations, 
it turned out that there are (at least) two appropriate generalizations of the
original notion: \Cref{def:admorphism} and \Cref{def:stradmorphism}. When restricted
to association schemes, both definitions coincide (\Cref{coro:defsCoincide}).
Let us consider the more general definition first.

\begin{definition}\label{def:admorphism}
An {\em admissible morphism} of coherent configurations $(X,S)\to (X',S')$ is
a morphism $(\phi_0,\phi_1)\colon (X,S)\to(X',S')$ such
that for any $x\in X$, $y'\in X'$, $a\in S$ such that $(x,a)\in\supp(\gamma)$ and
$(\phi_0(x),y')\in \phi_1(a)$,
there exists $y\in X$ such that $\phi_0(y)=y'$ and $(x,y)\in a$.
\end{definition}

The notion of a support endomorphism \eqref{eq:supportendo} allows us to express
\Cref{def:admorphism} in the form of a commutative diagram, as the following
proposition shows.

\begin{proposition}\label{prop:admorphism}
A morphism $(\phi_0,\phi_1)\colon (X,S,\gamma)\to(X',S',\gamma')$ of coherent configurations
is admissible if and only if
\begin{equation}\label{diag:admorphism}
\begin{diagram}
X\otimes S
    \ar[r]^-\gamma
    \ar[d]_{\based{\gamma}}
    \simeqcellrdd
&
X
    \ar[dd]^{\phi_0}
\\
X\otimes S
    \ar[d]_{\phi_0\otimes \phi_1}
    &
    ~
\\
X'\otimes S'
    \ar[r]_-{\gamma'}
&
X'
\end{diagram}
\end{equation}
\end{proposition}
\begin{proof}
By \Cref{prop:charmorphism} and the fact that $\gamma\circ\based\gamma=\gamma$,
we see that, for any morphism $(\phi_0,\phi_1)$,
\[
\begin{diagram}
X\otimes S
    \ar[r]^-\gamma
    \ar[d]_{\based{\gamma}}
&
X
    \ar[dd]^{\phi_0}
\\
X\otimes S
    \ar[d]_{\phi_0\otimes \phi_1}
    \ar[ru]_{\gamma}
    \gscellrd
    &
    ~
\\
X'\otimes S'
    \ar[r]_-{\gamma'}
&
X'
\end{diagram}
\]
Therefore, it suffices to prove that
the condition in \Cref{def:admorphism}
is equivalent to
\begin{equation}\label{diag:admap}
\begin{diagram}
X\otimes S
    \ar[r]^-\gamma
    \ar[d]_{\based{\gamma}}
    \lscellrdd
&
X
    \ar[dd]^{\phi_0}
\\
X\otimes S
    \ar[d]_{\phi_0\otimes \phi_1}
    &
    ~
\\
X'\otimes S'
    \ar[r]_-{\gamma'}
&
X'
\end{diagram}
\end{equation}
The value
\[
(\gamma'\circ(\phi_0\otimes \phi_1)\circ\based\gamma)_{xa}^{y'}=
\sum_{(w,p)\in X'\otimes S'}
{\gamma'}_{wp}^{y'}(\phi_0)_x^w(\phi_1)_a^p
\]
is nonzero if and only if $(x,a)\in\supp(\gamma)$ and $(\phi_0(x),y')\in \phi_1(a)$.
The value
\[
(\phi_0\circ\gamma)_{xa}^{y'}=
\sum_{y\in X}(\phi_0)_y^{y'}\gamma_{xa}^y
\]
is nonzero if and only if there is $y\in X$ such that $(x,y)\in a$ and $\phi_0(y)=y'$.
The rest is trivial.
\end{proof}

Clearly, for every coherent configuration $(X,S)$, $(\id_X,\id_S)\colon (X,S)\to
(X,S)$ is an admissible morphism. Moreover, admissible morphisms are closed with
respect to composition; before proving this fact, we need a lemma.
\begin{lemma}\label{lemma:paste}
Let $(\phi_0,\phi_1)\colon(X,Y,\gamma)\to(X',Y',\gamma')$ be a morphism of coherent
configurations. Then
\[
\begin{diagram}
X\otimes S
    \ar[r]^-{\based\gamma}
    \ar[dd]_{\based\gamma}
&
X\otimes S
    \ar[d]^{\phi_0\otimes\phi_1}
\\
~
&
X'\otimes S'
    \ar[d]^-{\based{\gamma'}}
\\
X\otimes S
    \ar[r]_{\phi_0\otimes\phi_1}
&
X'\otimes S'
\end{diagram}
\]
commutes.
\end{lemma}
\begin{proof}
Both composites in the diagram are $0,1$-matrices. Let us prove that 
\begin{equation}\label{eq:ineq}
(\based{\gamma'}\circ(\phi_0\otimes\phi_1)\circ\based\gamma)_{xa}^{x'a'}\geq
((\phi_0\otimes\phi_1)\circ\based\gamma)_{xa}^{x'a'}.
\end{equation}
We see that $((\phi_0\otimes\phi_1)\circ\based\gamma)_{xa}^{x'a'}=1$ if and
only if 
$(x,a)\in\supp(\gamma)$, $x'=\phi_0(x)$ and $a'=\phi_1(a)$. Since
$(x,a)\in\supp(\gamma)$, there is $y\in X$ such that $(x,y)\in a$ and
since $(\phi_0,\phi_1)$ is a morphism, $(\phi_0(x),\phi_0(y))\in\phi_1(a)$.
Therefore, $(x',a')=(\phi_0(a),\phi_1(a))\in\supp(\gamma')$ and the left-hand
side of \eqref{eq:ineq} is equal to $1$.

The opposite inequality
is trivially true, because $\based{\gamma'}$ is a diagonal matrix with $0,1$ on the
diagonal.
\end{proof}
\begin{proposition}
If
\begin{align*}
(\phi_0,\phi_1)\colon(X,S,\gamma)&\to(X',S',\gamma')\text{ and}\\
(\phi'_0,\phi'_1)\colon(X',S',\gamma')&\to(X'',S'',\gamma'')
\end{align*}
are admissible morphisms of
coherent configurations, then 
\[
(\phi'_0\circ\phi_0,\phi'_1\circ\phi_1)\colon(X,S,\gamma)\to(X'',S'',\gamma'')
\]
is an admissible
morphism.
\end{proposition}
\begin{proof}
\[
\begin{diagram}
X\otimes S
    \ar[rr]^-\gamma
    \ar[rd]^-{{\based{\gamma}}}
    \ar[d]_-{\based{\gamma}}
&
~
&
X
    \ar[dd]^{\phi_0}
\\
X\otimes S
    \ar[d]_-{\phi_0\otimes\phi_1}
&
X\otimes S
    \ar[d]^-{\phi_0\otimes\phi_1}
    \simeqcellr
&
~
\\
X'\otimes S'
    \ar[d]_-{\phi'_0\otimes\phi'_1}
    \simeqcellrrd
&
X'\otimes S'
    \ar[l]_-{\based{\gamma'}}
    \ar[r]^-{\gamma'}
&
X'
    \ar[d]^{\phi'_0}
\\
X''\otimes S''
    \ar[rr]_-{\gamma''}
&
~
&
X''
\end{diagram}
\]
The inner
cells in the diagram are just the two admissible morphisms $(\phi_0,\phi_1)$ and $(\phi_0',\phi_1')$ that we are composing, together with the third cell coming from \Cref{lemma:paste}. The outer border
of the diagram 
is just the admissible morphism $(\phi'_0\circ\phi_0,\phi'_1\circ\phi_1)\colon(X,S)\to(X'',S'')$.
\end{proof}
Suppose that $(\phi_0,\phi_1)\colon (X,S)\to (X',S')$ is an
admissible morphism. Can we find a matrix $\widehat{\phi_1}$ that
will make the diagram \eqref{diag:admorphism} commute precisely? The
answer to this question is positive, as the following proposition shows. 
\begin{proposition}\label{prop:phionehat}(c.f. \cite[Lemma 3.12, Corollary 3.13]{french2013functors})
Let $(\phi_0,\phi_1)\colon(X,S,\gamma)\to(X',S',\gamma')$ be an admissible morphism of
coherent configurations.
\begin{enumerate}[(a)]
\item For all $z,y\in X$, with $(z,y)\in b\in S$, we have $\phi_0(z)=\phi_0(y)$ if and only if
$\phi_1(b)\in E_{S'}$.
\item For every $a\in S$,
$x\in X$, $y'\in X'$ such that 
$(x,a)\in\supp(\gamma)$ and
$(\phi_0(x),y')\in\phi_1(a)$, the number of
elements in the set
\[
\{z\in X:(x,z)\in a\text{ and }\phi_0(z)=y'\}
\]
does not depend on the choice of $x,y'$, only on $a$ and $(\phi_0,\phi_1)$.
\item Denote the number from (b) by $\valphi{a}$. Let $\widehat{\phi_1}$ be a matrix
given by
\[
(\widehat{\phi_1})_a^b=
\begin{cases}
\valphi{a}&\phi_1(a)=b,\\
0&\text{otherwise}.
\end{cases}
\]
The diagram
\begin{equation}\label{diag:modulemorph}
\begin{diagram}
X\otimes S
    \ar[r]^-\gamma
    \ar[d]_{\based{\gamma}}
&
X
    \ar[dd]^{\phi_0}
\\
X\otimes S
    \ar[d]_{\phi_0\otimes\widehat{\phi_1}}
    &
    ~
\\
X'\otimes S'
    \ar[r]_-{\gamma'}
&
X'
\end{diagram}
\end{equation}
commutes.
\item For every $a\in S$,
\[
\valphi{a}=\frac{\val a}{\val{\phi_1(a)}}.
\]
\end{enumerate}
\end{proposition}
\begin{proof}~\par
\begin{enumerate}[(a)]
\item Suppose that \( \phi_0(z) = \phi_0(y) \). Since \( (z,y) \in b \), it follows that  
\( (\phi_0(z), \phi_0(y)) = (\phi_0(z), \phi_0(z)) \in \phi_1(b) \), so \( \phi_1(b) \) is a unit color.  
Conversely, suppose that \( \phi_1(b) \in E_{S'} \). As $(\phi_0(z),\phi_0(y))$ is colored by a
unit color $\phi_1(b)$, we see that $\phi_0(z)=\phi_0(y)$.
\item As $(\phi_0,\phi_1)$ is admissible, there exists at least one 
$y\in X$ such that $\phi_0(y)=y'$ and $(x,y)\in a$. Fix one of such $y$
in what follows, and consider the set 
\[
W=\{z\in X:(x,z)\in a\text{ and }\phi_0(z)=y'=\phi_0(y)\}.
\]
By (a), $z\in W$ if and only if $(z,y)$ is colored by a color $b\in S$ such that 
$\phi_1(b)\in E_{S'}$. Hence
$W$ is a disjoint union of nonempty sets
\[
W=\mathop{\dot\bigcup}\limits_{\phi_1(b)\in E_{S'}}^{} \{z\in X:(x,z)\in a\text{ and }(z,y)\in b\}
\]
and the cardinality of each of those sets if $\nabla_{ab}^a$. Hence the number of
elements in the set $W$ is equal to
\[
\sum_{\phi_1(b)\in E_{S'}}\nabla_{ab}^a
\]
and this clearly depends only on $a$ and $\phi_1$.
\item
The value
\[
(\phi_0\circ\gamma)_{xa}^{y'}=
\sum_{y\in X}(\phi_0)_y^{y'}\gamma_{xa}^y
\]
counts the number of elements in the set
\[
\{y\in X: (x,y)\in a\text{ and }\phi_0(y)=y'\}
\]
which is equal to $\valphi{a}$. On the other hand,
\[
(\gamma\circ(\phi_0\otimes\widehat{\phi_1}))_{xa}^{y'}=
\sum_{(w,p)\in X'\otimes S'}\gamma_{wp}^{y'}(\phi_0)_x^w(\widehat{\phi_1})_a^p
\]
is nonzero if and only if $(\phi_0(x),y')\in\phi_1(a))$, and in this case, it is equal
to $\valphi{a}$.
\item
Whenever $(x,a)\in \supp(\gamma)$, the cardinality of the set
\[
Y=\{y\in X:(x,y)\in a\}
\]
is equal to $\val{a}$. Since $(\phi_0,\phi_1)$ is a morphism, \Cref{prop:charmorphism}
implies that $(\phi_0(x),\phi_1(a))\in\supp(\gamma'))$, so the cardinality of the
set
\[
Y'=\{y'\in X':(\phi_0(x),y')\in\phi_1(a)\}
\]
is equal to $\val{\phi_1(a)}$. For every $y'\in Y'$, the number of
elements in the set
\[
W_{y'}=\{z\in X:(x,z)\in a\text{ and }\phi_0(z)=y'\}
\]
is equal to $\valphi{a}$, and the set $Y$ is a disjoint union of all those
sets:
\[
Y=\dot\bigcup_{y'\in Y'}W_{y'}.
\]
Therefore, $\val{a}=|Y|=|Y'|\cdot \valphi{a}=\val{\phi_1(a)}\cdot \valphi{a}$.
\end{enumerate}
\end{proof}

Since (as we will prove in the next section) admissible morphisms of coherent
configurations generalize admissible
morphisms of association schemes, all the examples from \cite{french2013functors}
apply.

\begin{example}\label{ex:terminalMorph}
It is easy to see that the terminal object of the category $\Conf$ is $(I,I)$, the
coherent configuration with both degree and rank equal to $1$. For every coherent
configuration $(X,S)$, the unique morphism $!\colon (X,S)\to (I,I)$ is admissible.
\end{example}

\begin{example}\label{ex:tensorProduct}
A \emph{tensor product} of coherent configurations $(X,S)$ and $(X',S')$ is the
coherent configuration $(X,S)\otimes(X',S')=(X\otimes X',S\otimes S')$, with
relations in $S\otimes S'$ given by the rule $((x,x'),(y,y'))\in(a,a')\in S$ if and only if
$(x,y)\in a$ and $(x',y')\in a'$. By \cite[Theorem 3.2.9]{chen2024lectures},
the tensor product of coherent configurations is a coherent configuration. Moreover,
the pair of mappings $\pi_0\colon X\otimes X'\to X'$, $\pi_1\colon S\otimes S'\to
S'$ given by the projection $\pi_0(x,x')=x'$, $\pi_1(a,a')=a'$ are admissible
morphisms. Indeed, suppose that $((x,x'),(a,a'))\in\supp(\gamma)$ and  
$(x',y')=(\pi_0(x,x'),y')\in\pi_1(a,a')=a'$. The first assumption implies that
there is some $(w,w')\in X\otimes X'$ such that $((x,x'),(w,w'))\in (a,a')$,
which implies that $(x,w)\in a$. Therefore, $((x,x'),(w,y'))\in (a,a')$ and
$\pi_0(w,y')=y'$.
\end{example}

\section{Strongly admissible morphisms}
Recall the ``lifting'' problem from Definition~\ref{def:admorphism}. If an admissible
morphism admits solutions to a broader class of lifting problems -- namely those obtained
by dropping the assumption $(x,a)\in\supp(\gamma)$ from Definition~\ref{def:admorphism} -- we call it a \emph{strongly admissible morphism}.
The following definition is a straightforward extension of \cite[Definition 3.1]{french2013functors} from association schemes to coherent configurations.
\begin{definition}\label{def:stradmorphism}
A {\em strongly admissible morphism} of coherent configurations $(X,S)\to (X',S')$ is
a morphism $(\phi_0,\phi_1)\colon (X,S)\to(X',S')$ such
that for any $x\in X$, $y'\in X'$, and $a\in S$ such that 
$(\phi_0(x),y')\in \phi_1(a)$,
there exists $y\in X$ with $\phi_0(y)=y'$ and $(x,y)\in a$.
\end{definition}
Clearly, every strongly admissible morphism is admissible and one can easily adjust the
proof of \Cref{prop:admorphism} to prove that a morphism of coherent configurations
$(\phi_0,\phi_1)$ is strongly admissible if and only if the diagram
\begin{equation}\label{diag:strongadmorphism}
\begin{diagram}
X\otimes S
    \ar[r]^{\gamma}
    \ar[d]_{\phi_0\otimes\phi_1}
    \simeqcellrd
&
X
    \ar[d]^{\phi_0}
\\
X'\otimes S'
    \ar[r]_{\gamma'}
&
X'
\end{diagram}
\end{equation}
commutes. One can also see this directly: omitting the assumption
$(x,a)\in\supp(\gamma)$ in \Cref{def:admorphism} corresponds exactly to replacing the $\based{\gamma}$
in the diagram \eqref{diag:admorphism} by $\id_{X\otimes S}$.


\begin{lemma}\label{lemma:charstradmissible}
Let $(\phi_0,\phi_1)\colon(X,S,\gamma)\rightarrow (X',S',\gamma')$ be an admissible morphism. 
The following are equivalent
\begin{enumerate}[(a)]
\item $(\phi_0,\phi_1)$ is strongly admissible.
\item $\phi_1$ is injective on $E_S$.
\item For all $x\in X$, $a\in S$ and $y'\in S'$: if $(\phi_0(x),y')\in\phi_1(a)$,
then $(x,a)\in\supp(\gamma)$.
\item The diagram
\begin{equation}\label{diag:charstrong}
\begin{diagram}
X\otimes S
    \ar[r]^{\based\gamma}
    \ar[dd]_{\phi_0\otimes\phi_1}
&
X\otimes S
    \ar[d]_{\phi_0\otimes\phi_1}
\\
&
X'\otimes S'
    \ar[d]^{\gamma'}
\\
X'\otimes S'
    \ar[r]_{\gamma'}
&
X'
\end{diagram}
\end{equation}
commutes. 
\end{enumerate}
\end{lemma}
\begin{proof}
(a)$\implies$(b): Suppose that $p,q\in E_S$ and that $\phi_1(p)=\phi_1(q)$. Let $x\in
X$ be such that $(x,x)\in p$. As $(\phi_0,\phi_1)$ is a morphism,
$(\phi_0(x),\phi_0(x))\in\phi_1(p)=\phi_1(q)$. Putting $y'=\phi_0(x)$ and 
$a=q$ in \Cref{def:stradmorphism} yields the existence of $y\in X$ such that
$\phi_0(y)=\phi_0(x)$ and $(x,y)\in q$. As $q\in E_S$, $x=y$ and thus $(x,x)\in q$,
so $p=q$.

(b)$\implies$(c):  
Let $(x,x)\in p\in E_S$. Then $(\phi_0(x),\phi_0(x))\in\phi_1(p)$.  
If $(\phi_0(x),y')\in\phi_1(a)$, then $s(\phi_1(a))=\phi_1(p)$.  
By \Cref{prop:laxmonoidmorphism}, we have $s(\phi_1(a))=\phi_1(s(a))$, and by the injectivity
of $\phi_1$ on $E_S$, it follows that $s(a)=p$, which means that $(x,x)\in s(a)$.  
Therefore, $(x,a)\in\supp(\gamma)$.

(c)$\implies$(d):
As $\based{\gamma}$ is a $0,1$-diagonal matrix,
\begin{equation}
\gamma'\circ(\phi_0\otimes\phi_1)
\geq
\gamma'\circ(\phi_0\otimes\phi_1)\otimes\based{\gamma}.
\end{equation}
To prove the reverse inequality, note that
for $(x,a)\in X\otimes S$ and $y'\in X'$, 
\begin{equation}
(\gamma'\circ(\phi_0\otimes\phi_1))_{xa}^{y'}=
\begin{cases}\label{eq:gammaprime}
1 & (\phi_0(x),y')\in\phi_1(a)\\
0 & \text {otherwise.}
\end{cases}
\end{equation}
By (c), $(\phi_0(x),y')\in\phi_1(a)$ implies that $(x,a)\in\supp(\gamma)$, which is
equivalent to $(\based\gamma)_{xa}^{xa}=1$ and the rest is trivial.

(d)$\implies$(a):
Pasting of \eqref{diag:charstrong} and \eqref{diag:admorphism} yields the diagram
\eqref{diag:strongadmorphism}.
\end{proof}
\begin{corollary}
Let $(\phi_0,\phi_1,\gamma)\colon(X,S)\to(X',S',\gamma')$ be a strongly admissible morphism
of coherent configurations. Let $\widehat{\phi_1}$ be as in \Cref{prop:phionehat}. Then  the diagram
\[
\begin{diagram}
X\otimes S
    \ar[r]^{\gamma}
    \ar[d]_{\phi_0\otimes\widehat{\phi_1}}
&
X
    \ar[d]^{\phi_0}
\\
X'\otimes S'
    \ar[r]_{\gamma'}
&
X'
\end{diagram}
\]
commutes.
\end{corollary}
\begin{proof}
This follows by pasting of the diagrams \eqref{diag:charstrong} and
\eqref{diag:modulemorph}.
\end{proof}
\begin{corollary}\label{coro:defsCoincide}
Every admissible morphism from an association scheme $(X,S)$ is strongly admissible.
\end{corollary}
\begin{proof}
If $(X,S)$ is an association scheme, then $|E_S|=1$, the rest follows by (b) of
\Cref{lemma:charstradmissible}.
\end{proof}
\begin{example}\label{ex:tau}
Let $(X,S)$ be a coherent configuration. Consider the discrete coherent configuration
on the set of all unit colors, $\mathcal D_{E_S}$ with vertices
$E_S$ and colors given by all singleton relations on $E_S$ (which we identify with
$E_S\times E_S$).
Consider a pair of maps
$(\tau_{X,S})_0\colon X\to E_S$ and 
$(\tau_{X,S})_1\colon S\to E_S\times E_S$, 
where $(\tau_{X,S})_0(x)$ is the loop color on the vertex $x$
and $(\tau_{X,S})_1(a)=(s(a),t(a))$. Then $((\tau_{X,S})_0,(\tau_{X,S})_1)$ is a
strongly admissible morphism.
\end{example}
\begin{example}
Following \Cref{ex:terminalMorph}, consider the admissible morphism 
$!_{(X,S)}$ to the terminal object. By \Cref{lemma:charstradmissible} (b), $!_{(X,S)}$ is strongly
admissible if and only if $(X,S)$ is an association scheme.
\end{example}
\begin{example}
By \Cref{lemma:charstradmissible} (b), the projection morphism
$(\pi_0,\pi_1)\colon(X,S)\otimes(X',S')\rightarrow (X,S)$ in \Cref{ex:tensorProduct}
is strongly admissible if and only if $(X',S')$ is an association scheme.
\end{example}
The identity~\eqref{eq:semHom} from the following lemma was proved for association schemes in~\cite[Lemma 6.3]{french2013functors}. In the case of coherent configurations, one needs to be more careful: equality holds for strongly admissible ones, while for general admissible morphisms, it holds only in a certain lax form. 
\begin{lemma}\label{lemma:semHom}
    Let $(\phi_0,\phi_1)\colon (X,S)\rightarrow (X',S')$ be a strongly admissible morphism of coherent configurations, $a,b\in S$, and $c'\in S'$. 
Then $\widehat{\phi_1}$ is a semigroup morphism $(S,\nabla)\rightarrow
(S',\nabla)$, meaning that for all $a,b\in S$ and $c'\in S'$,
    \begin{equation}\label{eq:semHom}
    \sum_{\phi_1(c)=c'}\nabla_{ab}^c\frac{\val{c}}{\val{\phi_1(c)}}=\nabla^{c'}_{\phi_1(a),\phi_1(b)}\frac{\val{a}}{\val{\phi_1(a)}}\frac{\val{b}}{\val{\phi_1(b)}}.
\end{equation}
In the case of a general admissible morphism, either the equality in~\eqref{eq:semHom} holds or the left-hand side equals zero. 
\end{lemma}
\begin{proof}
First, note that if $a,b,c$ are such that $\nabla_{ab}^c>0$ then we must have
$\nabla_{\phi_1(a)\phi_1(b)}^{\phi_1(c)}>0$ for any morphism, so that if the right-hand side is zero, then
the left-hand side is zero as well. 

Consider the case $t(a)\not=s(b)$. Then by \Cref{lemma:units}, $\nabla_{ab}^c=0$ for all $c\in S$,
hence the left-hand side is zero. If
$(\phi_0,\phi_1)$ is strongly admissible, $t(\phi_1(a))\not=s(\phi_1(b))$ due to \Cref{lemma:charstradmissible}, and the
right-hand side vanishes as well. Moreover, if $s(\phi_1(a))\not=s(c')$ or
$t(\phi_1(b))\not=t(c')$, then the right-hand side of~\eqref{eq:semHom} vanishes,
so the left-hand side must vanish as well.

For the rest of the proof we assume $t(a)=s(b)$, $s(\phi_1(a))=s(c')$, and
$t(\phi_1(b))=t(c')$. In particular, there is some vertex $x'\in\mathrm{Im}(\phi_0)$
of color $s(c')=s(\phi_1(a))$. We will show that in this case the equality
in~\eqref{eq:semHom} holds also for admissible $(\phi_0,\phi_1)$.  

    Let us fix any $(x',z')\in c'$ and $x\in X$, such that $\phi_0(x)=x'$ and $x$ has color $s(a)$ (we can find such vertices by the aforementioned assumptions). We will count in two ways the number of $(y,z)\in X^2$, satisfying $(x,y)\in a$, $(y,z)\in b$, and $\phi_0(z)=z'$. 
First, there are $\nabla^{c'}_{\phi_1(a),\phi_1(b)}$ ways to choose possible $y'=\phi_0(y)$. Since $(\phi_0,\phi_1)$ is admissible, for any chosen $y$, we have $\valphi{a}$ possibilities to find $y$, and then  $\valphi{b}$ possibilities to find $z$. This gives us the right-hand side.

For the other direction we may proceed as follows. 
We first find $z$ and discuss the color $c$ of $(x,z)$. Depending on color $c$,
we can choose $z$ in $\valphi{c}$ ways, due to $(\phi_0,\phi_1)$ being admissible.
After choosing $z$, we have $\nabla_{ab}^c$ possibilities to find a suitable
$y$. This counting gives us the left-hand side. Note that the left-hand side sum 
can be restricted to those $c$, for which $s(c)$ is the color of $x$, since the
corresponding $\nabla$ entry is zero otherwise.
\end{proof}



\section{Dagger Frobenius structures}

Examples \ref{ex:groupoids} and \ref{ex:hStarAlgebras} show that 
dagger Frobenius structures are important and interesting in the categories
$\FHilb$ and $\Rel$. This leads to a natural question: are the Frobenius
structures that we have associated to coherent configurations \emph{dagger} Frobenius?
It is very easy to see that, in general, the answer to this question is negative. Indeed, the
matrix $\nabla$ consists only of natural numbers, whereas $\Delta$ may contain
non-natural rational numbers, whenever there is a triple of colors
$a,b,c$ such that $\val c$ is not a divisor of $\nabla_{ac^{-1}}^b$. Clearly,
whenever this happens, $\Delta\neq\nabla^\dag$.
However, as we will prove in this section, there is another Frobenius structure
on the set of all colors, that happens to be  dagger Frobenius.
The ``daggerization'' can be understood as a ``normalization'' of
structure constants via valencies,
converting the Frobenius structure to a dagger Frobenius structure. 
As we will demonstrate in the following sections, the dagger variant has more
convenient properties, especially in connection with admissible morphisms.

Let $(X,S)$ be a coherent configuration. For each $a,b,c\in S$ we denote 
\begin{equation}\label{eq:daggerNabla}
    \dnabla{a}{b}{c}=\ddnabla{a}{b}{c}.
\end{equation}

We will prove in this section that $(S,\Dnabla,e)$ is a dagger Frobenius
structure. We will refer to it as \emph{dagger Frobenius structure associated
with $(X,S)$}.

\begin{lemma}\label{lem:rotations}
    Let $(X,S)$ be a coherent configuration and $a,b,c\in S$. Then
    \begin{align}
        \dnabla{a}{b}{c}&=\dnabla{b^{-1}}{a^{-1}}{c^{-1}},\label{eq:dagOp}\\
        \dnabla{a}{b}{c} &= \sqrt{\frac{\val{b^{-1}}}{\val{b}}}\dnabla{b}{c^{-1}}{a^{-1}},\label{eq:dagRotR}\\
        \dnabla{a}{b}{c} &= \sqrt{\frac{\val{a}}{\val{a^{-1}}}}\dnabla{c^{-1}}{a}{b^{-1}}.\label{eq:dagRotL}
    \end{align}
    In the case $(X,S)$ is $\frac{1}{2}$-homogeneous, we have $\dnabla{a}{b}{c}=\dnabla{b}{c^{-1}}{a^{-1}}=\dnabla{c^{-1}}{a^{-1}}{b^{-1}}$.
\end{lemma}
\begin{proof}
To prove \eqref{eq:dagOp}, observe that 
$\nabla_{ab}^{c}=\nabla_{b^{-1}a^{-1}}^{c^{-1}}$. Note that 
$\dnabla{a}{b}{c}$ is nonzero if and only if $\nabla_{ab}^c$ is nonzero, so we may 
assume that $\nabla_{ab}^c$ is nonzero.
Then, by \Cref{coro:nakayama},
\[
\frac{\val{a}}{\val{a^{-1}}}.
\frac{\val{b}}{\val{b^{-1}}}=
\frac{\val{c}}{\val{c^{-1}}},
\]
which implies
\begin{equation}\label{eq:tripRot}
    \frac{\val{c}}{\val{a}\val{b}}=\frac{\val{c^{-1}}}{\val{a^{-1}}\val{b^{-1}}},
\end{equation}
and this clearly implies~\eqref{eq:dagOp}.
The proof of \eqref{eq:dagRotR} is as follows:
    \begin{align*}
        \dnabla{a}{b}{c}&=\ddnabla{a}{b}{c}=\frac{\val{c}}{\sqrt{\val{a}\val{b}\val{c}}}\nabla_{a b}^c=\frac{\val{a}}{\sqrt{\val{a}\val{b}\val{c}}}\nabla_{c b^{-1}}^a\\
        &=\sqrt{\frac{\val{b^{-1}}}{\val{b}}}\sqrt{\frac{\val{a}}{\val{c}\val{b^{-1}}}}\nabla^{a}_{cb^{-1}}=\sqrt{\frac{\val{b^{-1}}}{\val{b}}}\dnabla{c}{b^{-1}}{a}=\sqrt{\frac{\val{b^{-1}}}{\val{b}}}\dnabla{b}{c^{-1}}{a^{-1}}.
    \end{align*}
The third equality follows by the double counting principle: the terms $\val c\nabla_{ab}^{c}$ and $\val
a\nabla_{cb^{-1}}^a$ are either both zero, or they count the number of triangles
\[
\xymatrix{
~
&
\cdot
    \ar[rd]^b
\\
x
    \ar[rr]_c
    \ar[ur]^a
&
~
&
\cdot
}
\]
for any vertex $x$, for which at least one such a triangle exists.
    
Equation~\eqref{eq:dagRotL} follows from~\eqref{eq:dagRotR} by first
changing the variables $a\mapsto b^{-1}$, $b\mapsto a^{-1}$, $c\mapsto c^{-1}$ and
then applying \eqref{eq:dagOp} to both sides.
\end{proof}
The ``normalization'' of the structure constants by \eqref{eq:daggerNabla} preserves
the monoid laws, as the following lemma shows.
\begin{lemma}\label{lem:daggerMonoid}
For a coherent configuration $(X,S)$, the triple $(S,\Dnabla,e)$ is a monoid in
$\MatR$.
\end{lemma}
\begin{proof} Recall that by Theorem~\ref{thm:monAction} $(S,\nabla,e)$ is a monoid.
    First, we prove the associativity law. For each quadruple of colors $a,b,c,d\in S$, we have:
    \begin{align*}
        \sum_f\dnabla{a}{b}{f}\dnabla{f}{c}{d}&=
        \sqrt{\frac{\val{d}}{\val{a}\cdot\val{b}\cdot\val{c}}}\sum_{f\in S}\nabla_{ab}^f\nabla_{fc}^d\\
        &=\sqrt{\frac{\val{d}}{\val{a}\cdot\val{b}\cdot\val{c}}}\sum_{f\in S}\nabla_{af}^d\nabla_{bc}^f=\sum_{f\in S}\dnabla{a}{f}{d}\dnabla{b}{c}{f}.
    \end{align*}
    For $a,b\in S$, we verify the unit rule $\Dnabla\circ (e\otimes \mathrm{id})=\lambda$:
    \begin{equation*}
        \sum_{u\in E_s}\dnabla{u}{a}{b}=\sum_{u\in E_s}\sqrt{\frac{\val{b}}{\val{u}\cdot\val{a}}}\nabla_{ua}^b=\sqrt{\frac{\val{b}}{\val{a}}}\sum_{u\in E_S}\nabla_{ua}^b=\sqrt{\frac{\val{b}}{\val{a}}}\delta_a^b=\delta_a^b.
    \end{equation*}
    We can show that $\Dnabla\circ (\mathrm{id}\otimes e)=\rho$ analogously.
\end{proof}

\begin{theorem}\label{thm:daggerFrob}
    Let $(X,S)$ be a coherent configuration. Then $(S;\Dnabla,e)$ is a dagger Frobenius structure in $\MatR$.
\end{theorem}
\begin{proof}
    It is enough to verify Frobenius identity $$(\Dnabla\otimes\mathrm{id})\circ(\mathrm{id}\otimes\Dnabla^\dagger)= (\mathrm{id}\otimes \Dnabla)\circ(\Dnabla^\dagger\otimes \mathrm{id}).$$
    For each $a, b, c, d\in S$, using formulas from Lemma~\ref{lem:rotations}, we derive:
    \begin{align*}
        \sum_{f\in S} \dnabla{f}{d}{b}\dnabla{a}{f}{c}&=\sum_{f\in S}\sqrt{\frac{\val{f}}{\val{f^{-1}}}}\dnabla{b^{-1}}{f}{d^{-1}}\sqrt{\frac{\val{f^{-1}}}{\val{f}}}\dnabla{f}{c^{-1}}{a^{-1}}=\sum_{f\in S}\dnabla{b^{-1}}{f}{d^{-1}}\dnabla{f}{c^{-1}}{a^{-1}}\\
        &=\sum_{f\in S}\dnabla{f^{-1}}{b}{d}\dnabla{c}{f^{-1}}{a}=\sum_{f\in S}\dnabla{f}{b}{d}\dnabla{c}{f}{a}.
    \end{align*}
\end{proof}
\begin{corollary}\label{coro:thinIsDaggerSpecial}
For a thin coherent configuration, $(S,\nabla,e)$ is a special dagger Frobenius structure
in $\MatR$.
\end{corollary}
\begin{proof}
If $(X,S)$ is thin, then $\Dnabla=\nabla$. By \Cref{thm:connections}, the Frobenius structure is special.
\end{proof}
By the main result of \cite{heunen2013relative}, see also \cite[Theorem
5.4.1]{heunen2019categories}, special dagger Frobenius structures in
$\Rel$ are in a one-to-one correspondence with groupoids. In this correspondence, the
underlying set of the Frobenius structure is the set of all morphisms of a groupoid,
the multiplication $\nabla$ is essentially the composition $\circ$ and the unit
relation $e$ selects the identity morphisms.

In \cite{hanaki2015thin}, the authors proved that every thin coherent configuration
$(X,S)$ can be presented as a groupoid with the set of objects $E_S$ and the set of
morphisms $S$. \Cref{coro:thinIsDaggerSpecial} provides an alternative proof of this fact.
\begin{corollary}\label{coro:thinCCIsGroupoid}
If $(X,S)$ is thin, then the monoid $(R(S),R(\nabla),R(e))$ in $\Rel$ is a groupoid.
\end{corollary}

Another consequence of the existence of dagger Frobenius structure is the following theorem.
\begin{theorem}\label{coro:hstar}
For every $\frac{1}{2}$-homogeneous coherent configuration $(X,S)$,
the algebra $(L(X),L(\Dnabla),L(e))$ in $\FHilb$ is an $H^*$-algebra.
\end{theorem}
\begin{proof}
By \Cref{thm:connections}, $(X,S)$ is $\frac{1}{2}$-homogeneous if and only if
the associated Frobenius structure is symmetric. Let us prove that
the associated \emph{dagger} Frobenius structure is symmetric, as well.
Indeed, similarly as in the proof of \Cref{thm:connections}, we obtain
after some simple algebraic manipulations,
\[
(e^\dag\circ\nabla\circ\sigma_{S,S})_{ab}^*=
\sum_{p\in{E_S}}\Dnabla_{ba}^p=\\
\sum_{p\in{E_S}}\sqrt{\frac{1}{\val b\val a}}\nabla_{ba}^p=
\begin{cases}
\sqrt{\frac{\val b}{\val a}}&b=a^{-1}\\
0&\text{otherwise}
\end{cases}
\]
and
\[
(e^\dag\circ\nabla)_{ab}^*=
\sum_{p\in{E_S}}\Dnabla_{ab}^p=\\
\sum_{p\in{E_S}}\sqrt{\frac{1}{\val a\val b}}\nabla_{ab}^p=
\begin{cases}
\sqrt{\frac{\val a}{\val b}}&b=a^{-1}\\
0&\text{otherwise}
\end{cases}
\]
and we see that the symmetry of the dagger Frobenius structure associated
with $(X,S)$ is equivalent to the $\frac{1}{2}$-homogeneity of $(X,S)$. Therefore,
$L(S)$ is a dagger symmetric Frobenius structure in $\FHilb$ and those 
are exactly $H^*$-algebras by \cite[Theorem 5.32]{heunen2019categories}.
\end{proof}

The $H^*$-algebra in the above theorem is actually obtained from the \emph{adjacency
algebra} of $(X,S)$, see \cite{chen2024lectures}. This is a subalgebra $\mathcal A$ of the algebra $M_n(\mathbb
C)$ of complex $n\times n$-matrices, where $n=|X|$. $\mathcal A$ has a linear basis
$\{M_a,\ a\in S\}$ labeled by the colors, such that  $M_a^*=M_a^{-1}$, $\sum_{p\in E_S}
M_s=I_n$,
and $\nabla_{ab}^c$ are the structural constants of $\mathcal A$ with respect to this
basis:
\[
M_aM_b=\sum_c \nabla_{ab}^c M_c.
\]
Introducing the inner product $(A,B)\mapsto \mathrm{Tr}\,[A^*B]$ in $\mathcal A$ endows
it with an $H^*$-algebra structure (see \cite[Sec. 5.4]{heunen2019categories}). Moreover,
one can show that 
$\{\bar M_a:= |a|^{-1/2}M_a\}$ is an orthonormal basis of $\mathcal A$. The structural
constants with respect to this new basis have the form:
\[
\sqrt{\frac{|c|}{|a||b|}} \nabla_{ab}^c=\sqrt{\frac{|s(c)|\val c}{|s(a)|\val a|s(b)|\val
b}}\nabla_{ab}^c=k \Dnabla_{ab}^c,
\]
where $k=|p|^{-1/2}$ for $p\in E_S$ (remember that $(X,S)$ is assumed to be
$\frac12$-homogeneous, so that all unit colors have the same cardinality). This shows
that, up to multiplication by a constant, the symmetric dagger Frobenius structure in $\FHilb$
$(L(X),L(\Dnabla), L(e))$ coincides with the structure of the $H^*$-algebra $\mathcal A$.


\section{Admissible morphisms and the dagger Frobenius structures}

A strongly admissible morphism induces a mapping of Frobenius structures which
preserves multiplication (\Cref{lemma:semHom}). In this section, we observe that
if we pass to the dagger variant, the comultiplication structure is to some extent
preserved as well. Note that one should not seek for a notion of morphisms which
preserve all the operations of Frobenius structures, since these are necessarily
isomorphisms as proved in~\cite[Lemma 5.11]{heunen2019categories}.

\begin{lemma}\label{lemma:dagSemHom}  
A strongly admissible morphism $(\phi_0,\phi_1)\colon (X,S)\to(X',S')$ 
induces a semigroup morphism $\pmb\phi$ of associated dagger Frobenius structures,
given by the rule
\begin{equation}\label{eq:daggerAdmissibleEntries}
    \pmb\phi_a^{b'}=\begin{cases}
        \sqrt{\valphi{a}} & \text{if\ } \phi_1(a)=b',\\
        0 & \text{otherwise}.
    \end{cases}
\end{equation}
\end{lemma}
\begin{proof}
Multiplying~\eqref{eq:semHom} by $\frac{\sqrt{\|c'\|}}{\sqrt{\|a\|}\sqrt{\|b\|}}$ results in
\begin{align*}
    &\sum_{{\phi_1}(c)=c'}\frac{\sqrt{\val{c}}}{\sqrt{\val{a}}\sqrt{\val{b}}}\nabla_{ab}^c\frac{\sqrt{\val{c}}}{\sqrt{\val{{\phi_1}(c)}}}=\\
    =&\frac{\sqrt{\val{c'}}}{\sqrt{\val{{\phi_1}(a)}}\sqrt{\val{{\phi_1}(b)}}}\nabla^{c'}_{{\phi_1}(a){\phi_1}(b)}\frac{\sqrt{\val{a}}}{\sqrt{\val{{\phi_1}(a)}}}\frac{\sqrt{\val{b}}}{\sqrt{\val{{\phi_1}(b)}}}.
\end{align*}
This gives us
\begin{equation}\label{eq:dMonHom}
\sum_{\phi_1(c)=c'}\dnabla{a}{b}{c}\sqrt{\valphi{c}}=\dnabla{\phi_1(a)}{\phi_1(b)}{c'}\sqrt{\valphi{a}}\sqrt{\valphi{b}}.
\end{equation}
\end{proof}

In the remainder of the paper, whenever $(\phi_0,\phi_1)$ is a strongly admissible
morphism of coherent configurations, $\pmb\phi$ will denote the semigroup
morphism of associated dagger Frobenius structures, as in \Cref{lemma:dagSemHom}. 

In the case of association schemes, the morphism $\pmb\phi$ from the previous lemma is even a monoid morphism. This may happen also for coherent configurations, as the next corollary shows.
\begin{corollary}\label{cor:daggerMonHom}
Let $(\phi_0,\phi_1)\colon(X,S)\rightarrow (X',S')$ be a strongly admissible
morphism. Then $\pmb\phi\colon (S,\Dnabla,e)\rightarrow (S',\Dnabla',e')$ is a
morphism of monoids if and only if  $E_{S'}\subseteq \mathrm{Im}(\phi_1)$.
\end{corollary}
\begin{proof}
By Lemma~\ref{lemma:dagSemHom}, we already know that $\pmb\phi$ preserves multiplication. Therefore, it is enough to focus on the preservation of the unit. 
First, assume $\pmb\phi$ is a monoid morphism. For each $u'\in E_{S'}$, we have
$1={(e')}^{u'}_*=(\pmb\phi\circ e)^{u'}_*$. This could happen only if $u'$ is in the image of $\phi_1$.

    Next, consider the other implication. By the assumptions, the mapping $\phi_1$
    restricts to a bijection between $E_S$ and $E_{S'}$. Moreover, for each $u\in
    E_S$, $\valphi{u}=1$. It follows that \[
    (\pmb\phi\circ e)_*^{b'}=\sum_{u\in
    E_S}\pmb\phi^{b'}_u=\sum_{u\in E_S}\delta^{b'}_{\phi_1(u)}=\sum_{u'\in
    E_{S'}}\delta^{b'}_{u'}={(e')}_*^{b'}.\]
\end{proof}

Let $(\phi_0,\phi_1)\colon(X,S)\to(X',S')$ be a strongly admissible morphism of
coherent configurations. For every $p'\in E_{S'}$ and $(x',x')\in p'$,
the number $|\phi_0^{-1}(x')|$ does not depend on the choice of $x'$. Indeed,
it is easy to see that
\begin{equation}\label{eq:kp}
|\phi_0^{-1}(x')|=\sum_{\substack{a\in S\\ \phi_1(a)=p'}}\valphi{a}.
\end{equation}
In what follows, we denote the quantity in \eqref{eq:kp} by $k_{p'}$.
\begin{definition}\label{def:ddager}
    For a strongly admissible morphism $(\phi_0,\phi_1)\colon(X,S)\to(X',S')$ 
    the matrix
    $\phi^\ddagger\colon S'\to S$ is given by the rule
    \begin{equation}\label{eq:daggerHom}
    (\phi^\ddagger)_{b'}^a=
        \frac{1}{k_{t(b')}}\pmb\phi_a^{b'},\qquad a\in S,\ b'\in S'.
\end{equation}
\end{definition}

It immediately follows from~\eqref{eq:daggerHom} that we have the following equations
\begin{equation}\label{eq:diagonalK}
    \pmb\phi^\dagger=\phi^\ddagger\circ k_{\phi} \quad\text{and}\quad\phi^\ddagger=\pmb\phi^\dagger\circ k^{-1}_{\phi}
\end{equation}
where $k_\phi\colon S'\to S'$ is a diagonal matrix having $k_{t(b')}$'s on the diagonal. 

Note that if $(\phi_0,\phi_1)$ is an admissible morphism between schemes, then $k_\phi$ is a constant diagonal matrix, hence
$(\phi^\ddagger)_{b'}^a=\frac{1}{k}\cdot\pmb\phi^{b'}_a$ for some integer $k\geq 1$.
\begin{proposition}\label{prop:semigroupHom}
Let $(\phi_0,\phi_1)\colon (X,S)\rightarrow (X',S')$ be a strongly admissible
morphism of coherent configurations, $a',b'\in S'$, and $c\in S$.
If $\phi_1$ is surjective, we have
\begin{equation}\label{eq:cosemHom}
    \sum_{p,q\in S}\dnabla{p}{q}{c}\cdot(\phi^\ddagger)^p_{a'}\cdot
(\phi^\ddagger)_{b'}^q= (\phi^\ddagger)_{\phi_1(c)}^{c}\cdot\dnabla{a'}{b'}{\phi_1(c)}.
\end{equation}
In this case $\phi^\ddagger\colon (S',\Dnabla)\rightarrow (S,\Dnabla)$ is a semigroup morphism.
    
    If $\phi_1$ is not surjective, equality in~\eqref{eq:cosemHom} holds whenever the left-hand side is non-zero.
\end{proposition}
\begin{proof}
    First, if  $a'$ or $b'$ is not in the image of $\phi_1$, the left-hand side is zero, while the right-hand side may be 
    non-zero. Further, if $\Dnabla_{a'b'}^{\phi_1(c)}=0$, then both sides vanish. For the
rest of the proof we assume that $\Dnabla_{a'b'}^{\phi_1(c)}\neq0$.

    We will prove by double-counting the following equation
    \begin{equation}\label{eq:help1}
\sum_{\substack{\phi_1(p)=a'\\\phi_1(q)=b'}}\nabla^c_{p q}=k_{t(a')}\cdot \nabla^{\phi_1(c)}_{a' b'}.
    \end{equation}
    For a fixed edge $(x,y)\in c$, the left-hand side equals the number of $z\in X$,
such that $(\phi_0(x),\phi_0(z))\in a'$ and $(\phi_0(z),\phi_0(y))\in b'$. Let us
count the same number in a different way. First, there are
$\nabla^{\phi_1(c)}_{a'b'}$ choices of an element $z'\in X'$ such that
$(\phi_0(x),z')\in a'$ and $(z',\phi_0(y))\in b'$.  Due to strong admissibility and the assumption
$a'\in\mathrm{Im}(\phi_1)$, each chosen $z'$ belongs to $\mathrm{Im}(\phi_0)$ and
there are $k_{t(a')}$ elements in $\phi_0^{-1}(z')$. This yields the right-hand side.

    Let us finally prove the equation~\eqref{eq:cosemHom}:
    \begin{multline*}
        \sum_{p,q\in S}\dnabla{p}{q}{c}\cdot(\phi^\ddagger)^p_{a'}\cdot
(\phi^\ddagger)_{b'}^q=\sum_{\substack{\phi_1(p)=a'\\ \phi_1(q)=b'}}\ddnabla{p}{q}{c}\sqrt{\frac{\val{p}}{\val{a'}}}\sqrt{\frac{\val{q}}{\val{b'}}}\frac{1}{k_{t(a')}k_{t(b')}}\\
        =\sqrt{\frac{\val{c}}{\val{a'}\val{b'}}}
\frac{1}{k_{t(a')}k_{t(b')}}\sum_{\substack{\phi_1(p)=a' \\ \phi_1(q)=b'}}\nabla_{pq}^{c}\\
        \overset{\eqref{eq:help1}}{=}\sqrt{\frac{\val{c}}{\val{a'}\val{b'}}}\nabla^{\phi_1(c)}_{a'b'}\frac{1}{k_{t(b')}}
        =\sqrt{\frac{\val{\phi_1(c)}}{\val{a'}\val{b'}}}\nabla^{\phi_1(c)}_{a'b'}\sqrt{\frac{\val{c}}{\val{\phi_1(c)}}}\frac{1}{k_{t(b')}}=(\phi^\ddagger)_{\phi_1(c)}^{c}\cdot\dnabla{a'}{b'}{\phi_1(c)}.
    \end{multline*}
    Note that in the last step we use that $t(b')=t(\phi_1(c))$, which follows from
the assumption $\dnabla{a'}{b'}{\phi_1(c)}>0$ and \Cref{lemma:sourcesTargets}.
\end{proof}
\begin{corollary}
    Let $(\phi_0,\phi_1)\colon (X,S)\rightarrow (X',S')$ be a strongly
admissible morphism with surjective $\phi_1$. Then $k_\phi^{-1}\circ\pmb\phi\colon
(S,\Dnabla^\dagger)\rightarrow(S',\Dnabla^\dagger)$ is a morphism of cosemigroups.
\end{corollary}
\begin{proof}
    Note that $k^{-1}_\phi\circ\pmb\phi=(\phi^\ddagger)^\dagger$. Hence the statement
$(\phi^\ddagger)^\dagger$ preserves comultiplication $\Dnabla^\dagger$ is equivalent
to the statement $\phi^\ddagger$ preserves multiplication $\Dnabla$ and this is Proposition~\ref{prop:semigroupHom}.
\end{proof}
\begin{lemma}\label{lem:ddagger}
Let $(\phi_0,\phi_1)\colon (X,S)\rightarrow (X',S')$ be a strongly admissible morphism. Then $\Dnabla\circ (\id\otimes
k_\phi)=k_\phi\circ \Dnabla$.
\end{lemma}
\begin{proof}
    For each $a',b',c'\in S'$, we have $\dnabla{a'}{b'}{c'}\cdot
k_{t(b')}=\dnabla{a'}{b'}{c'}\cdot k_{t(c')}$, since $t(b')=t(c')$ whenever the
$\nabla$ entry is nonzero.
\end{proof}
\begin{corollary}\label{cor:ddagger}
    Let $(\phi_0,\phi_1)\colon (X,S)\rightarrow (X',S')$ be a strongly admissible
morphism. If $\phi_1$ is surjective,
then $\Dnabla\circ (\pmb\phi^\dagger\otimes\pmb\phi^\dagger)=
\pmb\phi^\dagger\circ\Dnabla\circ(k_\phi\otimes\id)$.
\end{corollary}
\begin{proof}
We will prove the equality using string diagrams as follows. In the computation, the first step is by~\eqref{eq:diagonalK}, the second step follows from Proposition~\ref{prop:semigroupHom}, and the last step is by Lemma~\ref{lem:ddagger} and~\eqref{eq:diagonalK}:
\begin{center}
\vcbox{\includegraphics{dominik_stringDdagger1.pdf}}
\vcbox{$=$\hspace*{0.7em}}
\vcbox{\includegraphics{dominik_stringDdagger2.pdf}}
\vcbox{\hspace*{0.7em}$=$}
\vcbox{\includegraphics{dominik_stringDdagger3.pdf}}
\vcbox{$=$}
\vcbox{\includegraphics{dominik_stringDdagger4.pdf}}
\end{center}
\end{proof}
We can express the image and the kernel of a strongly admissible morphism in terms of $\phi^\ddagger$.
\begin{lemma}\label{lem:projections}
Let $(\phi_0,\phi_1)\colon (X,S)\rightarrow (X',S')$ be a strongly admissible morphism of coherent configurations. Then
both $\pmb\phi\circ\phi^\ddagger\colon (S',\Dnabla_{S'})\rightarrow (S',\Dnabla_{S'})$ and $\phi^\ddagger\circ\pmb\phi\colon
(S,\Dnabla_S)\rightarrow (S,\Dnabla_S)$ are idempotent endomorphisms of semigroups. Moreover, for each $a',b'\in S'$ we
have
    \begin{equation}\label{eq:im}
        (\pmb\phi\circ\phi^\ddagger)_{a'}^{b'}=\begin{cases}
        1 & \text{if\ } a'=b'\in\mathrm{Im}(\phi_1),\\
        0 & \text{otherwise}
    \end{cases}
    \end{equation}
    and for each $a,b\in S$:
    \begin{equation}\label{eq:ker}
        (\phi^\ddagger\circ\pmb\phi)_a^b=\begin{cases}
        \frac{1}{k_{t(\phi_1(a))}}\sqrt{\valphi{a}\valphi{b}} & \text{if\ } \phi_1(a)=\phi_1(b),\\
        0 & \text{otherwise}.
    \end{cases}
    \end{equation}
\end{lemma}
\begin{proof}
    We compute \begin{align*}
        (\pmb\phi\circ\phi^\ddagger)_{a'}^{b'}&=\sum_{p\in
        S,a'=\phi_1(p)=b'}\sqrt{\valphi{p}}\sqrt{\valphi{p}}\frac{1}{k_{t(b')}}
        =\delta_{a'}^{b'}\frac{1}{k_{t(a')}}\sum_{p\in S,\phi_1(p)=a'}\valphi{p}\\
        &=\begin{cases}
        \frac{1}{k_{t(a')}}k_{t(a')}=1 & \text{if\ } a'=b'\in\mathrm{Im}(\phi_1),\\
        0 & \text{otherwise}.
    \end{cases}
    \end{align*}
The equation~\eqref{eq:ker} is immediate by definition. Similarly, $\pmb\phi\circ\phi^\ddagger$ is obviously idempotent,
and from $\pmb\phi\circ\phi^\ddagger\circ\pmb\phi=\pmb\phi$ it easily follows that $\phi^\ddagger\circ\pmb\phi$ is an idempotent as well.   
\end{proof}
\begin{theorem}\label{thm:projOnCoimage}
    Let $(\phi_0,\phi_1)\colon (X,S)\rightarrow (X',S')$ be a strongly admissible morphism. Then
    $\phi^\ddagger\circ\pmb\phi$ is a semigroup endomorphism of $(S,\Dnabla)$ and a
    cosemigroup endomorphism $(S,\Dnabla^\dagger)$.
    %considered as a morphism $S\rightarrow S$ in $\MatR$ preserves both multiplication $\Dnabla$ and comultiplication $\Dnabla^\dagger $, that is 
    %\begin{align}
     %      \Dnabla\circ((\phi^\ddagger\circ\pmb\phi)\otimes(\phi^\ddagger\circ\pmb\phi))&=(\phi^\ddagger\circ\pmb\phi)\circ\Dnabla,\\
      %     ((\phi^\ddagger\circ\pmb\phi)\otimes(\phi^\ddagger\circ\pmb\phi))\circ\Dnabla^\dagger &=\Dnabla^\dagger \circ
       %    (\phi^\ddagger\circ\pmb\phi).
    %\end{align}
\end{theorem}
\begin{proof}
    Since $\phi^\ddagger\circ\pmb\phi$ is  self-adjoint by \eqref{eq:diagonalK}, it is
    enough to prove the first equality and get the second one by applying the dagger.
    However, the first equality is proved in Lemma~\ref{lem:projections}. 
\end{proof}
%\begin{lemma}
 %    Let $\phi\colon (X,S)\rightarrow (Y,T)$ be an admissible morphism of coherent configurations seen as a morphism of monoids $\phi\colon (S;\Dnabla^S,e^S)\rightarrow (T;\Dnabla^T,e^T)$. Then $\phi\circ e^S$ is an idempotent with respect to $\Dnabla^T$ and $\phi^\ddagger\circ e^S$ is an idempotent with respect to $\Dnabla^S$.
%\end{lemma}

\section{The valency vector}
Let $(X,S)$ be a coherent configuration and let $\mathcal{F}=(S,\Dnabla,e)$ be the
corresponding dagger Frobenius structure. Even though the dagger-structure constants
have better properties in computations (in particular when dealing with
admissible morphisms), their combinatorial meaning is not clear. It is a
natural question of whether we can reconstruct the original structure constants
from $\Dnabla$. In this section, we give an affirmative answer. Obviously,
knowing $\Dnabla$ and valencies of colors is sufficient to express $\nabla$. 
In this section, we are going to show how to extract the information
about valencies from $\Dnabla$ in the form of the \emph{valency vector} $\mathbf v\colon I\to S$ given by
the rule
\[
    {\mathbf v}_*^a=\sqrt{\val a}
\]
for all $a\in S$.
For each pair of "composable" colors $a,b$ ($t(a)=s(b)$), the corresponding valencies fit in the equation
\begin{equation}\label{eq:cop}
    \val{a}\val{b}=\sum_c \nabla^{c}_{ab}\val{c}. 
\end{equation}
This follows from an easy calculation, where the right-hand side is obtained by counting the number of two consecutive edges of color $a$ and $b$ starting at a given vertex. 

In the dagger Frobenius structure associated to a coherent configuration, ~\eqref{eq:cop} changes to
\begin{multline}\label{eq:copp}
    \sqrt{\val{a}\val{b}}=\frac{1}{\sqrt{\val{a}\val{b}}}\val{a}\val{b}=\frac{1}{\sqrt{\val{a}\val{b}}}\sum_c\nabla^c_{a b}\val{c}\\
    =\sum_c\sqrt{\frac{\val{c}}{\val{a}\val{b}}}\nabla^c_{a b}\sqrt{\val{c}}=\sum_c\Dnabla^c_{a b}\sqrt{\val{c}}.
\end{multline}
In the case of an association scheme, equation~\eqref{eq:copp} holds for each pair of colors $a$ and $b$, hence the
valency vector $\bf v$ is a \emph{copyable state}, \cite[Definition 4.23]{heunen2019categories}.

We can express valency vectors and gain insight into their properties
using the results from the previous section about strongly
admissible morphisms. 
Consider for some coherent configuration $(X,S)$ the admissible morphism
$!\colon(X,S)\rightarrow (I,I)$. Following pattern~\eqref{eq:daggerAdmissibleEntries} for
the induced semigroup morphism $\pmb !\colon S\to I$, we derive
\begin{equation}
    \pmb !_a^*=\sqrt{\valphi{a}}=\sqrt{\val{a}}.
\end{equation}
This way, we can identify the valency vector $\mathbf{v}$ with the matrix $\pmb !^\dagger$.
If $(X,S)$ is a scheme, then $!$ is a strongly admissible surjective morphism and we can apply results from the previous section. For a general coherent configuration $(X,S)$, we utilize the splitting of $!=!_S$ through the strongly admissible $(\tau_0,\tau_1)\colon(X,S)\rightarrow \mathcal{D}_{E_S}$ described in Example~\ref{ex:tau}. Consequently, we can express the valency vector as
\begin{equation}\label{eq:splitOf!}
    \mathbf{v}=(!_S)^\dagger=(!_\mathcal{D}\circ\tau)^\dagger=\tau^\dagger\circ {!_\mathcal{D}}^\dagger.
\end{equation}
\begin{lemma}\label{lemma:valency}
    The valency vector of a coherent configuration $(X,S)$ is an eigenvector of $\Dnabla\circ\Dnabla^\dagger$ with an eigenvalue $n=|X|$.
\end{lemma}
\begin{proof}
For the sake of simplicity, denote $\mathcal{D}=\mathcal{D}_{E_S}$, and let
$(\tau_0,\tau_1)\colon (X,S)\rightarrow \mathcal{D}$ be a strongly
admissible surjective morphism from Example~\ref{ex:tau}.  Applying
equation~\eqref{eq:splitOf!}, Proposition~\ref{prop:semigroupHom}, and
Corollary~\ref{cor:ddagger} yields the first three equalities in the following
string diagram computation:
\begin{equation}\label{eq:valencyIsEigen}
\vcbox{\includegraphics{dominik_valencyIsEigen1.pdf}}
\hspace{.8em}\vcbox{$=$}\hspace{.8em}
\vcbox{\includegraphics{dominik_valencyIsEigen2.pdf}}
\hspace{.8em}
\vcbox{$=$}
\hspace{.8em}
\vcbox{\includegraphics{dominik_valencyIsEigen3.pdf}}
\hspace{.8em}
\vcbox{$=$}
\hspace{.8em}
\vcbox{\includegraphics{dominik_valencyIsEigen4.pdf}}
\hspace{.8em}
\vcbox{$=$}
\hspace{.8em}
\vcbox{\includegraphics{dominik_valencyIsEigen5.pdf}}
\hspace{.8em}
\vcbox{$=$}
\hspace{.8em}
\vcbox{\includegraphics{dominik_valencyIsEigen6.pdf}}
\end{equation}
Let us 
prove the fourth equality. For this, we first describe the entries of the diagonal matrix 
$k_\tau$. For a color $b=\{(v,u)\}$ of $\mathcal{D}$, 
where $(v,u)\in (E_S)^2$, the corresponding matrix entry is 
\begin{equation}\label{eq:kOfTau}
{(k_\tau)}^b_b=k_{t(b)}=|\tau_0^{-1}(u)|=|u|.    
\end{equation}
 Next, note that in $\mathcal{D}$, each color has valency equal to $1$, hence
 $\mathcal{D}$ is thin, moreover, $\Dnabla^\mathcal{D}=\nabla^\mathcal{D}$ has all entries
 in $\{0,1\}$. One can prove by the same way as in \Cref{thm:connections}(i) that 
$$\Dnabla^{\mathcal{D}}\circ(k_\tau\otimes \id)\circ(\Dnabla^{\mathcal{D}})^\dagger=\nabla^{\mathcal{D}}\circ(k_\tau\otimes \id)\circ(\nabla^{\mathcal{D}})^\dagger$$
is a diagonal matrix. Computing the entries, we get
    \begin{equation}\label{eq:pipes}
        (\Dnabla^{\mathcal{D}}\circ(k_\tau\otimes
	\id)\circ(\Dnabla^{\mathcal{D}})^\dagger)_a^a=\sum_{b,c}{(k_\tau)}_b^b((\nabla^\mathcal{D})_{b
	c}^{a})^2=\sum_{u\in E_S}|u|=n,
    \end{equation}
    where we  used~\eqref{eq:kOfTau} in the second step.
\end{proof}
%Let us briefly sketch an alternative viewpoint at the last equation in the proof.
%Using the structure of a compact category and techniques from \cite{heunen2019categories} we can prove
%equation~\eqref{eq:pipes} with string diagrams as follows. We apply the fact that $\mathcal{D}$ is essentially the
%pair-of-pants Frobenius structure, and $k$ can be expressed as $k=\id\otimes l$, where $l_{u}^u=|u|$.
%\begin{equation}
%\label{eq:pants}
%\vcbox{\includegraphics{dominik_pipes1.pdf}}
%\hspace{.8em}\vcbox{$=$}\hspace{.8em}
%\vcbox{\includegraphics{dominik_pipes2.pdf}}
%\hspace{.8em}\vcbox{$=$}\hspace{.8em}
%\vcbox{\includegraphics{dominik_pipes3.pdf}}
%\end{equation}
%The right-hand diagram equals $\mathrm{Tr}(l)\cdot\id_{F\otimes F}=(\sum_{u\in E_s}|u|)\cdot\id_{F\otimes F}=n\cdot \id_{F\otimes F}$.

Our next aim is to show that we can identify the valency vector among the eigenvectors of $\Dnabla\circ\Dnabla^\dagger$. The key tool is the following:
\begin{theorem}[Perron-Frobenius]\label{thm:PF}\cite{meyer2023matrix}
    Let $A$ be a square matrix with entries in $\mathbb{R}^+$. Then $A$ has, up to scalar multiple, a unique positive eigenvector $\mathbf v$. Moreover, $\mathbf v$ corresponds to the largest eigenvalue.
\end{theorem}
We aim to apply the Perron-Frobenius theorem to $O=\Dnabla\circ\Dnabla^\dagger$ identify
the valency vector. Note that if $O$ has all entries strictly positive, the valency vector
is identified as the positive eigenvector of the eigenvalue
$n$. In general, however,  we need to deal with the fact that $O$ might have some zero entries.

\begin{theorem}\label{thm:determineStructureConstants}
    Let $(S,\Dnabla,e)$ be a dagger Frobenius structure induced by a coherent configuration. Then the data $(S,\Dnabla,e)$ uniquely determines the structure constants $\nabla$.
\end{theorem}
\begin{proof}
Obviously, it is enough to determine the valency vector $\mathbf{v}$.
 Denote $O=\Dnabla\circ\Dnabla^\dagger$ and define a relation $\sim$ on the set of colors $S$ by the prescription
\begin{equation}
    a\sim b \Longleftrightarrow O_a^b\not= 0,
\end{equation}
in other words, $a\sim b$ if and only if there are some $p,q\in S$ such that
$\nabla^a_{pq}\nabla^b_{pq}>0$. 
It is easy to see that $\sim$ is reflexive and symmetric and there is some
$k\in\mathbb N$ such that that $(O^k)_a^b\not=0$ if and only if $a$ and
$b$ belong to the same equivalence class of the transitive closure of $\sim$. Using the
corresponding decomposition of $S$, it is easy
to see that there is some permutation matrix $\Pi$ such that $\Pi O\Pi^\dagger$ is
block-diagonal, with blocks $O_1,\dots, O_m$, and we have 
\[
\Pi O^k\Pi^\dagger = (\Pi O\Pi^\dagger)^k=\mathrm{Diag}(O^k_1,\dots, O^k_m).
\]
Moreover,  for each $i=1,\ldots,m$, the block  $O_i^k$ has all entries positive.
As each 
$O_i$ is symmetric, the eigenvectors of $O_i$ are the same as the eigenvectors of
$O_i^k$. Consequently, by the Perron-Frobenius theorem, each $O_i$ has, up to a scalar
multiple, a unique positive eigenvector. By Lemma \ref{lemma:valency}, $\Pi\mathbf{v}$ is an
eigenvector of $\Pi O\Pi^\dagger$ with eigenvalue $n=|X|$, hence it decomposes as a direct sum
$\Pi\mathbf{v}=\mathbf{v}_1\oplus \cdots\oplus \mathbf{v}_m$, where each $\mathbf{v}_i$ is
an eigenvector of $O_i$ with eigenvalue $n$. Since each $\mathbf{v}_i$ consists of some
entries of the valency vector $\mathbf{v}$, it is strictly positive and hence  it is
given uniquely, up to a scalar multiple.  

Given any block $O_i$ such that the corresponding equivalence class contains at least one unit color, we can identify the
 part $\mathbf{v}_i$ of the valency vector as the
 unique positive eigenvector of $O_i$ having $1$'s on unit color coordinates. Hence, each
 color in the equivalence class of a unit color has determined valency. 


Let now $a$ be any color, and let $P=\{b\in S : \Dnabla_{ab}^a>0\}$. 
Note that for $b\in
P$, $\nabla^b_{a^{-1}a}\nabla^{t(a)}_{a^{-1}a}>0$, so that $b\sim t(a)$ and hence $b$ has
determined valency by the previous paragraph. The proof is completed by the following
computation.
\begin{equation}
        \val{a}=\sum_{b\in S}\nabla^{a}_{a b}=\sum_{b\in P}\nabla^{a}_{a b}=\sum_{b\in P}\sqrt{\frac{\val{a}\val{b}}{\val{a}\val{b}}}\nabla^{a}_{a b}=\sum_{b\in P}\sqrt{\val{b}}\dnabla{a}{b}{a}.
    \end{equation}

\end{proof}

\section{The spectrum of $O_{(X,S)}$}

In the previous section, we have shown that the valency vector is one of the eigenvectors of
$\Dnabla\circ\Dnabla^\dag$. In this section, we will examine the spectrum of this matrix and then apply
the results to prove a generalization of the (normal subgroup case of) Lagrange's theorem for schemes. For some simple
classes of schemes, we will be able to characterize the spectrum.

From now on, for a dagger Frobenius structure associated to a
coherent configuration $(X,S)$ we denote the matrix $\Dnabla\circ\Dnabla^\dagger$ by
$O_{(X,S)}$, or simply by $O$, and its spectrum by $\sigma(O)$.
Note that $O$, being a symmetric matrix, is diagonalizable over $\mathbb R$.

In order to study the spectral data of $O$, in particular the eigenvectors, we also need negative reals. Hence, it is convenient to pass from the category $\MatR$ to a larger category  $\mathbf{FHilb}$. Following the (monoidal, dagger-preserving) functor $L\colon\MatR\hookrightarrow\mathbf{FHilb}$ described in Section~\ref{sec:2.5}, each 
coherent configuration induces also a $\dagger$-Frobenius structure in $\mathbf{FHilb}$. For a moment we will work with a general $\dagger$-Frobenius structure $(V,\mu,\eta)$ in $\mathbf{FHilb}$, where $\mu$ denotes the multiplication and $\eta$ denotes the unit. 
\begin{lemma}\label{lem:specDec}
    Let $(V,\mu,\eta)$ be a $\dagger$-Frobenius structure in $\mathbf{FHilb}$ and $\mathbf{u}$, $\mathbf{v}$ be two eigenvectors of $\mu\circ\mu^\dagger$ with eigenvalues $\lambda_u$ and $\lambda_v$, respectively. If $\lambda_u=\lambda_v$, then $\mu\circ (\mathbf{u}\otimes\mathbf{v})$ is an eigenvector of $\mu\circ\mu^\dagger$ corresponding to $\lambda_u=\lambda_v$ as well. If $\lambda_u\not=\lambda_v$, then $\mu\circ (\mathbf{u}\otimes\mathbf{v})$ is the zero vector.
\end{lemma}
\begin{proof}
The statement is a consequence of the following equalities: 
\begin{equation}\label{eq:multiplicationPreservesEigenvectors}
    \lambda_u \cdot \mu\circ (\mathbf{u}\otimes\mathbf{v})=\mu\circ\mu^\dag\circ\mu\circ(\mathbf{u}\otimes\mathbf{v})=\lambda_v \cdot \mu\circ (\mathbf{u}\otimes\mathbf{v}).
\end{equation}
We will prove the first equality in~\eqref{eq:multiplicationPreservesEigenvectors} using string diagrams as follows:
\begin{center}
\vcbox{\includegraphics{dominik_eigenstates1.pdf}}
\vcbox{$=$}
\vcbox{\includegraphics{dominik_eigenstates2.pdf}}
\vcbox{$=$}
\vcbox{\includegraphics{dominik_eigenstates3.pdf}}
\vcbox{$=$}
\vcbox{\includegraphics{dominik_eigenstates4.pdf}}
\end{center}
%    \begin{figure}[H]
%        \centering
%        \includegraphics[width=0.9\linewidth]{eigenstates.pdf}
%        \caption{Caption}
%        \label{fig:enter-label}
%    \end{figure}
The first step follows by the Frobenius identity, the second one applies
associativity, and the last follows from the assumption that $\mathbf{u}$ is an
eigenvector of $\mu\circ \mu^\dagger$.
\end{proof}

\begin{remark} Note that in the above string diagram equalities we did not use the fact
that $\mathbf{v}$ is an eigenvector, the same computation goes trough for any vector in $V$.
Using similarly the other Frobenius identity, we obtain that for any eigenvector
$\mathbf{u}$
with eigenvalue $\lambda_u$, both $\mu\circ(\mathbf{x}\otimes \mathbf{u})$ and
$\mu\circ(\mathbf{u}\otimes \mathbf{x})$ are eigenvectors with the same eigenvalue, for
any vector $\mathbf{x}$. It
follows that the eigenspaces of $\mu\circ\mu^\dagger$ are two-sided ideals in the algebra
given  by the monoid structure $(V,\mu,e)$. 


\end{remark}




In $\mathbf{FHilb}$, each dagger Frobenius structure is a direct sum of special $\dagger$-Frobenius structures. 

\begin{proposition}\label{prop:decomposition}
    Let $\mathcal{F}=(V,\mu,\eta)$ be a $\dagger$-Frobenius structure in $\mathbf{FHilb}$ and 
    \begin{equation}\label{eq:eigenSpaceDec}
        V=\bigoplus_{\lambda\in\sigma(\mu\circ\mu^\dagger)} V_\lambda
    \end{equation} 
    be a spectral decomposition of $V$ with respect to $\mu\circ\mu^\dagger$. Then
    $\mathcal{F}$ admits a direct sum decomposition
\begin{equation}
    \mathcal{F}=\bigoplus_{\lambda\in\sigma(\mu\circ\mu^\dagger)} \mathcal{F}_\lambda.
\end{equation}
Here $\mathcal{F}_\lambda=(V_\lambda,\mu_\lambda,\eta_\lambda)$ is a $\dagger$-Frobenius structure on $V_\lambda$ satisfying $\mu_\lambda\circ\mu^\dagger_\lambda=\lambda\cdot\mathrm{id}$.
\end{proposition}
\begin{proof}
    From Lemma~\ref{lem:specDec} we obtain a decomposition $\mu=\bigoplus_{\lambda\in\sigma}\mu_\lambda$, where $\sigma=\sigma(\mu\circ\mu^\dagger)$. Applying dagger, we yield an analogous direct decomposition of the comultiplication $\mu^\dagger$. Similarly, consider decomposition $\eta=\bigoplus_{\lambda\in\sigma}\eta_\lambda$ with respect to~\eqref{eq:eigenSpaceDec}. It is easy to deduce from Lemma~\ref{lem:specDec} that for each $\lambda\in\sigma$, the summand $\eta_\lambda$ is a unit for $\mu_\lambda$.  Finally, since $\mu_\lambda$ and $\mu^\dagger_\lambda$ are essentially (co)restrictions of $\mu$ and $\mu^\dagger$, we get that associativity, coassociativity, and the Frobenius identity for $\mathcal{F}_\lambda$ follow from the corresponding rules for $\mathcal{F}$.
\end{proof}
\begin{proposition}
Let $(S,\Dnabla,e)$ be the $\dagger$-Frobenius structure associated to a  coherent
configuration on $n$ vertices. Then the spectrum satisfies $\sigma(\Dnabla\circ \Dnabla^\dagger)\subseteq (0,n]$.
\end{proposition}
\begin{proof}
   Clearly, $O=\Dnabla\circ \Dnabla^\dagger$ is a positive semi-definite matrix. By the proof of Theorem~\ref{thm:determineStructureConstants}, we deduce that $O$ has $n$ as its maximal eigenvalue. It remains to prove $0$ is not an eigenvalue. 

Write $(V,\mu,\eta)$ for $(L(S),L(\Dnabla),L(e))$, so that we obtain a $\dagger$-Frobenius
structure in $\FHilb$. Let us assume $V_0$ is an eigenspace corresponding to an eigenvalue $\lambda=0$. 
   According to Proposition~\ref{prop:decomposition}, we have a $\dagger$-Frobenius structure $\mathcal{F}_0=(V_0,\mu_0,\eta_0)$ 
   in $\FHilb$, such that $\mu_0\circ\mu^\dagger_0$ is the zero operator. Therefore,
   $\mu_0$ is a zero operator as well, which forces the Frobenius algebra $\mathcal{F}_0$
   to be  trivial, with $V_0=\{0\}$, but then $V_0$ is not an eigenspace.
\end{proof}
Recall the self-adjoint idempotent ${\phi}^\ddagger\circ{\pmb \phi}$ from
Lemma~\ref{lem:projections}. In $\mathbf{FHilb}$, it corresponds to an orthogonal
projection to $\mathrm{Im}(L({\pmb \phi}^\dagger))$. It turns out that this projection
commutes with the orthogonal projections to eigenspaces established in
Proposition~\ref{prop:decomposition}.
\begin{proposition}
    Let $(\phi_0,\phi_1)\colon(X,S)\rightarrow(X',S')$ be a strongly admissible morphism
    of coherent configurations and let  $\phi^\ddagger\circ{\pmb \phi}\colon S\rightarrow
    S$ be the idempotent from Proposition~\ref{lem:projections}. Let $V=L(S)$ and let
    $\pi:=L(\phi^\ddagger\circ {\pmb \phi}):V\to V$. Let  $V=\oplus_{\lambda} V_\lambda$ be the decomposition~\eqref{eq:eigenSpaceDec}. Then $\pi$ admits decomposition $\pi=\oplus_\lambda\pi_\lambda$, where $\pi_\lambda$ is an idempotent endomorphism on $V_\lambda$.  
\end{proposition}

\begin{proof}
    It is enough to prove that $\pi$ preserves each eigenspace $V_\lambda$. It easily
    follows from Theorem~\ref{thm:projOnCoimage} and properties of the functor $L$ that
    $\pi$ is a semigroup and cosemigroup endomorphism for the corresponding structures on
    $V$. Consequently, for each eigenvector $\mathbf{v}\in V_\lambda$ we have:
\begin{equation}
\vcbox{\includegraphics{dominik_comOfProj1.pdf}}
\hspace{.8em}\vcbox{$=$}\hspace{.8em}
\vcbox{\includegraphics{dominik_comOfProj2.pdf}}
\hspace{.8em}\vcbox{$=$}\hspace{.8em}
\vcbox{\includegraphics{dominik_comOfProj3.pdf}}
\hspace{.8em}\vcbox{$=$}\hspace{.8em}
\vcbox{\includegraphics{dominik_comOfProj4.pdf}}
\end{equation}
\end{proof}
Revealing the formula of a trace of $O$, we get a sum over three variables. The following lemma gives us a formula with only two variables. The formula has a nice form under an assumption of $\frac{1}{2}$-homogeneity. However, note that in the proof we use this assumption only at the very last step.
\begin{lemma}
If $(X,S)$ is $\frac{1}{2}$-homogeneous coherent configuration, then
    \begin{equation}\label{eq:trace}
    \mathrm{Tr}(\Dnabla\circ\Dnabla^\dagger)=\sum_{d\in S}\left(\sum_{a\in S}\Dnabla^d_{a a^{-1}}\right)^2.
\end{equation}
\end{lemma}
\begin{proof}
We compute
\begin{align*}
    \sum_{ab,c}\dnabla{a}{b}{c}\dnabla{a}{b}{c}&\overset{\eqref{eq:dagRotR}}{=} \sum_{ab,c}\sqrt{\frac{\val{b^{-1}}}{\val{b}}}\dnabla{b}{c^{-1}}{a^{-1}}\dnabla{a}{b}{c}\overset{\eqref{eq:dagOp}}{=}\sum_{ab,c}\sqrt{\frac{\val{b^{-1}}}{\val{b}}}\dnabla{c}{b^{-1}}{a}\dnabla{a}{b}{c}\\
    &\overset{(assoc.)}{=}\sum_{b,c,d}\sqrt{\frac{\val{b^{-1}}}{\val{b}}}\dnabla{c}{d}{c}\dnabla{b^{-1}}{b}{d}\\
    &\overset{\eqref{eq:dagRotL}}{=}
    \sum_{b,c,d}\sqrt{\frac{\val{b^{-1}}}{\val{b}}}\sqrt{\frac{\val{c}}{\val{c^{-1}}}}\dnabla{c^{-1}}{c}{d^{-1}}\dnabla{b^{-1}}{b}{d}\\
    &\overset{\eqref{eq:dagOp}}{=}
    \sum_{b,c,d}\sqrt{\frac{\val{b^{-1}}}{\val{b}}}\sqrt{\frac{\val{c}}{\val{c^{-1}}}}\dnabla{c^{-1}}{c}{d}\dnabla{b^{-1}}{b}{d}\\
    &=\sum_{d}\bigg(\sum_a \sqrt{\frac{\val{a^{-1}}}{\val{a}}} \dnabla{a^{-1}}{a}{d}\bigg)\bigg(\sum_a \sqrt{\frac{\val{a}}{\val{a^{-1}}}} \dnabla{a^{-1}}{a}{d}\bigg).
\end{align*}
    Since $(X,S)$ is $\frac{1}{2}$-homogeneous, then $\val{a}=\val{a^{-1}}$ and the last expression equals~\eqref{eq:trace}. 
\end{proof}
\begin{example}
Let $G$ be a finite group with $n$ elements and consider the corresponding thin
scheme $S(G)=(X,S)$ as in \Cref{ex:cayleyScheme}. In
$S(G)$, both vertices and colors can be addressed with the elements of $G$.
Then $\Dnabla\circ\Dnabla^\dagger$ is a diagonal $n\times n$ matrix in which
all diagonal elements are equal to $n$. One can deduce that by the following
computation
    \begin{equation}
        (\Dnabla\circ\Dnabla^\dagger)_a^b=(\nabla\circ\nabla^\dagger)_a^b=\sum_{c,d\in S, a=c\cdot d=b}\nabla^a_{cd}\nabla^b_{cd}=n\cdot\delta_a^b.
    \end{equation}
\end{example}
Next, consider a surjective morphism of finite groups $f\colon G\rightarrow H$.
According to~\cite{french2013functors}, $f$ induces an admissible morphism
of schemes $S(f)\colon S(G)\rightarrow S(H)$. In this case, it is well known that $|H|$ is
a divisor of $|G|$; equivalently, the largest eigenvalue of $O_{S(H)}$ divides the
largest eigenvalue of $O_{S(G)}$. The following theorem refines this observation
for association schemes  and can be viewed as a generalization of the Lagrange theorem for groups.
\begin{theorem}\label{thm:genLagrange}
Let $(\phi_0,\phi_1)\colon (X,S)\rightarrow (X',S')$ be a surjective
admissible morphism of schemes and \((\lambda_1,\ldots,\lambda_m)\),
$m=|S'|$, be the vector of the eigenvalues of $O_{(X',S')}$ (counting
multiplicities). Then there is a positive integer $k$, such that
$(k\cdot\lambda_1,\ldots,k\cdot\lambda_m)$ is a subvector of the vector of
the eigenvalues of $O_{(X,S)}$.
\end{theorem}
\begin{proof}
Since we have schemes, the diagonal matrix $k_\phi$ (see formulas~\eqref{eq:diagonalK}) has all its diagonal entries
equal to some $k\in\mathbb{N}$. We claim that if $\mathbf{v}$ is an eigenvector of $(X',S')$ corresponding to eigenvalue
$\lambda$, then $\pmb\phi^\dagger\circ\mathbf{v}$ is an eigenvector of $(X,S)$, corresponding to eigenvalue
$k\cdot\lambda$.  This is enough, since $\pmb\phi^\dagger$ has a left inverse $\frac{1}{k}\pmb\phi$ by
Lemma~\ref{lem:projections}, hence it has a trivial kernel. Let us prove the claim with the following string diagram
calculation:
\begin{center}
\vcbox{\includegraphics{dominik_genLag1.pdf}}
\vcbox{~$=$~}
\vcbox{\includegraphics{dominik_genLag2.pdf}}
\vcbox{~$=$~}
\vcbox{\includegraphics{dominik_genLag3.pdf}}
\vcbox{~$=$~}
\vcbox{\includegraphics{dominik_genLag4.pdf}}
\vcbox{~$=$~}
\vcbox{\includegraphics{dominik_genLag5.pdf}}
\end{center}
The first equality follows by (the dagger variant of)
Corollary~\ref{cor:daggerMonHom}, the second equality follows from
Corollary~\ref{cor:ddagger}, and the third one is due to $k_\phi$ being a constant
diagonal matrix.
\end{proof}
\begin{example}\label{ex:completeGraph}
    Consider a trivial scheme $\mathcal{T}_n$, $n\in\mathbb{N}$, which has one unit color and one non-unit color of valency $n-1$. A straightforward computation leads to
    \begin{equation}
        O= \Dnabla\circ\Dnabla^\dagger =
        \begin{bmatrix}
    2 & \frac{n-2}{\sqrt{n-1}} \\
    \frac{n-2}{\sqrt{n-1}} & \frac{(n-1)^2+1}{n-1}
    \end{bmatrix}.
    \end{equation}
The spectrum of $O$ equals $\sigma(O)=\{n,\frac{n}{n-1}\}$. Eigenvalue $n$
corresponds to a valency eigenvector $\mathbf{v}=(1,\sqrt{n-1})^T$ and
eigenvalue $\frac{n}{n-1}$ corresponds to an eigenvector
$\mathbf{u}=(\sqrt{n-1},-1)^T$. 
The multiplication and comultiplication act on the eigenvectors as follows
    \begin{align*}
        \Dnabla^\dagger\circ\mathbf{v}= \mathbf{v}\otimes\mathbf{v} &\quad\text{and}\quad \Dnabla\circ(\mathbf{v}\otimes\mathbf{v})= n\cdot \mathbf{v},\\
         \Dnabla^\dagger\circ\mathbf{u}= \frac{1}{\sqrt{n-1}}\cdot\mathbf{u}\otimes\mathbf{u} &\quad\text{and}\quad\Dnabla\circ(\mathbf{u}\otimes\mathbf{u})= \frac{n}{\sqrt{n-1}}\cdot \mathbf{v},\\
         \Dnabla \circ(\mathbf{v}\otimes\mathbf{u})&=0=\Dnabla\circ(\mathbf{u}\otimes\mathbf{v}).
    \end{align*}
\end{example}
\begin{example}\label{ex:wiki}
Consider a pair $(\phi_0,\phi_1)\colon (X,S)\rightarrow \mathcal{T}_3$ and $(\psi_0,\psi_1)\colon S\rightarrow \mathcal{T}_2$ of admissible morphisms of association schemes given by Figure~\ref{fig:exWiki}. Let us label the colors of scheme $(X,S)$ by integers $0,1,2,3$. 
\begin{figure}
    \centering
    \includegraphics[width=0.95\linewidth]{admissibleEx.pdf}
    \caption{The association scheme $\mathcal T_2\otimes\mathcal T_3$}
    \label{fig:exWiki}
\end{figure}
For this scheme, we have
\begin{equation}
    O = \begin{bmatrix}
    4 & \sqrt{2} & 0 & 0 \\
    \sqrt{2} & 5 & 0 & 0 \\
    0 & 0 & 4 & \sqrt{2} \\
    0 & 0 & \sqrt{2} & 5
    \end{bmatrix}.
    %\label{eq:four_times_four_matrix}
\end{equation}
The spectrum has two elements, $\sigma(O)=\{3,6\}$. The eigenvalue $6$ equals the number of vertices, and the corresponding eigenspace is generated by two vectors
$\mathbf{v}_1=(1,\sqrt{2},1,\sqrt{2})^T$ and $\mathbf{v}_2=(1,\sqrt{2},-1,-\sqrt{2})^T$. The eigenspace corresponding to $3$ is generated by $\mathbf{u}_1=(\sqrt{2},-1,\sqrt{2},-1)^T$ and $\mathbf{u}_2=(\sqrt{2},-1,-\sqrt{2},1)^T$.

The matrices $\phi$ and $\psi$ corresponding to the two admissible morphisms in concern are 
\begin{equation}
    \pmb\phi = \begin{bmatrix} 
    1 & 0 & 1 & 0 \\ 
    0 & 1 & 0 & 1
    \end{bmatrix}, \quad
    \pmb\psi = \begin{bmatrix} 
    1 & \sqrt{2} & 0 & 0 \\ 
    0 & 0 & 1 & \sqrt{2} 
    \end{bmatrix}
    \label{eq:matrices_2x4}
\end{equation}
Denote by $\mathbf{v}'=(1,1)^T$, $\mathbf{u}'=(1,-1)^T$ and $\mathbf{v}''=(1,\sqrt{2})^T$, $\mathbf{u}''=(\sqrt{2},-1)^T$ the eigenvectors from Example~\ref{ex:completeGraph}, for $\mathcal{T}_2$ and $\mathcal{T}_3$.
Note that $\mathbf{v}_1=\pmb\phi^\dagger\mathbf{v}''=\pmb\psi^\dagger\mathbf{v}'$ is the valency vector. Moreover, $\mathbf{v}_2=(1,\sqrt{2},-1,-\sqrt{2})^T=\pmb\psi^\dagger\mathbf{u}'$ and $\mathbf{u}_1=(\sqrt{2},-1,\sqrt{2},-1)^T=\pmb\phi^\dagger\mathbf{u}''$.

Moreover, observe that $k_\phi$ is a diagonal matrix having $2$'s on its
diagonal and the spectrum of $\mathcal{T}_2$ has two eigenvalues $3$ and
$\frac{3}{2}$. In accordance with Theorem~\ref{thm:genLagrange}, $2\cdot 3$ and
$2\cdot \frac{3}{2}$ belong to the spectrum of $O_{(X,S)}$. Similarly, $k_\psi$ has
$3$'s on the diagonal and $\mathcal{T}_2$ has in spectrum twice $2$. Indeed,
the value $3\cdot 2$ appears twice in the spectrum of $O_{(X,S)}$. 
\end{example}
\begin{lemma}
Let $(X,S)$ and $(Y,T)$ be coherent configurations with spectra $\sigma_S$
and $\sigma_T$. Then the spectrum of the tensor product configuration $(X,S)\otimes(Y,T)$
is the elementwise product: $\{\lambda.\rho\mid
\lambda\in\sigma_S,\rho\in\sigma_T\}$.
\end{lemma}
\begin{proof}
This is an easy consequence of the fact that $\Dnabla_{(X,S)\otimes(Y,T)}\simeq\Dnabla_{(X,S)}\otimes\Dnabla_{(Y,T)}$.
\end{proof}
In the Example~\ref{ex:wiki}, the association scheme $(X,S)$ is isomorphic to
$\mathcal{T}_3\otimes \mathcal{T}_2$, hence we get the spectral data for $S$
by multiplying those for $\mathcal{T}_3$ and $\mathcal{T}_2$.

In the following example, there is an obvious symmetry, which reflects the presence of an automorphism switching two colors. 
\begin{example}
Let $S$ be an association scheme defined by Figure~\ref{fig:pentagon}. We have
\begin{equation}
    O:=(\Dnabla\circ\Dnabla^\dagger)_a^b = \begin{bmatrix}
    3 & \frac{1}{\sqrt{2}} & \frac{1}{\sqrt{2}} \\
    \frac{1}{\sqrt{2}} & \frac{7}{2} & 1 \\
    \frac{1}{\sqrt{2}} & 1  & \frac{7}{2}
    \end{bmatrix}.
    %\label{eq:four_times_four_matrix}
\end{equation}
    The spectrum equals $\sigma=\{5,\frac{5}{2}\}$. Note that $O$ is invariant under  the symmetry $\Pi:
    \mathbb R^3\to \mathbb R^3$ given as $\Pi(x,y,z)=(x,z,y)$,  which
    implies that the eigenspaces are invariant under $\Pi$. 
The eigenvalue $5$ corresponds to valency vector $\mathbf{v}=(1,\sqrt{2},\sqrt{2})$ which
clearly satisfies $\Pi\mathbf{v}=\mathbf{v}$. The eigenvalue $\frac{5}{2}$ has eigenspace of dimension $2$ generated by 
the vectors $\mathbf{u}_1$ and $\mathbf{u}_2=\Pi\mathbf{u}_1$, where we may choose 
    \begin{equation}
        \mathbf{u}_1=(\frac{2}{5},\frac{-\sqrt{2}-\sqrt{10}}{\sqrt{10}},\frac{\sqrt{2}-\sqrt{10}}{\sqrt{10}}).
    \end{equation}
    Note that then 
    \begin{equation}
        \langle \mathbf{u}_i,\mathbf{u}_j\rangle =\frac{2}{5}\delta_i^j,\quad \mu(\mathbf{u}_i\otimes\mathbf{u}_j)=\delta_i^j \mathbf{u}_i.
    \end{equation}
\end{example}
\begin{example}
    Denote by $S$ the association scheme from Figure~\ref{fig:petersen} induced by the Petersen graph.
    \begin{figure}
        \centering
        \includegraphics[width=0.5\linewidth]{petersen.pdf}
        \caption{scheme induced by Petersen graph}
        \label{fig:petersen}
    \end{figure}
    Even though the matrix
    \[
            \Dnabla\circ\Dnabla^\dagger =
            \begin{bmatrix}
                3 & \frac{2}{3}\sqrt{3} & \frac{5}{6}\sqrt{6} \\
                \frac{2}{3}\sqrt{3} & \frac{14}{3} & \frac{14}{6}\sqrt{2} \\
                \frac{5}{6}\sqrt{6} & \frac{14}{6}\sqrt{2} & \frac{41}{6} & \\
            \end{bmatrix}
            \]
            is rather complicated, the spectrum consists of only rational numbers $\sigma=\{10,\frac{5}{2},2\}$. 
\end{example}
\begin{example}
The smallest non-Schurian coherent configuration was found in \cite{klin2017non}. It has rank 12 and degree
14. This example is not $\frac{1}{2}$-homogeneous, there are two unit colors with cardinalities
$6$ and $8$.
The spectrum of $\theO$ is
\begin{center}
\begin{tabular}{|r|r|}
\hline
eigenvalue & multiplicity\\
\hline
14& 4\\
 3& 1\\
 $\frac{14}{3}$& 4\\
 4& 2\\
\hline
\end{tabular}
\end{center}
\end{example}

\begin{example}
Let  $(X,S)$ be $\frac12$-homogeneous. As we observed in Corollary
\ref{coro:hstar} and the
paragraph below it, our dagger Frobenius structure is symmetric
and coincides with an $H^*$-algebra structure of the adjacency algebra $\mathcal A$. In
particular, the matrix $O_{(X,S)}$ becomes the matrix of the operator 
\[
    m\cdot(\nabla_{\mathcal A}\circ\nabla_{\mathcal A}^\dag)
\]
in the normalized basis $\{\bar M_a,\ a\in S\}$, where $m$ is the cardinality of
unit colors and $(\mathcal A,\nabla_{\mathcal
A},e_{\mathcal A})$ is the corresponding symmetric dagger Frobenius structure on $\mathcal A$.
We now show how the spectrum of $O_{(X,S)}$ is related to the structure of $\mathcal A$. 

Since $\mathcal A$ is a $*$-subalgebra of $M_n(\mathbb C)$, we have a direct sum
decomposition of the form $\mathbb C^n\simeq \bigoplus_i \mathbb C^{d_i}\otimes \mathbb C^{k_i}$, such that
(up to a unitary conjugation)
\[
\mathcal A= \bigoplus_i M_{d_i}(\mathbb C)\otimes I_{k_i}.
\]
The inner product induced in each of the factors is  
\[(A,B)\mapsto \mathrm{Tr}\,
[(A^*B)\otimes I_{k_i}]=k_i\mathrm{Tr}\,[A^*B],\qquad A,B\in M_{d_i}(\mathbb C).
\] The $H^*$-algebra $\mathcal A$ is thus
isomorphic to $\bigoplus_i B(\mathbb C^{d_i}, k_i)$, in the notation of
\cite[Section 5.4]{heunen2019categories}. By \cite[Prop.~5.33]{heunen2019categories}, the symmetric
dagger Frobenius structure in the factors is given by the scaled pair of pants structure on $(C^{d_i})^*\otimes
C^{d_i}$, with the multiplication and the unit given by
\begin{center}
\vcbox{\includegraphics{scaled_pants_nabla.pdf}}
\vcbox{\hspace{3em}}
\vcbox{\includegraphics{scaled_pants_unit.pdf}}
\end{center}
and the Frobenius structure in $\mathcal A$ is
then their direct sum. In particular,
\begin{center}
\vcbox{$\nabla_{\mathcal A}\circ\nabla_{\mathcal A}^\dag=\bigoplus_i$}
\vcbox{\includegraphics{scaled_pants_O.pdf}}
\vcbox{$=\bigoplus_i\frac{d_i}{k_i} I_{d_i}$}
\end{center}
This shows that $\sigma(O)=\{ m\frac{d_i}{k_i}\}_i$. In particular, the spectrum contains only rational numbers.
\end{example}


\begin{example}\label{ex:permut} 
Let us describe the spectrum of $O$ for Schurian coherent configurations arising from
an action of $\mathbb Z$, that means, from a single permutation.
Let $X$ be a finite set and let $\Gamma$ be a permutation
on $X$. Let 
$(x,y)\sim (x',y')$ if $x'=\Gamma^k(x)$, $y'=\Gamma^k(y)$ for some $k\in \mathbb Z$ and
let 
$S$ be the set of equivalence classes of $\sim$. Then $(X,S)$ is a coherent configuration on $X$. 
The permutation $\Gamma$ can be decomposed
into cycles $C_1,\dots, C_m$, with lengths $n_l=|C_l|$, $\sum_l n_l=n:=|X|$. Two vertices
in $X$ have the same unit color iff they are in the same cycle, so that $(X,S)$ is not
$\frac12$-homogeneous if $n_l\ne n_{l'}$ for some $l$, $l'$. 

Computing  the form of the matrices $\nabla$, $\Dnabla$, $O$ and the spectrum of $O$ in this case is not difficult, but somewhat lengthy. We will
only report the general description of  $\sigma(O)$.  Let $N:=LCM(n_1,\dots,n_m)$ and let
$\mathcal L$ be the sublattice generated by $n_1,\dots, n_m$ in the divisibility lattice
$\mathcal L_D(N)$. Any join irreducible element in $\mathcal L$ has the form $n_T:=\wedge_{l\in T} n_l$ for
some $T\subseteq \{1,\dots,m\}$, with the lattice operations in $\mathcal L_D(N)$. 
Let 
\[\mathcal T:=\{\emptyset \neq T\subseteq \{1,\dots,m\},\ n_T \text{ is join irreducible in }
\mathcal L \text{ or } T=\{1,\dots,m\}\}.
\]
We then have 
\[
\sigma(O)=\{\lambda_T:=\sum_{l\in T} n_l,\ T\in \mathcal T\}.
\]
Moreover, the multiplicity of each  eigenvalue $\lambda\in \sigma(O)$ is obtained as
\[
\mu(\lambda)=\sum_{T\in \mathcal T, \lambda=\lambda_T} \chi_T|T|^2,\qquad
\chi_T=\begin{dcases} (n_T-n_T\wedge (\vee_{l\not\in T} n_l)) &\text{if } T\neq
[m]\\
n_{[m]} & \text{otherwise}.
\end{dcases}
\]
For example, assume that all $n_l$ are mutually coprime. Then $\mathcal
T$ consists of all singletons and the whole set $\{1,\dots,m\}$. The eigenvalues are 
\[
\lambda=\begin{dcases} n_l,\ l=1,\dots,n & \text{with multiplicity } n_l-1\\
 n=\sum_l n_l & \text{with multiplicity } m^2.
 \end{dcases}
 \]
For the other extreme example $n_1=\dots=n_m=k$, the sublattice $\mathcal L$ consists of a
single point and hence has no join-irreducible elements. In this case, $O$ has a single
eigenvalue $n$, with multiplicity $km^2=|S|$. 
\end{example}

\section{Future research, open problems}

In future papers, we intend to consider a variety of topics and themes; for example, the following.
\begin{itemize}
\item Describe the relationships between various types of morphisms we introduced, from the categorical perspective.
\item A coherent configuration is a module. A module over a monoid in a compact category is the same thing as the
representation of that monoid. There is a theory of representations of Frobenius algebras (see for example
\cite{skowronski2017frobenius}), that could be applicable in the theory of coherent configurations.
\item When we constructed the Frobenius structure, our choice of a non-degenerate form was $e^\dag$. 
However, \Cref{thm:frobIsC4} allows us to use any form that is supported in $E_S$. There is at least
one other non-degenerated form that is worth further examination
\[
c_p^*=
\begin{cases}
\frac{1}{|p|}&p\in E_S\\
0&\text{otherwise}.
\end{cases}
\]
We already know that the Frobenius structure arising from this form is always symmetric.
\end{itemize}

\begin{problem}
Is it true that for every coherent configuration $(X,S)$, the spectrum of $O_{(X,S)}$ contains
only rational numbers?
\end{problem}
\begin{problem}
Is it true that for every coherent configuration $(X,S)$, the spectrum of $O_{(X,S)}$ is
lower bounded by $\frac{|X|}{|X|-1}$?
\end{problem}
\begin{problem}
Are there a pair of coherent configurations 
$(X,S)$, $(Y,T)$ with $|S|=|T|$ such that the valency vectors for them are the same up to a permutation, 
but the spectra of $O_{(X,S)}$ and $O_{(Y,T)}$ are different?
\end{problem}
\bibliography{mybib}
\bibliographystyle{elsarticle-num}
\end{document}
