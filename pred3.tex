\section{Injekcie, surjekcie a bijekcie}

V tejto časti si zavedieme dôležité vlastnosti zobrazení. Skladanie zobrazení
môžeme chápať ako nejaký typ binárnej operácie, pre ktoré sa identické zobrazenie chová neutrálne, viď
vety \ref{veta:asocSkladania} a \ref{veta:neutralitaId}.
Z dostatočného odstupu a zanedbávajúc isté rozdiely môžeme $g\circ f$ vidieť ako analógiu súčinu reálnych čísel a $\id$ ako analógiu
\footnote{Táto analógia sa dá spresniť, takže z istého abstraktného hľadiska sa dajú súčin a skladanie
naozaj pochopiť ako dve inštancie jediného abstraktného pojmu.}
jednotky:
\[
\begin{array}{c|c}
a.b & g\circ f\\
a.1 =a &  g\circ\id=g \\
1.b=b & \id\circ f= f
\end{array}
\]
Pre násobenie čísel vieme ku každému číslu $a\neq 0$ nájsť nejaké číslo $a^{-1}$ také, že $a.a^{-1}=a^{-1}.a=1$, voláme ho
prevrátená hodnota $a$. Prirodzene vzniká otázka, či a kedy vieme nájsť k nejakému zobrazeniu $f$ analógiu prevrátenej
hodnoty čísla, to znamená zobrazenie $g$ z vlastnosťou $g \circ f=\id$ alebo $f\circ g=\id$. Skúmanie tohto
problému vedie k pojmom injekcie, surjekcie a bijekcie. Situácia je však trochu komplikovanejšia ako v prípade čísel,
pretože zobrazenia sú trochu zložitejšie veci ako čísla. 

\begin{definition}[injekcia]\label{def:injekcia}
Nech $A,B$ sú množiny. Zobrazeniu $f\colon A\to B$ hovoríme,
\emph{injekcia/injektívne} ak pre každé dva prvky $a_1,a_2\in A$ také, že $a_1\neq a_2$ platí, že $f(a_1)\neq f(a_2)$.
\end{definition}

V jazyku formálnej logiky
\begin{equation}\label{eq:injekcia}
(\forall a_1,a_2\in A)\quad a_1\neq a_2\implies f(a_1)\neq f(a_2)
\end{equation}
\begin{example}
Zobrazenie $j\colon A\to B$ v príklade \ref{ex:jedlo} nie je injektívne. Platí totiž
\[
\text{Miško}\neq\text{Janka}\qquad j(\text{Miško})=j(\text{Janka})
\]
Dokázali sme teda negáciu formuly \eqref{eq:injekcia} (pre $f=j$, samozrejme), to znamená
\[
(\exists a_1,a_2\in A)\quad a_1\neq a_2\land j(a_1)=j(a_2)
\]
\end{example}
Všimnite si, že neinjektívnosť $j$ je vidno z obrázku.

Logicky ekvivalentná forma \eqref{eq:injekcia} je
\begin{equation}\label{eq:injekcia2}
(\forall a_1,a_2\in A)\quad f(a_1)=f(a_2)\implies a_1=a_2
\end{equation}
ktorá vznikne transpozíciou implikácie:
\[
a_1\neq a_2\implies f(a_1)\neq f(a_2)
\quad
\text{je to isté ako}
\quad
f(a_1)=f(a_2)\implies a_1=a_2
\]
\begin{definition}[surjekcia]\label{def:surjekcia}
Nech $A,B$ sú množiny. Zobrazeniu $f\colon A\to B$ hovoríme,
\emph{surjekcia/surjektívne} ak pre každý prvok $b\in B$ existuje nejaký prvok $a\in A$ taký, že
$f(a)=b$.
\end{definition}
\begin{example}
Zobrazenie $j\colon A\to B$ z príkladu \ref{ex:jedlo} nie je surjektívne. Na prvok jablko kooboru $B$
zobrazenia $j$ sa žiadny prvok definičného oboru $A$ zobrazenia $j$ nezobrazí. Inými slovami, pre všetky prvky $a\in A$
platí, že $j(a)\neq\text{jablko}$.
\end{example}
V tejto chvíli je užitočné uvedomiť si, že zobrazenie $f\colon A\to B$ je surjektívne práve vtedy, keď 
koobor $B$ je rovný oboru hodnôt $f$, $B=\mathcal H(f)$. To znamená, že z každého zobrazenia vieme spraviť surjektívne
zobrazenie, ak zúžime jeho koobor: tieto dve zobrazenia
\begin{align*}
f_1\colon\mathbb R\to \mathbb R&\qquad f_1(x)=x^2+1\\
f_2\colon\mathbb R\to \mathbb R&\qquad \langle 1,\infty)
\end{align*}
majú rovnaký definičný obor a predpis (ale nie koobor, teda sú to rôzne funkcie). Pritom $f_1$ nie je surjektívne, ale
$f_2$ je surjekcia. 
\begin{definition}[bijekcia]\label{def:bijekcia}
Nech $A,B$ sú množiny. Zobrazeniu $f\colon A\to B$ hovoríme,
\emph{bijekcia/bijektívne} ak pre každý prvok $b\in B$ existuje nejaký prvok $a\in A$ taký, že
$f(a)=b$.
\end{definition}
\subsection{Ľavé a pravé inverzné zobrazenie}

Definície injekcie a surjekcie vyzerajú veľmi odlišne. V tejto časti textu sa naučíme, že sú prepojené istou skrytou
symetriou, ktorá sa týka toho, ako sa správajú vzhľadom na skladanie (operácia $\circ$) a identické zobrazenia.

\begin{definition}[zľava/sprava inverzné zobrazenie]\label{def:zlavaSpravaInverzne}
Nech $A,B$ sú množiny, nech $f\colon A\to B$. Hovoríme, že zobrazenie $g\colon B\to A$ je
\begin{itemize}
\item \emph{zľava inverzné k zobrazeniu $f$} ak platí, že $g\circ f=\id_A$
\item \emph{sprava inverzné k zobrazeniu $f$} ak platí, že $f\circ g=\id_B$
\end{itemize}
\end{definition}

\begin{veta}\label{veta:sekcie}
Nech $A,B$ sú množiny, nech $f\colon A\to B$. Potom
\begin{enumerate}[(a)]
\item $f$ je injekcia práve vtedy, ak existuje aspoň jedno zľava inverzné zobrazenie k $f$.
\item $f$ je surjekcia práve vtedy, ak existuje aspoň jedno sprava inverzné zobrazenie k $f$.
\end{enumerate}
\end{veta}
\begin{proof}~
\begin{enumerate}[(a)]
\item Nech $f$ je injekcia. Chceme nájsť nejaké zobrazenie $g\colon B\to A$, pričom $g$ má byť také,
že $g\circ f=\id_A$, teda pre všetky $a\in A$ má platiť
$(g\circ f)(a)=\id_A(a)$, to znamená $g(f(a))=a$. Keďže $f$ je injekcia, pre $b\in\mathcal H(f)$ existuje práve jedno
$a\in A$ také, že $f(a)=b$. Naozaj, ak by sme mali nejaké $a_1,a_2\in A$ také, že $a_1\neq a_2$ a zároveň
$f(a_1)=f(a_2)$, $f$ by nebola injekcia.
Pre $b\in B$, zvoľme $g(b)$ tak, že pre $b\in\mathcal H(f)$ máme $g(b)=a$, kde $f(a)=b$ a pre $b\in B\setminus\mathcal
H(f)$ zvolíme $g(b)$ ľubovoľne. Máme potom $g(f(a))=a$, pre každé $a\in A$.

Predpokladajme teraz, že existuje $g\colon B\to A$ také, že $g\circ f=\id_A$. Použijeme
charakterizáciu injekcie \eqref{eq:injekcia2}. Nech $a_1,a_2\in A$ sú také, že $f(a_1)=f(a_2)$. Z tohto
predpokladu máme dokázať, že $a_1=a_2$.
Podľa predpokladu zrejme $g(f(a_1))=g(f(a_2))$, čo znamená 
\begin{equation}\label{eq:veta:sekcie:1}
(g\circ f)(a_1)=(g\circ f)(a_2)\tag{*}
\end{equation}
Ale my sme predpokladali, že
$g\circ f=\id_A$, teda \eqref{eq:veta:sekcie:1} znamená, že $\id_A(a_1)=\id_A(a_2)$ a z toho máme ihneď
$a_1=a_2$
\item Dôkaz vynechávame.
\end{enumerate}
\end{proof}

\subsection{Inverzné zobrazenie}

\begin{definition}[inverzné zobrazenie]\label{def:inverzneZobrazenie}
Nech $A,B$ sú množiny, nech $f\colon A\to B$. Hovoríme, že zobrazenie $g\colon B\to A$ je
\emph{inverzné} k zobrazeniu $f$ ak je zľava inverzné k $f$ a zároveň sprava inverzné k $f$.
\end{definition}

\begin{veta}\label{veta:bijekciaInverzne}
Nech $A,B$ sú množiny. Potom $f\colon A\to B$ má inverzné zobrazenie práve vtedy, keď $f$ je bijekcia.
\end{veta}
\begin{proof}
Z definície bijekcie, inverzného zobrazenia a vety \ref{veta:sekcie} ihneď vidno, že ak má nejaké zobrazenie 
$f$ inverzné zobrazenie, potom $f$ je bijekcia.

Naopak, nech $f$ je bijekcia. Podľa vety \ref{veta:sekcie} má potom nejaké ľavé inverzné zobrazenie 
$g_L\colon B\to A$ a aj nejaké pravé inverzné zobrazenie $g_R\colon B\to A$. 
Ak dokážeme z týchto predpokladov že $g_L=g_R$, potom to je už inverzné zobrazenie k $f$.
Použijeme elegantný trik: vezmeme výraz $g_L\circ f\circ g_R$ a zjednodušíme ho dvoma
rôznymi spôsobmi:
\begin{align*}
g_L\circ f\circ g_R&=(g_L\circ f)\circ g_R=\id_A\circ g_R=g_R\\
g_L\circ f\circ g_R&=g_L\circ (f\circ g_R)=g_L\circ\id_B=g_L.
\end{align*}
Ale z toho zrejme vyplýva, že $g_L=g_R$.
\end{proof}
Všimnime si, že v dôkaze predošlej vety sme ukázali aj čosi navyše: pokiaľ $f$ je bijekcia, nielenže má
nejaké inverzné zobrazenie, ale toto inverzné zobrazenie je dokonca presne jedno. Z toho vyplýva, že má zmysel
zaviesť operáciu ,,invertuj zobrazenie''
\[
f\mapsto f^{-1}
\]
ktorá bude definovaná iba ak $f$ je bijekcia. 

