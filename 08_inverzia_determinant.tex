\section{Determinant matice}

Determinant je reálne číslo, ktoré priradíme každej štvorcovej matici typu $n \times n$.
Jeho skutočný význam je nemožné vysvetliť v~tejto fáze, ničmenej intuitívne sa dá
pochopiť ako (až na znamienko) objem $n$-rozmerného rovnobežnostena vytýčeného stĺpcami matice.

% Placeholder pre obrázok z PDF
\begin{center}
\begin{tikzpicture}
    % Define the scaled coordinates
    \coordinate (O) at (0,0);
    \coordinate (A) at (1, 1.5);   % (a,b)
    \coordinate (D) at (2.5, 1.5); % Changed from (2.5, 0.5) to make angle smaller
    \coordinate (B) at ($ (A) + (D) $); % This will now be (3.5, 3.0)

    % Draw the axes (no arrows), adjusted for the new coordinates
    \draw (-0.5,0) -- (4,0) node[right] {};
    \draw (0,-0.5) -- (0,3.5) node[above] {}; % Adjusted y-axis limit

    % Draw the parallelogram and fill it with gray
    \fill[gray!30] (O) -- (A) -- (B) -- (D) -- cycle;
    \draw (O) -- (A) -- (B) -- (D) -- cycle;

    % Add labels for the coordinates in math mode
    \node[below left] at (O) {$(0,0)$};
    \node[above left] at (A) {$(a,b)$};
    \node[below right] at (D) {$(c,d)$};

    % Add arrows for the vectors with -Stealth style
    \draw[-Stealth, thick] (O) -- (A);
    \draw[-Stealth, thick] (O) -- (D);
\end{tikzpicture}
\end{center}

Pre $A = \begin{pmatrix} c & a \\ d & b \end{pmatrix}$, kde stĺpce sú vektory $(a,b)$
a $(c,d)$, je plocha rovnobežníka rovná $cb-ad$. Dôkaz obrázkom nasleduje.
\begin{center}
\begin{tikzpicture}
    % Define the scaled coordinates (smaller angle)
    \coordinate (O) at (0,0);
    \coordinate (A) at (1, 1.5);   
    \coordinate (D) at (2.5, 1.5); 
    \coordinate (B) at ($ (A) + (D) $); % (3.5, 3.0)

    % Draw the axes (no arrows)
    \draw (-0.5,0) -- (4,0) node[right] {}; 
    \draw (0,-0.5) -- (0,3.5) node[above] {}; 

    % --- Add specific dotted lines ---
    
    % Original line from A: Vertical line to x-axis
    \draw[dotted] (A) -- (A -| O); 
    
    % Original line from D: Horizontal line to y-axis
    \draw[dotted] (D) -- (D |- O); 
    
    % New line from A: Dotted vertical line upwards
    \draw[dotted] (A) -- (A |- 0, 3); 
    
    % New line from D: Dotted horizontal line to the right
    \draw[dotted] (D) -- (D -| 3.5,0); 
    
    % Both lines from B (vertical to x-axis, horizontal to y-axis)
    \draw[dotted] (B) -- (B -| O); % Vertical
    \draw[dotted] (B) -- (B |- O); % Horizontal
    % --- End of dotted lines ---

    % Draw the parallelogram and fill it with gray
    \fill[gray!30] (O) -- (A) -- (B) -- (D) -- cycle;
    \draw (O) -- (A) -- (B) -- (D) -- cycle;


    % Add arrows for the vectors with -Stealth style
    \draw[-Stealth, thick] (O) -- (A);
    \draw[-Stealth, thick] (O) -- (D);
    
    \node[below] at (1.25,0) {$c$};
    \node[below] at (3,0) {$a$};

    \node[right] at (3.5,0.75) {$d$};
    \node[right] at (3.5,2.25) {$b$};
    
    \node[left] at (0,0.75) {$b$};
    \node[left] at (0,2.25) {$d$};

    \node[above] at (0.5,3) {$a$};
    \node[above] at (2.25,3) {$c$};

    \node at (0.5,2.25) {$ad$};
    \node at (3,0.75) {$ad$};
    \node[above left] at (2.25,2.25) {$\frac{cd}{2}$};
    \node[below right] at (1.25,0.75) {$\frac{cd}{2}$};
    \node[above left,xshift=3pt] at (0.5,0.75) {$\frac{ab}{2}$};
    \node[below right,xshift=-3pt] at (3,2.25) {$\frac{ab}{2}$};
\end{tikzpicture}
\end{center}
\begin{multline*}
\text{plocha rovnobežníka}=(a+c).(b+d)-
(2\cdot\frac{ab}{2}+2\cdot\frac{cd}{2}+2ad)=\\
(ab+cb+ad+cd)-(ab+cd+2ad)=cb-ad
\end{multline*}
Toto funguje aj v~troch rozmeroch, pre objem rovobežnostena.
Skôr ako zadefinujeme determinant vo všeobecnosti, potrebujeme pomocný pojem.

\begin{definicia}[Minor]
Nech $A$ je matica typu $n \times n$. Minor $A_{ij}$ je matica typu $(n-1) \times (n-1)$, ktorá vznikne z~$A$ vynechaním $i$-teho riadku a $j$-teho stĺpca.
\end{definicia}

\begin{priklad}
Ak
\[A = \begin{pmatrix} 1 & 3 & 2 & -1 \\ 0 & 1 & 7 & -4 \\ 1 & 0 & 3 & 2 \\ 4 & 0 & 4
& 8 \end{pmatrix}\]
potom niektoré minory sú takéto
$$ A_{23} = \begin{pmatrix} 1 & 3 & -1 \\ 1 & 0 & 2 \\ 4 & 0 & 8 \end{pmatrix} \quad
 A_{11} = \begin{pmatrix} 1 & 7 & -4 \\ 0 & 3 & 2 \\ 0 & 4 & 8 \end{pmatrix} \quad
 A_{42} = \begin{pmatrix} 1 & 2 & -1 \\ 0 & 7 & -4 \\ 1 & 3 & 2 \end{pmatrix} $$
\end{priklad}

\begin{definicia}[Determinant]
Determinant matice $A$ typu $n \times n$ definujeme induktívne podľa $n$.
\begin{enumerate}[(1)]
\item Pre maticu $1 \times 1$: $\det((a)) = a$.
\item Nech $n > 1$, $A = (a_{ij})_{n \times n}$. Nech $i \in \{1, \dots, n\}$ je index nejakého riadku. Potom (toto je \textbf{rozvoj podľa $i$-teho riadku}):
$$ \det(A) = \sum_{j=1}^{n} (-1)^{i+j} a_{ij} \det(A_{ij}) $$
\end{enumerate}
\end{definicia}

V bode 2) môžeme pritom spraviť rozvoj podľa ľubovoľného riadku $i$, $\det(A)$ vyjde rovnako nezávisle na voľbe $i$. Toto dokazovať nebudeme.

Podobne môžeme v~bode 2) spraviť rozvoj podľa ľubovoľného stĺpca $j \in \{1, \dots, n\}$:
$$ \det(A) = \sum_{i=1}^{n} (-1)^{i+j} a_{ij} \det(A_{ij}) $$

Notácia: $\det(A)$ sa zvykne niekedy označovať ako $|A|$.

\subsection{Determinant matice 2x2}
\[
A = \begin{pmatrix} a_{11} & a_{12} \\ a_{21} & a_{22} \end{pmatrix}
\]
Rozvoj podľa riadku 1 nám dáva:
\begin{align*}
\det(A) &= (-1)^{1+1} a_{11} \det(A_{11}) + (-1)^{1+2} a_{12} \det(A_{21}) \\
&= 1 \cdot a_{11} \cdot (a_{22}) - 1 \cdot a_{12} \cdot (a_{21}) \\
&= a_{11}a_{22} - a_{12}a_{21}
\end{align*}

\begin{priklad}
$$ \det \begin{pmatrix} 1 & 2 \\ 2 & 3 \end{pmatrix} = 1 \cdot 3 - 2 \cdot 2 = 3 - 4 = -1 $$
\begin{align*}
\det \begin{pmatrix} 1 & 3 & 2 \\ 0 & 1 & -1 \\ 1 & 2 & 1 \end{pmatrix} &= (1\cdot 1 \cdot 1) + (3 \cdot (-1) \cdot 1) + (2 \cdot 0 \cdot 2) \\
&\quad - (2 \cdot 1 \cdot 1) - (1 \cdot (-1) \cdot 2) - (3 \cdot 0 \cdot 1) \\
&= 1 - 3 + 0 - 2 - (-2) - 0 = -2
\end{align*}
\end{priklad}

\subsection{Determinant matice 3x3}
Môžeme počítať rozvojom podľa riadku/stĺpca alebo použiť \textbf{Sarrusovo pravidlo} pre $A = (a_{ij})_{3 \times 3}$:
\begin{align*}
\det(A) = & (a_{11}a_{22}a_{33} + a_{12}a_{23}a_{31} + a_{13}a_{21}a_{32}) \\
& - (a_{13}a_{22}a_{31} + a_{11}a_{23}a_{32} + a_{12}a_{21}a_{33})
\end{align*}

\[
\begin{NiceArray}{(ccc)>{}c>{}c}
\CodeBefore [create-cell-nodes]
    \begin{tikzpicture} [shorten < = 2pt,shorten > = 2pt]
    \draw [red,->] (1-1) -- (3-3) ;
    \draw [red,->] (1-2) -- (3-4) ;
    \draw [red,->] (1-3) -- (3-5) ;
    \draw [blue,->] (1-3) -- (3-1) ;
    \draw [blue,->] (1-4) -- (3-2) ;
    \draw [blue,->] (1-5) -- (3-3) ;
    \end{tikzpicture}
\Body
    a_{11} & a_{12} & a_{13} & a_{11} & a_{12} \\[2mm]
    a_{21} & a_{22} & a_{23} & a_{21} & a_{22} \\[2mm]
    a_{31} & a_{32} & a_{33} & a_{31} & a_{32} \\
\end{NiceArray}
\]
Červené šípky sú kladný smer, modré záporný. Sarrusovo pravidlo síce vyzerá
jednoducho, ale v skutočnosti je zložitejšie ako počítať determinant rozvojom podľa
riadku alebo stĺpca. Vzorec pre Sarrusovo pravidlo má 12 násobení,
rozvoj podľa riadku má násobení 9.

\textbf{POZOR!} Sarrusovo pravidlo nefunguje na matice $4 \times 4$ a~väčšie!

\subsection{Vlastnosti determinantu}
Dôležitá vlastnosť determinantu je tá, že detekuje, kedy je matica singulárna.

\begin{veta}
Štvorcová matica $A$ je singulárna práve vtedy, keď $\det(A) = 0$.
\end{veta}

\begin{veta}
Nech $A, B$ sú štvorcové matice rovnakého typu. Potom:
\begin{enumerate}[(a)]
\item $\det(AB) = \det(A) \det(B)$
\item $\det(A^T) = \det(A)$
\item $\det(I) = 1$
\item Ak $A$ je regulárna, $\det(A^{-1}) = 1 / \det(A)$
\end{enumerate}
\end{veta}

\subsection{Elementárne riadkové operácie a determinant}
Nie je pravda, že elementárne riadkové operácie zachovávajú determinant. Avšak pravda je, že vieme presne povedať, ako menia determinant.

\begin{veta}
Nech $A$ je štvorcová matica.
\begin{enumerate}[(a)]
\item Ak $A'$ je matica, ktorá vznikla z $A$ pripočítaním skalárneho násobku niektorého riadku k~inému riadku, potom $\det(A') = \det(A)$.
\item Ak $A'$ je matica, ktorá vznikla z~matice $A$ vynásobením niektorého riadku skalárom $c \in \R$, potom $\det(A') = c \cdot \det(A)$.
\item Ak $A'$ je matica, ktorá vznikla z $A$ výmenou riadkov, potom $\det(A') = - \det(A)$.
\end{enumerate}
Analogické tvrdenie platí pre stĺpce a~elementárne stĺpcové operácie.
\end{veta}

\begin{priklad}
Vypočítajme determinant matice typu $4\times 4$
\[
A=(a_{ij})_{4\times 4}=
\begin{pmatrix}
-2 & 1 & -3 & \tikzmarknode{r1a}{-2} \\
-1 & -1 & 2 & 1 \\
-3 & -1 & 2 & 2 \\
-2 & 1 & -3 & \tikzmarknode{r4a}{-1}
\end{pmatrix}
\]
Samozrejme, môžeme použiť priamo vzorec na rozvoj podľa riadku,
ale to je nerozumné, vedie to na komplikovaný výpočet. Miesto toho použijeme
kombinovanú techniku: využijeme fakt, že pripočítanie násobku
nejakého riadku k inému determinant nemení.
\[
\det\begin{pmatrix}
-2 & 1 & -3 & \tikzmarknode{r1a}{-2} \\
-1 & -1 & 2 & 1 \\
-3 & -1 & 2 & 2 \\
-2 & 1 & -3 & \tikzmarknode{r4a}{-1}
\end{pmatrix}
\begin{tikzpicture}[remember picture, overlay]
    \draw[->, thick, shorten >=2pt]
        ([xshift=1.5em]r1a.east)
        -- ++(1em,0) coordinate (corner)
        -- (corner |- r4a.east)
        node [midway, right] {$-1$}
        -- ([xshift=1.5em]r4a.east);
\end{tikzpicture}
\hspace{3.5em} = \quad
\det
\begin{pmatrix}
-2 & 1 & -3 & -2 \\
-1 & -1 & 2 & 1 \\
-3 & -1 & 2 & 2 \\
0 & 0 & 0 & 1
\end{pmatrix}
\]
Teraz môžeme rozvinúť podľa posledného riadku:
\begin{multline*}
(-1)^{1+4}.0.\det(\dots)+(-1)^{2+4}.0.\det(\dots)+\\(-1)^{3+4}.0.\det(\dots)+(-1)^{4+4}.1.\det
\begin{pmatrix}
-2 & 1 & -3 \\
-1 & -1 & 2 \\
-3 & -1 & 2 
\end{pmatrix}
\end{multline*}
pričom prvé tri minory ani nemusíme písať, keďže ich determinanty vo vzorci sú
vynásobené $0$. Determinant $A$ je teda rovný
\[
\begin{pmatrix}
-2 & 1 & -3 \\
-1 & -1 & 2 \\
-3 & -1 & 2 
\end{pmatrix}
\]
Tento môžeme vypočítať Sarrusovým pravidlom, ale ide to aj šikovnejšie.
\[
\det
\begin{pmatrix}
-2 & 1 & -3 \\
-1 & -1 & \tikzmarknode{r2b}{2} \\
-3 & -1 & \tikzmarknode{r3b}{2}
\end{pmatrix}
\begin{tikzpicture}[remember picture, overlay]
    \draw[->, thick, shorten >=2pt]
        ([xshift=1.5em]r3b.east)
        -- ++(1em,0) coordinate (corner)
        -- (corner |- r2b.east)
        node [midway, right] {$-1$}
        -- ([xshift=1.5em]r2b.east);
\end{tikzpicture}
\hspace{3.5em} = \quad
\det
\begin{pmatrix}
-2 & 1 & -3 \\
2 & 0 & 0 \\
-3 & -1 & 2
\end{pmatrix}
\]
a rozvoj podľa druhého riadku nám dá (nulové členy vynechávame)
\begin{multline*}
\det
\begin{pmatrix}
-2 & 1 & -3 \\
2 & 0 & 0 \\
-3 & -1 & 2
\end{pmatrix}
=
(-1)^{2+1}.2.
\det
\begin{pmatrix}
1 & -3\\
-1 & 2
\end{pmatrix}
\\
=(-2).(1.2-(-3).(-1))=(-2).(2-3)=(-2).(-1)=2
\end{multline*}
\end{priklad}
Uvedomme si teraz, ako nám rozumné použitie elementárnych riadkových operácií v
predošlom príklade umožnilo zjednodušiť výpočet. Ak by sme spravili rozvoj podľa
riadku/stĺpca hneď, počet násobení ktorý bu sme museli vykonať by bol (pri použití
Sarrusovho pravidla) rovný $4*12=48$.

\subsection{Determinanty trojuholníkových matíc}
Horná trojuholníková matica je taká štvorcová matica, ktorá má pod diagonálou samé nuly. Analogicky definujeme dolnú trojuholníkovú maticu.
Všimnime si, že matica je diagonálna práve vtedy, keď je zároveň horná aj dolná trojuholníková.

\begin{veta}
Determinant hornej trojuholníkovej matice je súčin jej diagonálnych prvkov.
\end{veta}
\begin{proof}
Tvrdenie platí pre matice $1\times 1$. 

Nech $k\in\Nat$, $k>1$ a nech $A=(a_{ij})_{k\times k}$ je horná trojuholníková
matica. Predpokladajme, že tvrdenie vety platí pre 
horné trojuholníkové matice rozmeru $k-1$.

Rozvojom podľa prvého stĺpca dostaneme
\[
\det(A)=a_{11}\det(A_{11})
\]
ale pre minor $A_{11}\in\R^{(k-1)\times(k-1)}$ tvrdenie platí
podľa predpokladu, teda
\[
\det(A_{11})=a_{22}.\dots.a_{nn}
\]
Teda
\[
\det(A)=a_{11}.\det(A_{11})=a_{11}.a_{22}.\dots.a_{nn}
\]
\end{proof}

\begin{priklad}
$$ \det \begin{pmatrix} 1 & 3 & 2 & 0 \\ 0 & -1 & 5 & 8 \\ 0 & 0 & 2 & -1 \\ 0 & 0 & 0 & 7 \end{pmatrix} = 1 \cdot (-1) \cdot 2 \cdot 7 = -14 $$
\end{priklad}

\subsection{Cramerovo pravidlo}
Nech $A$ je regulárna matica typu $n \times n$. Uvažujme sústavu $n$ lineárnych rovníc o~$n$ neznámych $A\ora{x} = \ora{b}$.
Označme $A_i$ maticu, ktorá vznikne nahradením $i$-teho stĺpca matice $A$ pravou stranou $\ora{b}$.
Potom sústava má jediné riešenie $\ora{x} = (x_1, \dots, x_n)$, kde
$$ x_i = \frac{\det(A_i)}{\det(A)} $$

\subsection{Invertovanie matíc pomocou determinantov}
Nech $A$ je matica typu $n \times n$. \emph{Kofaktorová matica} $C$ je matica typu
$n \times n$ definovaná ako $C = (c_{ij})$, kde:
$$ c_{ij} = (-1)^{i+j} \det(A_{ij}) $$

\begin{veta}
Nech $A$ je regulárna matica. Potom
$$ A^{-1} = \frac{1}{\det(A)} C^T $$
kde $C$ je kofaktorová matica $A$.
\end{veta}

\begin{priklad}
Nájdime inverznú maticu k $A = \begin{pmatrix} 1 & 2 \\ 3 & 4 \end{pmatrix}$.
\begin{enumerate}[Krok 1:]
    \item Determinant: $\det(A) = 1 \cdot 4 - 2 \cdot 3 = 4 - 6 = -2$.
    \item Kofaktory:
    \begin{itemize}
        \item $c_{11} = (-1)^{1+1} \det(A_{11}) = 1 \cdot \det((4)) = 4$
        \item $c_{12} = (-1)^{1+2} \det(A_{12}) = -1 \cdot \det((3)) = -3$
        \item $c_{21} = (-1)^{2+1} \det(A_{21}) = -1 \cdot \det((2)) = -2$
        \item $c_{22} = (-1)^{2+2} \det(A_{22}) = 1 \cdot \det((1)) = 1$
    \end{itemize}
    \item Kofaktorová matica: $C = \begin{pmatrix} 4 & -3 \\ -2 & 1 \end{pmatrix}$.
    \item Transponovaná matica kofaktorov: $C^T = \begin{pmatrix} 4 & -2 \\ -3 & 1 \end{pmatrix}$.
    \item Inverzná matica: $A^{-1} = \frac{1}{\det(A)} C^T = \frac{1}{-2} \begin{pmatrix} 4 & -2 \\ -3 & 1 \end{pmatrix} = \begin{pmatrix} -2 & 1 \\ 3/2 & -1/2 \end{pmatrix}$.
\end{enumerate}
\end{priklad}
Poznamenajme, že tento spôsob počítania inverznej matice je veľmi
neefektívny v porovnaní s eliminačnou metódou pre ľubovoľnú maticu väčšiu ako
$2\times 2$.
